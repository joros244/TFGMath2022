\documentclass[12pt]{article}
\usepackage[utf8]{inputenc}
\usepackage[spanish,es-noshorthands]{babel}

\usepackage{amsmath,amssymb,amsthm}
\usepackage{amsfonts}
\usepackage{authblk}
\usepackage{mathtools}

\usepackage{graphicx}
\graphicspath{{Images/}}
\usepackage{float}
\usepackage[a4paper, total={426pt,674pt}]{geometry}
\usepackage{tikz}
\usetikzlibrary{positioning}
\usetikzlibrary{babel}
\pgfdeclarelayer{background}
\pgfdeclarelayer{foreground}
\pgfsetlayers{background,main,foreground}

\numberwithin{equation}{section}

\theoremstyle{definition}
\newtheorem{defi}{Definición}
\newtheorem{ejem}{Ejemplo}

\theoremstyle{remark}
\newtheorem*{remark}{Nota}

\theoremstyle{plain}
\newtheorem{prop}{Proposición}

\title{Sobre la homología persistente en redes neuronales}


\author{José Manuel Ros Rodrigo}

\affil{Facultad de Ciencia y Tecnología\\
  Universidad de La Rioja}

\date{Marzo 2022}

\begin{document}
	
	\maketitle

	\section{Nociones teóricas previas}
	
		Comenzamos dando algunas nociones básicas que nos serán muy útiles a lo largo del trabajo.
	
	\subsection{Complejos simpliciales}
		
		\begin{defi}
			Un \textit{complejo simplicial o complejo simplicial abstracto} es una colección, $\mathcal{K}$, de subconjuntos no vacíos de un conjunto, $\mathcal{K}_{0}$, 
			tal que verifica las siguientes propiedades:
			\begin{enumerate}
				\item $v \in \mathcal{K}_{0} \implies \{v\} \in \mathcal{K}$
				\item $ \sigma \in \mathcal{K} \text{ y } \tau \subset \sigma \implies \tau \in \mathcal{K}$	
			\end{enumerate}
			A los elementos de  $\mathcal{K}_{0}$ los llamamos \textit{vértices de $\mathcal{K}$} y a los elementos de $\mathcal{K}$ los llamamos \textit{símplices}. 
		\end{defi}

		\begin{remark}
			Diremos que un símplice tiene \textit{dimensión p o es un p-símplice} si tiene longitud $p+1$, y definimos la \textit{dimensión de $\mathcal{K}$} como el máximo
			de las dimensiones de sus símplices. Denotaremos por $\mathcal{K}_{p}$ a la colección de los p-símplices de $\mathcal{K}$.
		\end{remark}
			
		\begin{defi}
			Sean $\sigma$ y $\tau$ dos símplices de $\mathcal{K}$ tales que $\tau \subset \sigma$. Entonces diremos que $\tau$ es una \textit{cara} de $\sigma$, y
			si las dimensiones de $\sigma$ y $\tau$ difieren por a, diremos que $\tau$ es una cara de $\sigma$ de \textit{codimensión a}.
		\end{defi}

		\begin{defi}
			Una \textit{aplicación entre complejos simpliciales}, $f:\mathcal{K} \rightarrow \mathcal{L}$, es una aplicación $f:\mathcal{K}_{0} \rightarrow \mathcal{L}_{0}$ tal
			que $f(\sigma) \in \mathcal{L} \hspace{4pt} \forall \sigma \in \mathcal{K}$.
		\end{defi}
		
		Veamos un pequeño ejemplo para ilustrar todas estas nociones.

		\begin{ejem}
			Ejemplo conveniente de complejos simpliciales. (En construcción).	
		\end{ejem}

	\subsection{Homología}	

		Habiendo interiorizado los conceptos previos, vamos a usarlos en nuestros siguientes propósitos. Así pues, consideremos $\mathbb{F}_{2}$ el cuerpo de dos elementos. Dado un
		complejo simplicial $\mathcal{K}$, denotaremos por $C_{p}(\mathcal{K})$ al $\mathbb{F}_{2}$-espacio vectorial cuya base viene dada por los p-símplices de $\mathcal{K}$. Ahora,
		para cualquier $p \in \{1,2,...\}$ definimos la siguiente aplicación: 
		\begin{flalign*}
			& \left.
			\begin{array}{l}
				\partial_{p}:C_{p}(\mathcal{K}) \rightarrow C_{p-1}(\mathcal{K})\\[2pt] 
				\hspace{1.5cm} \sigma \mapsto \displaystyle \sum_{\mathclap{\tau \subset \sigma,\tau \in \mathcal{K}_{p-1}}}\tau
			\end{array}
			\right. &
		\end{flalign*}
		
		Si $p=0$ entonces definimos $\partial_{0}=0$. Observemos que $\partial_{p}$ le asigna a cada p-símplice su borde, esto es, la suma de sus caras de codimensión 1. Adicionalmente, podemos
		observar que $\forall p \in \{0,1,2,..\}$, $\partial_{p}\circ \partial_{p+1}=0$. De manera informal decimos que el borde del borde es vacío. Veamos un ejemplo para clarificar esta noción.

		\begin{ejem}
		
			Ejemplo conveniente de ''el borde del borde es vacío''. (En construcción).

		\end{ejem}

		De la observación anterior se desprende que $Im(\partial_{p+1}) \subset Ker(\partial_{p})$. Esta propiedad motiva la siguiente definición.
		\begin{defi}
			Sea $p \in \{0,1,...\}$. Definimos la \textit{p-ésima homología} de un complejo simplicial $\mathcal{K}$ como el espacio vectorial cociente:	
			\begin{flalign*}
				& \left.
				H_{p}(\mathcal{K}):=Ker(\partial_{p})/Im(\partial_{p+1})
				\right. &
			\end{flalign*}
		\end{defi}

		\begin{remark}
			A la dimensión de $H_{p}(\mathcal{K})$ la denotaremos por $B_{p}(\mathcal{K}):=dim H_{p}(\mathcal{K})=dim Ker(\partial_{p})-dim Im(\partial_{p+1})$ y la llamaremos
			\textit{p-ésimo número de Betti}.\\
			A los elementos de $Im(\partial_{p+1})$ los llamaremos \textit{p-bordes}, y a los de $Ker(\partial_{p})$ \textit{p-ciclos}.
		\end{remark}

		De la anterior consideración, deducimos que el p-ésimo número de Betti representa el número de agujeros p-dimensionales de K. Además, si $\mathcal{K}$ es un complejo simplicial
		de dimensión n, entonces $\forall p > n,\hspace{0.1cm} H_{p}(\mathcal{K})=0$ ya que $\mathcal{K}_{p}=\emptyset$. Esto nos lleva a la construcción de la siguiente cadena de 
		espacios vectoriales:
		\begin{flalign*}
			& \left.
			\begin{array}{l}
				\hspace{0.1cm} \partial_{n+1} \hspace{1.1cm} \partial_{n} \hspace{1.4cm} \partial_{n-1} \hspace{0.9cm} \partial_{2} \hspace{1.3cm} \partial_{1} \hspace{1.2cm} \partial_{0}\\
				0 \rightarrow C_{n}(\mathcal{K}) \rightarrow C_{n-1}(\mathcal{K}) \rightarrow \hspace{0.25cm} ... \hspace{0.25cm} \rightarrow C_{1}(\mathcal{K}) \rightarrow C_{0}(\mathcal{K}) \rightarrow 0
			\end{array}
			\right. &
		\end{flalign*}

		\begin{ejem}
			Ejemplo conveniente de homología simplicial. (En construcción).
		\end{ejem}	

	\subsection{Aplicaciones entre complejos simpliciales}

		Consideremos una aplicación entre complejos simpliciales, $f:\mathcal{K}\rightarrow\mathcal{K}'$. Tal aplicación induce una aplicación lineal en $\mathbb{F}_{2}$:
		\begin{flalign*}
			& \left.
			\begin{array}{l}
				\overline{f_{p}}:C_{p}(\mathcal{K})\rightarrow C_{p}(\mathcal{K}')\\[2pt] 
				\hspace{0.45cm} \displaystyle \sum_{\mathclap{\sigma \in \mathcal{K}_{p}}}a_{\sigma}\sigma \mapsto 
				\displaystyle \sum_{\mathclap{f(\sigma) \in \mathcal{K}'}}a_{\sigma}f(\sigma)
			\end{array}
			\right. ; \hspace{0.25cm} a_{\sigma} \in \mathbb{F}_{2} &
		\end{flalign*}
		
		Observamos también que $\overline{f_{p}}\circ\partial_{p+1}=\partial_{p+1}'\circ\overline{f_{p+1}}$, y en consecuencia, $\overline{f_{p}}$ induce una aplicación lineal
		entre espacios vectoriales de homología:	
		\begin{flalign*}
			& \left.
			\begin{array}{l}
				f_{p}:H_{p}(\mathcal{K})\rightarrow H_{p}(\mathcal{K}')\\[2pt] 
				\hspace{1.35cm} [x] \mapsto [\overline{f_{p}(x)}]
			\end{array}
			\right.&
		\end{flalign*}

		Concluimos que, dada una aplicación $f$ entre complejos simpliciales, siempre es posible asociarle una aplicación $f_{p}$ entre espacios vectoriales de homología.
	
	\subsection{Homología persistente}	
		
		Tras todos los conceptos previos, ya estamos en una buena posición para definir el concepto central del trabajo: \textbf{la homología persistente}. Sin embargo, aún
		vamos a necesitar una definición más.

		\begin{defi}
		
			Sea $\mathcal{K}$ un complejo simplicial finito y $\mathcal{K}_{1}\subset\mathcal{K}_{2}\subset...\subset\mathcal{K}_{n-1}\subset\mathcal{K}_{n}=\mathcal{K}$
			una cadena de subcomplejos simpliciales de $\mathcal{K}$. Al complejo simplicial $\mathcal{K}$ y a su cadena asociada los llamaremos \textit{complejo simplicial filtrado}.

		\end{defi}

		Ahora, y sin más dilación, introducimos el concepto que da nombre a este trabajo.

		\begin{defi}
		
			Sea $\mathcal{K}_{1}\subset\mathcal{K}_{2}\subset...\subset\mathcal{K}_{n-1}\subset\mathcal{K}_{n}=\mathcal{K}$ un complejo simplicial filtrado. La 
			\textit{p-ésima homología persistente} de $\mathcal{K}$ es el par:
			\begin{flalign*}
				& \left.	
				(\{H_{p}(\mathcal{K}_{i})\}_{1\leq i \leq r},\{f_{i,j}\}_{1\leq i\leq j \leq r})
				\right. &
			\end{flalign*}
			Donde $\forall i,j \in \{1,2,...,r\}$ con $i \leq j$, las aplicaciones lineales 
			$f_{i,j}:H_{p}(\mathcal{K}_{i})\rightarrow H_{p}(\mathcal{K}_{j})$ son las inducidas por las inclusiones $\mathcal{K}_{i}\xhookrightarrow{} \mathcal{K}_{j}$.
	

		\end{defi}
		
		

		De la definición anterior, observamos que el concepto de homología persistente es más ``rico" que el de homología, ya que, la homología peristente nos ofrece más
		información acerca de un complejo simplicial filtrado que la consideración de la homología de los subcomplejos simpliciales. Para visualizar la homolgía persistente
		nos serán de mucha utilidad los \textit{diagramas de persistencia}, así como los \textit{códigos de barras}.\\
		Veamos un ejemplo que ilustre lo que hemos definido.
		
		\begin{ejem}	
			Ejemplo adecuado de complejo simplicial filtrado + código de barras + diagrama de persistencia. (En construcción).
		\end{ejem}

	\section{Homología peristente en redes neuronales}

		En la sección anterior hemos discutido todas las cuestiones 
		acerca de la homología persistente en complejos simpliciales. 
		Pero, ¿Cómo aplica toda esta teoría a las redes neuronales? \\
		\\
		Para responder a esta cuestión debemos pensar en las redes 
		neuronales como si fueran grafos, cuyos nodos serán las neuronas 
		de la red, y las aristas, las relaciones entre las neuronas.De 
		este modo podremos construir complejos simpliciales asociados 
		al grafo y aplicarles toda la teoría que ya conocemos. \\
		\\
		A lo largo de esta sección veremos con detalle como hacer esta 
		correspondencia, así como algunos ejemplos ilustrativos de los 
		conceptos que irán apareciendo.\\
		\\
		(En el futuro habrá que añadir alguna disquisición previa sobre redes neuronales a la sección anterior).

	\subsection{Construcción de un complejo simplicial a partir de una red neuronal}

		Consideraremos el conjunto de las neuronas como el conjunto de vértices, es decir, $\mathcal{K}_{0}=\{k_{0},...,k_{n}\}$ con n+1 el número de neuronas. Pensando de esta manera, vemos las redes
		neuronales como grafos dirigidos con pesos $\omega_{ij}$, con $\omega_{ij}$ el peso entre $k_{i}$ y $k_{j}$. Notemos que $\omega_{ij}=0$ si y sólo si $k_{i}$ y $k_{j}$ no están conectadas.
		Con esto en mente, definimos la importancia de $k_{i}$ (salida) para $k_{j}$ (llegada) como:
		\begin{equation}
			R_{ij}=  \left \{ 
				\begin{array}{ll}
					1 & \text{si } i=j \\
					\omega_{ij}^{+}/\sum_{k,k \neq j}\omega_{kj}^{+} & \text{si } i \neq j
				\end{array}
			\right. 
		\end{equation}
		Donde $\omega_{ij}^{+}$ es la parte positiva del peso $\omega_{ij}$, esto es, $\omega_{ij}^{+}:=\text{máx}\{0,\omega_{ij}\}$.\\ 
		Observamos que la importancia de una neurona para sí misma es de 1, y la importancia entre neuronas distintas es la proporción del peso entre ellas con respecto 
		al resto de pesos de la neurona de llegada.\\
		(La elección de $\omega_{ij}^{+}$ se toma motivada por la \textit{regla-}$z^{+}$ definida en la 
		\textit{``descomposición profunda de Taylor"}.Preguntar). \\
		\\
		Para poder definir los complejos simpliciales tenemos que extender la definición de la importancia entre neuronas, para aquellas que no estén directamente conectadas. Consideremos 
		las neuronas $k_{0}$ y $k_{2}$ conectadas por el camino: $k_{2} \rightarrow k_{1} \rightarrow k_{0}$, la importancia de $k_{2}$ para $k_{0}$ es, según el camino entre ellas, $R_{21} \cdot R_{10}$.
		Por lo tanto, definimos:
		\begin{equation}
			\overline{R_{ij}}=\text{máx}\{R_{k_{i}k_{m_{1}}} \cdot\cdot\cdot R_{k_{m_{n}}k_{j}} \mid (k_{i},k_{m_{1}},...,k_{m_{n}},k_{j}) \in C_{ij} \}
			\label{def:ie}
		\end{equation}	
		Donde $C_{ij}$ denota el conjunto de todos los posibles caminos de $k_{i}$ a $k_{j}$. \\
		(Se puede definir $\overline{R_{ij}}$ considerando varios caminos en $C_{ij}$. Elegimos el máximo 
		por eficiencia computacional. Preguntar). \\
		\\
		De aquí en adelante, numeraremos las neuronas de una red neuronal en orden ascendente, desde las neuronas de llegada hasta las de salida.\\
		
		\newpage
		
		Veamos un ejemplo sencillo para interiorizar estas definiciones que serán clave a lo largo de esta sección.
		\begin{ejem}
			\label{ej:primerRel}	
			Supongamos la siguiente representación de una red neuronal con sus correspondientes pesos: 	
			\begin{figure}[H]
				\centering
				\begin{tikzpicture}[
					roundnode/.style={circle, draw=black, thick, fill=white, minimum size=7mm},
					]
					%Nodes
					\node[roundnode]      (n0) 	at 	(2,0)           {0};
					\node[roundnode]      (n1)      at 	(0,1)		{1};
					\node[roundnode]      (n2)      at 	(0,-1)		{2};
					\node[roundnode]      (n3)      at 	(-2,0)		{3};
					
					\begin{pgfonlayer}{background}
					%Lines
					\draw[thick,->] (n3.mid) -- node[above,sloped] {6.9} (n1.west);
					\draw[thick,->] (n3.mid) -- node[below,sloped] {5.8} (n2.west);
					\draw[thick,->] (n1.mid) -- node[above,sloped] {2.6} (n0.west);
					\draw[thick,->] (n2.mid) -- node[below,sloped] {3.9} (n0.west);
					
					\end{pgfonlayer}
				\end{tikzpicture}
				\caption{Representación de una red neuronal de 4 neuronas y 1 capa.}
			\end{figure}
			
			Tal y como vemos, las neuronas ya han sido ordenadas de manera correcta. Además en este caso todos los pesos son positivos, 
			por lo que no nos tenemos que preocupar escoger la parte positiva. Vamos a calcular algunas importancias entre neuronas:			
			\begin{flalign*}
           			& \left.
            				\begin{array}{l}
                				R_{31}=\frac{6.9}{6.9}=1 \hspace{0.5cm} R_{32}=\frac{5.8}{5.8}=1 \\[3pt]
						R_{10}=\frac{2.6}{6.5}=0.4 \hspace{0.55cm}  R_{20}=\frac{3.9}{6.5}=0.6 \\[3pt]
						\overline{R_{30}}=\text{máx}\{R_{31} \cdot R_{10}, R_{32}\cdot R_{20}\}=R_{32} \cdot R_{20}=0.6  
            				\end{array}
            			\right. & 
        		\end{flalign*}
			Este ejemplo pone de manifiesto la intuición detrás de la definición de la importancia entre neuronas. Lo que hace es medir 
			la aportación de la neurona emisora, $k_{i}$, con respecto al resto de neuronas emisoras de $k_{j}$.
		\end{ejem}
		Teninedo en cuenta la definición de $\overline{R_{ij}}$, y el orden en una red, ya podemos construir un complejo simplicial filtrado. 
		En primer lugar, definimos los p-símplices a partir de $\mathcal{K}_{0}$ como sigue:
		\begin{equation}
           		\mathcal{K}_{p}^{t}=
			\left \{
				\begin{array}{ll}
					\mathcal{K}_{0} & \text{si } p=0\\[3pt]
            				\{(k_{a_{0}},...,k_{a_{p}}) \mid k_{a_{s}} \in \mathcal{K}_{0}, \overline{R_{a_{s}a_{r}}} \geq t, 
					\forall a_{s} > a_{r} \} & \text{si } p \geq 1
				\end{array}
            		\right.  
			\label{def:sn}
        	\end{equation}
		Donde $0 \leq t \leq 1$ es un parámetro. \\
		\\
		Notemos que de la construcción que hemos hecho de los complejos simpliciales caben dos interpretaciones: la primera consite en tomar 
		un símplice si la importancia entre dos vértices calculada \emph{localmente} pasa el filtro $t$; la segunda se distingue 
		de la primera en que la importancia entre dos vértices se calcula \emph{globalmente}.\\
		\\
		Veamos la diferencia entre ambas interpretaciones con el siguiente ejemplo.
		\begin{ejem}
			\label{ej:int}
			Supongamos la siguiente representación de una red neuronal con las importancias entre nueronas ya calculadas: 	
			\begin{figure}[H]
				\centering
				\begin{tikzpicture}[
					roundnode/.style={circle, draw=black, thick, fill=white, minimum size=7mm},
					]
					%Nodes
					\node[roundnode]      (n0) 	at 	(2,0)           {0};
					\node[roundnode]      (n1)      at 	(0,1)		{1};
					\node[roundnode]      (n2)      at 	(0,-1)		{2};
					\node[roundnode]      (n4)      at 	(-2,-1)		{4};
					\node[roundnode]      (n3)      at 	(-2,1)		{3};
					
					\begin{pgfonlayer}{background}
					%Lines
					\draw[thick,->] (n3.mid) -- node[above] {0.5} (n1.west);
					\draw[thick,->] (n4.mid) -- node[above,sloped] {0.5} (n1.west);
					\draw[thick,->] (n4.mid) -- node[below] {1} (n2.west);
					\draw[thick,->] (n1.mid) -- node[above,sloped] {0.4} (n0.west);
					\draw[thick,->] (n2.mid) -- node[below,sloped] {0.6} (n0.west);
					
					\end{pgfonlayer}
				\end{tikzpicture}
				\caption{Representación de una red neuronal de 5 neuronas y 1 capa.}
			\end{figure}
			Vamos calcular $\mathcal{K}^{0.4}_{2}$ para ver las diferencias entre las interpretaciones. En primer lugar, 
			listamos los posibles 2-símplices: 
			$$
			\{4,2,0\},\{4,1,0\},\{3,1,0\}
			$$
			\begin{itemize}
				\item{Interpretación local}\\
					Como la importancia entre 4-2, 2-0 y 4-0 es mayor que 0.4 $\Rightarrow\{4,2,0\} \in \mathcal{K}^{0.4}_{2}$,
					donde importancia entre 4-0 viene dada por:
					$$\overline{R_{40}}=\text{máx}\{R_{42} \cdot R_{20}\}=R_{42} \cdot R_{20}=0.6$$
					Siguiendo la misma regla tenemos que $\{4,1,0\},\{3,1,0\} \notin \mathcal{K}^{0.4}_{2}$. Por lo tanto,
					$\mathcal{K}^{0.4}_{2}=\{\{4,2,0\}\}$.
				\item{Interpretación global}\\
					Razonando igual que antes,$\hspace{0.15cm}\{4,2,0\} \in \mathcal{K}^{0.4}_{2}$. Ahora bien, como la importancia
					entre 4-1 y 1-0 es mayor o igual que 0.4, y 
					$$\overline{R_{40}}=\text{máx}\{R_{42} \cdot R_{20},R_{41} \cdot R_{10}\}=R_{42} \cdot R_{20}=0.6$$ 
					Entonces $\{4,1,0\} \in \mathcal{K}^{0.4}_{2}$. Sin embargo, $\{3,2,0\} \notin \mathcal{K}^{0.4}_{2}$ y así,\\ 
					$\mathcal{K}^{0.4}_{2}=\{\{4,2,0\},\{4,1,0\}\}$
			\end{itemize}
			El ejemplo pone de manifiesto la principal diferencia entre ambas interpretaciones: en la primera, el máximo se calcula sobre los caminos
			que aparecen en el p-símplice; en la segunda, el máximo se calcula sobre los caminos que aparecen en todos los p-símplices.
			\begin{remark}
				Mientras que la interpretación global es consistente para los 1-símplices, la interpretación local no lo es. Para el cálculo
				de los 1-símplices en la interpretación local se toma el máximo entre todos los posibles caminos. 
			\end{remark}
		\end{ejem}
		
		Ahora que ya tenemos definidos los p-símplices, vamos con la construcción del complejo simplicial abstracto. Para ello 
		damos el siguiente resultado:
		\begin{prop}
			Sea $\mathcal{K}_{0}=\{k_{0},...,k_{n}\}$ un conjunto finito, y $\{\omega_{ij}\}_{0\leq j \leq i \leq n}$ un 
			conjunto de números reales. Sea $\overline{R_{ij}}$ la importancia entre neuronas 
			definida en (\ref{def:ie}), y $\mathcal{K}_{p}^{t}$ los p-símplices definidos en (\ref{def:sn}) con t
			parámetro real entre 0 y 1. Entonces, \\ 
			$\mathcal{K}^{t}=\bigcup_{s=0}^{s=n}\mathcal{K}_{s}^{t}$ es un complejo simplicial. 
			\label{prop:cs}
		\end{prop}

		\newpage

		\begin{proof}
			Supongamos hipótesis generales. \\
			Para probar que $\mathcal{K}^{t}$ es un complejo simplicial debemos ver: 
			\begin{enumerate}
				\item $v \in \mathcal{K}_{0} \implies \{v\} \in \mathcal{K}^{t}$
				\item $ \sigma \in \mathcal{K}^{t} \land \tau \subset \sigma \implies \tau \in \mathcal{K}^{t}$	
			\end{enumerate}
			Notemos que la primera propiedad se deduce inmediatamente de (\ref{def:sn}) y de la definición 
			de $\mathcal{K}^{t}$. \\
			Así pues, vamos a probar la segunda.\\
			$\sigma=(k_{m_{0}},...,k_{m_{p}}) \in \mathcal{K}^{t} \implies \overline{R_{m_{i}m_{j}}}\geq t \hspace{5pt} 
			\forall m_{i} \geq m_{j}$. Ahora sea $\tau \subset \sigma$, entonces, \\
			$\tau = (k_{n_{0}},...,k_{n_{q}})$, y como $\{n_{0},...,n_{q}\} \subset \{m_{0},...,m_{p}\}$, se tendrá que $\overline{R_{n_{i}n_{j}}} \geq t, \\
			\forall n_{i} \geq n_{j}$. El resultado se sigue inmediatamente. 
		\end{proof}

		Ahora que ya tenemos construido nuestro complejo simplicial vamos a proceder a su filtración. Para ello será necesario
		el siguiente resultado:

		\begin{prop}
			Sea $(t_{i})_{i=1}^{n}$ una sucesión, monótona decreciente, de números reales entre 1 y 0. Entonces, 
			$\mathcal{K}_{0}=\emptyset$ y $\mathcal{K}_{i}=\mathcal{K}^{t_{i}}$ con $1\leq i \leq n$, es un complejo simplicial filtrado.   
		\end{prop}

		\begin{proof}
			Supongamos hipótesis generales.\\
			Por la proposición anterior, sabemos que $\mathcal{K}^{t_{n}}$ es un complejo simplicial. Ahora bien, \\
			$t_{i} > t_{j} \implies \mathcal{K}_{p}^{t_{i}} \subset \mathcal{K}_{p}^{t_{j}}$ por la definición (\ref{def:sn}). Entonces, 
			$\emptyset=\mathcal{K}_{0} \subset \mathcal{K}_{1} \subset \cdot \cdot \cdot \subset \mathcal{K}_{n}=\mathcal{K}^{t_{n}}$. Se sigue
			inmediatamente el resultado.
		\end{proof}

		Ahora que ya contamos con todas las herramientas necesarias, veamos un ejemplo completo en el que calculemos los números de Betti y 
		los diagramas correspondientes. Haremos el desarrollo del ejemplo siguiendo las dos interpretaciones vistas en el ejemplo \ref{ej:int}.  

		\begin{ejem}
			Supongamos la siguiente representación de una red neuronal con las importancias entre neuronas ya calculadas:	
			\begin{figure}[H]
					\centering
					\begin{tikzpicture}[
						roundnode/.style={circle, draw=black, thick, fill=white, minimum size=7mm},
						]
						%Nodes
						\node[roundnode]      (n0) 	at 	(4,1)           {0};
						\node[roundnode]      (n1) 	at 	(4,-1)          {1};
						\node[roundnode]      (n2) 	at 	(2,2)           {2};
						\node[roundnode]      (n3) 	at 	(2,0)           {3};
						\node[roundnode]      (n4)      at 	(2,-2)		{4};
						\node[roundnode]      (n5)      at 	(0,3)		{5};
						\node[roundnode]      (n6)      at 	(0,1)		{6};
						\node[roundnode]      (n7) 	at 	(0,-1)           {7};
						\node[roundnode]      (n8) 	at 	(0,-3)           {8};
						\node[roundnode]      (n9) 	at 	(-2,2)           {9};
						\node[roundnode]      (n10) 	at 	(-2,0)           {10};
						\node[roundnode]      (n11) 	at 	(-2,-2)           {11};
						
						\begin{pgfonlayer}{background}
						%Lines
						\draw[thick,->] (n9.mid) -- node[above,sloped] {1.0} (n5.west);
						\draw[thick,->] (n9.mid) -- node[above,sloped] {0.5} (n6.west);
						\draw[thick,->] (n10.mid) -- node[above,sloped] {0.5} (n6.west);
						\draw[thick,->] (n10.mid) -- node[above,sloped] {0.55} (n7.west);
						\draw[thick,->] (n11.mid) -- node[above,sloped] {0.45} (n7.west);
						\draw[thick,->] (n11.mid) -- node[above,sloped] {1.0} (n8.west);
						\draw[thick,->] (n5.mid) -- node[above,sloped] {0.4} (n2.west);
						\draw[thick,->] (n6.mid) -- node[above,sloped] {0.6} (n2.west);
						\draw[thick,->] (n6.mid) -- node[above,sloped] {0.58} (n3.west);
						\draw[thick,->] (n7.mid) -- node[above,sloped] {0.42} (n3.west);
						\draw[thick,->] (n7.mid) -- node[above,sloped] {0.56} (n4.west);
						\draw[thick,->] (n8.mid) -- node[above,sloped] {0.44} (n4.west);
						\draw[thick,->] (n2.mid) -- node[above,sloped] {0.57} (n0.west);
						\draw[thick,->] (n3.mid) -- node[above,sloped] {0.43} (n0.west);
						\draw[thick,->] (n3.mid) -- node[above,sloped] {0.59} (n1.west);
						\draw[thick,->] (n4.mid) -- node[above,sloped] {0.41} (n1.west);
						\end{pgfonlayer}
					\end{tikzpicture}
					\caption{Representación de una red neuronal de 12 neuronas y 2 capas.}
			\end{figure}
			Vamos a calcularle el complejo simplicial filtrado asociado. Para ello vamos a ilustrar unos cuantos pasos en la filtración
			con los correspondientes números de Betti asociados. También añadimos los correspondientes diagramas de persistencia y de barras
			realizados con \emph{GUDHI} y \emph{Dionysus}.
			\begin{itemize}
				\item{\textbf{Interpretación local}}
			\end{itemize}
			\begin{figure}[H]
				\fbox{\minipage{0.225\textwidth}
				    	\begin{figure}[H]
						\resizebox{1.0\textwidth}{!}{\begin{tikzpicture}[
							roundnode/.style={circle, draw=black, thick, fill=white, minimum size=7mm},
							]
							%Nodes
							\node           (label) 	at	(1,4) 		{Filtración 0 (t=1.0)};
							\node[roundnode]      (n0) 	at 	(4,1)           {0};
							\node[roundnode]      (n1) 	at 	(4,-1)          {1};
							\node[roundnode]      (n2) 	at 	(2,2)           {2};
							\node[roundnode]      (n3) 	at 	(2,0)           {3};
							\node[roundnode]      (n4)      at 	(2,-2)		{4};
							\node[roundnode]      (n5)      at 	(0,3)		{5};
							\node[roundnode]      (n6)      at 	(0,1)		{6};
							\node[roundnode]      (n7) 	at 	(0,-1)           {7};
							\node[roundnode]      (n8) 	at 	(0,-3)           {8};
							\node[roundnode]      (n9) 	at 	(-2,2)           {9};
							\node[roundnode]      (n10) 	at 	(-2,0)           {10};
							\node[roundnode]      (n11) 	at 	(-2,-2)           {11};
							\node 		(label2) 	at 	(1,-4) 		{$\beta_{0}=10,\beta_{1}=0$};
							
							\begin{pgfonlayer}{background}
							%Lines
							\draw[thick] (n11.mid) --  (n8);
							\draw[thick] (n9.mid) --  (n5);
							\end{pgfonlayer}
						\end{tikzpicture}}	
					\end{figure}	
				\endminipage}
				\fbox{\minipage{0.225\textwidth}
				    	\begin{figure}[H]
						\resizebox{1.0\textwidth}{!}{\begin{tikzpicture}[
							roundnode/.style={circle, draw=black, thick, fill=white, minimum size=7mm},
							]
							%Nodes
							\node           (label) 	at	(1,4) 		{Filtración 4 (t=0.6)};
							\node[roundnode]      (n0) 	at 	(4,1)           {0};
							\node[roundnode]      (n1) 	at 	(4,-1)          {1};
							\node[roundnode]      (n2) 	at 	(2,2)           {2};
							\node[roundnode]      (n3) 	at 	(2,0)           {3};
							\node[roundnode]      (n4)      at 	(2,-2)		{4};
							\node[roundnode]      (n5)      at 	(0,3)		{5};
							\node[roundnode]      (n6)      at 	(0,1)		{6};
							\node[roundnode]      (n7) 	at 	(0,-1)           {7};
							\node[roundnode]      (n8) 	at 	(0,-3)           {8};
							\node[roundnode]      (n9) 	at 	(-2,2)           {9};
							\node[roundnode]      (n10) 	at 	(-2,0)           {10};
							\node[roundnode]      (n11) 	at 	(-2,-2)           {11};
							\node 		(label2) 	at 	(1,-4) 		{$\beta_{0}=9,\beta_{1}=0$};
							
							\begin{pgfonlayer}{background}
							%Lines
							\draw[thick] (n11.mid) --  (n8);
							\draw[thick] (n9.mid) --  (n5);
							\draw[thick] (n6.mid) --  (n2);
							\end{pgfonlayer}
						\end{tikzpicture}}	
					\end{figure}	
				\endminipage}
				\fbox{\minipage{0.225\textwidth}
				    	\begin{figure}[H]
						\resizebox{1.0\textwidth}{!}{\begin{tikzpicture}[
							roundnode/.style={circle, draw=black, thick, fill=white, minimum size=7mm},
							]
							%Nodes
							\node           (label) 	at	(1,4) 		{Filtración 5 (t=0.5)};
							\node[roundnode]      (n0) 	at 	(4,1)           {0};
							\node[roundnode]      (n1) 	at 	(4,-1)          {1};
							\node[roundnode]      (n2) 	at 	(2,2)           {2};
							\node[roundnode]      (n3) 	at 	(2,0)           {3};
							\node[roundnode]      (n4)      at 	(2,-2)		{4};
							\node[roundnode]      (n5)      at 	(0,3)		{5};
							\node[roundnode]      (n6)      at 	(0,1)		{6};
							\node[roundnode]      (n7) 	at 	(0,-1)           {7};
							\node[roundnode]      (n8) 	at 	(0,-3)           {8};
							\node[roundnode]      (n9) 	at 	(-2,2)           {9};
							\node[roundnode]      (n10) 	at 	(-2,0)           {10};
							\node[roundnode]      (n11) 	at 	(-2,-2)           {11};
							\node 		(label2) 	at 	(1,-4) 		{$\beta_{0}=2,\beta_{1}=0$};
							
							\begin{pgfonlayer}{background}
							%Lines
							\draw[thick] (n11.mid) --  (n8);
							\draw[thick] (n9.mid) --  (n5);
							\draw[thick] (n6.mid) --  (n2);
							\draw[thick] (n10.mid) --  (n7);
							\draw[thick] (n10.mid) --  (n6);
							\draw[thick] (n9.mid) --  (n6);
							\draw[thick] (n7.mid) --  (n4);
							\draw[thick] (n6.mid) --  (n3);
							\draw[thick] (n3.mid) --  (n1);
							\draw[thick] (n2.mid) --  (n0);
							\end{pgfonlayer}
						\end{tikzpicture}}	
					\end{figure}	
				\endminipage}
				\fbox{\minipage{0.225\textwidth}
				    	\begin{figure}[H]
						\resizebox{1.0\textwidth}{!}{\begin{tikzpicture}[
							roundnode/.style={circle, draw=black, fill=white, thick, minimum size=7mm},
							]
							%Nodes
							\node           (label) 	at	(1,4) 		{Filtración 6 (t=0.4)};
							\node[roundnode]      (n0) 	at 	(4,1)           {0};
							\node[roundnode]      (n1) 	at 	(4,-1)          {1};
							\node[roundnode]      (n2) 	at 	(2,2)           {2};
							\node[roundnode]      (n3) 	at 	(2,0)           {3};
							\node[roundnode]      (n4)      at 	(2,-2)		{4};
							\node[roundnode]      (n5)      at 	(0,3)		{5};
							\node[roundnode]      (n6)      at 	(0,1)		{6};
							\node[roundnode]      (n7) 	at 	(0,-1)           {7};
							\node[roundnode]      (n8) 	at 	(0,-3)           {8};
							\node[roundnode]      (n9) 	at 	(-2,2)           {9};
							\node[roundnode]      (n10) 	at 	(-2,0)           {10};
							\node[roundnode]      (n11) 	at 	(-2,-2)           {11};
							\node 		(label2) 	at 	(1,-4) 		{$\beta_{0}=1,\beta_{1}=5$};
							
							\begin{pgfonlayer}{background}
							\draw[thick] (n11.mid) --  (n7);
							\draw[thick] (n10.mid) --  (n7);
							\draw[thick] (n10.mid) --  (n6);
							\draw[thick] (n9.mid) --  (n6);
							\draw[thick] (n7.mid) --  (n4);
							\draw[thick] (n7.mid) --  (n3);
							\draw[thick] (n6.mid) --  (n3);
							\draw[thick] (n6.mid) --  (n2);
							\draw[thick] (n4.mid) --  (n1);
							\draw[thick] (n3.mid) --  (n1);
							\draw[thick] (n3.mid) --  (n0);
							\draw[thick] (n2.mid) --  (n0);
						

							\filldraw[fill=black!20,opacity=1] (n11.mid) -- (n8.mid) -- (n4.mid) -- cycle;
							\filldraw[fill=black!20,opacity=1] (n9.mid) -- (n5.mid) -- (n2.mid) -- cycle;
							
							\end{pgfonlayer}
						\end{tikzpicture}}	
					\end{figure}	
				\endminipage}
    			\end{figure}
			\begin{figure}[H]
				\fbox{\minipage{0.225\textwidth}
				    	\begin{figure}[H]
						\resizebox{1.0\textwidth}{!}{\begin{tikzpicture}[
							roundnode/.style={circle, draw=black, thick, fill=white, minimum size=7mm},
							]
							%Nodes
							\node           (label) 	at	(1,4) 		{Filtración 7 (t=0.3)};
							\node[roundnode]      (n0) 	at 	(4,1)           {0};
							\node[roundnode]      (n1) 	at 	(4,-1)          {1};
							\node[roundnode]      (n2) 	at 	(2,2)           {2};
							\node[roundnode]      (n3) 	at 	(2,0)           {3};
							\node[roundnode]      (n4)      at 	(2,-2)		{4};
							\node[roundnode]      (n5)      at 	(0,3)		{5};
							\node[roundnode]      (n6)      at 	(0,1)		{6};
							\node[roundnode]      (n7) 	at 	(0,-1)           {7};
							\node[roundnode]      (n8) 	at 	(0,-3)           {8};
							\node[roundnode]      (n9) 	at 	(-2,2)           {9};
							\node[roundnode]      (n10) 	at 	(-2,0)           {10};
							\node[roundnode]      (n11) 	at 	(-2,-2)           {11};
							\node 		(label2) 	at 	(1,-4) 		{$\beta_{0}=1,\beta_{1}=4$};
							
							\begin{pgfonlayer}{background}							
						

							\fill[fill=black!20,opacity=1] (n11.mid) -- (n8.mid) -- (n4.mid) -- cycle;
							\fill[fill=black!20,opacity=1] (n10.mid) to[bend right] (n4.mid) -- (n7.mid) -- cycle;
							\fill[fill=black!20,opacity=1] (n10.mid) to[bend left] (n2.mid) -- (n6.mid) -- cycle;
							\fill[fill=black!20,opacity=1] (n9.mid) -- (n6.mid) -- (n2.mid) -- cycle;
							\fill[fill=black!20,opacity=1] (n9.mid) -- (n5.mid) -- (n2.mid) -- cycle;
							\fill[fill=black!20,opacity=1] (n6.mid) to[bend left] (n1.mid) -- (n3.mid) -- cycle;
							\fill[fill=black!20,opacity=1] (n6.mid) -- (n2.mid) -- (n0.mid) -- cycle;
							
							\draw[thick] (n11.mid) --  (n8);
							\draw[thick] (n11.mid) --  (n7);
							\draw[thick] (n11.mid) --  (n4.mid);
							\draw[thick] (n10.mid) --  (n7);
							\draw[thick] (n10.mid) --  (n6);
							\draw[thick] (n9.mid) --  (n6);
							\draw[thick] (n9.mid) --  (n5);
							\draw[thick] (n9.mid) --  (n2.mid);
							\draw[thick] (n8.mid) --  (n4);
							\draw[thick] (n7.mid) --  (n4);
							\draw[thick] (n7.mid) --  (n3);
							\draw[thick] (n6.mid) --  (n3);
							\draw[thick] (n6.mid) --  (n2);
							\draw[thick] (n6.mid) --  (n0);
							\draw[thick] (n5.mid) --  (n2);
							\draw[thick] (n4.mid) --  (n1);
							\draw[thick] (n3.mid) --  (n1);
							\draw[thick] (n3.mid) --  (n0);
							\draw[thick] (n2.mid) --  (n0);
							\end{pgfonlayer}
						\end{tikzpicture}}	
					\end{figure}	
				\endminipage}
				\fbox{\minipage{0.225\textwidth}
				    	\begin{figure}[H]
						\resizebox{1.0\textwidth}{!}{\begin{tikzpicture}[
							roundnode/.style={circle, draw=black, thick, fill=white, minimum size=7mm},
							]
							%Nodes
							\node           (label) 	at	(1,4) 		{Filtración 8 (t=0.2)};
							\node[roundnode]      (n0) 	at 	(4,1)           {0};
							\node[roundnode]      (n1) 	at 	(4,-1)          {1};
							\node[roundnode]      (n2) 	at 	(2,2)           {2};
							\node[roundnode]      (n3) 	at 	(2,0)           {3};
							\node[roundnode]      (n4)      at 	(2,-2)		{4};
							\node[roundnode]      (n5)      at 	(0,3)		{5};
							\node[roundnode]      (n6)      at 	(0,1)		{6};
							\node[roundnode]      (n7) 	at 	(0,-1)           {7};
							\node[roundnode]      (n8) 	at 	(0,-3)           {8};
							\node[roundnode]      (n9) 	at 	(-2,2)           {9};
							\node[roundnode]      (n10) 	at 	(-2,0)           {10};
							\node[roundnode]      (n11) 	at 	(-2,-2)           {11};
							\node 		(label2) 	at 	(1,-4) 		{$\beta_{0}=1,\beta_{1}=0$};
							
							\begin{pgfonlayer}{background}							
						

							\fill[fill=black!20,opacity=1] (n11.mid) -- (n8.mid) -- (n4.mid) -- cycle;
							\fill[fill=black!20,opacity=1] (n11.mid) -- (n7.mid) -- (n4.mid) -- cycle;
							\fill[fill=black!20,opacity=1] (n10.mid) to[bend right] (n4.mid) -- (n7.mid) -- cycle;
							\fill[fill=black!20,opacity=1] (n10.mid) -- (n7.mid) -- (n3.mid) -- cycle;
							\fill[fill=black!20,opacity=1] (n10.mid) -- (n6.mid) -- (n3.mid) -- cycle;
							\fill[fill=black!20,opacity=1] (n10.mid) to[bend left] (n2.mid) -- (n6.mid) -- cycle;
							\fill[fill=black!20,opacity=1] (n9.mid) to[bend right] (n3.mid) -- (n6.mid) -- cycle;
							\fill[fill=black!20,opacity=1] (n9.mid) -- (n6.mid) -- (n2.mid) -- cycle;
							\fill[fill=black!20,opacity=1] (n9.mid) -- (n5.mid) -- (n2.mid) -- cycle;
							\fill[fill=black!20,opacity=1] (n7.mid) -- (n4.mid) -- (n1.mid) -- cycle;
							\fill[fill=black!20,opacity=1] (n7.mid) -- (n3.mid) -- (n1.mid) -- cycle;
							\fill[fill=black!20,opacity=1] (n6.mid) to[bend left] (n1.mid) -- (n3.mid) -- cycle;
							\fill[fill=black!20,opacity=1] (n6.mid) -- (n3.mid) -- (n0.mid) -- cycle;
							\fill[fill=black!20,opacity=1] (n6.mid) -- (n2.mid) -- (n0.mid) -- cycle;
							\fill[fill=black!20,opacity=1] (n5.mid) to[bend left] (n0.mid) -- (n2.mid) -- cycle;
							
							\draw[thick] (n11.mid) --  (n8.mid);
							\draw[thick] (n11.mid) --  (n7.mid);
							\draw[thick] (n11.mid) --  (n4.mid);
							\draw[thick] (n10.mid) --  (n7.mid);
							\draw[thick] (n10.mid) --  (n6.mid);
							\draw[thick] (n10.mid) --  (n3.mid);
							\draw[thick] (n9.mid) --  (n6.mid);
							\draw[thick] (n9.mid) --  (n5.mid);
							\draw[thick] (n9.mid) --  (n2.mid);
							\draw[thick] (n8.mid) --  (n4.mid);
							\draw[thick] (n7.mid) --  (n4.mid);
							\draw[thick] (n7.mid) --  (n3.mid);
							\draw[thick] (n7.mid) --  (n1.mid);
							\draw[thick] (n6.mid) --  (n3.mid);
							\draw[thick] (n6.mid) --  (n2.mid);
							\draw[thick] (n6.mid) --  (n0.mid);
							\draw[thick] (n5.mid) --  (n2.mid);
							\draw[thick] (n4.mid) --  (n1.mid);
							\draw[thick] (n3.mid) --  (n1.mid);
							\draw[thick] (n3.mid) --  (n0.mid);
							\draw[thick] (n2.mid) --  (n0.mid);
							\end{pgfonlayer}
						\end{tikzpicture}}	
					\end{figure}	
				\endminipage}
				\fbox{\minipage{0.225\textwidth}
				    	\begin{figure}[H]
						\resizebox{1.0\textwidth}{!}{\begin{tikzpicture}[
							roundnode/.style={circle, draw=black, thick, fill=white, minimum size=7mm},
							]
							%Nodes
							\node           (label) 	at	(1,4) 		{Filtración 9 (t=0.1)};
							\node[roundnode]      (n0) 	at 	(4,1)           {0};
							\node[roundnode]      (n1) 	at 	(4,-1)          {1};
							\node[roundnode]      (n2) 	at 	(2,2)           {2};
							\node[roundnode]      (n3) 	at 	(2,0)           {3};
							\node[roundnode]      (n4)      at 	(2,-2)		{4};
							\node[roundnode]      (n5)      at 	(0,3)		{5};
							\node[roundnode]      (n6)      at 	(0,1)		{6};
							\node[roundnode]      (n7) 	at 	(0,-1)           {7};
							\node[roundnode]      (n8) 	at 	(0,-3)           {8};
							\node[roundnode]      (n9) 	at 	(-2,2)           {9};
							\node[roundnode]      (n10) 	at 	(-2,0)           {10};
							\node[roundnode]      (n11) 	at 	(-2,-2)           {11};
							\node 		(label2) 	at 	(1,-4) 		{$\beta_{0}=1,\beta_{1}=0$};
							
							\begin{pgfonlayer}{background}							
						

							\fill[fill=black!20,opacity=1] (n11.mid) -- (n8.mid) -- (n4.mid) -- cycle;
							\fill[fill=black!20,opacity=1] (n11.mid) -- (n7.mid) -- (n4.mid) -- cycle;
							\fill[fill=black!20,opacity=1] (n11.mid) to[bend left] (n3.mid) -- (n7.mid) -- cycle;
							\fill[fill=black!20,opacity=1] (n10.mid) to[bend right] (n4.mid) -- (n7.mid) -- cycle;
							\fill[fill=black!20,opacity=1] (n10.mid) -- (n7.mid) -- (n3.mid) -- cycle;
							\fill[fill=black!20,opacity=1] (n10.mid) -- (n6.mid) -- (n3.mid) -- cycle;
							\fill[fill=black!20,opacity=1] (n10.mid) to[bend left] (n2.mid) -- (n6.mid) -- cycle;
							\fill[fill=black!20,opacity=1] (n9.mid) to[bend right] (n3.mid) -- (n6.mid) -- cycle;
							\fill[fill=black!20,opacity=1] (n9.mid) -- (n6.mid) -- (n2.mid) -- cycle;
							\fill[fill=black!20,opacity=1] (n9.mid) -- (n5.mid) -- (n2.mid) -- cycle;
							\fill[fill=black!20,opacity=1] (n8.mid) to[bend right] (n1.mid) -- (n4.mid) -- cycle;
							\fill[fill=black!20,opacity=1] (n7.mid) -- (n4.mid) -- (n1.mid) -- cycle;
							\fill[fill=black!20,opacity=1] (n7.mid) -- (n3.mid) -- (n1.mid) -- cycle;
							\fill[fill=black!20,opacity=1] (n7.mid) to[bend right] (n0.mid) -- (n3.mid) -- cycle;
							\fill[fill=black!20,opacity=1] (n6.mid) to[bend left] (n1.mid) -- (n3.mid) -- cycle;
							\fill[fill=black!20,opacity=1] (n6.mid) -- (n3.mid) -- (n0.mid) -- cycle;
							\fill[fill=black!20,opacity=1] (n6.mid) -- (n2.mid) -- (n0.mid) -- cycle;
							\fill[fill=black!20,opacity=1] (n5.mid) to[bend left] (n0.mid) -- (n2.mid) -- cycle;
							
							\draw[thick] (n11.mid) --  (n8.mid);
							\draw[thick] (n11.mid) --  (n7.mid);
							\draw[thick] (n11.mid) --  (n4.mid);
							\draw[thick] (n10.mid) --  (n7.mid);
							\draw[thick] (n10.mid) --  (n6.mid);
							\draw[thick] (n10.mid) --  (n3.mid);
							\draw[thick] (n9.mid) --  (n6.mid);
							\draw[thick] (n9.mid) --  (n5.mid);
							\draw[thick] (n9.mid) --  (n2.mid);
							\draw[thick] (n8.mid) --  (n4.mid);
							\draw[thick] (n7.mid) --  (n4.mid);
							\draw[thick] (n7.mid) --  (n3.mid);
							\draw[thick] (n7.mid) --  (n1.mid);
							\draw[thick] (n6.mid) --  (n3.mid);
							\draw[thick] (n6.mid) --  (n2.mid);
							\draw[thick] (n6.mid) --  (n0.mid);
							\draw[thick] (n5.mid) --  (n2.mid);
							\draw[thick] (n4.mid) --  (n1.mid);
							\draw[thick] (n3.mid) --  (n1.mid);
							\draw[thick] (n3.mid) --  (n0.mid);
							\draw[thick] (n2.mid) --  (n0.mid);
							\end{pgfonlayer}
						\end{tikzpicture}}	
					\end{figure}	
				\endminipage}
				\fbox{\minipage{0.225\textwidth}
				    	\begin{figure}[H]
						\resizebox{1.0\textwidth}{!}{\begin{tikzpicture}[
							roundnode/.style={circle, draw=black, thick, fill=white, minimum size=7mm},
							]
							%Nodes
							\node           (label) 	at	(1,4) 		{Filtración 10 (t=0.0)};
							\node[roundnode]      (n0) 	at 	(4,1)           {0};
							\node[roundnode]      (n1) 	at 	(4,-1)          {1};
							\node[roundnode]      (n2) 	at 	(2,2)           {2};
							\node[roundnode]      (n3) 	at 	(2,0)           {3};
							\node[roundnode]      (n4)      at 	(2,-2)		{4};
							\node[roundnode]      (n5)      at 	(0,3)		{5};
							\node[roundnode]      (n6)      at 	(0,1)		{6};
							\node[roundnode]      (n7) 	at 	(0,-1)           {7};
							\node[roundnode]      (n8) 	at 	(0,-3)           {8};
							\node[roundnode]      (n9) 	at 	(-2,2)           {9};
							\node[roundnode]      (n10) 	at 	(-2,0)           {10};
							\node[roundnode]      (n11) 	at 	(-2,-2)           {11};
							\node 		(label2) 	at 	(1,-4) 		{$\beta_{0}=1,\beta_{1}=0$};
							
							\begin{pgfonlayer}{background}							
						

							\fill[fill=black!20,opacity=1] (n11.mid) -- (n8.mid) -- (n4.mid) -- cycle;
							\fill[fill=black!20,opacity=1] (n11.mid) -- (n7.mid) -- (n4.mid) -- cycle;
							\fill[fill=black!20,opacity=1] (n11.mid) to[bend left] (n3.mid) -- (n7.mid) -- cycle;
							\fill[fill=black!20,opacity=1] (n10.mid) to[bend right] (n4.mid) -- (n7.mid) -- cycle;
							\fill[fill=black!20,opacity=1] (n10.mid) -- (n7.mid) -- (n3.mid) -- cycle;
							\fill[fill=black!20,opacity=1] (n10.mid) -- (n6.mid) -- (n3.mid) -- cycle;
							\fill[fill=black!20,opacity=1] (n10.mid) to[bend left] (n2.mid) -- (n6.mid) -- cycle;
							\fill[fill=black!20,opacity=1] (n9.mid) to[bend right] (n3.mid) -- (n6.mid) -- cycle;
							\fill[fill=black!20,opacity=1] (n9.mid) -- (n6.mid) -- (n2.mid) -- cycle;
							\fill[fill=black!20,opacity=1] (n9.mid) -- (n5.mid) -- (n2.mid) -- cycle;
							\fill[fill=black!20,opacity=1] (n8.mid) to[bend right] (n1.mid) -- (n4.mid) -- cycle;
							\fill[fill=black!20,opacity=1] (n7.mid) -- (n4.mid) -- (n1.mid) -- cycle;
							\fill[fill=black!20,opacity=1] (n7.mid) -- (n3.mid) -- (n1.mid) -- cycle;
							\fill[fill=black!20,opacity=1] (n7.mid) to[bend right] (n0.mid) -- (n3.mid) -- cycle;
							\fill[fill=black!20,opacity=1] (n6.mid) to[bend left] (n1.mid) -- (n3.mid) -- cycle;
							\fill[fill=black!20,opacity=1] (n6.mid) -- (n3.mid) -- (n0.mid) -- cycle;
							\fill[fill=black!20,opacity=1] (n6.mid) -- (n2.mid) -- (n0.mid) -- cycle;
							\fill[fill=black!20,opacity=1] (n5.mid) to[bend left] (n0.mid) -- (n2.mid) -- cycle;
							
							\draw[thick] (n11.mid) --  (n8.mid);
							\draw[thick] (n11.mid) --  (n7.mid);
							\draw[thick] (n11.mid) --  (n4.mid);
							\draw[thick] (n10.mid) --  (n7.mid);
							\draw[thick] (n10.mid) --  (n6.mid);
							\draw[thick] (n10.mid) --  (n3.mid);
							\draw[thick] (n9.mid) --  (n6.mid);
							\draw[thick] (n9.mid) --  (n5.mid);
							\draw[thick] (n9.mid) --  (n2.mid);
							\draw[thick] (n8.mid) --  (n4.mid);
							\draw[thick] (n7.mid) --  (n4.mid);
							\draw[thick] (n7.mid) --  (n3.mid);
							\draw[thick] (n7.mid) --  (n1.mid);
							\draw[thick] (n6.mid) --  (n3.mid);
							\draw[thick] (n6.mid) --  (n2.mid);
							\draw[thick] (n6.mid) --  (n0.mid);
							\draw[thick] (n5.mid) --  (n2.mid);
							\draw[thick] (n4.mid) --  (n1.mid);
							\draw[thick] (n3.mid) --  (n1.mid);
							\draw[thick] (n3.mid) --  (n0.mid);
							\draw[thick] (n2.mid) --  (n0.mid);
							\end{pgfonlayer}
						\end{tikzpicture}}	
					\end{figure}	
				\endminipage}
    			\end{figure}
			\begin{figure}[H]
				\minipage{0.5\textwidth}
					\begin{figure}[H]
						\resizebox{1.0\textwidth}{!}{\includegraphics{Images/DiagramaBarrasEj7LOCAL.png}}
					\end{figure}
				\endminipage
				\minipage{0.5\textwidth}
					\begin{figure}[H]
						\resizebox{1.0\textwidth}{!}{\includegraphics{Images/DiagramaPersistenciaEj7LOCAL.png}}
					\end{figure}
				\endminipage
			\end{figure}
			
			\begin{itemize}
				\item{\textbf{Interpretación global}}
			\end{itemize}
			\begin{figure}[H]
				\fbox{\minipage{0.225\textwidth}
				    	\begin{figure}[H]
						\resizebox{1.0\textwidth}{!}{\begin{tikzpicture}[
							roundnode/.style={circle, draw=black, thick, fill=white, minimum size=7mm},
							]
							%Nodes
							\node           (label) 	at	(1,4) 		{Filtración 0 (t=1.0)};
							\node[roundnode]      (n0) 	at 	(4,1)           {0};
							\node[roundnode]      (n1) 	at 	(4,-1)          {1};
							\node[roundnode]      (n2) 	at 	(2,2)           {2};
							\node[roundnode]      (n3) 	at 	(2,0)           {3};
							\node[roundnode]      (n4)      at 	(2,-2)		{4};
							\node[roundnode]      (n5)      at 	(0,3)		{5};
							\node[roundnode]      (n6)      at 	(0,1)		{6};
							\node[roundnode]      (n7) 	at 	(0,-1)           {7};
							\node[roundnode]      (n8) 	at 	(0,-3)           {8};
							\node[roundnode]      (n9) 	at 	(-2,2)           {9};
							\node[roundnode]      (n10) 	at 	(-2,0)           {10};
							\node[roundnode]      (n11) 	at 	(-2,-2)           {11};
							\node 		(label2) 	at 	(1,-4) 		{$\beta_{0}=10,\beta_{1}=0$};
							
							\begin{pgfonlayer}{background}
							%Lines
							\draw[thick] (n11.mid) --  (n8);
							\draw[thick] (n9.mid) --  (n5);
							\end{pgfonlayer}
						\end{tikzpicture}}	
					\end{figure}	
				\endminipage}
				\fbox{\minipage{0.225\textwidth}
				    	\begin{figure}[H]
						\resizebox{1.0\textwidth}{!}{\begin{tikzpicture}[
							roundnode/.style={circle, draw=black, thick, fill=white, minimum size=7mm},
							]
							%Nodes
							\node           (label) 	at	(1,4) 		{Filtración 4 (t=0.6)};
							\node[roundnode]      (n0) 	at 	(4,1)           {0};
							\node[roundnode]      (n1) 	at 	(4,-1)          {1};
							\node[roundnode]      (n2) 	at 	(2,2)           {2};
							\node[roundnode]      (n3) 	at 	(2,0)           {3};
							\node[roundnode]      (n4)      at 	(2,-2)		{4};
							\node[roundnode]      (n5)      at 	(0,3)		{5};
							\node[roundnode]      (n6)      at 	(0,1)		{6};
							\node[roundnode]      (n7) 	at 	(0,-1)           {7};
							\node[roundnode]      (n8) 	at 	(0,-3)           {8};
							\node[roundnode]      (n9) 	at 	(-2,2)           {9};
							\node[roundnode]      (n10) 	at 	(-2,0)           {10};
							\node[roundnode]      (n11) 	at 	(-2,-2)           {11};
							\node 		(label2) 	at 	(1,-4) 		{$\beta_{0}=9,\beta_{1}=0$};
							
							\begin{pgfonlayer}{background}
							%Lines
							\draw[thick] (n11.mid) --  (n8);
							\draw[thick] (n9.mid) --  (n5);
							\draw[thick] (n6.mid) --  (n2);
							\end{pgfonlayer}
						\end{tikzpicture}}	
					\end{figure}	
				\endminipage}
				\fbox{\minipage{0.225\textwidth}
				    	\begin{figure}[H]
						\resizebox{1.0\textwidth}{!}{\begin{tikzpicture}[
							roundnode/.style={circle, draw=black, thick, fill=white, minimum size=7mm},
							]
							%Nodes
							\node           (label) 	at	(1,4) 		{Filtración 5 (t=0.5)};
							\node[roundnode]      (n0) 	at 	(4,1)           {0};
							\node[roundnode]      (n1) 	at 	(4,-1)          {1};
							\node[roundnode]      (n2) 	at 	(2,2)           {2};
							\node[roundnode]      (n3) 	at 	(2,0)           {3};
							\node[roundnode]      (n4)      at 	(2,-2)		{4};
							\node[roundnode]      (n5)      at 	(0,3)		{5};
							\node[roundnode]      (n6)      at 	(0,1)		{6};
							\node[roundnode]      (n7) 	at 	(0,-1)           {7};
							\node[roundnode]      (n8) 	at 	(0,-3)           {8};
							\node[roundnode]      (n9) 	at 	(-2,2)           {9};
							\node[roundnode]      (n10) 	at 	(-2,0)           {10};
							\node[roundnode]      (n11) 	at 	(-2,-2)           {11};
							\node 		(label2) 	at 	(1,-4) 		{$\beta_{0}=2,\beta_{1}=0$};
							
							\begin{pgfonlayer}{background}
							%Lines
							\draw[thick] (n11.mid) --  (n8);
							\draw[thick] (n9.mid) --  (n5);
							\draw[thick] (n6.mid) --  (n2);
							\draw[thick] (n10.mid) --  (n7);
							\draw[thick] (n10.mid) --  (n6);
							\draw[thick] (n9.mid) --  (n6);
							\draw[thick] (n7.mid) --  (n4);
							\draw[thick] (n6.mid) --  (n3);
							\draw[thick] (n3.mid) --  (n1);
							\draw[thick] (n2.mid) --  (n0);
							\end{pgfonlayer}
						\end{tikzpicture}}	
					\end{figure}	
				\endminipage}
				\fbox{\minipage{0.225\textwidth}
				    	\begin{figure}[H]
						\resizebox{1.0\textwidth}{!}{\begin{tikzpicture}[
							roundnode/.style={circle, draw=black, fill=white, thick, minimum size=7mm},
							]
							%Nodes
							\node           (label) 	at	(1,4) 		{Filtración 6 (t=0.4)};
							\node[roundnode]      (n0) 	at 	(4,1)           {0};
							\node[roundnode]      (n1) 	at 	(4,-1)          {1};
							\node[roundnode]      (n2) 	at 	(2,2)           {2};
							\node[roundnode]      (n3) 	at 	(2,0)           {3};
							\node[roundnode]      (n4)      at 	(2,-2)		{4};
							\node[roundnode]      (n5)      at 	(0,3)		{5};
							\node[roundnode]      (n6)      at 	(0,1)		{6};
							\node[roundnode]      (n7) 	at 	(0,-1)           {7};
							\node[roundnode]      (n8) 	at 	(0,-3)           {8};
							\node[roundnode]      (n9) 	at 	(-2,2)           {9};
							\node[roundnode]      (n10) 	at 	(-2,0)           {10};
							\node[roundnode]      (n11) 	at 	(-2,-2)           {11};
							\node 		(label2) 	at 	(1,-4) 		{$\beta_{0}=1,\beta_{1}=3$};
							
							\begin{pgfonlayer}{background}
						
							\fill[fill=black!20,opacity=1] (n11.mid) -- (n8.mid) -- (n4.mid) -- cycle;
							\fill[fill=black!20,opacity=1] (n11.mid) -- (n7.mid) -- (n4.mid) -- cycle;
							\fill[fill=black!20,opacity=1] (n9.mid) -- (n6.mid) -- (n2.mid) -- cycle;
							\fill[fill=black!20,opacity=1] (n9.mid) -- (n5.mid) -- (n2.mid) -- cycle;
							
							\draw[thick] (n11.mid) --  (n8.mid);
							\draw[thick] (n11.mid) --  (n7.mid);
							\draw[thick] (n11.mid) --  (n4.mid);
							\draw[thick] (n10.mid) --  (n7.mid);
							\draw[thick] (n10.mid) --  (n6);
							\draw[thick] (n9.mid) --  (n6.mid);
							\draw[thick] (n9.mid) --  (n5.mid);
							\draw[thick] (n9.mid) --  (n2.mid);
							\draw[thick] (n8.mid) --  (n4);
							\draw[thick] (n7.mid) --  (n4);
							\draw[thick] (n7.mid) --  (n3);
							\draw[thick] (n6.mid) --  (n3);
							\draw[thick] (n6.mid) --  (n2);
							\draw[thick] (n5.mid) --  (n2);
							\draw[thick] (n4.mid) --  (n1);
							\draw[thick] (n3.mid) --  (n1);
							\draw[thick] (n3.mid) --  (n0);
							\draw[thick] (n2.mid) --  (n0);
							
							\end{pgfonlayer}
						\end{tikzpicture}}	
					\end{figure}	
				\endminipage}
    			\end{figure}
			\begin{figure}[H]
				\fbox{\minipage{0.225\textwidth}
				    	\begin{figure}[H]
						\resizebox{1.0\textwidth}{!}{\begin{tikzpicture}[
							roundnode/.style={circle, draw=black, thick, fill=white, minimum size=7mm},
							]
							%Nodes
							\node           (label) 	at	(1,4) 		{Filtración 7 (t=0.3)};
							\node[roundnode]      (n0) 	at 	(4,1)           {0};
							\node[roundnode]      (n1) 	at 	(4,-1)          {1};
							\node[roundnode]      (n2) 	at 	(2,2)           {2};
							\node[roundnode]      (n3) 	at 	(2,0)           {3};
							\node[roundnode]      (n4)      at 	(2,-2)		{4};
							\node[roundnode]      (n5)      at 	(0,3)		{5};
							\node[roundnode]      (n6)      at 	(0,1)		{6};
							\node[roundnode]      (n7) 	at 	(0,-1)           {7};
							\node[roundnode]      (n8) 	at 	(0,-3)           {8};
							\node[roundnode]      (n9) 	at 	(-2,2)           {9};
							\node[roundnode]      (n10) 	at 	(-2,0)           {10};
							\node[roundnode]      (n11) 	at 	(-2,-2)           {11};
							\node 		(label2) 	at 	(1,-4) 		{$\beta_{0}=1,\beta_{1}=2$};
							
							\begin{pgfonlayer}{background}							
						

							\fill[fill=black!20,opacity=1] (n11.mid) -- (n8.mid) -- (n4.mid) -- cycle;
							\fill[fill=black!20,opacity=1] (n11.mid) -- (n7.mid) -- (n4.mid) -- cycle;
							\fill[fill=black!20,opacity=1] (n10.mid) to[bend right] (n4.mid) -- (n7.mid) -- cycle;
							\fill[fill=black!20,opacity=1] (n10.mid) to[bend left] (n2.mid) -- (n6.mid) -- cycle;
							\fill[fill=black!20,opacity=1] (n9.mid) -- (n6.mid) -- (n2.mid) -- cycle;
							\fill[fill=black!20,opacity=1] (n9.mid) -- (n5.mid) -- (n2.mid) -- cycle;
							\fill[fill=black!20,opacity=1] (n6.mid) to[bend left] (n1.mid) -- (n3.mid) -- cycle;
							\fill[fill=black!20,opacity=1] (n6.mid) -- (n3.mid) -- (n0.mid) -- cycle;
							\fill[fill=black!20,opacity=1] (n6.mid) -- (n2.mid) -- (n0.mid) -- cycle;
							
							\draw[thick] (n11.mid) --  (n8);
							\draw[thick] (n11.mid) --  (n7);
							\draw[thick] (n11.mid) --  (n4.mid);
							\draw[thick] (n10.mid) --  (n7);
							\draw[thick] (n10.mid) --  (n6);
							\draw[thick] (n9.mid) --  (n6);
							\draw[thick] (n9.mid) --  (n5);
							\draw[thick] (n9.mid) --  (n2.mid);
							\draw[thick] (n8.mid) --  (n4);
							\draw[thick] (n7.mid) --  (n4);
							\draw[thick] (n7.mid) --  (n3);
							\draw[thick] (n6.mid) --  (n3);
							\draw[thick] (n6.mid) --  (n2);
							\draw[thick] (n6.mid) --  (n0);
							\draw[thick] (n5.mid) --  (n2);
							\draw[thick] (n4.mid) --  (n1);
							\draw[thick] (n3.mid) --  (n1);
							\draw[thick] (n3.mid) --  (n0);
							\draw[thick] (n2.mid) --  (n0);
							\end{pgfonlayer}
						\end{tikzpicture}}	
					\end{figure}	
				\endminipage}
				\fbox{\minipage{0.225\textwidth}
				    	\begin{figure}[H]
						\resizebox{1.0\textwidth}{!}{\begin{tikzpicture}[
							roundnode/.style={circle, draw=black, thick, fill=white, minimum size=7mm},
							]
							%Nodes
							\node           (label) 	at	(1,4) 		{Filtración 8 (t=0.2)};
							\node[roundnode]      (n0) 	at 	(4,1)           {0};
							\node[roundnode]      (n1) 	at 	(4,-1)          {1};
							\node[roundnode]      (n2) 	at 	(2,2)           {2};
							\node[roundnode]      (n3) 	at 	(2,0)           {3};
							\node[roundnode]      (n4)      at 	(2,-2)		{4};
							\node[roundnode]      (n5)      at 	(0,3)		{5};
							\node[roundnode]      (n6)      at 	(0,1)		{6};
							\node[roundnode]      (n7) 	at 	(0,-1)           {7};
							\node[roundnode]      (n8) 	at 	(0,-3)           {8};
							\node[roundnode]      (n9) 	at 	(-2,2)           {9};
							\node[roundnode]      (n10) 	at 	(-2,0)           {10};
							\node[roundnode]      (n11) 	at 	(-2,-2)           {11};
							\node 		(label2) 	at 	(1,-4) 		{$\beta_{0}=1,\beta_{1}=0$};
							
							\begin{pgfonlayer}{background}							
						

							\fill[fill=black!20,opacity=1] (n11.mid) -- (n8.mid) -- (n4.mid) -- cycle;
							\fill[fill=black!20,opacity=1] (n11.mid) -- (n7.mid) -- (n4.mid) -- cycle;
							\fill[fill=black!20,opacity=1] (n10.mid) to[bend right] (n4.mid) -- (n7.mid) -- cycle;
							\fill[fill=black!20,opacity=1] (n10.mid) -- (n7.mid) -- (n3.mid) -- cycle;
							\fill[fill=black!20,opacity=1] (n10.mid) -- (n6.mid) -- (n3.mid) -- cycle;
							\fill[fill=black!20,opacity=1] (n10.mid) to[bend left] (n2.mid) -- (n6.mid) -- cycle;
							\fill[fill=black!20,opacity=1] (n9.mid) to[bend right] (n3.mid) -- (n6.mid) -- cycle;
							\fill[fill=black!20,opacity=1] (n9.mid) -- (n6.mid) -- (n2.mid) -- cycle;
							\fill[fill=black!20,opacity=1] (n9.mid) -- (n5.mid) -- (n2.mid) -- cycle;
							\fill[fill=black!20,opacity=1] (n7.mid) -- (n4.mid) -- (n1.mid) -- cycle;
							\fill[fill=black!20,opacity=1] (n7.mid) -- (n3.mid) -- (n1.mid) -- cycle;
							\fill[fill=black!20,opacity=1] (n6.mid) to[bend left] (n1.mid) -- (n3.mid) -- cycle;
							\fill[fill=black!20,opacity=1] (n6.mid) -- (n3.mid) -- (n0.mid) -- cycle;
							\fill[fill=black!20,opacity=1] (n6.mid) -- (n2.mid) -- (n0.mid) -- cycle;
							\fill[fill=black!20,opacity=1] (n5.mid) to[bend left] (n0.mid) -- (n2.mid) -- cycle;
							
							\draw[thick] (n11.mid) --  (n8.mid);
							\draw[thick] (n11.mid) --  (n7.mid);
							\draw[thick] (n11.mid) --  (n4.mid);
							\draw[thick] (n10.mid) --  (n7.mid);
							\draw[thick] (n10.mid) --  (n6.mid);
							\draw[thick] (n10.mid) --  (n3.mid);
							\draw[thick] (n9.mid) --  (n6.mid);
							\draw[thick] (n9.mid) --  (n5.mid);
							\draw[thick] (n9.mid) --  (n2.mid);
							\draw[thick] (n8.mid) --  (n4.mid);
							\draw[thick] (n7.mid) --  (n4.mid);
							\draw[thick] (n7.mid) --  (n3.mid);
							\draw[thick] (n7.mid) --  (n1.mid);
							\draw[thick] (n6.mid) --  (n3.mid);
							\draw[thick] (n6.mid) --  (n2.mid);
							\draw[thick] (n6.mid) --  (n0.mid);
							\draw[thick] (n5.mid) --  (n2.mid);
							\draw[thick] (n4.mid) --  (n1.mid);
							\draw[thick] (n3.mid) --  (n1.mid);
							\draw[thick] (n3.mid) --  (n0.mid);
							\draw[thick] (n2.mid) --  (n0.mid);
							\end{pgfonlayer}
						\end{tikzpicture}}	
					\end{figure}	
				\endminipage}
				\fbox{\minipage{0.225\textwidth}
				    	\begin{figure}[H]
						\resizebox{1.0\textwidth}{!}{\begin{tikzpicture}[
							roundnode/.style={circle, draw=black, thick, fill=white, minimum size=7mm},
							]
							%Nodes
							\node           (label) 	at	(1,4) 		{Filtración 9 (t=0.1)};
							\node[roundnode]      (n0) 	at 	(4,1)           {0};
							\node[roundnode]      (n1) 	at 	(4,-1)          {1};
							\node[roundnode]      (n2) 	at 	(2,2)           {2};
							\node[roundnode]      (n3) 	at 	(2,0)           {3};
							\node[roundnode]      (n4)      at 	(2,-2)		{4};
							\node[roundnode]      (n5)      at 	(0,3)		{5};
							\node[roundnode]      (n6)      at 	(0,1)		{6};
							\node[roundnode]      (n7) 	at 	(0,-1)           {7};
							\node[roundnode]      (n8) 	at 	(0,-3)           {8};
							\node[roundnode]      (n9) 	at 	(-2,2)           {9};
							\node[roundnode]      (n10) 	at 	(-2,0)           {10};
							\node[roundnode]      (n11) 	at 	(-2,-2)           {11};
							\node 		(label2) 	at 	(1,-4) 		{$\beta_{0}=1,\beta_{1}=0$};
							
							\begin{pgfonlayer}{background}							
						

							\fill[fill=black!20,opacity=1] (n11.mid) -- (n8.mid) -- (n4.mid) -- cycle;
							\fill[fill=black!20,opacity=1] (n11.mid) -- (n7.mid) -- (n4.mid) -- cycle;
							\fill[fill=black!20,opacity=1] (n11.mid) to[bend left] (n3.mid) -- (n7.mid) -- cycle;
							\fill[fill=black!20,opacity=1] (n10.mid) to[bend right] (n4.mid) -- (n7.mid) -- cycle;
							\fill[fill=black!20,opacity=1] (n10.mid) -- (n7.mid) -- (n3.mid) -- cycle;
							\fill[fill=black!20,opacity=1] (n10.mid) -- (n6.mid) -- (n3.mid) -- cycle;
							\fill[fill=black!20,opacity=1] (n10.mid) to[bend left] (n2.mid) -- (n6.mid) -- cycle;
							\fill[fill=black!20,opacity=1] (n9.mid) to[bend right] (n3.mid) -- (n6.mid) -- cycle;
							\fill[fill=black!20,opacity=1] (n9.mid) -- (n6.mid) -- (n2.mid) -- cycle;
							\fill[fill=black!20,opacity=1] (n9.mid) -- (n5.mid) -- (n2.mid) -- cycle;
							\fill[fill=black!20,opacity=1] (n8.mid) to[bend right] (n1.mid) -- (n4.mid) -- cycle;
							\fill[fill=black!20,opacity=1] (n7.mid) -- (n4.mid) -- (n1.mid) -- cycle;
							\fill[fill=black!20,opacity=1] (n7.mid) -- (n3.mid) -- (n1.mid) -- cycle;
							\fill[fill=black!20,opacity=1] (n7.mid) to[bend right] (n0.mid) -- (n3.mid) -- cycle;
							\fill[fill=black!20,opacity=1] (n6.mid) to[bend left] (n1.mid) -- (n3.mid) -- cycle;
							\fill[fill=black!20,opacity=1] (n6.mid) -- (n3.mid) -- (n0.mid) -- cycle;
							\fill[fill=black!20,opacity=1] (n6.mid) -- (n2.mid) -- (n0.mid) -- cycle;
							\fill[fill=black!20,opacity=1] (n5.mid) to[bend left] (n0.mid) -- (n2.mid) -- cycle;
							
							\draw[thick] (n11.mid) --  (n8.mid);
							\draw[thick] (n11.mid) --  (n7.mid);
							\draw[thick] (n11.mid) --  (n4.mid);
							\draw[thick] (n10.mid) --  (n7.mid);
							\draw[thick] (n10.mid) --  (n6.mid);
							\draw[thick] (n10.mid) --  (n3.mid);
							\draw[thick] (n9.mid) --  (n6.mid);
							\draw[thick] (n9.mid) --  (n5.mid);
							\draw[thick] (n9.mid) --  (n2.mid);
							\draw[thick] (n8.mid) --  (n4.mid);
							\draw[thick] (n7.mid) --  (n4.mid);
							\draw[thick] (n7.mid) --  (n3.mid);
							\draw[thick] (n7.mid) --  (n1.mid);
							\draw[thick] (n6.mid) --  (n3.mid);
							\draw[thick] (n6.mid) --  (n2.mid);
							\draw[thick] (n6.mid) --  (n0.mid);
							\draw[thick] (n5.mid) --  (n2.mid);
							\draw[thick] (n4.mid) --  (n1.mid);
							\draw[thick] (n3.mid) --  (n1.mid);
							\draw[thick] (n3.mid) --  (n0.mid);
							\draw[thick] (n2.mid) --  (n0.mid);
							\end{pgfonlayer}
						\end{tikzpicture}}	
					\end{figure}	
				\endminipage}
				\fbox{\minipage{0.225\textwidth}
				    	\begin{figure}[H]
						\resizebox{1.0\textwidth}{!}{\begin{tikzpicture}[
							roundnode/.style={circle, draw=black, thick, fill=white, minimum size=7mm},
							]
							%Nodes
							\node           (label) 	at	(1,4) 		{Filtración 10 (t=0.0)};
							\node[roundnode]      (n0) 	at 	(4,1)           {0};
							\node[roundnode]      (n1) 	at 	(4,-1)          {1};
							\node[roundnode]      (n2) 	at 	(2,2)           {2};
							\node[roundnode]      (n3) 	at 	(2,0)           {3};
							\node[roundnode]      (n4)      at 	(2,-2)		{4};
							\node[roundnode]      (n5)      at 	(0,3)		{5};
							\node[roundnode]      (n6)      at 	(0,1)		{6};
							\node[roundnode]      (n7) 	at 	(0,-1)           {7};
							\node[roundnode]      (n8) 	at 	(0,-3)           {8};
							\node[roundnode]      (n9) 	at 	(-2,2)           {9};
							\node[roundnode]      (n10) 	at 	(-2,0)           {10};
							\node[roundnode]      (n11) 	at 	(-2,-2)           {11};
							\node 		(label2) 	at 	(1,-4) 		{$\beta_{0}=1,\beta_{1}=0$};
							
							\begin{pgfonlayer}{background}							
						

							\fill[fill=black!20,opacity=1] (n11.mid) -- (n8.mid) -- (n4.mid) -- cycle;
							\fill[fill=black!20,opacity=1] (n11.mid) -- (n7.mid) -- (n4.mid) -- cycle;
							\fill[fill=black!20,opacity=1] (n11.mid) to[bend left] (n3.mid) -- (n7.mid) -- cycle;
							\fill[fill=black!20,opacity=1] (n10.mid) to[bend right] (n4.mid) -- (n7.mid) -- cycle;
							\fill[fill=black!20,opacity=1] (n10.mid) -- (n7.mid) -- (n3.mid) -- cycle;
							\fill[fill=black!20,opacity=1] (n10.mid) -- (n6.mid) -- (n3.mid) -- cycle;
							\fill[fill=black!20,opacity=1] (n10.mid) to[bend left] (n2.mid) -- (n6.mid) -- cycle;
							\fill[fill=black!20,opacity=1] (n9.mid) to[bend right] (n3.mid) -- (n6.mid) -- cycle;
							\fill[fill=black!20,opacity=1] (n9.mid) -- (n6.mid) -- (n2.mid) -- cycle;
							\fill[fill=black!20,opacity=1] (n9.mid) -- (n5.mid) -- (n2.mid) -- cycle;
							\fill[fill=black!20,opacity=1] (n8.mid) to[bend right] (n1.mid) -- (n4.mid) -- cycle;
							\fill[fill=black!20,opacity=1] (n7.mid) -- (n4.mid) -- (n1.mid) -- cycle;
							\fill[fill=black!20,opacity=1] (n7.mid) -- (n3.mid) -- (n1.mid) -- cycle;
							\fill[fill=black!20,opacity=1] (n7.mid) to[bend right] (n0.mid) -- (n3.mid) -- cycle;
							\fill[fill=black!20,opacity=1] (n6.mid) to[bend left] (n1.mid) -- (n3.mid) -- cycle;
							\fill[fill=black!20,opacity=1] (n6.mid) -- (n3.mid) -- (n0.mid) -- cycle;
							\fill[fill=black!20,opacity=1] (n6.mid) -- (n2.mid) -- (n0.mid) -- cycle;
							\fill[fill=black!20,opacity=1] (n5.mid) to[bend left] (n0.mid) -- (n2.mid) -- cycle;
							
							\draw[thick] (n11.mid) --  (n8.mid);
							\draw[thick] (n11.mid) --  (n7.mid);
							\draw[thick] (n11.mid) --  (n4.mid);
							\draw[thick] (n10.mid) --  (n7.mid);
							\draw[thick] (n10.mid) --  (n6.mid);
							\draw[thick] (n10.mid) --  (n3.mid);
							\draw[thick] (n9.mid) --  (n6.mid);
							\draw[thick] (n9.mid) --  (n5.mid);
							\draw[thick] (n9.mid) --  (n2.mid);
							\draw[thick] (n8.mid) --  (n4.mid);
							\draw[thick] (n7.mid) --  (n4.mid);
							\draw[thick] (n7.mid) --  (n3.mid);
							\draw[thick] (n7.mid) --  (n1.mid);
							\draw[thick] (n6.mid) --  (n3.mid);
							\draw[thick] (n6.mid) --  (n2.mid);
							\draw[thick] (n6.mid) --  (n0.mid);
							\draw[thick] (n5.mid) --  (n2.mid);
							\draw[thick] (n4.mid) --  (n1.mid);
							\draw[thick] (n3.mid) --  (n1.mid);
							\draw[thick] (n3.mid) --  (n0.mid);
							\draw[thick] (n2.mid) --  (n0.mid);
							\end{pgfonlayer}
						\end{tikzpicture}}	
					\end{figure}	
				\endminipage}
    			\end{figure}
			\begin{figure}[H]
				\minipage{0.5\textwidth}
					\begin{figure}[H]
						\resizebox{1.0\textwidth}{!}{\includegraphics{Images/DiagramaBarrasEj7GLOBAL.png}}
					\end{figure}
				\endminipage
				\minipage{0.5\textwidth}
					\begin{figure}[H]
						\resizebox{1.0\textwidth}{!}{\includegraphics{Images/DiagramaPersistenciaEj7GLOBAL.png}}
					\end{figure}
				\endminipage
			\end{figure}
			\begin{remark}
				Para una mayor claridad, en los dibujos de las filtraciones se han omitido algunas aristas.
			\end{remark}
		\end{ejem}	
		Del ejemplo anterior podemos extraer unas conclusiones muy importantes. Por una parte, observamos que si las neuronas de entrada se conectan
		directamente a las de salida, el conocimiento de la red será ``pobre" ya que será equivalente a la detección de patrones. Por otra parte,
		el incremento del número de Betti $\beta_{1}$ indica que la red determina la neurona de llegada por combinación de las neuronas de salida. De este modo,
		podemos suponer que el aumento de $\beta_{1}$ releja la complejidad del conocimiento adquirido por la red. Por lo tanto, mediante el uso de la homología persistente
		seremos capaces de medir la complejidad del conocimiento adquirido por la red.
	
	\section{Experimentos}
		(En construcción.)
\end{document} 
