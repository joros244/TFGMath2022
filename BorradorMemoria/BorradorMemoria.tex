\documentclass[12pt, a4paper, twoside]{book}
\usepackage[utf8]{inputenc}
\usepackage[T1]{fontenc}
\usepackage{lmodern}


\usepackage[spanish, es-noshorthands, english]{babel}

\usepackage{amsmath,amssymb,amsthm}
\usepackage{amsfonts}
\usepackage{authblk}
\usepackage{mathtools}
\usepackage[nottoc]{tocbibind}
\usepackage{hyperref}
\usepackage{pdfpages}
\usepackage{babelbib}
\usepackage{titlesec}
\usepackage{titleps}
\usepackage{wrapfig}
\usepackage{caption}
\usepackage{enumitem}

\newpagestyle{main}[\sffamily]{
	\setheadrule{.55pt}
	\sethead[\thepage][][\chaptertitlename\space\thechapter.\space\chaptertitle]
	{\thesection\space\sectiontitle}{}{\thepage}}

\titleformat{\chapter}[display]{\bfseries\huge}{\LARGE Capítulo \thechapter}{0pt}{}[]
\titlespacing{\chapter}{0pt}{0pt}{10pt}

\let\cleardoublepageOriginal\cleardoublepage
\renewcommand{\cleardoublepage}%
{\newpage{\pagestyle{empty}\cleardoublepageOriginal}}

\usepackage{graphicx}
\graphicspath{{Images/}}
\usepackage{float}
\usepackage[headings]{fullpage}
\parskip = 3pt plus 1.5pt minus 1pt
\usepackage{tikz}
\usetikzlibrary{positioning}
\usetikzlibrary{babel}
\pgfdeclarelayer{background}
\pgfdeclarelayer{foreground}
\pgfsetlayers{background,main,foreground}

\newcommand\hlightr[1]{\tikz[overlay, remember picture,baseline=-
\the\dimexpr\fontdimen22\textfont2\relax]\node[rectangle,fill=red!50,rounded 
corners,fill opacity = 0.2,draw,thick,text opacity =1] {$#1$};}

\numberwithin{equation}{section}

\newtheorem{theorem}{Teorema}
\theoremstyle{definition}
\newtheorem{defi}{Definición}[section]
\newtheorem{ejemplo}{Ejemplo}[section]
\newenvironment{ejem}
  {\pushQED{\qed}\renewcommand{\qedsymbol}{$\blacktriangleleft$}\ejemplo}
  {\popQED\endejemplo}

\theoremstyle{remark}
\newtheorem*{remark}{Nota}

\theoremstyle{plain}
\newtheorem{prop}{Proposición}

\usepackage{nicematrix}
\setcounter{MaxMatrixCols}{20}

\usepackage{listings}
\usepackage{xcolor}
\usepackage{soul}
\sethlcolor{green}
\renewcommand{\lstlistingname}{BC}

\definecolor{codegreen}{rgb}{0,0.6,0}
\definecolor{codegray}{rgb}{0.5,0.5,0.5}
\definecolor{codepurple}{rgb}{0.58,0,0.82}
\definecolor{backcolour}{rgb}{0.95,0.95,0.92}

\lstdefinestyle{mystyle}{
    backgroundcolor=\color{backcolour},   
    commentstyle=\color{codegreen},
    keywordstyle=\color{magenta},
    numberstyle=\tiny\color{codegray},
    stringstyle=\color{codepurple},
    basicstyle=\ttfamily\footnotesize,
    breakatwhitespace=false,         
    breaklines=true,                 
    captionpos=b,                    
    keepspaces=true,                 
    showspaces=false,                
    showstringspaces=false,
    showtabs=false,
    columns=fixed,
    tabsize=2
}

\lstset{style=mystyle,
	inputencoding=utf8,
	extendedchars=true,
literate={á}{{\'a}}1 {é}{{\'e}}1 {í}{{\'i}}1 {ñ}{{\~n}}1 {ó}{{\'o}}1
{Ó}{{\'O}}1 {Á}{{\'A}}1 {Ú}{{\'U}}1}

\DeclareMathOperator{\Ima}{im}

\begin{document}
	\pagenumbering{Roman}
	
	\selectlanguage{spanish}
	
	\pagestyle{main}

	\includepdf{Portada.pdf}
	
	\chapter*{Resumen/Abstract}
	\addcontentsline{toc}{chapter}{Resumen/Abstract}
	\section*{Resumen}
	En este trabajo fin de grado se realiza un análisis de una novedosa 
	herramienta empleada en la depuración y explicabilidad de las redes 
	neuronales artificiales: \emph{la homología persistente}. Esta 
	herramienta, publicada en un artículo de investigación del que haremos 
	un análisis en profundidad, permite medir la complejidad del 
	conocimiento que podrá obtener una red neuronal. 

	En primer lugar, para la comprensión de esta herramienta, presentamos 
	un capítulo de preliminares: en él exponemos todos los fundamentos 
	teóricos de la homología persistente y hacemos una breve introducción a 
	las redes neuronales. En segundo lugar, nos centramos en el desarrollo 
	del artículo de investigación, así como en los problemas encontrados 
	durante el estudio del mismo. Finalmente, detallamos el código 
	producido para la aplicación de la herramienta, y la empleamos sobre 
	una colección de ejemplos. 
	\section*{Abstract}
	In this final degree project we will analyse a novel tool used in the 
	debugging and explainability of artificial neural networks: 
	\emph{persistent homology}. This tool, published in a research article 
	which we will analyse in depth, allows us to measure the complexity of 
	the knowledge that a neural network will be able to obtain. 

	Firstly, in order to understand this tool, we present 
	a preliminary chapter in which we set out all the theoretical 
	foundations of the persistent homology and we do a brief introduction 
	to artificial neural networks. Secondly, we focus on the development 
	of the research paper, as well as on the problems encountered during 
	the study of it. Finally, we detail the code developed in this project, 
	and we use the tool on a collection of examples.  

	\tableofcontents

	\pagenumbering{arabic}
	
	\chapter{Introducción}
	Supongamos que nos entregan las imágenes \ref{fig:mist} y nos 
	solicitan que identifiquemos la figura geométrica de la que proceden.

	\begin{figure}[!htbp]
		\minipage{0.5\textwidth}
			\begin{figure}[H]
				\resizebox{1.0\textwidth}{!}{\includegraphics{Images/DosCirculos.png}}
			\end{figure}
		\endminipage
		\minipage{0.5\textwidth}
			\begin{figure}[H]
				\resizebox{1.0\textwidth}{!}{\includegraphics{Images/Toro.png}}
			\end{figure}
		\endminipage
		\caption{Imágenes misteriosas}
		\label{fig:mist}
	\end{figure}

	Observando las imágenes podemos apreciar que la primera proviene de 
	dos circunferencias concéntricas, mientras que la segunda proviene de 
	un toro. Reflexionemos momentáneamente acerca de cómo hemos alcanzado
	esta conclusión.

	Es probable que, de una manera inconsciente, hayamos analizado la 
	forma de las imágenes, el número de agujeros que poseen y 
	la dimensión de dichos agujeros. Este análisis es el que nos ha 
	permitido encontrar las figuras de procedencia. Este proceso, que en 
	2 y 3 dimensiones nos puede resultar inmediato, en dimensiones 
	superiores resulta más complicado, pues nuestra intuición comienza a 
	fallar. Por ello, se hace necesario el uso de las matemáticas para 
	vencer en este desafío a la intuición.

	\begin{wrapfigure}[11]{r}{0.27\textwidth}
	  \begin{center}
		  \raisebox{16pt}[\dimexpr\height-0.6\baselineskip\relax]{\includegraphics[width=0.25\textwidth]{Images/344px-Betti_Enrico.jpg}}
	  \vspace*{-9mm}
	  \caption*{Enrico Betti}
	  \end{center}
	\end{wrapfigure}
	
	En matemáticas existe toda una rama 
	relacionada con la idea de estudiar las propiedades invariantes 
	(como los agujeros) de los objetos mediante el uso de técnicas 
	algebraicas. Esta rama se conoce como 
	topología algebraica y tiene su origen con el matemático italiano 
	Enrico Betti (1823-1892). E. Betti, quien en un principio trabajó 
	sobre la teoría de Galois, se hizo célebre por su estudio topológico 
	de la conexión entre figuras de dimensiones altas (mayores que tres) y 
	por caracterizar la igualdad entre superficies.

	Más tarde llegaría Henri Poincaré (1854-1912) quien, inspirado por el 
	trabajo de E. Betti, formalizaría toda 
	esta teoría en su famoso <<Analysis Situs>>, acuñando el término 
	\emph{número de Betti} que precisamente representa el número de 
	agujeros de un objeto. Esta teoría fue, en un comienzo, puramente 
	topológica y para el siguiente avance significativo hubo que esperar 
	a la llegada de la matemática alemana 
	Emmy Noether (1882-1935), quien apreció la estructura de grupo 
	subyacente a los ciclos definidos en la teoría previa. Esto originó 
	una algebraización de esta teoría.

	\begin{wrapfigure}[14]{l}{0.27\textwidth}
	  \begin{center}
		  \raisebox{16pt}[\dimexpr\height-0.6\baselineskip\relax]{\includegraphics[width=0.25\textwidth]{Images/315px-Noether.jpg}}
	  \vspace*{-9mm}
	  \caption*{Emmy Noether}
	  \end{center}
	\end{wrapfigure}

	En los subsiguientes años esta teoría se desarrolla rápidamente de la 
	mano de grandes matemáticos como Pável Alexandroff, Leopold Vietoris, 
	Georges de Rham, y Eduard \v{C}ech entre otros. Finalmente, Samuel 
	Eilenberg (1913-1998) y 
	Saunders Mac Lane (1909-2005) axiomatizaron esta teoría y la 
	relacionaron con su \emph{teoría de categorías}.

	Enmarcada en la topología algebraica tenemos la noción principal 
	sobre la que versa el presente trabajo: \emph{la homología 
	persistente}. Este concepto surge en la década de los 90 y los 2000 
	de la mano de varios autores: Patrizio Frosini, Massimo Ferri, 
	Vanessa Robins, y Herbert Edelsbrunner. Estos autores llegaron de 
	manera independiente al concepto de persistencia que se desarrollaría 
	en profundidad durante los siguientes 15 años. 

	Estos fundamentos teóricos nos permiten, entre otras cosas, el 
	cálculo sistemático de los agujeros de un objeto en cualquier 
	dimensión. 

	En el presente trabajo, los objetos sobre los que estudiaremos los
	agujeros serán las redes neuronales. Siendo más precisos, vamos a 
	estudiar los agujeros de los grafos subyacentes a dichas redes.
	Como veremos, estos agujeros, más que simples 
	vacíos dentro de los grafos, representarán de cierta manera el grado 
	de conocimiento que podrá ser adquirido por la red. Por lo tanto, 
	mediante el uso de 
	la topología algebraica podremos extraer conclusiones valiosas acerca 
	de las redes neuronales que habitualmente son vistas como <<cajas 
	negras>>.

	En el capítulo 2 del presente trabajo haremos una introducción a los 
	conceptos necesarios para el recuento de agujeros mencionado 
	anteriormente. En el mismo capítulo haremos una breve introducción a 
	las redes neuronales. Finalmente, en el capítulo 3, aplicaremos toda
	la teoría introducida y veremos una serie de ejemplos.
	
	\chapter{Preliminares}

	A lo largo de este capítulo vamos a presentar todas las nociones teóricas 
	necesarias para el uso de la homología persistente en redes neuronales.	
	
	\section{Complejos simpliciales}
	Comenzamos con el primer concepto fundamental de todo el trabajo, los 
	\emph{complejos simpliciales}. Esta noción admite dos enfoques 
	diferentes, por lo que debemos distinguir entre dos definiciones 
	relacionadas: los complejos simpliciales \emph{abstractos} y los 
	complejos simpliciales \emph{geométricos}. Para el desarrollo de estas 
	nociones seguiremos la guía proporcionada por 
	\cite{TopoAlg-Edelsbrunner,Intro-Otter}.

	Partiendo del enfoque combinatorio, comenzamos definiendo los complejos 
	simpliciales abstractos y algunas nociones relacionadas. 

	\begin{defi}
	
	Un \textit{complejo simplicial abstracto} es una colección 
	\begin{Large}$\nu$\end{Large}, de subconjuntos no vacíos de un 
	conjunto {\Large $\nu$}$_{0}$, que verifica las siguientes 
	propiedades:
	
	\begin{enumerate}
		\item Si $v \in $ {\Large $ \nu$}$_{0}$, entonces $\{v\} \in$
			\begin{Large}$ \nu$\end{Large}.
		\item Si $\sigma \in $ {\Large $ \nu$}$ \text{ y } \tau 
			\subset \sigma$, entonces $ \tau \in $
			\begin{Large}$ \nu$\end{Large}.
	\end{enumerate}
	
	A los elementos de {\Large $\nu$} los llamaremos \textit{símplices};
	más concretamente: dado $\sigma \in $ {\Large $\nu$}, diremos que 
	$\sigma$ tiene \textit{dimensión $p$}, y que $\sigma$ es un 
	\textit{$p$-símplice}, si $|\sigma|=p+1$. Asimismo, definimos la 
	\textit{dimensión de {\Large $\nu$}} como el máximo de las dimensiones 
	de sus símplices y denotaremos por {\Large $\nu$}$_{p}$ a la colección 
	de los $p$-símplices de {\Large $\nu$}.	
	
	\end{defi}

	Para los propósitos del presente trabajo consideraremos que 
	{\Large $\nu$}$_{0}$ es finito, lo que supondrá que los complejos 
	simpliciales definidos a partir de él también serán finitos, así como 
	sus correspondientes símplices. Nótese que la definición de los 
	$p$-símplices es coherente con {\Large $\nu$}$_{0}$, es decir, los 
	elementos de {\Large $\nu$}$_{0}$ los podemos considerar como 
	$0$-símplices. 
	
	En relación con el concepto de símplice y de dimensión surge la 
	siguiente noción: 

	\begin{defi}
		Sean $\sigma$ y $\tau$ dos símplices de {\Large $\nu$} tales 
		que $\tau \subset \sigma$. Entonces diremos que $\tau$ es una 
		\textit{cara} de $\sigma$. Además, si las dimensiones de 
		$\sigma$ y $\tau$ difieren por un número natural $a$, 
		diremos que $\tau$ es una cara de $\sigma$ de 
		\textit{codimensión a}.
	\end{defi}

	Ahora que hemos definido los complejos simpliciales abstractos veamos 
	un pequeño ejemplo para fijar ideas.

	\begin{ejem}
		\label{ejem:1}
		Supongamos el siguiente complejo simplicial abstracto:
		\begin{multline*} 
			\text{{\Large $\nu$}}=\{\{a\},\{b\},\{c\},\{d\},
			\{a,b\},\{a,c\},\{a,d\},\{b,c\},\{b,d\},\{c,d\},
			\{a,b,c\},\{a,b,d\},\\
			\{a,c,d\},\{a,b,c,d\}\}.
		\end{multline*}
		Así, tenemos que la dimensión de {\Large $\nu$} es 3. También 
		observamos que el 3-símplice $\{a,b,c,d\}$ tiene por caras de 
		codimensión 1 a los 2-símplices $\{a,b,c\},\{a,b,d\}$ y 
		$\{a,c,d\}$. En la \autoref{fig:tetra} ilustramos una 
		representación geométrica de {\Large $\nu$}.

		\begin{figure}[!htbp]
			\centering
			\begin{tikzpicture}
				%Nodes
				\coordinate      (a) 	at 	(4.5,2.5);
				\coordinate      (b) 	at 	(3,0.8);
				\coordinate     (c) 	at 	(4.4,0.1);
				\coordinate      (d) 	at 	(6,0.8);
				
				%Lines
		    		\draw[thick, fill=black!20] (a) -- (b) -- (c) -- (d) -- cycle;
				\draw[thick, dashed] (b) -- (d);
				\draw[thick] (a) -- (c);

				\fill[black!20, draw=black, thick] (a) circle (3pt) node[black, above right] {a};
				\fill[black!20, draw=black, thick] (b) circle (3pt) node[black, above left] {b};
				\fill[black!20, draw=black, thick] (c) circle (3pt) node[black, below right] {c};
				\fill[black!20, draw=black, thick] (d) circle (3pt) node[black, above right] {d};

			\end{tikzpicture}
			\caption{Representación geométrica del complejo simplicial {\Large $\nu$}.}
			\label{fig:tetra}
		\end{figure}

		La representación recogida en la \autoref{fig:tetra}, en la
		que cada símplice corresponde con un poliedro regular (cada 
		$0$-símplice corresponde a un punto, cada $1$-símplice a una 
		arista, cada $2$-símplice a un triángulo, cada $3$-símplice a 
		un tetraedro, etc.) es única salvo homeomorfismo. Observamos 
		que interpretando {\Large $\nu$} como un subconjunto de 
		$\mathbb{R}^{3}$ obtenemos un tetraedro. Esta idea motiva el 
		otro enfoque de los complejos simpliciales: el enfoque 
		geométrico.
	\end{ejem}

	Siguiendo el enfoque geométrico es necesario que, antes de llegar a la
	definición de complejo simplicial geométrico, veamos unos conceptos 
	previos relacionados con la propia definición.

	\begin{defi}
		Sean $\{u_{0},u_{1},...,u_{k}\}\subset\mathbb{R}^{n}$. Diremos
		que los $k+1$ puntos son \textit{afínmente independientes} si 
		los $k$ vectores $u_{1}-u_{0},u_{2}-u_{0},...,u_{k}-u_{0}$ son
		linealmente independientes.

		Sea $x \in \mathbb{R}^{n}$. Diremos que $x$ es una 
		\textit{combinación afín} de $\{u_{0},u_{1},...,u_{k}\}$ si 
		$\exists \lambda_{0},...,\lambda_{k}$ tales que 
		$x=\sum_{i=0}^{k}\lambda_{i}u_{i}$ y 
		$\sum_{i=0}^{k}\lambda_{i}=1$.
	\end{defi}

	\begin{defi}
		Sean $\{u_{0},u_{1},...,u_{k}\}\subset\mathbb{R}^{n}$ $k+1$ 
		puntos afínmente independientes y $x=\sum_{i=0}^{k}
		\lambda_{i}u_{i}$ una combinación afín. Diremos que $x$ es una
		\textit{combinación convexa} de $\{u_{0},u_{1},...,u_{k}\}$ si 
		$\{\lambda_{0},\lambda_{1},...,\lambda_{k}\}$ son 
		no negativos.

		Definimos la \textit{clausura convexa} de $\{u_{0},u_{1},...,
		u_{k}\}$ como el conjunto de todas sus posibles combinaciones 
		convexas.
	\end{defi}

	Ahora que ya contamos con estas nociones previas pasamos a definir la 
	pieza clave en la definición de complejo simplicial geométrico: el 
	\emph{símplice}.

	\begin{defi}
		Definimos un \textit{k-símplice} como la clausura convexa de 
		$k+1$ puntos afínmente independientes. Lo denotaremos por 
		$\sigma=conv\{u_{0},u_{1},\dots,u_{k}\}$, y diremos que la 
		\textit{dimensión de $\sigma$} es $k$. A los puntos $u_{0},
		\dots,u_{k}$ los llamamos \textit{vértices} del símplice. 
		Observamos que, al ser afínmente independientes, el conjunto 
		de vértices de un símplice es único.

		Definimos \textit{cara} de $\sigma$ como cualquier combinación 
		convexa de un subconjunto no vacío de $\{u_{0},u_{1},...,
		u_{k}\}$. A la relación <<ser cara de>> la denotaremos por 
		$\leq$.

		Para los casos $k=1,2,3$ diremos que $\sigma$ es una arista, 
		un triángulo o un tetraedro respectivamente.
	\end{defi}

	Habiendo definido todos los conceptos previos necesarios pasamos a 
	definir \emph{complejo simplicial geométrico}.

	\begin{defi}
		Llamamos \textit{complejo simplicial geométrico} a la 
		colección finita de símplices {\Large $\nu$} verificando las
		siguientes propiedades:
		\begin{enumerate}
			\item Si $\sigma\in \text{{\Large $\nu$} y }
				\tau \leq \sigma, \text{ entonces }  \tau \in 
				\text{{\Large $\nu$}}$,
			\item Si $\sigma_{1},\sigma_{2} \in 
				\text{{\Large $\nu$}}, \text{ entonces } 
				\sigma_{1}\cap\sigma_{2}=\emptyset, 
				\text{ o bien, }
				\sigma_{1}\cap\sigma_{2} \text{ es una cara 
				común a ambos.}$
		\end{enumerate}
	\end{defi}
	
	La relación entre los complejos simpliciales abstractos y los 
	geométricos viene dada por la construcción de la \emph{realización
	geométrica} de un complejo simplicial abstracto, que es un complejo 
	simplicial geométrico definido tal y como se ilustra en la 
	\autoref{fig:tetra} del ejemplo \ref{ejem:1} (para más detalles véase 
	\cite{TopoAlg-Edelsbrunner}). Siendo más precisos, los conjuntos de 
	vértices de un símplice geométrico forman un símplice abstracto.

	De aquí en adelante emplearemos la definición de complejo simplicial 
	abstracto, pues es la más adecuada para el presente trabajo.

	Ahora que ya hemos definido los objetos con los que vamos a trabajar, 
	procedemos a definir las aplicaciones entre ellos.
	\begin{defi}
		Una \textit{aplicación simplicial entre complejos 
		simpliciales}, $f\colon\text{{\Large $\nu$}} \rightarrow 
		\text{{\Large $\nu$}$^{\prime}$}$, es una aplicación 
		tal que $f(\sigma)=\{g(u_{1}),g(u_{2}),...,g(u_{k})\}=
		\{v_{1},v_{2},...,v_{k}\}$; donde
		$g\colon\text{{\Large $\nu$}$_{0}$} \rightarrow \text{{\Large 
		$\nu$}$_{0}^{\prime}$}$ es una aplicación entre $0$-símplices,
		$\sigma=\{u_{1},u_{2},...,u_{k}\} \in \text{{\Large $\nu$}}$
		y $\{v_{1},v_{2},...,v_{k}\} \in \text{{\Large $\nu$}$^{
		\prime}$}$.	
	\end{defi}

	\begin{remark}
	Notemos que en la anterior definición la función $g$ podría no ser 
	inyectiva, esto provocaría que la función $f$ no respetara las 
	dimensiones de los símplices. Este caso particular es muy interesante 
	y veremos sus implicaciones en la siguiente sección.
	\end{remark}
	
	\section{Homología. Homología persistente}
	
	En la sección anterior hemos fijado el concepto de complejo 
	simplicial, que nos será muy útil a lo largo de esta sección para 
	desarrollar la noción de \emph{espacio vectorial de homología}. A 
	diferencia de como surgió el concepto de \emph{espacio vectorial de
	homología} en la historia de las matemáticas, en el presente trabajo 
	desarrollaremos primero la noción general para luego reducir al caso 
	particular de la \emph{homología simplicial}. Para ello emplearemos la 
	guía proporcionada por \cite{TopoAlg-Edelsbrunner,Homologia-Macho,
	Algebra-Elduque}.

	Comenzamos la sección con algunas definiciones básicas que serán 
	necesarias para alcanzar la definición de los espacios vectoriales de 
	homología.

	\begin{defi}
		Sea $R$ un anillo. Definimos el \textit{R-módulo a izquierda} 
		sobre
		$R$ como el conjunto $M$ junto con las operaciones:
		\begin{itemize}
			\item Suma: $M \times M \rightarrow M, (x,y) \mapsto 
				x+y$, y
			\item Producto por escalares: $R \times M \rightarrow 
				M, (r,x) \mapsto rx$,	
		\end{itemize}
		que satisfacen las siguientes propiedades:
		\begin{enumerate}
			\item La suma es asociativa, conmutativa, $M$ contiene
			      un elemento neutro para ella y todo elemento 
			      tiene opuesto. Es decir, $(M,+)$ es un grupo 
			      abeliano.
		      \item Para cualesquiera $x,y$ de $M$ y $r,s$ de $R$: 
		      	\begin{enumerate}
			 	\item $(r+s)x=rx+sx$ (distributiva respecto a 
					la suma de $R$).
				\item $(rs)x=r(sx)$ (asociativa).
				\item $r(x+y)=rx+ry$ (distributiva respecto a 
					la suma de $M$).
				\item Si $R$ es unitario, $1x=x$.	
			\end{enumerate}
		\end{enumerate}
		De manera análoga definimos el \textit{R-módulo a derecha}. Si 
		$R$ es conmutativo, entonces el $R$-módulo a izquierda es el 
		mismo que el $R$-módulo a derecha. En tal caso nos referiremos 
		a
		él simplemente como \textit{R-módulo}.
	\end{defi}

	\begin{remark}
		Notamos que un grupo abeliano se identifica de manera natural 
		con un $\mathbb{Z}$-módulo. Asimismo, un módulo sobre un 
		cuerpo no es otra cosa que un espacio vectorial. 	
	\end{remark}

	Como es natural, a la noción de $R$-módulo le sigue la definición de
	\emph{R-submódulo}.

	\begin{defi}
	Sea $M$ un $R$-módulo. Definimos el \textit{R-submódulo} de $M$ como 
	el subconjunto, no vacío, $N$ de $M$ tal que es cerrado para opuestos 
	y para las operaciones heredadas de $M$. A la relación <<ser submódulo 
	de>> la denotaremos por $\leq$.
	\end{defi}

	Ahora que hemos definido los $R$-submódulos, vamos a definir los 
	cocientes asociados.

	\begin{defi}
		Sea $M$ un módulo sobre un anillo $R$ y $N$ un submódulo de 
		$M$. Consideremos la relación de equivalencia en $M$ dada por:
		$$
		x \sim y \text{ si } x-y \in N,
		$$
		con $x,y \in M$. Mediante esta relación de equivalencia 
		definimos el conjunto cociente $M/N$. Ahora bien, mediante las 
		operaciones suma y producto por escalares de $M$ podemos 
		inducir las correspondientes operaciones sobre $M/N$ de la 
		siguiente manera:
		\begin{equation*}
		\begin{array}{rll}
			(x+N)+(y+N) & = & (x+y)+N,\\
			r(x+N) & = & rx+N,
		\end{array}
		\end{equation*}
		donde $x,y \in M$ y $r \in R$. Así, al conjunto cociente 
		$M/N$ junto con las operaciones previas lo llamaremos 
		\textit{módulo cociente} de $M$ sobre $N$.
	\end{defi}

	Tras estas consideraciones básicas, comenzamos el camino que nos 
	conducirá a la definición de los \emph{espacios vectoriales de 
	homología}. Empezamos el camino con la definición de \emph{complejo de 
	cadenas}.

	\begin{defi}
	\label{def:complejo-cadenas}
	Sea $R$ un anillo. Decimos que un \textit{complejo de cadenas} sobre 
	$R$ 
	es un conjunto $\mathcal{C}_{*}=\{(C_{p},\partial_{p}) \mid p \in 
	\mathbb{Z}\}$
	de $R$-módulos y $R$-homomorfismos $\{\partial_{p}\colon C_{p} 
		\rightarrow 
	C_{p-1} \mid p \in \mathbb{Z}\}$, que satisfacen la siguiente 
	relación: $\partial_{p}\circ 
	\partial_{p+1}=0$. Se denota por $(\mathcal{C}_{*},\partial)$ y a 
	$\partial$ se le llama el \textit{diferencial} del complejo.
	\end{defi}

	Notemos que la relación $\partial_{p}\circ \partial_{p+1}=0$ es 
	equivalente a que 
	$\Ima \partial_{p+1} \leq \ker \partial_{p},\ p \in \mathbb{Z}$. Es 
	habitual pensar en $\mathcal{C}_{*}$ como una sucesión infinita cuya
	representación es como sigue:

	\begin{equation*}
		 \left.
		\begin{array}{ccccccccc}
			\cdots & \overset{\partial_{2}}{\longrightarrow} & C_{1}& 
			\overset{\partial_{1}}{\longrightarrow} & C_{0} 
					& \overset{\partial_{0}}{\longrightarrow}
			& C_{-1} & 
			\overset{\partial_{-1}}{\longrightarrow} & \cdots 
		\end{array}
		\right. 
	\end{equation*}

	Fijemos ahora nuestra atención en los $R$-submódulos de $C_{p}$: 
	$\Ima \partial_{p+1}$ y $\ker \partial_{p}$.

	\begin{defi}
		Sea $(\mathcal{C}_{*},\partial)$ un complejo de cadenas sobre 
		$R$ anillo. 
		Definimos los siguientes $R$-submódulos de $C_{p}$:
		\begin{enumerate}
			\item $Z_{p}(\mathcal{C}_{*})=\ker \partial_{p}$. A 
				sus elementos los llamaremos $p$-ciclos.
			\item $B_{p}(\mathcal{C}_{*})=\Ima\partial_{p+1}$. A 
				sus elementos los llamaremos $p$-bordes.
		\end{enumerate}
	\end{defi}

	Ahora bien, en virtud de la relación de inclusión entre ambos 
	submódulos y de las operaciones heredadas, podemos considerar el 
	\emph{$R$-módulo cociente} $Z_{p}(\mathcal{C}_{*})/B_{p}(
	\mathcal{C}_{*})$. Este cociente constituye una pieza fundamental del 
	presente trabajo y merece una definición detallada:

	\begin{defi}
		Sea $(\mathcal{C}_{*},\partial)$ un complejo de cadenas sobre 
		$R$ anillo. Definimos el $p$-ésimo $R$-módulo de homología de
		$\mathcal{C}_{*}$ como el cociente:
		$$H_{p}(\mathcal{C}_{*}):=Z_{p}(\mathcal{C}_{*})/B_{p}(
		\mathcal{C}_{*}).$$
	\end{defi}

	Ahora que ya tenemos una noción general de los $R$-módulos de 
	homología, pasamos al caso particular que nos ocupa en el presente 
	trabajo: \emph{la homología simplicial}. El primer paso será definir 
	el complejo de cadenas asociado a un complejo simplicial, tras esto, 
	fijaremos los $R$-módulos y $R$-homomorfismos sobre los que 
	finalmente calcularemos la homología. A continuación veamos la 
	definición de los complejos de cadenas:

	\begin{defi}
		Sea $R$ un anillo, {\Large $\nu$} un complejo simplicial, y $p 
		\in \mathbb{N}\cup\{0\}$ 
		tal que $p\leq dim \text{{\Large $\nu$}}$. Una 
		\textit{$p$-cadena} es una suma formal de $p$-símplices de 
		{\Large $\nu$}. Es decir, si $c$ es una $p$-cadena, entonces 
		$c=\sum a_{i}\sigma_{i},\ \sigma_{i} \in 
		\text{{\Large $\nu$}}\text{ y } 
		a_{i} \in R$. 
	\end{defi}
	
	Con la noción de $p$-cadena, pasamos a la definición de los 
	$R$-módulos que hemos mencionado anteriormente:
	\begin{defi}
		Sea $R$ un anillo y {\Large $\nu$} un complejo simplicial 
		abstracto. Definimos
		el \textit{$R$-módulo de p-cadenas} de 
		{\Large $\nu$} como el conjunto de todas las $R$-cadenas de 
		{\Large $\nu$}, con la operación suma componente a componente 
		con coeficientes en $R$, y el producto por escalares 
		por elementos de $R$. Lo denotaremos por 
		$C_{p}(\text{{\Large $\nu$}})$. 	
	\end{defi}

	Naturalmente, una decisión que debemos tomar a la hora de calcular la 
	homología simplicial es elegir el anillo de escalares sobre el que 
	vamos a trabajar. 

	Lo habitual suele ser fijar una orientación para los 
	símplices (por ejemplo, la dada por el orden lexicográfico de sus 
	vértices) y escoger el anillo $\mathbb{Z}$. Otra opción, que ha 
	demostrado dar información significativa en las aplicaciones, es 
	escoger el anillo $\mathbb{Z}_{2}$. Al tener dos elementos, el trabajo 
	sobre él resulta especialmente sencillo, sobre todo desde un punto de 
	vista computacional.

	Observamos que con la segunda opción, una $p$-cadena puede verse como
	un conjunto de $p$-símplices y la suma de ellas consiste en hacer la 
	unión de los conjuntos y eliminar los símplices repetidos un número 
	par de veces. Además, la suma coincide con la resta, y el producto por 
	escalares es nulo o la propia $p$-cadena.
	
	Habiendo especificado los $R$-módulos que vamos a emplear, pasamos a
	definir los $R$-homomorfismos asociados. Para la definición de tales 
	homomorfismos de manera general es necesario considerar un 
	ordenamiento en el conjunto {\Large $\nu$}$_{0}$. 

	\begin{defi}
		Sea $R$ un anillo, {\Large $\nu$} un complejo simplicial 
		abstracto, $\sigma \in \text{{\Large $\nu$}}$ y 
		$\sigma = \{u_{0},...,u_{p}\}$ un 
		$p$-símplice escogido de manera ordenada. 
		Definimos el \textit{operador borde para un símplice}
		como:
		$$
		B_{p}(\sigma)=\displaystyle 
		\sum_{j=0}^{p}(-1)^{j}\{u_{0},...,\widehat{u_{j}},...,u_{p}\}.
		$$
 		La suma anterior es una suma formal, donde $\widehat{u_{j}}$ 
		indica que omitimos $u_{j}$. Ahora extendemos dicho operador 
		para cadenas, en concreto:
	\begin{equation}
		\label{def:borde_form}
		\begin{array}{lll}
			\partial_{p}\colon C_{p}(\text{{\Large $\nu$}}) & 
				\rightarrow & C_{p-1}(\text{{\Large $\nu$}})
				\\[3pt] 
				\multicolumn{1}{r}{c=\sum a_{i}\sigma_{i}} & 
				\mapsto & 
				\partial_{p}(c)=\sum a_{i}B_{p}(\sigma_{i})
		\end{array}
	\end{equation}
	donde $a_{i} \in R$. Notemos que, si escogemos $R=\mathbb{Z}_{2}$, no 
	es necesaria la alternancia de signos ni el uso de coeficientes 
	escalares. A $\partial_{p}$ lo llamaremos \textit{homomorfismo borde}.	
	\end{defi}

	\begin{remark}
	Puesto que las $p$-cadenas son sumas formales, es decir, estamos 
	considerando $R$-módulos libres sobre el conjunto de $p$-símplices, la 
	aplicación borde está bien definida, y es lineal, o sea, un 
	homomorfismo entre $R$-módulos.
	\end{remark}


	Por lo tanto, ya tenemos casi construido nuestro complejo de cadenas 
	asociado a un complejo simplicial. Nótese que en la definición 
	\ref{def:complejo-cadenas} admitimos complejos de cadenas para 
	cualquier índice $p \in \mathbb{Z}$, no obstante, tomaremos 
	$C_{p}=\emptyset$ para todo $p\leq0$ con $p \in \mathbb{Z}$. Tras esta 
	consideración, sólo resta enunciar el siguiente resultado:

	\begin{theorem}
		Sea $\partial_{p}$ definida como en la expresión 
		\ref{def:borde_form}. 
		Entonces para todo $p \in \mathbb{N}\cup\{0\}$, $\partial_{p}\circ 
		\partial_{p+1}=0$.
	\end{theorem}
	\begin{proof}
		Sea $c \in C_{p+1}(\text{{\Large $\nu$}})$ y, sin pérdida
		de generalidad (en virtud de la definición 
		\ref{def:borde_form}), supongamos que $c$ es un elemento de la 
		base del $R$-módulo libre de cadenas $C_{p+1}(\text{
		{\Large $\nu$}})$.Veamos que $\partial_{p}(
		\partial_{p+1}(c))=0$.

		En efecto, notemos que, si $p\geq1$ (si $p=0$ es trivial), 
		$\partial_{p}(\partial_{p+1}(c))$ 
		posee $\binom{p+2}{p}=(p+2)(p+1)$ sumandos, y cada uno de 
		ellos es el producto de un escalar por un $(p-1)$-símplice. 
		Sea $\tau$ uno de ellos, y sean $u_{i}$, $u_{j}$ los 
		$0$-símplices presentes en el $(p+1)$-símplice de $c$ pero no 
		en $\tau$. Supondremos también que $u_{i}$, $u_{j}$ ocupan las
		posiciones $i$ y $j$ respectivamente en el $(p+1)$-símplice de
		$c$ escrito de manera ordenada.

		Observamos que $\tau$ aparece en dos sumandos de 
		$\partial_{p}(\partial_{p+1}(c))$ con las expresiones 
		opuestas:
		\begin{equation*}
			(-1)^{i}(-1)^{j}\tau, (-1)^{j-1}(-1)^{i}\tau.
		\end{equation*}

		La primera expresión se obtiene al omitir el elemento en la 
		posición $j$ y luego el de la posición $i$ del 
		$(p+1)$-símplice de $c$ (véase la definición 
		\ref{def:borde_form}); la segunda expresión se obtiene al 
		omitir el elemento en la posición $i$, con lo que se adelanta 
		una posición $u_{j}$, y luego el de la posición $j-1$.

		Notemos que al ser expresiones opuestas la suma es nula y se 
		concluye el resultado.
	\end{proof}
	
	\begin{remark}
		Coloquialmente, diremos que <<el borde del borde es vacío>>. 
		Notemos que el teorema anterior denota lo significativo de la 
		elección del anillo sobre el que se toman los coeficientes, 
		pues la demostración sería distinta y los cálculos posteriores 
		se complican. Veremos este hecho en los siguientes ejemplos.
	\end{remark}

	Veamos un ejemplo que ilustre el teorema anterior, es decir, que 
	<<el borde del borde es vacío>>. En él, como hemos sugerido, tomaremos 
	como anillo el cuerpo $\mathbb{Z}_{2}$.

	\begin{ejem}
		Supongamos el complejo simplicial {\Large $\nu$} del ejemplo 
		anterior y $\sigma = \{a,b,c\}$ $\in \text{{\Large $\nu$}}$.

		Así pues, tendremos $c \in C_{2}(\text{{\Large $\nu$}})$, con
		$c=\begin{pmatrix}
			1 \\
			0 \\
			0 
		\end{pmatrix}$ la cadena asociada a $\sigma$ en $C_{2}(
		\text{{\Large $\nu$}})$. Ahora expresamos las aplicaciones 
		$\partial_{2}$ y $\partial_{1}$ en forma matricial:
		\begin{equation*}
			\partial_{2}= \begin{pmatrix}
				1 & 1 & 0 \\
				1 & 0 & 1 \\
				0 & 1 & 1 \\
				1 & 0 & 0 \\
				0 & 1 & 0 \\
				0 & 0 & 1
			\end{pmatrix}
			\hspace{0.5cm}\text{y}\hspace{0.5cm}
			\partial_{1}= \begin{pmatrix}
				1 & 1 & 1 & 0 & 0 &0 \\
				1 & 0 & 0 & 1 & 1 &0 \\
				0 & 1 & 0 & 1 & 0 &1 \\
				0 & 0 & 1 & 0 & 1 &1
			\end{pmatrix}.
		\end{equation*}

		Ahora, teniendo en cuenta que estamos operando en un cuerpo de 
		característica 2, hacemos $\partial_{1}(\partial_{2}(c))$:
		\begin{equation*}
			\begin{split}
				\partial_{1}(\partial_{2}(c))=\begin{pmatrix}
				1 & 1 & 1 & 0 & 0 &0 \\
				1 & 0 & 0 & 1 & 1 &0 \\
				0 & 1 & 0 & 1 & 0 &1 \\
				0 & 0 & 1 & 0 & 1 &1
			\end{pmatrix}\cdot \Bigg( 
				\begin{pmatrix}
				1 & 1 & 0 \\
				1 & 0 & 1 \\
				0 & 1 & 1 \\
				1 & 0 & 0 \\
				0 & 1 & 0 \\
				0 & 0 & 1
				\end{pmatrix}
				\cdot
				\begin{pmatrix}
				1 \\
				0 \\
				0 
				\end{pmatrix} \Bigg ) = \\
				=\begin{pmatrix}
				1 & 1 & 1 & 0 & 0 &0 \\
				1 & 0 & 0 & 1 & 1 &0 \\
				0 & 1 & 0 & 1 & 0 &1 \\
				0 & 0 & 1 & 0 & 1 &1
				\end{pmatrix}
				\cdot
				\begin{pmatrix}
				1 \\
				1 \\
				0 \\
				1 \\
				0 \\
				0
				\end{pmatrix} =
				\begin{pmatrix}
                                0 \\
				0 \\
				0 \\
				0
				\end{pmatrix}=
				\vec{0}.
			\end{split}
		\end{equation*}

		Hemos comprobado que, en efecto, <<el borde del borde>> de 
		$c$ es 0. Para comprobarlo para cualquier vector bastará 
		observar que:
		\begin{equation*}
		\begin{pmatrix}
				1 & 1 & 1 & 0 & 0 &0 \\
				1 & 0 & 0 & 1 & 1 &0 \\
				0 & 1 & 0 & 1 & 0 &1 \\
				0 & 0 & 1 & 0 & 1 &1
			\end{pmatrix}\cdot 
				\begin{pmatrix}
				1 & 1 & 0 \\
				1 & 0 & 1 \\
				0 & 1 & 1 \\
				1 & 0 & 0 \\
				0 & 1 & 0 \\
				0 & 0 & 1
				\end{pmatrix}
				=
				\begin{pmatrix}
				0 & 0 & 0 \\
				0 & 0 & 0 \\
				0 & 0 & 0 \\
				0 & 0 & 0 
				\end{pmatrix}.
		\end{equation*}
	\end{ejem}

	Tras el resultado anterior y las construcciones previas, hemos acabado
	con la construcción del complejo de cadenas asociado a un complejo 
	simplicial. En consecuencia, ya estamos en posición de definir los
	\emph{$R$-módulos de homología}.

	\begin{defi}
		\label{defi:homology}
		Sea $R$ un anillo, $p \in \mathbb{N}\cup\{0\}$ y 
		{\Large $\nu$}, un complejo 
		simplicial. Definimos el \textit{p-ésimo $R$-módulo de 
		homología simplicial} del complejo simplicial {\Large $\nu$} 
		como el $R$-módulo cociente $\ker \partial_{p}/
		\Ima\partial_{p+1}$, donde $\partial_{p}$ está definida como
		en la expresión \ref{def:borde_form}. Lo denotaremos por 
		$H_{p}(\text{{\Large $\nu$}})$.
	\end{defi}

	La definición de $R$-módulos de homología tiene sentido para cualquier 
	anillo unitario y conmutativo $R$, pero se puede decir más sobre su 
	estructura si exigimos a $R$ propiedades adicionales. En particular, 
	si $R$ es un dominio de ideales principales, el teorema de estructura 
	de $R$-módulos finitamente generados (véase \cite{Algebra-Elduque}) 
	asegura que cada $R$-módulo de homología tendrá una parte libre 
	(su rango) y 
	una parte de torsión. Puesto que este estudio general alargaría en 
	exceso esta memoria, particularizamos a partir de este punto los 
	coeficientes para que sean el cuerpo con dos elementos 
	$\mathbb{Z}_{2}$, 
	lo que es suficiente para las aplicaciones de las siguientes 
	secciones. En este nuevo contexto, los $R$-módulos de homología, que 
	ahora serán $\mathbb{Z}_{2}$-espacios vectoriales, quedan 
	caracterizados por su dimensión, que se denomina \emph{número de 
	Betti} (en honor al matemático del mismo nombre, y que también puede 
	ser definido en el caso general como diferencia entre rangos).

	\begin{defi}
		Sea $H_{p}(\text{{\Large $\nu$}})$ el $p$-ésimo 
		$\mathbb{Z}_{2}$-espacio vectorial de homología de un complejo
		simplicial {\Large $\nu$}.
		A su dimensión, $dimH_{p}(\text{{\Large $\nu$}})=
		dim \ker\partial_{p}-dim\Ima\partial_{p+1}$, la 
		llamaremos \textit{p-ésimo número de Betti}, y la denotaremos
		por $\beta_{p}(\text{{\Large $\nu$}})$.
	\end{defi}

	Intuitivamente, los $p$-ciclos que no son $p$-bordes representan agujeros
	$p$-dimensionales. Por lo tanto, $\beta_{p}(\text{{\Large $\nu$}})$ 
	representa el número de $p$-agujeros de {\Large $\nu$}. Además, notemos 
	que si $dim\text{{\Large $\nu$}}=n$, entonces $\forall p >n 
	\hspace{0.25cm} H_{p}(\text{{\Large $\nu$}})=[0]$, pues 
	{\Large $\nu$}$_{p}=\emptyset$.

	Ahora veamos un ejemplo (propuesto en \cite{Intro-Otter}) en el que calculamos 
	los números de Betti dado un complejo simplicial abstracto.

	\begin{ejem}
		Supongamos el siguiente complejo simplicial abstracto:
		\begin{equation*} 
			\text{{\Large $\nu$}}=\{\{a\},\{b\},\{c\},\{d\},\{e\},
			\{a,b\},\{a,c\},\{a,d\},\{b,c\},\{c,d\},\{a,b,c\}\}.
		\end{equation*}

		Construimos la secuencia de espacios de cadenas asociados:
		\begin{equation*}
			 \left.
			\begin{array}{ccccccccc}
				\emptyset & 
				\longrightarrow & C_{2}(
				\text{{\Large $\nu$}}) & 
				\overset{\partial_{2}}{
				\longrightarrow} & C_{1}(\text{{\Large $\nu$}}) 
						& \overset{\partial_{1}}{
				\longrightarrow} 
				& C_{0}(\text{{\Large $\nu$}}) & 
				\overset{\partial_{0}}{
				\longrightarrow} & \emptyset. 
			\end{array}
			\right. 
		\end{equation*}

		Calculamos $\partial_{2}$ y $\partial_{1}$, y los expresamos 
		de manera matricial:
		\begin{equation*}
			\partial_{2}=
			\begin{pmatrix}
			1 \\
			1 \\
			0 \\
			1 \\
			0
			\end{pmatrix},
			\hspace{1cm}
			\partial_{1}=
			\begin{pmatrix}
			1 & 1 & 1 & 0 & 0\\
			1 & 0 & 0 & 1 & 0\\
			0 & 1 & 0 & 1 & 1\\
			0 & 0 & 1 & 0 & 1\\
			0 & 0 & 0 & 0 & 0
			\end{pmatrix}.
		\end{equation*}

		Ahora para calcular $\beta_{0}(\text{{\Large $\nu$}}) 
		\text{ y } \beta_{1}(\text{{\Large $\nu$}})$ bastará
		calcular el rango de las anteriores matrices. Observamos que:
		\begin{equation*}
			\partial_{1}=
			\begin{pmatrix}
			1 & 1 & 1 & 0 & 0\\
			\hlightr{1} & 0 & 0 & 1 & 0\\
			0 & \hlightr{1} & 0 & 1 & 1\\
			0 & 0 & \hlightr{1} & 0 & 1\\
			0 & 0 & 0 & 0 & 0
			\end{pmatrix}.
		\end{equation*}

		Con lo que, $dim \Ima \partial_{1}=3 \text{ y } 
		dim \ker \partial_{1}=2$, y por lo tanto, $\beta_{1}(\text{
		{\Large $\nu$}})=1$ y $\beta_{0}(\text{{\Large $\nu$}})=2$. La
		representación gráfica de {\Large $\nu$} en $\mathbb{R}^{2}$
		puede verse en la \autoref{fig:homs}.
		
		\begin{figure}[!htbp]
			\centering
			\begin{tikzpicture}
				%Nodes
				\coordinate      (a) 	at 	(0,0);
				\coordinate      (b) 	at 	(2,0);
				\coordinate      (c) 	at 	(2,2);
				\coordinate      (d) 	at 	(0,2);
				\coordinate      (e) 	at 	(4,1);
				
				%Lines
				\draw[thick] (a) -- (c);
				\draw[thick] (a) -- (b);
				\draw[thick] (a) -- (d);
				\draw[thick] (b) -- (c);
				\draw[thick] (d) -- (c);


				\fill[black!20, draw=black, thick] (a) -- (c) -- (b) -- cycle;
				\fill[black!20, draw=black, thick] (a) circle (3pt) node[black, left] {a};
				\fill[black!20, draw=black, thick] (b) circle (3pt) node[black, right]  {b};
				\fill[black!20, draw=black, thick] (c) circle (3pt) node[black, right] {c};
				\fill[black!20, draw=black, thick] (d) circle (3pt) node[black, left] {d};
				\fill[black!20, draw=black, thick] (e) circle (3pt) node[black, right] {e};

			\end{tikzpicture}
			\caption{Representación geométrica del complejo simplicial {\Large $\nu$}.}
			\label{fig:homs}
		\end{figure}
		
		Tal y como ya hemos comentado, los números de Betti nos 
		cuentan los agujeros $p$-dimensionales. En este caso, si 
		observamos la representación anterior, vemos que tenemos un 
		agujero 1-dimensional y dos componentes conexas, que se 
		corresponde con los números de Betti que hemos calculado. 

		Notemos que, al igual que en el ejemplo anterior, si 
		trabajamos sobre otro cuerpo que no sea $\mathbb{Z}_{2}$, el 
		cálculo de las matrices y los rangos sería más costoso.	
	\end{ejem}
	
	Al igual que hicimos para los complejos simpliciales, veamos cómo 
	podemos definir morfismos, de manera más general, entre los objetos 
	que estamos manejando.
	
	Consideremos una aplicación entre complejos simpliciales, $f\colon
	\text{
	{\Large $\nu$}}\rightarrow\text{{\Large $\nu$}}'$. Aplicando el mismo 
	razonamiento que para la definición del homomorfismo borde, tenemos 
	que $f$ induce un homomorfismo entre $R$-módulos de cadenas:
	\begin{equation}
		\label{hom:induc}
		\left.
		\begin{array}{lll}
			\overline{f_{p}}\colon C_{p}(\text{{\Large $\nu$}}) & 
				\rightarrow & C_{p}(\text{{\Large $\nu$}}')
				\\[3pt] 
			\multicolumn{1}{r}{c=\displaystyle \sum_{\mathclap{
			\sigma \in \text{{\Large $\nu$}$_{p}$}}}\sigma} & 
			\mapsto & 
			\overline{f_{p}}(c)=\displaystyle \sum_{\mathclap{
			f(\sigma) \in \text{{\Large $\nu$}$_{p}'$}}}f(\sigma).
		\end{array}
		\right. 
	\end{equation}
	Además, tal $f$ nos permite construir la secuencia:
	\begin{equation}
		\label{hom: gruphom}
		 \left.
		\begin{array}{ccccccccccccc}
			\emptyset & \rightarrow & C_{p}(
			\text{{\Large $\nu$}}) & \overset{\partial_{p}}{
			\rightarrow} & C_{p-1}(
			\text{{\Large $\nu$}}) & \overset{\partial_{p-1}}{
			\rightarrow} &\cdots & 
			\overset{\partial_{2}}{
			\rightarrow} & C_{1}(
			\text{{\Large $\nu$}})& \overset{\partial_{1}}{
			\rightarrow} & C_{0}(
			\text{{\Large $\nu$}})& \overset{\partial_{0}}{
			\rightarrow} & \emptyset\\[3pt]

			 & & \downarrow \overline{f_{p}}& & \downarrow 
			\overline{f_{p-1}}& 
			 & & & \downarrow \overline{f_{1}}
			 & & \downarrow \overline{f_{0}} & & \\ [3pt]
			\emptyset & \rightarrow & C_{p}(
			\text{{\Large $\nu$}}') &\underset{\partial_{p}'}{
			\rightarrow}  & C_{p-1}(
			\text{{\Large $\nu$}}') & \underset{\partial_{p-1}'}{
			\rightarrow}&\cdots & 
			\underset{\partial_{2}'}{
			\rightarrow} & C_{1}(
			\text{{\Large $\nu$}}')& \underset{\partial_{1}'}{
			\rightarrow} & C_{0}(
			\text{{\Large $\nu$}}')& \underset{\partial_{0}'}{
			\rightarrow} & \emptyset. \\
		\end{array}
		\right. 
	\end{equation}
	De esta secuencia observamos que: 
	\begin{equation*}
		\overline{f_{p-1}}\circ\partial_{p}=
		\partial_{p}'\circ\overline{f_{p}}. 
	\end{equation*}
	En consecuencia, $\overline{f_{p}}$ induce un 
	homomorfismo entre $R$-módulos de homología:	
		\begin{align*}
			\left.
			\begin{array}{l}
				f_{p}\colon H_{p}(\text{{\Large $\nu$}})
				\rightarrow 
				H_{p}(\text{{\Large $\nu$}}')\\[3pt] 
				\hspace{1.35cm} [c] \mapsto [\overline{f_{p}
				(c)}].
			\end{array}
			\right.
		\end{align*}
	Concluimos que, dada una aplicación $f$ entre complejos simpliciales, 
	siempre es posible asociarle una aplicación $f_{p}$ entre grupos de 
	homología.
	
	\begin{remark}
		Esta propiedad es muy importante, de hecho, se conoce como 
		\emph{funtorialidad} y pertenece al ámbito de la teoría de 
		categorías que queda fuera del alcance del presente trabajo. 
	\end{remark}
	
	\begin{remark}
		Si consideramos el caso particular $f=\iota$ (la inclusión) 
		tendremos que $f$ induce una inclusión entre los complejos de 
		cadenas asociados a {\Large $\nu$} y 
		{\Large $\nu$}$^{\prime}$ de manera natural. Por supuesto, 
		esta propiedad ya no se hereda al considerar las aplicaciones 
		entre $R$-módulos de homología. Es decir, $\iota_{p}$ no es 
		necesariamente una aplicación inyectiva. Muy al contrario, 
		la posibilidad de que no lo sea juega un papel fundamental en 
		el estudio de la homología persistente, que estudiaremos a 
		continuación.	
	\end{remark}

	Si bien los espacios vectoriales de homología de un complejo 
	simplicial abstracto nos aportan mucha información acerca de sus 
	características topológicas, esta información tiene bastante margen de 
	mejora pues no nos dice nada de la variable <<tiempo>>, es decir, 
	sobre el carácter dinámico de la evolución de un complejo simplicial 
	si este evolucionara en el tiempo. ¿Cómo 
	introducimos la noción de tiempo en un complejo simplicial abstracto? 
	Esta pregunta motiva la siguiente definición:

	\begin{defi}
		Sea {\Large $\nu$} un complejo simplicial abstracto finito. 
		Consideremos la secuencia $ \text{{\Large $\nu$}$^{1}$} 
		\subset \text{{\Large $\nu$}$^{2}$} \subset \cdots \subset
		\text{{\Large $\nu$}$^{k-1}$} \subset 
		\text{{\Large $\nu$}$^{k}$} = \text{{\Large $\nu$}}$ de 
		subcomplejos simpliciales cualesquiera de {\Large $\nu$}. A
		{\Large $\nu$} junto con su secuencia de subcomplejos 
		simpliciales encajados lo llamaremos \textit{complejo 
		simplicial filtrado}. 
	\end{defi}

	Esta noción nos habilita la variable <<tiempo>>, pues nos permite 
	preguntarnos en que momento de la secuencia aparecerá una cierta 
	característica topológica, y cuanto <<tiempo>> sobrevivirá dicha 
	característica.

	Existen varias maneras de construir la secuencia de complejos 
	simpliciales, por ejemplo, empleando el \emph{complejo simplicial de 
	Čech}. Su construcción se realiza de la siguiente manera:

	Sea {\Large $\nu$} un complejo simplicial y $\mathcal{U}$ un 
	cubrimiento de {\Large $\nu$}. Los $p$-símplices del complejo simplicial
	de Čech vendrán dados por la intersección no vacía de $p+1$ conjuntos de
	$\mathcal{U}$.

	Lo interesante de este método es que si $\mathcal{U}$ verifica ciertas
	condiciones, el \emph{Teorema del nervio} garantiza que el complejo de
	Čech recupera la homología de {\Large $\nu$} (para más detalles véase 
	\cite{TeoremaNervio-Ghrist}). ¿Cómo podemos capturar y visualizar esta 
	nueva información? Empleando la \emph{homología persistente}.

	\begin{remark}
	En el presente trabajo nos ceñiremos a la homología persistente 
	definida sobre un complejo simplicial filtrado. No obstante, cabe 
	destacar que esta noción tiene una generalización conocida como 
	\emph{módulo de persistencia} que se define sobre un conjunto 
	parcialmente ordenado.
	\end{remark}

	\begin{defi}
	Sea $ \text{{\Large $\nu$}$^{1}$} 
		\subset \text{{\Large $\nu$}$^{2}$} \subset \cdots \subset
		\text{{\Large $\nu$}$^{k-1}$} \subset 
		\text{{\Large $\nu$}$^{k}$} = \text{{\Large $\nu$}}$ un 
		complejo simplicial filtrado. Definimos los \textit{p-ésimos
		espacios vectoriales de homología persistente} como las 
		imágenes de los homomorfismos inducidos por la inclusión, 
		$H_{p}^{i,j}=\Ima f_{p}^{i,j}$, con $0\leq i \leq j \leq k$.

		A su dimensión, $dimH_{p}^{i,j}$, la llamaremos \textit{p-ésimo 
		número de Betti persistente} y la denotaremos por 
		$\beta_{p}^{i,j}$.
	\end{defi}

	Los homomorfismos $f_{p}^{i,j}$ los definimos siguiendo la idea dada
	por la funtorialidad. Es decir, tendremos el diagrama:
	\begin{equation*}
		\resizebox{\textwidth}{!}{$ 
		\begin{array}{ccccccccccccc}
			\emptyset & \rightarrow & C_{p}(
			\text{{\Large $\nu$}$^{1}$}) & \overset{\partial_{p}^{1}}{\rightarrow} & C_{p-1}(
			\text{{\Large $\nu$}$^{1}$}) & \overset{\partial_{p-1}^{1}}{\rightarrow} &\cdots & 
			\overset{\partial_{2}^{1}}{\rightarrow} & C_{1}(
			\text{{\Large $\nu$}$^{1}$})& \overset{\partial_{1}^{1}}{\rightarrow} & C_{0}(
			\text{{\Large $\nu$}$^{1}$})& \overset{\partial_{0}^{1}}{\rightarrow} & 
			\emptyset\\[3pt]

			& & \downarrow \overline{f_{p}^{1,2}}& & 
			\downarrow \overline{f_{p-1}^{1,2}}&  
			& &  & \downarrow \overline{f_{1}^{1,2}}
			&  & \downarrow \overline{f_{0}^{1,2}} & 
			 & \\ 	
			\emptyset & \rightarrow & C_{p}(
			\text{{\Large $\nu$}$^{2}$}) & \overset{\partial_{p}^{2}}{\rightarrow} & C_{p-1}(
			\text{{\Large $\nu$}$^{2}$}) & \overset{\partial_{p-1}^{2}}{\rightarrow} &\cdots & 
			\overset{\partial_{2}^{2}}{\rightarrow} & C_{1}(
			\text{{\Large $\nu$}$^{2}$})& \overset{\partial_{1}^{2}}{\rightarrow} & C_{0}(
			\text{{\Large $\nu$}$^{2}$})& \overset{\partial_{0}^{2}}{\rightarrow} & 
			\emptyset \\[5pt]
			
			\vdots & \vdots & \vdots & \vdots & \vdots & \vdots & 
			\vdots & \vdots & \vdots & \vdots & \vdots & \vdots & 
			\vdots \\[5pt]
			
			\emptyset & \rightarrow & C_{p}(
			\text{{\Large $\nu$}$^{k-1}$}) & \overset{\partial_{p}^{k-1}}{\rightarrow} & C_{p-1}(
			\text{{\Large $\nu$}$^{k-1}$}) & \overset{\partial_{p-1}^{k-1}}{\rightarrow} &\cdots & 
			\overset{\partial_{2}^{k-1}}{\rightarrow} & C_{1}(
			\text{{\Large $\nu$}$^{k-1}$})& \overset{\partial_{1}^{k-1}}{\rightarrow} & C_{0}(
			\text{{\Large $\nu$}$^{k-1}$})& \overset{\partial_{0}^{k-1}}{\rightarrow} & 
			\emptyset\\[3pt]

			& & \downarrow \overline{f_{p}^{k-1,k}}& & 
			\downarrow \overline{f_{p-1}^{k-1,k}}&  
			& &  & \downarrow \overline{f_{1}^{k-1,k}}
			&  & \downarrow \overline{f_{0}^{k-1,k}} &  & \\

			\emptyset & \rightarrow & C_{p}(
			\text{{\Large $\nu$}$^{k}$}) & \overset{\partial_{p}^{k}}{\rightarrow} & C_{p-1}(
			\text{{\Large $\nu$}$^{k}$}) & \overset{\partial_{p-1}^{k}}{\rightarrow} &\cdots & 
			\overset{\partial_{2}^{k}}{\rightarrow} & C_{1}(
			\text{{\Large $\nu$}$^{k}$})& \overset{\partial_{1}^{k}}{\rightarrow} & C_{0}(
			\text{{\Large $\nu$}$^{k}$})& \overset{\partial_{0}^{k}}{\rightarrow} & \emptyset. \\
		\end{array}$}
	\end{equation*}
	Donde los homomorfismos $\overline{f_{p}^{i,j}}$ entre espacios 
	vectoriales de 
	cadenas vienen inducidos por el homomorfismo inclusión $f^{i,j}\colon 
	\text{{\Large $\nu$}$^{i}$} \xhookrightarrow{} 
	\text{{\Large $\nu$}$^{j}$}$ con $0\leq i \leq j \leq k$, como en la 
	expresión
	\ref{hom:induc}. Ahora, aplicando el mismo razonamiento que en la 
	expresión 
	\ref{hom: gruphom} definimos los homomorfismos entre espacios 
	vectoriales de 
	homología $f^{i,j}_{p}$.

	Ahora que ya tenemos una herramienta que nos captura las 
	características topológicas junto con la variable <<tiempo>> en un 
	complejo simplicial abstracto filtrado, necesitamos una manera gráfica 
	de visualizar esta información. Para ello, emplearemos los  
	\emph{diagramas de barras} y los \emph{diagramas de persistencia}.
	
	Para construir los diagramas de barras elegimos, para cada $i$, una 
	base de cada espacio vectorial $H_{p}^{i,i+1}$, que representamos en el 
	diagrama mediante $\beta_{p}^{i,i+1}$ puntos. Entonces, conectamos los
	$\beta_{p}^{i,i+1}$ puntos de la $i$-ésima filtración con los 
	$\beta_{p}^{i+1,i+2}$ puntos de la filtración $i+1$ atendiendo al 
	siguiente criterio: unimos los puntos $a$ de la filtración $i$ y $b$ de
	la filtración $i+1$, si la clase del elemento que genera a $a$ es 
	preimagen por $f_{p}^{i,i+1}$ de la clase del elemento que genera a 
	$b$. Si la clase del elemento que genera a $a$ es enviada a 0 por 
	$f_{p}^{i,i+1}$, dibujaremos una linea que sale de $a$ en la filtración
	$i$ hasta la filtración $i+1$. En este caso, diremos que la clase del 
	elemento que genera a $a$ \emph{muere} en la filtración $i+1$. Si la 
	preimagen de la clase del elemento que genera a $a$ por 
	$f_{p}^{i-1,i}$ es el 0, diremos que la clase \emph{nace} en la 
	filtración $i$.

	Para los diagramas de persistencia, dibujaremos $\sum_{p}
	\beta_{p}^{i,j}$ puntos cuyas coordenadas en $\mathbb{R}^{2}$ vendrán 
	dadas por su filtración de nacimiento y de muerte en ese orden. 

	\begin{remark}
		Estos diagramas dependen de la elección de la base de los 
		espacios vectoriales subyacentes. Una mala elección nos 
		llevará a diagramas ilegibles. No obstante, la existencia de 
		una <<buena>> base está garantizada (véase 
		\cite{BaseDiagExt-Carlsson}) y la 
		coherencia de los diagramas también (véase 
		\cite{BaseDiag-Cavanna}). Adicionalmente, debemos destacar que
		en las aplicaciones el cálculo de estos diagramas y la 
		elección de una base apropiada se ha 
		realizado de manera automática (véase la
		\autoref{sec:local+prog}). 
	\end{remark}

	Veamos un ejemplo (propuesto en \cite{Intro-Otter}) ilustrativo de estos 
	conceptos:

	\begin{ejem}
	Consideremos el siguiente complejo simplicial filtrado:  
	$ \text{{\Large $\nu$}$^{1}$} 
		\subset \text{{\Large $\nu$}$^{2}$} \subset
		\text{{\Large $\nu$}$^{3}$} \subset 
		\text{{\Large $\nu$}$^{4}$} = \text{{\Large $\nu$}}$. Donde:
	\begin{itemize}
		
		\item
			$\text{{\Large $\nu$}$^{1}$}=\{\{a\},\{b\},\{c\},\{d\}
			,\{a,b\}\}$.
		\item
			$\text{{\Large $\nu$}$^{2}$}=\{\{a\},\{b\},\{c\},
				\{d\},\{e\},\{f\},\{a,b\},\{a,c\},\{b,c\}\}$.
		\item 
			$\text{{\Large $\nu$}$^{3}$}=\{\{a\},\{b\},\{c\},
				\{d\}, \{e\},\{f\},\{a,b\},\{a,c\},\{a,e\},
				\{b,c\},\{b,f\},\{c,f\},\{c,e\},$\\
			$\{a,b,c\}\}$.
		\item 
			$\text{{\Large $\nu$}$^{4}$}=\{\{a\},\{b\},\{c\},
				\{d\},\{e\},\{f\},\{a,b\},\{a,c\},\{a,e\},
				\{b,c\},\{b,f\},\{c,f\},\{c,e\},$\\
				$\{a,b,c\},\{a,c,e\}\}$.
	\end{itemize}
	En la \autoref{fig:EjBas} podemos ver su representación gráfica en 
	$\mathbb{R}^{2}$.
	\begin{figure}[!htbp]
		\fbox{\minipage{0.225\textwidth}
			\begin{figure}[H]
				\resizebox{\textwidth}{!}{

			\begin{tikzpicture}
				%Nodes
				\node 		 (label) at (2,5) {{\Large $\nu$}$^{1}$};
				\coordinate      (a) 	at 	(0,0);
				\coordinate      (b) 	at 	(2,0);
				\coordinate      (c) 	at 	(2,2);
				\coordinate      (d) 	at 	(4,2);
				\coordinate      (e) 	at 	(2,4);
				\coordinate      (f) 	at 	(4,0);
				
				%Lines
				\draw[thick] (a) -- (b);


				\fill[black!20, draw=black, thick] (a) circle (3pt) node[black, below] {a};
				\fill[black!20, draw=black, thick] (b) circle (3pt) node[black, below]  {b};
				\fill[black!20, draw=black, thick] (c) circle (3pt) node[black, right] {c};
				\fill[black!20, draw=black, thick] (d) circle (3pt) node[black, right] {d};

		\end{tikzpicture}}
			\end{figure}	
		\endminipage}
		\fbox{\minipage{0.225\textwidth}
			\begin{figure}[H]
				\resizebox{1.0\textwidth}{!}{
				\begin{tikzpicture}
					%Nodes
					\node 		 (label) at (2,5) {{\Large $\nu$}$^{2}$};
					\coordinate      (a) 	at 	(0,0);
					\coordinate      (b) 	at 	(2,0);
					\coordinate      (c) 	at 	(2,2);
					\coordinate      (d) 	at 	(4,2);
					\coordinate      (e) 	at 	(2,4);
					\coordinate      (f) 	at 	(4,0);
					
					%Lines
					\draw[thick] (a) -- (c);
					\draw[thick] (a) -- (b);
					\draw[thick] (b) -- (c);

					\fill[black!20, draw=black, thick] (a) circle (3pt) node[black, below] {a};
					\fill[black!20, draw=black, thick] (b) circle (3pt) node[black, below]  {b};
					\fill[black!20, draw=black, thick] (c) circle (3pt) node[black, right] {c};
					\fill[black!20, draw=black, thick] (d) circle (3pt) node[black, right] {d};
					\fill[black!20, draw=black, thick] (e) circle (3pt) node[black, right] {e};
					\fill[black!20, draw=black, thick] (f) circle (3pt) node[black, right] {f};

				\end{tikzpicture}}
			\end{figure}	
		\endminipage}
		\fbox{\minipage{0.225\textwidth}
			\begin{figure}[H]
				\resizebox{1.0\textwidth}{!}{
				\begin{tikzpicture}
					%Nodes
					\node 		 (label) at (2,5) {{\Large $\nu$}$^{3}$};
					\coordinate      (a) 	at 	(0,0);
					\coordinate      (b) 	at 	(2,0);
					\coordinate      (c) 	at 	(2,2);
					\coordinate      (d) 	at 	(4,2);
					\coordinate      (e) 	at 	(2,4);
					\coordinate      (f) 	at 	(4,0);
					
					%Lines
					\draw[thick] (a) -- (e);
					\draw[thick] (c) -- (f);
					\draw[thick] (c) -- (e);
					\draw[thick] (b) -- (f);

					\fill[black!20, draw=black, thick] (a) -- (b) -- (c) -- cycle;
					\fill[black!20, draw=black, thick] (a) circle (3pt) node[black, below] {a};
					\fill[black!20, draw=black, thick] (b) circle (3pt) node[black, below]  {b};
					\fill[black!20, draw=black, thick] (c) circle (3pt) node[black, right] {c};
					\fill[black!20, draw=black, thick] (d) circle (3pt) node[black, right] {d};
					\fill[black!20, draw=black, thick] (e) circle (3pt) node[black, right] {e};
					\fill[black!20, draw=black, thick] (f) circle (3pt) node[black, right] {f};

				\end{tikzpicture}}
			\end{figure}	
		\endminipage}
		\fbox{\minipage{0.225\textwidth}
			\begin{figure}[H]
				\resizebox{1.0\textwidth}{!}{
				\begin{tikzpicture}
					%Nodes
					\node 		 (label) at (2,5) {{\Large $\nu$}$^{4}$};
					\coordinate      (a) 	at 	(0,0);
					\coordinate      (b) 	at 	(2,0);
					\coordinate      (c) 	at 	(2,2);
					\coordinate      (d) 	at 	(4,2);
					\coordinate      (e) 	at 	(2,4);
					\coordinate      (f) 	at 	(4,0);
					
					%Lines
					\draw[thick] (c) -- (f);
					\draw[thick] (b) -- (f);

					\fill[black!20, draw=black, thick] (a) -- (b) -- (c) -- cycle;
					\fill[black!20, draw=black, thick] (a) -- (e) -- (c) -- cycle;
					\fill[black!20, draw=black, thick] (a) circle (3pt) node[black, below] {a};
					\fill[black!20, draw=black, thick] (b) circle (3pt) node[black, below]  {b};
					\fill[black!20, draw=black, thick] (c) circle (3pt) node[black, right] {c};
					\fill[black!20, draw=black, thick] (d) circle (3pt) node[black, right] {d};
					\fill[black!20, draw=black, thick] (e) circle (3pt) node[black, right] {e};
					\fill[black!20, draw=black, thick] (f) circle (3pt) node[black, right] {f};

				\end{tikzpicture}}
			\end{figure}	
		\endminipage}
		\caption{Representación gráfica de los pasos en la filtración 
		de {\Large $\nu$}.}
		\label{fig:EjBas}
	\end{figure}
	En la \autoref{fig:EjBasDiagB} podemos ver los diagramas de barras 
	correspondientes.
	
	\begin{figure}[!htbp]
		\fbox{\minipage{0.48\textwidth}
			\begin{figure}[H]
				\resizebox{\textwidth}{!}{
				
				\begin{tikzpicture}[scale=2.5]
				%Nodes
				\node 		 (label) at (2,3) {Grado 0};
				\node 		 (label) at (2,-0.5) {Filtración};
				\coordinate      (n0) 	at 	(0,0);
				\coordinate      (n1) 	at 	(1,0);
				\coordinate      (n2) 	at 	(2,0);
				\coordinate      (n3) 	at 	(3,0);
				\coordinate      (n4) 	at 	(4,0);
				\coordinate      (n5) 	at 	(0,0.5);
				\coordinate      (n6) 	at 	(1,0.5);
				\coordinate      (n7) 	at 	(2,0.5);
				\coordinate      (n8) 	at 	(3,0.5);
				\coordinate      (n9) 	at 	(4,0.5);
				\coordinate      (n10) 	at 	(0,1);
				\coordinate      (n11) 	at 	(1,1);
				\coordinate      (n12) 	at 	(2,1);
				\coordinate      (n13) 	at 	(3,1);
				\coordinate      (n14) 	at 	(4,1);
				\coordinate      (n15) 	at 	(0,1.5);
				\coordinate      (n16) 	at 	(1,1.5);
				\coordinate      (n17) 	at 	(2,1.5);
				\coordinate      (n18) 	at 	(3,1.5);
				\coordinate      (n19) 	at 	(4,1.5);
				\coordinate      (n20) 	at 	(0,2);
				\coordinate      (n21) 	at 	(1,2);
				\coordinate      (n22) 	at 	(2,2);
				\coordinate      (n23) 	at 	(3,2);
				\coordinate      (n24) 	at 	(4,2);
				\coordinate      (n25) 	at 	(0,2.5);
				\coordinate      (n26) 	at 	(1,2.5);
				\coordinate      (n27) 	at 	(2,2.5);
				\coordinate      (n28) 	at 	(3,2.5);
				\coordinate      (n29) 	at 	(4,2.5);
				
				%Lines
				\draw[thick] (n0) -- (n3);
				\draw[thick,->] (n3) -- (n4);
				\draw[thick] (0,3pt) -- (0,-3pt) node[anchor=north] {1};
				\draw[thick] (1,3pt) -- (1,-3pt) node[anchor=north] {2};
				\draw[thick] (2,3pt) -- (2,-3pt) node[anchor=north] {3};
				\draw[thick] (3,3pt) -- (3,-3pt) node[anchor=north] {4};


				\draw[thick,->,color=blue] (n5) -- (n9);
				\draw[thick,color=blue] (n10) -- (n11);
				\draw[thick,->,color=blue] (n15) -- (n19);
				\draw[thick,color=blue] (n21) -- (n22);
				\draw[thick,color=blue] (n26) -- (n27);

		\end{tikzpicture}}
			\end{figure}	
		\endminipage}
		\fbox{\minipage{0.48\textwidth}
			\begin{figure}[H]
				\resizebox{\textwidth}{!}{
				\begin{tikzpicture}[scale=2.5]
					%Nodes
					\node 		 (label) at (2,3) {Grado 1};
					\node 		 (label) at (2,-0.5) {Filtración};
				\coordinate      (n0) 	at 	(0,0);
				\coordinate      (n1) 	at 	(1,0);
				\coordinate      (n2) 	at 	(2,0);
				\coordinate      (n3) 	at 	(3,0);
				\coordinate      (n4) 	at 	(4,0);
				\coordinate      (n5) 	at 	(0,0.5);
				\coordinate      (n6) 	at 	(1,0.5);
				\coordinate      (n7) 	at 	(2,0.5);
				\coordinate      (n8) 	at 	(3,0.5);
				\coordinate      (n9) 	at 	(4,0.5);
				\coordinate      (n10) 	at 	(0,1);
				\coordinate      (n11) 	at 	(1,1);
				\coordinate      (n12) 	at 	(2,1);
				\coordinate      (n13) 	at 	(3,1);
				\coordinate      (n14) 	at 	(4,1);
				\coordinate      (n15) 	at 	(0,1.5);
				\coordinate      (n16) 	at 	(1,1.5);
				\coordinate      (n17) 	at 	(2,1.5);
				\coordinate      (n18) 	at 	(3,1.5);
				\coordinate      (n19) 	at 	(4,1.5);
				\coordinate      (n20) 	at 	(0,2);
				\coordinate      (n21) 	at 	(1,2);
				\coordinate      (n22) 	at 	(2,2);
				\coordinate      (n23) 	at 	(3,2);
				\coordinate      (n24) 	at 	(4,2);
				\coordinate      (n25) 	at 	(0,2.5);
				\coordinate      (n26) 	at 	(1,2.5);
				\coordinate      (n27) 	at 	(2,2.5);
				\coordinate      (n28) 	at 	(3,2.5);
				\coordinate      (n29) 	at 	(4,2.5);
				
				%Lines
				\draw[thick] (n0) -- (n3);
				\draw[thick,->] (n3) -- (n4);
				\draw[thick] (0,3pt) -- (0,-3pt) node[anchor=north] {1};
				\draw[thick] (1,3pt) -- (1,-3pt) node[anchor=north] {2};
				\draw[thick] (2,3pt) -- (2,-3pt) node[anchor=north] {3};
				\draw[thick] (3,3pt) -- (3,-3pt) node[anchor=north] {4};


				\draw[thick,color=red] (n6) -- (n7);
				\draw[thick,->,color=red] (n12) -- (n14);
				\draw[thick,color=red] (n17) -- (n18);

				\end{tikzpicture}}
			\end{figure}	
		\endminipage}
		\caption{Diagramas de barras asociados a la filtración de 
		{\Large $\nu$}.}
		\label{fig:EjBasDiagB}
	\end{figure}

	Como podemos observar en la \autoref{fig:EjBasDiagB} diagrama 
	izquierdo (grado 0), 
	tenemos inicialmente (filtración 1 que corresponde al dibujo de 
	{\Large $\nu$}$^{1}$) tres clases de equivalencia que son, en orden 
	ascendente 
	en el diagrama, las clases $[d],[a+b],$ y $[c]$. Como ya hemos visto, 
	esta representación se corresponde con las componentes conexas de 
	{\Large $\nu$}$^{1}$. A medida que avanzamos en las 
	filtraciones podemos ver cómo la clase $[a+b]$ muere en la filtración 
	2
	(que se corresponde con el dibujo de {\Large $\nu$}$^{2}$), nacen dos 
	clases más ($[f]$ y $[e]$ en orden ascendente en el diagrama), ambas 
	clases mueren en la filtración 3 (véase el dibujo de 
	{\Large $\nu$}$^{3}$), y finalmente, las clases $[d]$ y $[c]$ 
	persisten.
	
	Observemos ahora la \autoref{fig:EjBasDiagB} diagrama derecho 
	(grado 1). Como se puede 
	ver, en la filtración 1 no tenemos ninguna clase de equivalencia, 
	tendremos que esperar a la filtración 2 en la que nace la clase 
	$[ab+bc+ac]$. Esta clase muere en la siguiente filtración, pero nacen 
	otras dos: la clase $[bc+cf+bf]$ y la clase $[ac+ae+ce]$. Esta última 
	morirá en la filtración 4, no obstante, la clase $[bc+cf+bf]$ 
	persiste. Como ya mencionamos, estas clases de equivalencia se 
	corresponden con agujeros $2$-dimensionales, cuya evolución podemos ir
	viendo en los dibujos anteriores que representan los pasos en la 
	filtración del complejo simplicial {\Large $\nu$}.

	De los anteriores diagramas podemos extraer información topológica 
	muy importante acerca de nuestro complejo simplicial: vemos que 
	presenta dos componentes conexas y un agujero $2$-dimensional. Esta 
	información nos la aporta la homología persistente, ya que las 
	clases de 
	equivalencia que persisten representan características topológicas, 
	mientras que las que mueren en alguna filtración representan ruido.

	Por último, en la \autoref{fig:EjBasDiagP} podemos visualizar los 
	diagramas de persistencia.
	\begin{figure}[!htbp]
		\fbox{\minipage{0.48\textwidth}
			\begin{figure}[H]
				\resizebox{\textwidth}{!}{
				
				\begin{tikzpicture}[scale=2.5]
				%Nodes
				\node 		 (label) at (2,4) {Grado 0};
				\node 		 (label) at (2,-0.5) {Nacimiento};
				\node[rotate=90] 		 (label) at (-0.5,1.5) {Muerte};
				\coordinate      (n0) 	at 	(0,0);
				\coordinate      (n1) 	at 	(1,0);
				\coordinate      (n2) 	at 	(2,0);
				\coordinate      (n3) 	at 	(3,0);
				\coordinate      (n4) 	at 	(4,0);
				\coordinate      (n5) 	at 	(0,0.5);
				\coordinate      (n6) 	at 	(1,0.5);
				\coordinate      (n7) 	at 	(2,0.5);
				\coordinate      (n8) 	at 	(3,0.5);
				\coordinate      (n9) 	at 	(4,0.5);
				\coordinate      (n10) 	at 	(0,1);
				\coordinate      (n11) 	at 	(1,1);
				\coordinate      (n12) 	at 	(2,1);
				\coordinate      (n13) 	at 	(3,1);
				\coordinate      (n14) 	at 	(4,1);
				\coordinate      (n15) 	at 	(0,1.5);
				\coordinate      (n16) 	at 	(1,1.5);
				\coordinate      (n17) 	at 	(2,1.5);
				\coordinate      (n18) 	at 	(3,1.5);
				\coordinate      (n19) 	at 	(4,1.5);
				\coordinate      (n20) 	at 	(0,2);
				\coordinate      (n21) 	at 	(1,2);
				\coordinate      (n22) 	at 	(2,2);
				\coordinate      (n23) 	at 	(3,2);
				\coordinate      (n24) 	at 	(4,2);
				\coordinate      (n25) 	at 	(0,2.5);
				\coordinate      (n26) 	at 	(1,2.5);
				\coordinate      (n27) 	at 	(2,2.5);
				\coordinate      (n28) 	at 	(3,2.5);
				\coordinate      (n29) 	at 	(4,2.5);
				
				%Lines
				\draw[thick] (n0) -- (n3);
				\draw[thick,->] (n3) -- (n4);
				\draw[thick] (0,2pt) -- (0,-2pt) node[anchor=north] {1};
				\draw[thick] (1,2pt) -- (1,-2pt) node[anchor=north] {2};
				\draw[thick] (2,2pt) -- (2,-2pt) node[anchor=north] {3};
				\draw[thick] (3,2pt) -- (3,-2pt) node[anchor=north] {4};
				
				\draw[thick,->] (n0) -- (0,3.7);
				\draw[thick] (2pt,1) -- (-2pt,1) node[anchor=north] {2};
				\draw[thick] (2pt,2) -- (-2pt,2) node[anchor=north] {3};
				\draw[thick] (2pt,3) -- (-2pt,3) node[anchor=north] {4};
				\node[anchor=east] at (-0.05,3.5) {$\infty$};


				\draw[thick,color=blue] (n0) -- (4,3.5);
				\draw[thick] (0,3.5) -- (4,3.5);
				\fill[blue] (0,1) circle (1pt);
				\fill[blue] (1,2) circle (2pt);
				\fill[blue] (0,3.5) circle (2pt);

		\end{tikzpicture}}
			\end{figure}	
		\endminipage}
		\fbox{\minipage{0.48\textwidth}
			\begin{figure}[H]
				\resizebox{\textwidth}{!}{
				
				\begin{tikzpicture}[scale=2.5]
				%Nodes
				\node 		 (label) at (2,4) {Grado 1};
				\node 		 (label) at (2,-0.5) {Nacimiento};
				\node[rotate=90] 		 (label) at (-0.5,1.5) {Muerte};
				\coordinate      (n0) 	at 	(0,0);
				\coordinate      (n1) 	at 	(1,0);
				\coordinate      (n2) 	at 	(2,0);
				\coordinate      (n3) 	at 	(3,0);
				\coordinate      (n4) 	at 	(4,0);
				\coordinate      (n5) 	at 	(0,0.5);
				\coordinate      (n6) 	at 	(1,0.5);
				\coordinate      (n7) 	at 	(2,0.5);
				\coordinate      (n8) 	at 	(3,0.5);
				\coordinate      (n9) 	at 	(4,0.5);
				\coordinate      (n10) 	at 	(0,1);
				\coordinate      (n11) 	at 	(1,1);
				\coordinate      (n12) 	at 	(2,1);
				\coordinate      (n13) 	at 	(3,1);
				\coordinate      (n14) 	at 	(4,1);
				\coordinate      (n15) 	at 	(0,1.5);
				\coordinate      (n16) 	at 	(1,1.5);
				\coordinate      (n17) 	at 	(2,1.5);
				\coordinate      (n18) 	at 	(3,1.5);
				\coordinate      (n19) 	at 	(4,1.5);
				\coordinate      (n20) 	at 	(0,2);
				\coordinate      (n21) 	at 	(1,2);
				\coordinate      (n22) 	at 	(2,2);
				\coordinate      (n23) 	at 	(3,2);
				\coordinate      (n24) 	at 	(4,2);
				\coordinate      (n25) 	at 	(0,2.5);
				\coordinate      (n26) 	at 	(1,2.5);
				\coordinate      (n27) 	at 	(2,2.5);
				\coordinate      (n28) 	at 	(3,2.5);
				\coordinate      (n29) 	at 	(4,2.5);
				
				%Lines
				\draw[thick] (n0) -- (n3);
				\draw[thick,->] (n3) -- (n4);
				\draw[thick] (0,2pt) -- (0,-2pt) node[anchor=north] {1};
				\draw[thick] (1,2pt) -- (1,-2pt) node[anchor=north] {2};
				\draw[thick] (2,2pt) -- (2,-2pt) node[anchor=north] {3};
				\draw[thick] (3,2pt) -- (3,-2pt) node[anchor=north] {4};
				
				\draw[thick,->] (n0) -- (0,3.7);
				\draw[thick] (2pt,1) -- (-2pt,1) node[anchor=north] {2};
				\draw[thick] (2pt,2) -- (-2pt,2) node[anchor=north] {3};
				\draw[thick] (2pt,3) -- (-2pt,3) node[anchor=north] {4};
				\node[anchor=east] at (0,3.5) {$\infty$};


				\draw[thick,color=red] (n0) -- (4,3.5);
				\draw[thick] (0,3.5) -- (4,3.5);
				\fill[red] (2,3) circle (1pt);
				\fill[red] (1,2) circle (1pt);
				\fill[red] (2,3.5) circle (1pt);

		\end{tikzpicture}}
			\end{figure}	
		\endminipage}
		\caption{Diagramas de persistencia asociados a la filtración 
		de {\Large $\nu$}.}
		\label{fig:EjBasDiagP}
	\end{figure}

	En este caso, si nos fijamos en la \autoref{fig:EjBasDiagP} 
	diagrama izquierdo (grado 0), vemos
	que hay una clase de equivalencia ($[a+b]$) que nace en la 
	filtración 1 y muere en la filtración 2, dos clases ($[f]$ y $[e]$) 
	que 
	nacen en la filtración 2 y mueren en la filtración 3, y dos clases que
	nacen en la filtración 1 y persisten. Para representar esta 
	persistencia se ha dibujado en el diagrama una línea horizontal que 
	simboliza el infinito. Como se puede ver en los diagramas, para 
	remarcar que hay varias clases que comparten nacimiento y muerte se 
	han señalizado con un punto grueso.

	Si nos fijamos ahora en la \autoref{fig:EjBasDiagP} diagrama 
	derecho (grado 1), observamos cómo
	hay una clase de equivalencia ($[ab+bc+ac]$) que nace en la filtración 
	2
	y muere en la filtración 3, una clase ($[ac+ae+ce]$) que nace en la 
	filtración 3 y muere en la filtración 4, y una clase ($[bc+cf+bf]$) 
	que nace en la filtración 3 y persiste.
	\end{ejem}

	\section{Redes neuronales. Grafos subyacentes}
	
	Ahora cambiamos completamente de tema: nos alejamos momentáneamente de 
	la topología algebraica y nos acercamos al mundo 
	de las ciencias de la información. En esta sección haremos una
	introducción hacia el otro concepto fundamental del presente trabajo: 
	\emph{las redes neuronales artificiales}. A diferencia de la sección 
	anterior, en la presente sección haremos una aproximación un poco menos 
	formal de las nociones que se introducen. Para el desarrollo de esta 
	sección emplearemos la guía proporcionada por 
	\cite{Goodfellow-et-al-2016,IA-Jonathan,MDiscreta-Guti}.

	La historia del \emph{aprendizaje profundo} y de las redes neuronales 
	es muy amplia y queda fuera del alcance del presente trabajo. No 
	obstante, conviene reseñar que, aunque los términos aprendizaje 
	profundo y red neuronal nos puedan parecer algo recientes, provienen 
	de una disciplina con un largo recorrido temporal. El origen de esta 
	disciplina data de los años 40 de la mano de Warren McCulloch y Walter 
	Pitts, quienes inspirados en el cerebro humano propusieron el primer 
	modelo de red neuronal artificial. Tras ellos, en los años 50, Frank 
	Rosenblatt propuso el primer modelo del que más tarde surgieron las
	redes neuronales modernas: \emph{el perceptrón}. En esa misma época se
	comenzó a usar el algoritmo del \emph{descenso del gradiente}, que 
	se emplea para optimizar de manera automática los modelos de redes 
	neuronales. Tras esto se alcanzaron multitud de hitos importantes, 
	como la victoria en 1997 de la computadora Deep Blue ante el campeón 
	mundial de ajedrez Gari Kaspárov, que hicieron avanzar la disciplina 
	hasta nuestros días. Actualmente, gracias a la potencia de cómputo 
	moderna la inteligencia artificial tiene multitud de aplicaciones: la 
	conducción autónoma, la generación de imágenes realistas, la 
	traducción automática, la detección de tumores, etc. No obstante, cabe
	reseñar que, al igual que con otras líneas de investigación, la 
	inteligencia artificial (y en particular el 
	aprendizaje profundo) sufrió varios estancamientos y contratiempos 
	hasta llegar a nuestros días (<<inviernos de la IA>>).

	Supongamos que queremos que nuestro ordenador nos escriba con palabras
	un número entre 0 y 9, dado dicho número en cifras. Aparentemente es 
	una tarea sencilla, bastará con desarrollar un programa que a partir 
	de una entrada (la cifra) nos produzca una salida (la palabra) de 
	entre las nueve posibilidades; resumidamente, unas pocas sentencias 
	condicionales bastarán para obtener el resultado esperado.  

	Pensemos ahora en la 
	misma tarea, pero esta vez le proporcionaremos como entrada una imagen
	de la cifra manuscrita. Ahora el problema es 
	significativamente más difícil. La aproximación que propone el 
	aprendizaje automático es dejar que el ordenador <<aprenda>> a hacer 
	esa tarea. La idea intuitiva es desarrollar un programa que tome dos 
	entradas: la imagen y unos parámetros; y que a partir de estas 
	entradas proporcione una salida. Tras esto, ajustar automáticamente 
	los parámetros para mejorar el desempeño del programa. Iterando este 
	proceso de predicción-ajuste, llegaremos a proporcionar la salida 
	correcta. Una vez que hemos encontrado los parámetros correctos,
	podemos abandonar este proceso de ajuste y tendremos un 
	programa clásico que a partir de unas entradas produce unas salidas.

	Teniendo clara la idea intuitiva, vamos a ver su desarrollo hasta 
	llegar a las redes neuronales.

	Supongamos que queremos modelar la relación lineal entre una variable
	objetivo ($y$) y $p$ variables independientes ($x_{1},\dots,x_{p}$). 
	Este modelo se conoce como regresión lineal y puede ser expresado 
	como:
	\begin{equation*}
		\hat{y}=w_{1}x_{1}+w_{2}x_{2}+\cdots+w_{p}x_{p}+b,
	\end{equation*}
	donde $\hat{y}$ es valor estimado de $y$; $w_{1},w_{2},\cdots,w_{p}$ 
	son los pesos que indican la importancia de cada variable 
	independiente para el modelo; y $b$ es el término independiente. Aquí 
	los parámetros $w_{1},w_{2},\cdots,w_{p}$ y el término independiente 
	$b$ toman valores en $\mathbb{R}$.
	Volviendo a la terminología introducida por la idea intuitiva, 
	$\hat{y}$ es nuestra salida, las 
	variables independientes son nuestras entradas, y $w_{1},w_{2},\cdots,
	w_{p}$ son los parámetros a ajustar. 

	Para realizar este ajuste será necesario tener una métrica que nos 
	indique si un conjunto de parámetros es mejor que otro. Para ello,
	usaremos el error cuadrático medio cuya expresión es la siguiente:

	\begin{equation*}
		MSE(w_{1},w_{2},\cdots,w_{p},b)=\frac{1}{n}\sum_{i=1}^{n}(y_{i}-\hat{y}_{i})^2,
	\end{equation*}

	\noindent donde $n$ es el número de datos (cada dato es un vector de $p$ 
	entradas), $y_{i}$ es la salida real y 
	$\hat{y}_{i}$ es la salida estimada por el modelo. Nuestro objetivo
	es el de minimizar $MSE(w_{1},w_{2},\cdots,w_{p},b)$ para que el modelo 
	sea el mejor posible. 
	Para minimizar esta función será necesario ir ajustando los parámetros 
	correctamente. Para ello se emplea el algoritmo conocido como 
	\emph{el descenso del gradiente}. A continuación, veamos cómo funciona 
	este algoritmo.

	Supongamos una función continua y suave. Por ser ella continua sabemos
	que mapea puntos <<cercanos>> a puntos <<cercanos>>, y por ser suave,
	sabemos que podemos aproximar valores <<cercanos>> a un punto de la 
	función por una función lineal cuya pendiente sea la diferencial de la 
	función. Este resultado se conoce como el \emph{teorema de Taylor} (un 
	enunciado más riguroso, así como su demostración, puede verse en 
	\cite{Diferencial-Bello}). El gradiente de una función en un punto 
	nos indica cuanto aumenta o disminuye el valor 
	de dicha función ante un pequeño cambio en su entrada. Lo que haremos 
	será <<movernos un poco>> en la dirección marcada por el gradiente 
	pero en sentido opuesto, pues nuestro objetivo 
	es el de minimizar la función. Iterando este proceso,
	iremos dando pequeños <<pasos>> minimizando $MSE(w_{1},w_{2},\cdots,
	w_{p},b)$. En concreto, la 
	actualización de los parámetros es como sigue:
	
	\begin{equation*}
	\begin{array}{c}
		w_{i}^{siguiente}=w_{i}^{actual}-\alpha\frac{\partial MSE(w,b)}
		{\partial w_{i}}\\
		b^{siguiente}=b^{actual}-\alpha\frac{\partial MSE(w,b)}
		{\partial w_{i}}
	\end{array},
	\end{equation*}
	donde $\alpha$ es el ratio de aprendizaje: un hiperparámetro que 
	tenemos que elegir. Este paso de actualización se conoce como 
	\emph{propagación hacia atrás}, y el paso del cálculo de $\hat{y}$ se 
	conoce como \emph{propagación hacia adelante}. Una vez hemos logrado 
	la 
	configuración óptima de los parámetros ya podemos emplear este modelo 
	para la predicción de $y$.
	
	\begin{remark}
	Notemos que el proceso presentado logra teóricamente la convergencia 
	al mínimo local de $MSE(w_{1},w_{2},\cdots,w_{p},b)$. Para converger de 
	manera teórica al 
	mínimo global son necesarias técnicas más avanzadas cuya descripción 
	queda fuera del alcance del presente trabajo. Para más información 
	véase \cite{Goodfellow-et-al-2016}.
	\end{remark}

	Desafortunadamente, el modelo de regresión lineal es muy limitado: 
	supone una relación lineal entre la variable objetivo y las variables
	independientes. Se hace necesaria una generalización que conseguiremos
	introduciendo una función no lineal a la regresión lineal. El primer
	modelo conocido con esta definición es el \emph{perceptrón}. Este 
	modelo se expresa de la siguiente manera:
	\begin{equation*}
		\hat{y}=r(w_{1}x_{1}+w_{2}x_{2}+\cdots+w_{p}x_{p}+b),
	\end{equation*}
	donde, en general, $r(x)$ es una función no lineal que se conoce como 
	\emph{función de activación} y el resto de variables son como en la 
	regresión lineal. En el caso particular del perceptrón, $r(x)$ está 
	definida de la siguiente manera:
	
	\begin{equation}
		\label{def:r}
		r(x)=\left \{
			\begin{array}{ll}
				1&\text{si }x>0\\
				0&\text{si }x\leq0
			\end{array} 
		     \right .
	\end{equation}

	Observamos que la función $r(x)$ no es continua, por lo que no 
	podemos aplicar el algoritmo del descenso del gradiente para la 
	actualización de los parámetros. En su lugar se utiliza la siguiente 
	fórmula:
	\begin{equation*}
		w_{i}^{siguiente}=w_{i}^{actual}+\alpha(y-\hat{y})x_{i}
	\end{equation*}
	donde $\alpha$ es el ratio de aprendizaje, y $y$ es la salida 
	esperada. Obtenemos así un modelo más general que no presupone una 
	relación lineal entre la variable objetivo y las variables 
	independientes. 
	
	La función $r(x)$ puede ser considerada como un hiperparámetro, es decir,
	puede ser instanciada a una función no lineal distinta de la considerada 
	anteriormente. Podemos, por ejemplo, considerar la función sigmoide, 
	la tangente hiperbólica o la función ReLU (véase 
	\cite{Goodfellow-et-al-2016}). Dado que habitualmente se utiliza la función ReLU, vamos a detenernos 
	brevemente a analizar su uso. La función ReLU está definida de la 
	siguiente manera:

	\begin{equation}
		\label{def:ReLU}
		ReLU(x)=\left \{
			\begin{array}{ll}
				x&\text{si }x>0\\
				0&\text{si }x\leq0
			\end{array} 
		     \right .
	\end{equation}
	
	Observamos que la función ReLU es continua pero no suave, pues no es 
	diferenciable en $x=0$. Por lo tanto, teóricamente no podremos 
	asegurar que $\hat{y}$ lo sea (y por tanto $MSE$ tampoco) y no 
	podremos utilizar el algoritmo del 
	descenso del gradiente. Sin embargo, por 
	cuestiones prácticas, este incidente no se tiene en cuenta ya que 
	habitualmente no se alcanza el mínimo local de $\hat{y}$, luego a 
	priori, no hay problemas en que la función no sea diferenciable en 
	dicho punto. Adicionalmente, cabe 
	destacar que la mayoría de los algoritmos empleados para el cálculo de 
	la derivada en el descenso del gradiente sólo calculan una de las 
	derivadas laterales.

	A fin de facilitar el entendimiento de nuestros siguientes pasos, 
	en la \autoref{fig:perceptron} representamos gráficamente la 
	estructura subyacente del perceptrón.
	\begin{figure}[!htbp]
		\centering
		\begin{tikzpicture}[
			roundnode/.style={circle, draw=black, thick, fill=white, minimum size=7mm},
			]
			%Nodes
			\node[roundnode]      (x1) 	at 	(-1,2)           {$x_{1}$};
			\node[roundnode]      (x2)      at 	(-1,1)		{$x_{2}$};
			\node      (p)      at 	(-1,0)		{$\vdots$};
			\node[roundnode]      (xn)      at 	(-1,-1)		{$x_{p}$};
			\node[roundnode]      (b)      at 	(-1,-2)		{$1$};
			\node[roundnode, text width=2.2cm, align=center]      (ac)      at 	(3,0)		{Función de activación $r$};
			\node[roundnode]      (dest)      at 	(11,0) {$\hat{y}$};
			
			\begin{pgfonlayer}{background}
			%Lines
			\draw[thick,->,rounded corners] (x1.mid) -| node[above,pos=0.2] {$w_{1}$} (1,0) -- (ac.west);
			\draw[thick,->,rounded corners] (x2.mid) -| node[above,pos=0.2] {$w_{2}$} (1,0) --(ac.west);
			\draw[thick,->,rounded corners] (xn.mid) -| node[above,pos=0.2] {$w_{p}$} (1,0) --(ac.west);
			\draw[thick,->,rounded corners] (b.mid) -| node[above,pos=0.2] {$b$} (1,0) --(ac.west);
			\draw[thick,->] (ac.east) -- node[above,align=left] {$r(w_{1}x_{1}+w_{2}x_{2}+\cdots+w_{p}x_{p}+b)$} (dest.west);
			
			\end{pgfonlayer}
		\end{tikzpicture}
		\caption{Representación gráfica de un perceptrón.}
		\label{fig:perceptron}
	\end{figure}
	
	
	
	Esta representación gráfica y las limitaciones del perceptrón (para 
	más detalles véase \cite{Perceptron-Convergencia}), motivan el 
	siguiente paso en la 
	construcción de las \emph{redes neuronales}. Este paso consiste en 
	añadir más perceptrones en paralelo (nodos) y en serie (capas), 
	consiguiendo lo que se conoce como \emph{perceptrón multicapa}.

	El perceptrón multicapa es un caso particular de arquitectura de una red 
	neuronal. En esta arquitectura las conexiones entre nodos (neuronas) 
	se realizan únicamente hacia adelante, no se permite la conexión 
	entre nodos de la misma capa, y la función de activación es 
	$r(x)$ definida en \ref{def:r}. Para los propósitos del presente 
	trabajo, la función de activación no es relevante; pero sí lo es la 
	dirección de las conexiones entre los nodos. Por lo tanto, nos 
	ceñiremos al caso particular de las redes neuronales cuyas conexiones
	se realizan hacia adelante y sin conexiones dentro de la misma capa, 
	estas redes neuronales se conocen como \emph{redes neuronales 
	prealimentadas}. 
	
	\begin{remark}
	Aunque puede parecer que nos 
	estamos restringiendo a un caso muy particular de las redes 
	neuronales, el \emph{teorema de aproximación universal} nos garantiza
	que una red de estas características puede aproximar cualquier función
	Borel-medible cuyo dominio tenga dimensión finita (para más 
	información véase \cite{TeoremaAproxUn-Kurt}).
	\end{remark}

	En el presente trabajo nos interesa especialmente la estructura 
	combinatoria asociada a las redes neuronales prealimentadas, es decir, 
	los grafos subyacentes. En este sentido, veamos las siguientes 
	definiciones:

	\begin{defi}
		Un \textit{grafo} $G=(V,E)$ consiste en un conjunto finito V, 
		cuyos 
		elementos reciben el nombre de \textit{vértices}, y un 
		conjunto 
		E de pares de elementos de V, cuyos elementos se conocen como 
		\textit{aristas}. Si $\{u,v\}$ es una arista de $G$, se dice 
		que los vértices $u$ y $v$ son \textit{adyacentes} y 
		llamaremos a $u$ y a $v$ \textit{extremos} de la arista.
	\end{defi}

	Para el caso las redes neuronales prealimentadas necesitamos un caso 
	particular de grafo:
	
	\begin{defi}
		Un \textit{grafo dirigido} es un grafo en el que se asigna un 
		orden a los extremos de cada arista. Las aristas dirigidas se
		denotan $(u,v)$, y nótese que $(u,v)\neq(v,u)$. En el 
		caso de una arista dirigida $(u,v)$, se dice que $u$ es el 
		\textit{origen} y al vértice $v$ se le llama \textit{destino} 
		de la arista. Un grafo dirigido se dice \textit{acíclico} si 
		para cada vértice $u$, no existe una sucesión de aristas cuyo
		origen y destino sea $u$.
	\end{defi}

	Así pues, vamos a estudiar los grafos dirigidos acíclicos 
	subyacentes a las redes neuronales prealimentadas. No obstante, dadas 
	las características de las redes neuronales prealimentadas, también 
	podemos aproximarnos a sus grafos subyacentes desde otro punto de 
	vista. 

	Como ya hemos visto, las neuronas de una red neuronal prealimentada 
	están
	distribuidas por capas. Además, podemos identificar claramente estas 
	capas, y ya hemos visto que, en el caso que nos ocupa, el flujo de 
	información de la red se produce en un único sentido. Estas 
	propiedades nos inducen a pensar, de una manera natural, en un orden 
	sobre las neuronas de la red. 
	En este sentido, veamos la otra aproximación a los grafos subyacentes
	de una red neuronal prealimentada:

	\begin{defi}
		\label{def:gNivelado}
		Sea $G=(V, E)$ un grafo. Diremos que $G$ es 
		\textit{grafo nivelado} si tomando una partición adecuada de 
		$V$ en 
		subconjuntos $L_{1},\dots,L_{h}$, se cumple que si $(u,v) \in 
		E$ con $u \in L_{i}$ y $v \in L_{j}$, entonces $j=i+1$. A los 
		conjuntos $L_{1},\dots,L_{h}$ se los conoce como 
		\textit{niveles} y a $h$ como \textit{altura}.
	\end{defi}

	\begin{remark}
		La condición $j=i+1$ de la definición anterior puede relajarse
		a $j>i$. Esta relajación implicaría que en nuestra red podrían
		haber saltos de nivel. Hemos optado por la primera condición 
		pues es suficiente para los propósitos del presente trabajo.
	\end{remark}

	Observamos que, la definición de 
	grafo nivelado generalizado (admitiendo saltos de más de un nivel) y 
	la definición de grafo dirigido acíclico son 
	equivalentes. En efecto, un vértice $u$ de un grafo dirigido acíclico 
	está en el nivel $L_{i}$ cuando la sucesión de aristas más larga cuyo 
	destino es $u$ tiene $i$ vértices. Con lo que, siempre que tengamos 
	una red neuronal 
	prealimentada podremos identificar de manera natural su grafo nivelado
	subyacente, y por tanto, su grafo dirigido acíclico asociado. 

	Notemos que de la definición \ref{def:gNivelado} se deduce 
	naturalmente un orden parcial sobre los vértices del grafo: si $j>i$, 
	$u \in L_{i}$ y $v \in L_{j}$, entonces $v>u$. Así pues, este será el 
	orden que consideraremos para nuestras próximas definiciones y 
	resultados.

	Recapitulando, dada una red neuronal, le asociaremos un grafo dirigido
	acíclico (nivelado) sobre el que aplicaremos la teoría de homología 
	vista anteriormente. A fin de facilitar la comprensión, veamos un 
	ejemplo:

	\begin{ejem}
	 Supondremos una red neuronal prealimentada de 2 capas y 13 neuronas.
	 Su grafo asociado puede verse en la \autoref{fig:redEjem}.
	 
	 Hemos etiquetado las neuronas (vértices) en orden ascendente desde la 
	 primera neurona de la capa de salida hasta la última de la capa de 
	 entrada. En las próximas representaciones seguiremos usando esta 
	 notación. Nótese que los cálculos y los resultados a los que 
	 llegaremos en este trabajo son independientes de esta notación, 
	 seguiremos este convenio pues es el elegido en 
	 \cite{Articulo-Watanabe}.

		\begin{figure}[!htbp]
			\centering
			\begin{tikzpicture}[
				roundnode/.style={circle, draw=black, thick, fill=white, minimum size=7mm},
				]
				%Nodes
				\node[roundnode]      (n0) 	at 	(6,1)           {0};
				\node[roundnode]      (n1) 	at 	(6,-1)          {1};
				\node[roundnode]      (n2) 	at 	(4,3)           {2};
				\node[roundnode]      (n3) 	at 	(4,1)           {3};
				\node[roundnode]      (n4)      at 	(4,-1)		{4};
				\node[roundnode]      (n5)      at 	(4,-3)		{5};
				\node[roundnode]      (n6)      at 	(2,3)		{6};
				\node[roundnode]      (n7) 	at 	(2,1)           {7};
				\node[roundnode]      (n8) 	at 	(2,-1)           {8};
				\node[roundnode]      (n9)      at 	(2,-3)		{9};
				\node[roundnode]      (n10)      at 	(0,2)		{10};
				\node[roundnode]      (n11) 	at 	(0,0)           {11};
				\node[roundnode]      (n12) 	at 	(0,-2)           {12};
				
				\begin{pgfonlayer}{background}
				%Lines
				\draw[thick,->] (n10.mid) -- (n6.west);
				\draw[thick,->] (n10.mid) -- (n7.west);
				\draw[thick,->] (n10.mid) -- (n8.west);
				\draw[thick,->] (n10.mid) -- (n9.west);
				\draw[thick,->] (n11.mid) -- (n6.west);
				\draw[thick,->] (n11.mid) -- (n7.west);
				\draw[thick,->] (n11.mid) -- (n8.west);
				\draw[thick,->] (n11.mid) -- (n9.west);
				\draw[thick,->] (n12.mid) -- (n6.west);
				\draw[thick,->] (n12.mid) -- (n7.west);
				\draw[thick,->] (n12.mid) -- (n8.west);
				\draw[thick,->] (n12.mid) -- (n9.west);
				
				\draw[thick,->] (n6.mid) -- (n2.west);
				\draw[thick,->] (n6.mid) -- (n3.west);
				\draw[thick,->] (n6.mid) -- (n4.west);
				\draw[thick,->] (n6.mid) -- (n5.west);
				\draw[thick,->] (n7.mid) -- (n2.west);
				\draw[thick,->] (n7.mid) -- (n3.west);
				\draw[thick,->] (n7.mid) -- (n4.west);
				\draw[thick,->] (n7.mid) -- (n5.west);
				\draw[thick,->] (n8.mid) -- (n2.west);
				\draw[thick,->] (n8.mid) -- (n3.west);
				\draw[thick,->] (n8.mid) -- (n4.west);
				\draw[thick,->] (n8.mid) -- (n5.west);
				\draw[thick,->] (n9.mid) -- (n2.west);
				\draw[thick,->] (n9.mid) -- (n3.west);
				\draw[thick,->] (n9.mid) -- (n4.west);
				\draw[thick,->] (n9.mid) -- (n5.west);

				\draw[thick,->] (n2.mid) -- (n0.west);
				\draw[thick,->] (n2.mid) -- (n1.west);
				\draw[thick,->] (n3.mid) -- (n0.west);
				\draw[thick,->] (n3.mid) -- (n1.west);
				\draw[thick,->] (n4.mid) -- (n0.west);
				\draw[thick,->] (n4.mid) -- (n1.west);
				\draw[thick,->] (n5.mid) -- (n0.west);
				\draw[thick,->] (n5.mid) -- (n1.west);
				\end{pgfonlayer}
			\end{tikzpicture}
			\caption{Grafo dirigido acíclico (nivelado) asociado a 
			una red neuronal prealimentada de 13 neuronas y 2 capas.}
			\label{fig:redEjem}
		\end{figure}
	\end{ejem}

	Habiendo fijado los conceptos de red neuronal prealimentada y de su 
	grafo asociado, no parece evidente la aplicación de la homología 
	persistente a estas nociones. Lo que haremos será, dada una red 
	neuronal prealimentada, construir el complejo simplicial asociado a su
	grafo dirigido acíclico (DAG por sus siglas en inglés); a partir del 
	complejo simplicial construiremos un complejo simplicial filtrado; y 
	sobre éste consideraremos la homología persistente. En esta sección
	ilustraremos únicamente el primer paso.

	Si bien existen varias maneras de construir un complejo simplicial a 
	partir de un DAG (véase \cite{Clique-Jakob}), en el presente trabajo 
	desarrollaremos la más simple
	de todas ellas: la construcción del complejo simplicial \emph{clique}. 
	Esta construcción, aunque simple, nos servirá de inspiración para la 
	construcción que realizaremos más adelante. 
	
	\begin{defi}
		Definimos el \textit{complejo simplicial clique} asociado a un
		grafo a partir de la construcción de sus $p$-símplices, $p \in 
		\mathbb{N}\cup\{0\}$. Los $p$-símplices de un complejo 
		simplicial clique están construidos de la siguiente manera: el 
		conjunto de vértices $\{u_{0},\dots,u_{p}\}$ es un 
		$p$-símplice si y sólo si cada par de vértices está conectado 
		por una arista. Por simplicidad del lenguaje, nos referiremos
		a un complejo simplicial clique como clique.
	\end{defi}

	\begin{remark}
		Aunque la definición de clique es sencilla, el cálculo de un 
		clique a partir de un grafo es un problema complejo. De hecho,
		este problema se conoce como \emph{el problema del clique} y 
		es NP-completo. Para más detalles véase \cite{NP-Karp}.
	\end{remark}

	Nótese que, por ser un complejo simplicial, un clique es cerrado por 
	subconjuntos, esto es, cualquier subcomplejo simplicial de un clique 
	es un clique. Veamos un ejemplo:

	\begin{ejem}
		Vamos a construir el clique asociado al ejemplo anterior. Lo
		denotaremos por {\Large $\nu$}. Atendiendo a la definición de 
		clique y a la \autoref{fig:redEjem} tendremos que:
		\begin{multline*} 
			\text{{\Large $\nu$}}=\{\{0\},\{1\},\{2\},\{3\},
			\{4\},\{5\},\{6\},\{7\},\{8\},\{9\},
			\{10\},\{11\},\{12\},\{0,2\},\{0,3\},\\
			\{0,4\},\{0,5\},
			\{1,2\},\{1,3\},\{1,4\},\{1,5\},\{2,6\},\{2,7\},
			\{2,8\},\{2,9\},\{3,6\},\{3,7\},\{3,8\},\\\{3,9\},
			\{4,6\},\{4,7\},\{4,8\},\{4,9\},\{5,6\},\{5,7\},
			\{5,8\},\{5,9\},\{6,10\},\{6,11\},
			\{6,12\},\\\{7,10\},\{7,11\},
			\{7,12\},\{8,10\},\{8,11\},
			\{8,12\},\{9,10\},\{9,11\},
			\{9,12\}\}.
		\end{multline*}

		Observamos que {\Large $\nu$} es un clique muy particular, 
		pues está compuesto únicamente por vértices ($0$-símplices) y 
		aristas (1-símplices). Su representación gráfica puede verse 
		en la \autoref{fig:cliqueEjem}.
		
		\begin{figure}[!htbp]
			\centering
			\begin{tikzpicture}[
				roundnode/.style={circle, draw=black, thick, fill=white, minimum size=7mm},
				]
				%Nodes
				\node[roundnode]      (n0) 	at 	(6,1)           {0};
				\node[roundnode]      (n1) 	at 	(6,-1)          {1};
				\node[roundnode]      (n2) 	at 	(4,3)           {2};
				\node[roundnode]      (n3) 	at 	(4,1)           {3};
				\node[roundnode]      (n4)      at 	(4,-1)		{4};
				\node[roundnode]      (n5)      at 	(4,-3)		{5};
				\node[roundnode]      (n6)      at 	(2,3)		{6};
				\node[roundnode]      (n7) 	at 	(2,1)           {7};
				\node[roundnode]      (n8) 	at 	(2,-1)           {8};
				\node[roundnode]      (n9)      at 	(2,-3)		{9};
				\node[roundnode]      (n10)      at 	(0,2)		{10};
				\node[roundnode]      (n11) 	at 	(0,0)           {11};
				\node[roundnode]      (n12) 	at 	(0,-2)           {12};
				
				\begin{pgfonlayer}{background}
				%Lines
				\draw[thick] (n10.mid) -- (n6.mid);
				\draw[thick] (n10.mid) -- (n7.mid);
				\draw[thick] (n10.mid) -- (n8.mid);
				\draw[thick] (n10.mid) -- (n9.mid);
				\draw[thick] (n11.mid) -- (n6.mid);
				\draw[thick] (n11.mid) -- (n7.mid);
				\draw[thick] (n11.mid) -- (n8.mid);
				\draw[thick] (n11.mid) -- (n9.mid);
				\draw[thick] (n12.mid) -- (n6.mid);
				\draw[thick] (n12.mid) -- (n7.mid);
				\draw[thick] (n12.mid) -- (n8.mid);
				\draw[thick] (n12.mid) -- (n9.mid);
				
				\draw[thick] (n6.mid) -- (n2.mid);
				\draw[thick] (n6.mid) -- (n3.mid);
				\draw[thick] (n6.mid) -- (n4.mid);
				\draw[thick] (n6.mid) -- (n5.mid);
				\draw[thick] (n7.mid) -- (n2.mid);
				\draw[thick] (n7.mid) -- (n3.mid);
				\draw[thick] (n7.mid) -- (n4.mid);
				\draw[thick] (n7.mid) -- (n5.mid);
				\draw[thick] (n8.mid) -- (n2.mid);
				\draw[thick] (n8.mid) -- (n3.mid);
				\draw[thick] (n8.mid) -- (n4.mid);
				\draw[thick] (n8.mid) -- (n5.mid);
				\draw[thick] (n9.mid) -- (n2.mid);
				\draw[thick] (n9.mid) -- (n3.mid);
				\draw[thick] (n9.mid) -- (n4.mid);
				\draw[thick] (n9.mid) -- (n5.mid);

				\draw[thick] (n2.mid) -- (n0.mid);
				\draw[thick] (n2.mid) -- (n1.mid);
				\draw[thick] (n3.mid) -- (n0.mid);
				\draw[thick] (n3.mid) -- (n1.mid);
				\draw[thick] (n4.mid) -- (n0.mid);
				\draw[thick] (n4.mid) -- (n1.mid);
				\draw[thick] (n5.mid) -- (n0.mid);
				\draw[thick] (n5.mid) -- (n1.mid);
				\end{pgfonlayer}
		\end{tikzpicture}
			\caption{Representación gráfica del clique asociado a una red neuronal prealimentada de 13 neuronas y 2 capas.}
			\label{fig:cliqueEjem}
		\end{figure}
	\end{ejem}

	El ejemplo anterior pone de manifiesto que considerar el clique 
	asociado a un grafo por niveles de una red neuronal prealimentada no 
	es suficiente, pues el resultado es un clique trivial que nos aporta 
	muy poca información sobre la red. Reflexionando sobre ello, se hace 
	notar que debemos hacer una extensión mediante una cierta relación de 
	transitividad. 

	No obstante, estamos adelantando acontecimientos pues en el siguiente 
	capítulo veremos detalladamente el proceso necesario para la 
	aplicación de la homología persistente a una red neuronal 
	prealimentada. Así pues, finalizamos este capítulo de preliminares en
	el que se han expuesto todas las nociones teóricas necesarias para la 
	comprensión del artículo sobre el que versa el presente trabajo.

	\newpage

	\chapter{¿Es posible aplicar la homología persistente a las redes 
	neuronales?}
	
	La respuesta a la pregunta del título es afirmativa, sin embargo, a lo 
	largo de este capítulo iremos viendo que esta respuesta tiene sus 
	matices. En este capítulo nos vamos a centrar en el desarrollo del 
	artículo \cite{Articulo-Watanabe} sobre el que está basado el presente 
	trabajo. 
	La motivación principal de las técnicas que expondremos es esclarecer
	el funcionamiento interno de las redes neuronales prealimentadas, que 
	habitualmente son vistas como <<cajas negras>>. Mediante el uso de 
	estas técnicas podremos extraer información relevante acerca de la 
	complejidad y el grado de aprendizaje de las redes neuronales 
	prealimentadas. 

	\section{Descripción matemática}
	
	Partimos de la base del capítulo anterior, en el que se hizo notoria 
	la necesidad de extender la noción de clique asociado a una red 
	neuronal mediate una cierta idea de transitividad. Adicionalmente, 
	debemos construir el complejo simplicial filtrado para poder aplicar 
	la homología persistente.
	
	Como bien sabemos, dada una red neuronal prealimentada, es posible 
	asociarle un DAG. Ademas, aunque en el capítulo anterior no lo 
	mencionamos, a este grafo podemos asociarle unos pesos positivos o 
	negativos que serán los 
	parámetros de la red neuronal prealimentada. Pensando de esta manera, 
	tendremos un DAG con pesos $w_{ij}$, con $w_{ij}$ el parámetro de la 
	red entre $u_{i}$ y $u_{j}$. Notemos que $w_{ij}=0$ si y sólo si 
	$u_{i}$ y $u_{j}$ no están conectadas.
	
	Observamos que, en cierto modo, un peso positivo indica una 
	predisposición para pasar de $u_{i}$ a $u_{j}$, mientras que un peso 
	negativo indica una obstrucción. Puesto que la complejidad de la red 
	la medimos en términos de esa predisposición de pasar de una neurona a 
	otra, en la siguiente definición obviaremos los pesos negativos. Para 
	ello, tomamos la función parte positiva del peso asociado: 
	$w_{ij}^{+}:=\max\{w_{ij},0\}$. Asimismo, adoptamos el axioma de que 
	la predisposición de pasar de una neurona $u_{i}$ a otra $u_{j}$ hay 
	que contextualizarla en el sentido de compararla con la predisposición
	de pasar de $u_{i}$ a otra neurona de destino $u_{k}$. En cierto 
	sentido <<probabilizamos>> nuestros pesos. Con esto en mente, damos la 
	siguiente definición: 
	
	\begin{defi}
	Definimos la \textit{importancia} de $u_{i}$ (salida) para $u_{j}$ 
	(llegada) como:
	\begin{equation*}
		R_{ij}=  \left \{ 
			\begin{array}{ll}
				1 & \text{si } i=j \\
				w_{ij}^{+}/\sum_{k,k \neq j}w_{kj}^{+} & \text{si } i \neq j
			\end{array}
		\right. 
	\end{equation*}
	\end{defi}
	Observamos que la importancia de una neurona para sí misma es de 1, y 
	la importancia entre neuronas distintas es la proporción del peso 
	entre ellas con respecto al resto de pesos de la neurona de llegada. 
	Además, es claro que la importancia entre dos neuronas conectadas es 
	un valor entre $0$ y $1$.

	La noción de importancia entre neuronas nos habilita la construcción 
	de un complejo simplicial filtrado a partir del DAG asociado a la red. 
	Sin embargo, teniendo en mente la idea de la transitividad, todavía 
	tenemos que extender la definición de la importancia entre neuronas 
	para aquellas que no estén directamente conectadas. Así pues, damos la 
	siguiente definición: 
	\begin{defi}
		\label{def:impExt}	
	Consideremos las neuronas $u_{0}$ y $u_{2}$ conectadas por el camino: 
	$u_{2} 
	\rightarrow u_{1} \rightarrow u_{0}$, la importancia de $u_{2}$ para 
	$u_{0}$ es, según el camino entre ellas, $R_{21} \cdot R_{10}$.
	Por lo tanto, definimos la \textit{importancia extendida} entre 
	neuronas, y la denotamos por $\overline{R_{ij}}$, como:
	\begin{equation*}
		\overline{R_{ij}}=\max\{R_{u_{i}u_{m_{1}}} \cdots
		R_{u_{m_{n}}u_{j}} \mid (u_{i},u_{m_{1}},...,u_{m_{n}},u_{j}) 
	\in C_{ij} \}
	\end{equation*}	
	donde $C_{ij}$ denota el conjunto de todos los posibles caminos de 
	$u_{i}$ a $u_{j}$. 
	\end{defi}

	\begin{remark}
	Nótese que es posible definir $\overline{R_{ij}}$ considerando 
	funciones distintas al máximo, tales como la suma o la suma 
	cuadrática. El autor (\cite{Articulo-Watanabe}) justifica la 
	elección del máximo por eficiencia computacional. Además, notemos que 
	el máximo tiene una ventaja evidente: mantiene los números de 
	importancia entre 0 y 1.
	\end{remark}

	Como ya mencionamos en el capítulo anterior, numeraremos los nodos 
	(vértices) de un DAG con pesos, asociado a una red neuronal 
	prealimentada, en orden ascendente, desde las neuronas de llegada 
	hasta las de salida. Veamos, a continuación, un ejemplo sencillo para 
	interiorizar estas definiciones que serán clave a lo largo de este 
	capítulo.
	\begin{ejem}
		\label{ej:primerRel}	
		Supondremos la representación de una red neuronal, que puede 
		verse en la \autoref{fig:ejCalcImp}, con sus correspondientes 
		pesos.

		\begin{figure}[!htbp]
			\centering
			\begin{tikzpicture}[
				roundnode/.style={circle, draw=black, thick, fill=white, minimum size=7mm},
				]
				%Nodes
				\node[roundnode]      (n0) 	at 	(2,0)           {0};
				\node[roundnode]      (n1)      at 	(0,1)		{1};
				\node[roundnode]      (n2)      at 	(0,-1)		{2};
				\node[roundnode]      (n3)      at 	(-2,0)		{3};
				
				\begin{pgfonlayer}{background}
				%Lines
				\draw[thick,->] (n3.mid) -- node[above,sloped] {6.9} (n1.west);
				\draw[thick,->] (n3.mid) -- node[below,sloped] {5.8} (n2.west);
				\draw[thick,->] (n1.mid) -- node[above,sloped] {2.6} (n0.west);
				\draw[thick,->] (n2.mid) -- node[below,sloped] {3.9} (n0.west);
				
				\end{pgfonlayer}
			\end{tikzpicture}
			\caption{Representación de una red neuronal de 4 neuronas y 1 capa.}
			\label{fig:ejCalcImp}
		\end{figure}
		
		Tal y como vemos, las neuronas ya han sido ordenadas de manera 
		correcta. Además, en este caso, todos los pesos son positivos, 
		por lo que no nos tenemos que preocupar escoger la parte 
		positiva. Vamos a calcular algunas importancias entre 
		neuronas:			
		\begin{flalign*}
			& \left.
				\begin{array}{l}
					R_{31}=\frac{6.9}{6.9}=1 \hspace{0.5cm} R_{32}=\frac{5.8}{5.8}=1 \\[3pt]
					R_{10}=\frac{2.6}{6.5}=0.4 \hspace{0.55cm}  R_{20}=\frac{3.9}{6.5}=0.6 \\[3pt]
					\overline{R_{30}}=\max\{R_{31} \cdot R_{10}, R_{32}\cdot R_{20}\}=R_{32} \cdot R_{20}=0.6  
				\end{array}
			\right. & 
		\end{flalign*}
	\end{ejem}
	El ejemplo anterior pone de manifiesto la intuición detrás de la 
	definición de la importancia entre neuronas: lo que hace es medir 
	la aportación de la neurona emisora, $u_{i}$, con respecto al resto de 
	neuronas emisoras de $u_{j}$.
	
	La definición de importancia extendida nos habilita la construcción 
	del complejo simplicial filtrado asociado a un DAG con pesos de una 
	manera muy natural. El primer paso será definir el complejo 
	simplicial asociado, pues no vamos a usar un clique. Esencialmente, 
	vamos a definir umbrales de importancia para determinar los complejos
	simpliciales asociados a una red neuronal.

	\begin{defi}
		\label{def:SimpAut}
		Sea $0\leq t \leq 1$ un parámetro real. Definimos los 
		\textit{$p$-símplices asociados a un DAG con pesos} a partir 
		del conjunto de vértices del DAG (que por simplificar notación 
		denotaremos por {\Large $\nu$}$_{0}$) 
		como sigue:
	\begin{equation*}
		\text{{\Large $\nu$}}_{p}^{t}=
		\left \{
			\begin{array}{ll}
				\text{{\Large $\nu$}}_{0} & \text{si } p=0\\[3pt]
				\{\{u_{a_{0}},\dots,u_{a_{p}}\} \mid u_{a_{i}} \in 
					\text{{\Large $\nu$}}_{0},\ 
					\overline{R_{a_{i}a_{j}}} \geq t,\ 
				\forall a_{i} > a_{j} \} & \text{si } p \geq 1
			\end{array}
		\right.  
	\end{equation*}
	donde cada $u_{a_{i}}$ pertenece al nivel $a_{i}$ del DAG y $a_{0} 
	< \cdots < a_{i} < \cdots < a_{p}$.
	\end{defi}
	Ahora que ya tenemos definidos los p-símplices, vamos con la 
	construcción del complejo simplicial asociado. Para ello damos el 
	siguiente resultado:
	\begin{prop}
		Sea {\Large $\nu$}$_{0}=\{u_{0},\dots,u_{n}\}$ un conjunto 
		finito, y $\{w_{ij}\}_{0\leq j \leq i \leq n}$ un 
		conjunto de números reales. Sea $\overline{R_{ij}}$ la 
		importancia entre neuronas definida en \ref{def:impExt}, y 
		{\Large $\nu$}$_{p}^{t}$ los p-símplices definidos en 
		\ref{def:SimpAut} con t parámetro real entre 0 y 1. Entonces 
		{\Large $\nu$}$^{t}=\bigcup_{s=0}^{s=n}$
		{\Large $\nu$}$_{s}^{t}$ 
		es un complejo simplicial abstracto. 
		\label{prop:cs}
	\end{prop}

	\begin{proof}
		Supongamos hipótesis generales. \\
		Para probar que {\Large $\nu$}$^{t}$ es un complejo simplicial 
		abstracto debemos ver: 
		\begin{enumerate}
			\item $u \in \text{{\Large $\nu$}}_{0} \text{ implica 
				que } \{u\} \in \text{{\Large $\nu$}}^{t}$
			\item $ \sigma \in \text{{\Large $\nu$}}^{t} 
				\text{ y } \tau \subset \sigma \text{ implica 
				que } \tau \in \text{{\Large $\nu$}}^{t}$	
		\end{enumerate}
		Notemos que la primera propiedad se deduce inmediatamente de 
		la definición \ref{def:SimpAut} y de la definición de 
		{\Large $\nu$}$^{t}$.

		Así pues, vamos a probar la segunda propiedad:
		$\sigma=\{u_{a_{0}},\dots,u_{a_{p}}\} \in 
		\text{{\Large $\nu$}}^{t}$ 
		 implica que $\overline{R_{a_{i}a_{j}}}\geq t, \  
		\forall a_{i} \geq a_{j}$. 
		Ahora sea $\tau \subset \sigma$, entonces, 
		$\tau = \{u_{b_{0}},\dots,u_{b_{q}}\}$, y como $\{b_{0},\dots,
		b_{q}\} 
		\subset \{a_{0},\dots,a_{p}\}$, se tendrá que 
		$\overline{R_{b_{i}b_{j}}} \geq t,\ \forall n_{i} \geq n_{j}$. 
		El resultado se sigue inmediatamente. 
	\end{proof}

	Ahora que ya tenemos construido nuestro complejo simplicial 
	{\Large $\nu$} procedemos con la construcción del complejo simplicial 
	filtrado. Para ello, una vez más, damos el siguiente resultado:

	\begin{prop}
		Sea $(t_{i})_{i=1}^{n}$ una sucesión, monótona decreciente, 
		de números reales entre 1 y 0, e indexada sobre los naturales. 
		Entonces {\Large $\nu$}$_{0}=\emptyset$ y {\Large $\nu$}$_{i}=
		\text{{\Large $\nu$}}^{t_{i}}$ con $1\leq i \leq n$, es un 
		complejo simplicial filtrado.   
	\end{prop}

	\begin{proof}
		Supongamos hipótesis generales.\\
		Por la proposición anterior, sabemos que 
		{\Large $\nu$}$^{t_{n}}$ es un complejo simplicial. Ahora 
		bien, $t_{i} > t_{j} \text{ implica que } 
		\text{{\Large $\nu$}}_{p}^{t_{i}} \subset 
		\text{{\Large $\nu$}}_{p}^{t_{j}}$ por la definición 
		\ref{def:SimpAut}. Entonces 
		$\emptyset=\text{{\Large $\nu$}}_{0} 
		\subset \text{{\Large $\nu$}}_{1} \subset \cdots \subset 
		\text{{\Large $\nu$}}_{n}=\text{{\Large $\nu$}}^{t_{n}}$. Se 
		sigue inmediatamente el resultado.
	\end{proof}

	Hasta ahora, hemos conseguido construir el complejo simplicial 
	filtrado asociado a un DAG con pesos, con lo que podemos pensar que ya 
	estamos listos para aplicar la homología persistente y extraer 
	conclusiones. No obstante, nuestro análisis del artículo de 
	investigación \cite{Articulo-Watanabe} nos lleva a considerar una 
	forma diferente de definir complejos simpliciales a partir de DAG con 
	pesos.
	
	Si leemos detenidamente las definiciones de la presente sección, 
	observamos que la definición \ref{def:impExt} tiene un pequeño 
	defecto: no 
	se especifica el significado exacto de $C_{ij}$. Este pequeño defecto, 
	que en un principio puede pasar desapercibido, supone que haya al 
	menos dos maneras distintas de calcular los $p$-símplices dado un DAG 
	con pesos.

	Esta disyuntiva da lugar a las dos interpretaciones posibles sobre el
	cálculo de los $p$-símplices, las hemos bautizado como: la 
	interpretación \emph{local} y la interpretación \emph{global}. De los 
	mensajes que hemos podido intercambiar con los autores de 
	\cite{Articulo-Watanabe}, hemos constatado que los cálculos que 
	realizan se corresponden con la interpretación que hemos bautizado 
	como \emph{local}.

	Veamos un pequeño ejemplo que ilustre la diferencia entre ambas 
	interpretaciones.
	\begin{ejem}
		\label{ej:int}
		Supondremos la representación de una red neuronal con 
		las importancias entre neuronas ya calculadas que puede verse 
		en la \autoref{fig:ejemInter}.

		\begin{figure}[!htbp]
			\centering
			\begin{tikzpicture}[
				roundnode/.style={circle, draw=black, thick, fill=white, minimum size=7mm},
				]
				%Nodes
				\node[roundnode]      (n0) 	at 	(2,0)           {0};
				\node[roundnode]      (n1)      at 	(0,1)		{1};
				\node[roundnode]      (n2)      at 	(0,-1)		{2};
				\node[roundnode]      (n4)      at 	(-2,-1)		{4};
				\node[roundnode]      (n3)      at 	(-2,1)		{3};
				
				\begin{pgfonlayer}{background}
				%Lines
				\draw[thick,->] (n3.mid) -- node[above] {0.5} (n1.west);
				\draw[thick,->] (n4.mid) -- node[above,sloped] {0.5} (n1.west);
				\draw[thick,->] (n4.mid) -- node[below] {1} (n2.west);
				\draw[thick,->] (n1.mid) -- node[above,sloped] {0.4} (n0.west);
				\draw[thick,->] (n2.mid) -- node[below,sloped] {0.6} (n0.west);
				
				\end{pgfonlayer}
			\end{tikzpicture}
			\caption{Representación de una red neuronal de 5 neuronas y 1 capa.}
			\label{fig:ejemInter}
		\end{figure}
		Vamos calcular {\Large $\nu$}$^{0.4}_{2}$ para ver las 
		diferencias entre las dos interpretaciones. En primer lugar, 
		listamos los posibles 2-símplices: 
		$$
		\{4,2,0\},\{4,1,0\},\{3,1,0\}
		$$
		\begin{itemize}
			\item{Interpretación local}\\
				Como la importancia entre 4-2, 2-0 y 4-0 es 
				mayor que 0.4, entonces $\{4,2,0\} \in 
				\text{{\Large $\nu$}}^{0.4}_{2}$,
				donde importancia entre 4-0 viene dada por:
				$$\overline{R_{40}}=\max\{R_{42} \cdot 
				R_{20}\}=R_{42} \cdot R_{20}=0.6$$
				Siguiendo la misma regla tenemos que 
				$\{4,1,0\},\{3,1,0\} \notin 
				\text{{\Large $\nu$}}^{0.4}_{2}$. Por lo 
				tanto,
				$\text{{\Large $\nu$}}^{0.4}_{2}=
				\{\{4,2,0\}\}$.
			\item{Interpretación global}\\
				Razonando igual que antes,$\ \{4,2,0\} \in 
				\text{{\Large $\nu$}}^{0.4}_{2}$. Ahora bien, 
				como la importancia entre 4-1 y 1-0 es mayor o 
				igual que 0.4, y 
				$$\overline{R_{40}}=\max\{R_{42} \cdot 
				R_{20},R_{41} \cdot R_{10}\}=R_{42} \cdot 
				R_{20}=0.6$$ 
				Entonces $\{4,1,0\} \in 
				\text{{\Large $\nu$}}^{0.4}_{2}$. Sin embargo, 
				$\{3,2,0\} \notin 
				\text{{\Large $\nu$}}^{0.4}_{2}$ y así, 
				{\Large $\nu$}$^{0.4}_{2}=\{\{4,2,0\},\\  
				\{4,1,0\}\}$
		\end{itemize}
	\end{ejem}
	El ejemplo anterior pone de manifiesto la principal diferencia 
	entre ambas interpretaciones: en la primera, el máximo se 
	calcula sobre los caminos que aparecen en el p-símplice; en la 
	segunda, el máximo se calcula sobre todos los caminos posibles entre el 
	vértice de origen y de destino. 
	\begin{remark}
		Mientras que la interpretación global es consistente 
		para los 1-símplices (aristas), la interpretación 
		local no lo es. Es decir, si seguimos el razonamiento local, 
		no podremos calcular una arista entre dos vértices no 
		conectados, pues no existirá camino local sobre el que tomar 
		el máximo. Este hecho imposibilitaría el cálculo del complejo 
		simplicial asociado a la red. Por lo tanto, para el cálculo en 
		la interpretación local se calcula el producto a través de cada 
		camino, y se adopta el criterio de que cuando ese producto alcanza 
		el umbral establecido el símplice formado por los nodos de ese 
		camino y todos sus subsímplices pertenecen a nuestro complejo 
		simplicial. Notamos que esta interpretación permite construir 
		los complejos simpliciales sin realizar el cálculo de 
		$\overline{R}$ que introducen los autores de 
		\cite{Articulo-Watanabe}.	
	\end{remark}
	
	A lo largo de las siguientes secciones veremos más en detalle la 
	diferencia entre ambas interpretaciones y cómo afecta esta diferencia
	al cálculo de la homología persistente.

	\section{Interpretación global. Ejemplo}
	
	A lo largo de esta sección desarrollaremos el cálculo de la homología
	persistente en el ejemplo propuesto en \cite{Articulo-Watanabe} 
	mediante la interpretación que hemos bautizado como global. Comenzamos
	con esta interpretación pues es la que hicimos en un primer momento y 
	es la que consideramos natural.

	En primer lugar, vamos a detenernos momentáneamente a analizar la 
	interpretación global. Si bien esta interpretación no es la original, 
	como ya mencionamos, esta interpretación es consistente para el 
	cálculo de los 1-símplices. Adicionalmente, esta interpretación es 
	coherente con la definición de clique y con la definición 
	\ref{def:impExt}, donde $C_{ij}$ se considera globalmente: se toma el 
	DAG
	inicial y se le calcula el cierre transitivo. Sobre este nuevo DAG se
	realizan el resto de cálculos.

	No obstante, aunque esta interpretación nos pueda parecer más 
	coherente desde el punto de vista teórico, habrá que realizar una 
	serie de experimentos y comparativas para validar la información 
	obtenida siguiendo esta interpretación. A continuación describimos, de
	manera general, los pasos que hemos seguido para el cálculo de los 
	símplices bajo esta interpretación: 

	\begin{enumerate}[label=\arabic*)]
		\item Se calculan recursivamente todos los posibles caminos 
			para un vértice de origen. 
		\item Se calcula la importancia de cada camino, es decir, la 
			importancia entre el vértice de origen y de destino de 
			la siguiente manera: se van multiplicando las 
			importancias entre cada par de vértices; y si dos 
			caminos comparten un vértice intermedio, entonces la 
			importancia hasta ese vértice será la máxima producida 
			por ambos caminos hasta ese vértice. La importancia 
			obtenida tras esta multiplicación será la importancia 
			del camino, y por ser un número entre 0 y 1, todas las
			importancias intermedias serán mayores o iguales que 
			la importancia del camino.
		\item Habiendo calculado las importancias de cada camino, se 
			descartan aquellos caminos que no superan el
			umbral fijado.
		\item Como cada camino constituye un símplice maximal, se 
			calculan todos los subsímplices
			asociados al camino.
	\end{enumerate}

	Aunque no lo hemos mencionado, un DAG con pesos no es más que un DAG
	al que le asociamos una matriz de pesos (recordemos que, en el caso 
	que nos ocupa, estos pesos vienen dados por la importancia extendida 
	entre neuronas). Sobre esta matriz de pesos y sobre una lista de 
	adyacencia se siguen una serie de pasos para el cálculo del complejo 
	simplicial filtrado. Veremos más en detalle estos pasos en la 
	siguiente sección. 

	Habiendo descrito de manera general la interpretación global, vamos a 
	aplicarla al ejemplo propuesto en \cite{Articulo-Watanabe}, en el que 
	vamos a calcular los números de Betti y vamos a ver los diagramas 
	correspondientes.

	\begin{ejem}
		Supondremos la representación de una red neuronal con las 
		importancias entre neuronas ya calculadas que puede verse en 
		la \autoref{fig:ejemGlobal}.

		\begin{figure}[!htbp]
				\centering
				\begin{tikzpicture}[
					roundnode/.style={circle, draw=black, thick, fill=white, minimum size=7mm},
					]
					%Nodes
					\node[roundnode]      (n0) 	at 	(4,0)           {0};
					\node[roundnode]      (n1) 	at 	(2,2)           {1};
					\node[roundnode]      (n2) 	at 	(2,0)           {2};
					\node[roundnode]      (n3)      at 	(2,-2)		{3};
					\node[roundnode]      (n4)      at 	(0,3)		{4};
					\node[roundnode]      (n5)      at 	(0,1)		{5};
					\node[roundnode]      (n6) 	at 	(0,-1)           {6};
					\node[roundnode]      (n7) 	at 	(0,-3)           {7};
					\node[roundnode]      (n8) 	at 	(-2,2)           {8};
					\node[roundnode]      (n9) 	at 	(-2,0)           {9};
					\node[roundnode]      (n10) 	at 	(-2,-2)           {10};
					
					\begin{pgfonlayer}{background}
					%Lines
					\draw[thick,->] (n9.mid) -- node[above,sloped] {0.8} (n5.west);
					\draw[thick,->] (n9.mid) -- node[above,sloped] {0.7} (n6.west);
					\draw[thick,->] (n10.mid) -- node[above,sloped] {0.3} (n6.west);
					\draw[thick,->] (n10.mid) -- node[above,sloped] {1.0} (n7.west);

					\draw[thick,->] (n1.mid) -- node[above,sloped] {0.3} (n0.west);
					
					\draw[thick,->] (n5.mid) -- node[above,sloped] {0.2} (n2.west);
					\draw[thick,->] (n5.mid) -- node[above,sloped] {0.8} (n1.west);
					\draw[thick,->] (n6.mid) -- node[above,sloped] {0.8} (n2.west);
					\draw[thick,->] (n6.mid) -- node[above,sloped] {0.7} (n3.west);
					\draw[thick,->] (n7.mid) -- node[above,sloped] {0.3} (n3.west);
					\draw[thick,->] (n8.mid) -- node[above,sloped] {1.0} (n4.west);
					\draw[thick,->] (n8.mid) -- node[above,sloped] {0.2} (n5.west);
					\draw[thick,->] (n2.mid) -- node[above,sloped] {0.1} (n0.west);
					\draw[thick,->] (n3.mid) -- node[above,sloped] {0.6} (n0.west);
					\draw[thick,->] (n4.mid) -- node[above,sloped] {0.2} (n1.west);
					\end{pgfonlayer}
				\end{tikzpicture}
				\caption{Representación de una red neuronal de 11 neuronas y 2 capas.}
				\label{fig:ejemGlobal}
		\end{figure}
		Dada la anterior representación, tenemos la matriz de pesos 
		asociada:
		\begin{equation*}
			M=\begin{pNiceMatrix}
				1 & 0 & \Cdots & \Cdots & \Cdots & \Cdots &\Cdots  & \Cdots & \Cdots & \Cdots & 0 \\
				0.3 & 1 & 0 & \Cdots & \Cdots & \Cdots & \Cdots & \Cdots & \Cdots & \Cdots & 0 \\
				0.1 & 0 & 1 & 0 & \Cdots & \Cdots & \Cdots & \Cdots & \Cdots & \Cdots & 0 \\
				0.6 & 0 & 0 & 1 & 0 & \Cdots & \Cdots & \Cdots & \Cdots & \Cdots & 0 \\
				0 & 0.2 & 0 & 0 & 1 & 0 & \Cdots & \Cdots & \Cdots & \Cdots & 0 \\
				0 & 0.8 & 0.2 & 0 & 0 & 1 & 0 & \Cdots & \Cdots & \Cdots & 0 \\
				0 & 0 & 0.8 & 0.7 & 0 & 0 & 1 & 0 & \Cdots & \Cdots & 0 \\
				0 & 0 & 0 & 0.3 & 0 & 0 & 0 & 1 & 0 & \Cdots & 0 \\
				0 & 0 & 0 & 0 & 1 & 0.2 & 0 & 0 & 1 & 0 & 0 \\
				0 & 0 & 0 & 0 & 0 & 0.8 & 0.7 & 0 & 0 & 1 & 0 \\
				0 & 0 & 0 & 0 & 0 & 0 & 0.3 & 1 & 0 & 0 & 1 
			\end{pNiceMatrix}
		\end{equation*}
		donde cada entrada de la matriz, $a_{ij}$, es la importancia 
		de la neurona $i$ para la neurona $j$ ($i \geq j$).

		Ahora vamos con el cálculo del complejo simplicial filtrado 
		asociado. Para ello vamos a ilustrar unos cuantos pasos en la 
		filtración con los correspondientes números de Betti 
		asociados. Tras esto, mostraremos los complejos
		simpliciales asociados a cada filtración. Podemos ver estos 
		pasos en la \autoref{fig:CompSimpGlobal}.

		\begin{figure}[!htbp]
			\fbox{\minipage{0.225\textwidth}
				\begin{figure}[H]
					\resizebox{1.0\textwidth}{!}{\begin{tikzpicture}[
						roundnode/.style={circle, draw=black, thick, fill=white, minimum size=7mm},
						]
						%Nodes
						\node           (label) 	at	(1,4) 		{Filtración 0 (t=1.0)};
						\node[roundnode]      (n0) 	at 	(4,0)           {0};
						\node[roundnode]      (n1) 	at 	(2,2)           {1};
						\node[roundnode]      (n2) 	at 	(2,0)           {2};
						\node[roundnode]      (n3)      at 	(2,-2)		{3};
						\node[roundnode]      (n4)      at 	(0,3)		{4};
						\node[roundnode]      (n5)      at 	(0,1)		{5};
						\node[roundnode]      (n6) 	at 	(0,-1)           {6};
						\node[roundnode]      (n7) 	at 	(0,-3)           {7};
						\node[roundnode]      (n8) 	at 	(-2,2)           {8};
						\node[roundnode]      (n9) 	at 	(-2,0)           {9};
						\node[roundnode]      (n10) 	at 	(-2,-2)           {10};
						\node 		(label2) 	at 	(1,-4) 		{$\beta_{0}=9,\beta_{1}=0$};
						
						\begin{pgfonlayer}{background}
						%Lines
						\draw[thick] (n10.mid) -- (n7.mid);
						\draw[thick] (n8.mid) -- (n4.mid);
						\end{pgfonlayer}
					\end{tikzpicture}}	
				\end{figure}	
				\begin{figure}[H]
					\resizebox{1.0\textwidth}{!}{\begin{tikzpicture}[
						roundnode/.style={circle, draw=black, thick, fill=white, minimum size=7mm},
						]
						%Nodes
						\node           (label) 	at	(1,4) 		{Filtración 7 (t=0.3)};
						\node[roundnode]      (n0) 	at 	(4,0)           {0};
						\node[roundnode]      (n1) 	at 	(2,2)           {1};
						\node[roundnode]      (n2) 	at 	(2,0)           {2};
						\node[roundnode]      (n3)      at 	(2,-2)		{3};
						\node[roundnode]      (n4)      at 	(0,3)		{4};
						\node[roundnode]      (n5)      at 	(0,1)		{5};
						\node[roundnode]      (n6) 	at 	(0,-1)           {6};
						\node[roundnode]      (n7) 	at 	(0,-3)           {7};
						\node[roundnode]      (n8) 	at 	(-2,2)           {8};
						\node[roundnode]      (n9) 	at 	(-2,0)           {9};
						\node[roundnode]      (n10) 	at 	(-2,-2)           {10};
						\node 		(label2) 	at 	(1,-4) 		{$\beta_{0}=2,\beta_{1}=1$};
						
						\begin{pgfonlayer}{background}							
						\fill[fill=black!20,opacity=1] (n10.mid) -- (n7.mid) -- (n3.mid) -- cycle;
						\fill[fill=black!20,opacity=1] (n10.mid) -- (n6.mid) -- (n3.mid) -- cycle;
						\fill[fill=black!20,opacity=1] (n9.mid) to[bend left] (n1.mid) -- (n5.mid) -- cycle;
						\fill[fill=black!20,opacity=1] (n9.mid) to[bend right] (n3.mid) -- (n6.mid) -- cycle;
						\fill[fill=black!20,opacity=1] (n9.mid) -- (n2.mid) -- (n6.mid) -- cycle;
						\fill[fill=black!20,opacity=1] (n6.mid) -- (n0.mid) -- (n3.mid) -- cycle;
						
						\draw[thick] (n10.mid) -- (n7.mid);
						\draw[thick] (n10.mid) -- (n6.mid);
						\draw[thick] (n10.mid) -- (n3.mid);
						\draw[thick] (n9.mid) -- (n6.mid);
						\draw[thick] (n9.mid) -- (n5.mid);
						\draw[thick] (n9.mid) -- (n3.mid);
						\draw[thick] (n9.mid) -- (n2.mid);
						\draw[thick] (n9.mid) -- (n1.mid);
						\draw[thick] (n8.mid) -- (n4.mid);
						\draw[thick] (n7.mid) -- (n3.mid);
						\draw[thick] (n6.mid) -- (n3.mid);
						\draw[thick] (n6.mid) -- (n2.mid);
						\draw[thick] (n6.mid) -- (n0.mid);
						\draw[thick] (n5.mid) -- (n1.mid);
						\draw[thick] (n3.mid) -- (n0.mid);
						\draw[thick] (n1.mid) -- (n0.mid);
					

						\end{pgfonlayer}
					\end{tikzpicture}}	
				\end{figure}
			\endminipage}
			\fbox{\minipage{0.225\textwidth}
				\begin{figure}[H]
					\resizebox{1.0\textwidth}{!}{\begin{tikzpicture}[
						roundnode/.style={circle, draw=black, thick, fill=white, minimum size=7mm},
						]
						%Nodes
						\node           (label) 	at	(1,4) 		{Filtración 3 (t=0.7)};
						\node[roundnode]      (n0) 	at 	(4,0)           {0};
						\node[roundnode]      (n1) 	at 	(2,2)           {1};
						\node[roundnode]      (n2) 	at 	(2,0)           {2};
						\node[roundnode]      (n3)      at 	(2,-2)		{3};
						\node[roundnode]      (n4)      at 	(0,3)		{4};
						\node[roundnode]      (n5)      at 	(0,1)		{5};
						\node[roundnode]      (n6) 	at 	(0,-1)           {6};
						\node[roundnode]      (n7) 	at 	(0,-3)           {7};
						\node[roundnode]      (n8) 	at 	(-2,2)           {8};
						\node[roundnode]      (n9) 	at 	(-2,0)           {9};
						\node[roundnode]      (n10) 	at 	(-2,-2)           {10};
						\node 		(label2) 	at 	(1,-4) 		{$\beta_{0}=4,\beta_{1}=0$};
						
						\begin{pgfonlayer}{background}
						%Lines
						\draw[thick] (n10.mid) -- (n7.mid);
						\draw[thick] (n8.mid) -- (n4.mid);
						\draw[thick] (n9.mid) -- (n6.mid);
						\draw[thick] (n9.mid) -- (n5.mid);
						\draw[thick] (n6.mid) -- (n3.mid);
						\draw[thick] (n6.mid) -- (n2.mid);
						\draw[thick] (n5.mid) -- (n1.mid);
						\end{pgfonlayer}
					\end{tikzpicture}}	
				\end{figure}	
				\begin{figure}[H]
					\resizebox{1.0\textwidth}{!}{\begin{tikzpicture}[
						roundnode/.style={circle, draw=black, thick, fill=white, minimum size=7mm},
						]
						%Nodes
						\node           (label) 	at	(1,4) 		{Filtración 8 (t=0.2)};
						\node[roundnode]      (n0) 	at 	(4,0)           {0};
						\node[roundnode]      (n1) 	at 	(2,2)           {1};
						\node[roundnode]      (n2) 	at 	(2,0)           {2};
						\node[roundnode]      (n3)      at 	(2,-2)		{3};
						\node[roundnode]      (n4)      at 	(0,3)		{4};
						\node[roundnode]      (n5)      at 	(0,1)		{5};
						\node[roundnode]      (n6) 	at 	(0,-1)           {6};
						\node[roundnode]      (n7) 	at 	(0,-3)           {7};
						\node[roundnode]      (n8) 	at 	(-2,2)           {8};
						\node[roundnode]      (n9) 	at 	(-2,0)           {9};
						\node[roundnode]      (n10) 	at 	(-2,-2)           {10};
						\node 		(label2) 	at 	(1,-4) 		{$\beta_{0}=1,\beta_{1}=0$};
						
						\begin{pgfonlayer}{background}							
						\fill[fill=black!20,opacity=1] (n10.mid) -- (n7.mid) -- (n3.mid) -- cycle;
						\fill[fill=black!20,opacity=1] (n10.mid) -- (n6.mid) -- (n3.mid) -- cycle;
						\fill[fill=black!20,opacity=1] (n10.mid) to[bend left] (n2.mid) -- (n6.mid) -- cycle;
						\fill[fill=black!20,opacity=1] (n9.mid) to[bend left] (n1.mid) -- (n5.mid) -- cycle;
						\fill[fill=black!20,opacity=1] (n9.mid) to[bend right] (n3.mid) -- (n6.mid) -- cycle;
						\fill[fill=black!20,opacity=1] (n9.mid) -- (n2.mid) -- (n6.mid) -- cycle;
						\fill[fill=black!20,opacity=1] (n9.mid) -- (n0.mid) -- (n6.mid) -- cycle;
						\fill[fill=black!20,opacity=1] (n9.mid) -- (n2.mid) -- (n5.mid) -- cycle;
						\fill[fill=black!20,opacity=1] (n9.mid) -- (n0.mid) -- (n5.mid) -- cycle;
						\fill[fill=black!20,opacity=1] (n9.mid) -- (n0.mid) -- (n3.mid) -- cycle;
						\fill[fill=black!20,opacity=1] (n9.mid) -- (n0.mid) -- (n1.mid) -- cycle;
						\fill[fill=black!20,opacity=1] (n8.mid) -- (n1.mid) -- (n5.mid) -- cycle;
						\fill[fill=black!20,opacity=1] (n8.mid) -- (n1.mid) -- (n4.mid) -- cycle;
						\fill[fill=black!20,opacity=1] (n6.mid) -- (n0.mid) -- (n3.mid) -- cycle;
						\fill[fill=black!20,opacity=1] (n5.mid) -- (n0.mid) -- (n1.mid) -- cycle;
						
						\draw[thick] (n10.mid) -- (n7.mid);
						\draw[thick] (n10.mid) -- (n6.mid);
						\draw[thick] (n10.mid) -- (n3.mid);
						\draw[thick] (n10.mid) -- (n2.mid);
						\draw[thick] (n9.mid) -- (n6.mid);
						\draw[thick] (n9.mid) -- (n5.mid);
						\draw[thick] (n9.mid) -- (n3.mid);
						\draw[thick] (n9.mid) -- (n2.mid);
						\draw[thick] (n9.mid) -- (n1.mid);
						\draw[thick] (n9.mid) -- (n0.mid);
						\draw[thick] (n8.mid) -- (n5.mid);
						\draw[thick] (n8.mid) -- (n4.mid);
						\draw[thick] (n8.mid) -- (n1.mid);
						\draw[thick] (n7.mid) -- (n3.mid);
						\draw[thick] (n6.mid) -- (n3.mid);
						\draw[thick] (n6.mid) -- (n2.mid);
						\draw[thick] (n6.mid) -- (n0.mid);
						\draw[thick] (n5.mid) -- (n2.mid);
						\draw[thick] (n5.mid) -- (n1.mid);
						\draw[thick] (n5.mid) -- (n0.mid);
						\draw[thick] (n4.mid) -- (n1.mid);
						\draw[thick] (n3.mid) -- (n0.mid);
						\draw[thick] (n1.mid) -- (n0.mid);
						\end{pgfonlayer}
					\end{tikzpicture}}	
				\end{figure}	
			\endminipage}
			\fbox{\minipage{0.225\textwidth}
				\begin{figure}[H]
					\resizebox{1.0\textwidth}{!}{\begin{tikzpicture}[
						roundnode/.style={circle, draw=black, thick, fill=white, minimum size=7mm},
						]
						%Nodes
						\node           (label) 	at	(1,4) 		{Filtración 4 (t=0.6)};
						\node[roundnode]      (n0) 	at 	(4,0)           {0};
						\node[roundnode]      (n1) 	at 	(2,2)           {1};
						\node[roundnode]      (n2) 	at 	(2,0)           {2};
						\node[roundnode]      (n3)      at 	(2,-2)		{3};
						\node[roundnode]      (n4)      at 	(0,3)		{4};
						\node[roundnode]      (n5)      at 	(0,1)		{5};
						\node[roundnode]      (n6) 	at 	(0,-1)           {6};
						\node[roundnode]      (n7) 	at 	(0,-3)           {7};
						\node[roundnode]      (n8) 	at 	(-2,2)           {8};
						\node[roundnode]      (n9) 	at 	(-2,0)           {9};
						\node[roundnode]      (n10) 	at 	(-2,-2)           {10};
						\node 		(label2) 	at 	(1,-4) 		{$\beta_{0}=3,\beta_{1}=0$};
						
						\begin{pgfonlayer}{background}
						%Lines
						\fill[fill=black!20,opacity=1] (n9.mid) to[bend right] (n1.mid) -- (n5.mid) -- cycle;
						
						\draw[thick] (n10.mid) -- (n7.mid);
						\draw[thick] (n8.mid) -- (n4.mid);
						\draw[thick] (n9.mid) -- (n6.mid);
						\draw[thick] (n9.mid) -- (n5.mid);
						\draw[thick] (n6.mid) -- (n3.mid);
						\draw[thick] (n6.mid) -- (n2.mid);
						\draw[thick] (n5.mid) -- (n1.mid);
						\draw[thick] (n3.mid) -- (n0.mid);

						
						\end{pgfonlayer}
					\end{tikzpicture}}	
				\end{figure}	
				\begin{figure}[H]
					\resizebox{1.0\textwidth}{!}{\begin{tikzpicture}[
						roundnode/.style={circle, draw=black, thick, fill=white, minimum size=7mm},
						]
						%Nodes
						\node           (label) 	at	(1,4) 		{Filtración 9 (t=0.1)};
						\node[roundnode]      (n0) 	at 	(4,0)           {0};
						\node[roundnode]      (n1) 	at 	(2,2)           {1};
						\node[roundnode]      (n2) 	at 	(2,0)           {2};
						\node[roundnode]      (n3)      at 	(2,-2)		{3};
						\node[roundnode]      (n4)      at 	(0,3)		{4};
						\node[roundnode]      (n5)      at 	(0,1)		{5};
						\node[roundnode]      (n6) 	at 	(0,-1)           {6};
						\node[roundnode]      (n7) 	at 	(0,-3)           {7};
						\node[roundnode]      (n8) 	at 	(-2,2)           {8};
						\node[roundnode]      (n9) 	at 	(-2,0)           {9};
						\node[roundnode]      (n10) 	at 	(-2,-2)           {10};
						\node 		(label2) 	at 	(1,-4) 		{$\beta_{0}=1,\beta_{1}=0$};
						
						\begin{pgfonlayer}{background}							
						\fill[fill=black!20,opacity=1] (n10.mid) -- (n7.mid) -- (n3.mid) -- cycle;
						\fill[fill=black!20,opacity=1] (n10.mid) -- (n6.mid) -- (n3.mid) -- cycle;
						\fill[fill=black!20,opacity=1] (n10.mid) to[bend left] (n2.mid) -- (n6.mid) -- cycle;
						\fill[fill=black!20,opacity=1] (n9.mid) to[bend left] (n1.mid) -- (n5.mid) -- cycle;
						\fill[fill=black!20,opacity=1] (n9.mid) to[bend right] (n3.mid) -- (n6.mid) -- cycle;
						\fill[fill=black!20,opacity=1] (n9.mid) -- (n2.mid) -- (n6.mid) -- cycle;
						\fill[fill=black!20,opacity=1] (n9.mid) -- (n2.mid) -- (n5.mid) -- cycle;
						\fill[fill=black!20,opacity=1] (n8.mid) -- (n1.mid) -- (n5.mid) -- cycle;
						\fill[fill=black!20,opacity=1] (n8.mid) -- (n1.mid) -- (n4.mid) -- cycle;
						\fill[fill=black!20,opacity=1] (n7.mid) to[bend right] (n0.mid) -- (n3.mid) -- cycle;
						\fill[fill=black!20,opacity=1] (n6.mid) -- (n0.mid) -- (n3.mid) -- cycle;
						\fill[fill=black!20,opacity=1] (n6.mid) -- (n0.mid) -- (n2.mid) -- cycle;
						\fill[fill=black!20,opacity=1] (n5.mid) -- (n0.mid) -- (n1.mid) -- cycle;
						\fill[fill=black!20,opacity=1] (n5.mid) -- (n0.mid) -- (n2.mid) -- cycle;
						
						\draw[thick] (n10.mid) -- (n7.mid);
						\draw[thick] (n10.mid) -- (n6.mid);
						\draw[thick] (n10.mid) -- (n3.mid);
						\draw[thick] (n10.mid) -- (n0.mid);
						\draw[thick] (n9.mid) -- (n6.mid);
						\draw[thick] (n9.mid) -- (n5.mid);
						\draw[thick] (n9.mid) -- (n2.mid);
						\draw[thick] (n8.mid) -- (n5.mid);
						\draw[thick] (n8.mid) -- (n4.mid);
						\draw[thick] (n8.mid) -- (n1.mid);
						\draw[thick] (n7.mid) -- (n3.mid);
						\draw[thick] (n6.mid) -- (n3.mid);
						\draw[thick] (n6.mid) -- (n2.mid);
						\draw[thick] (n6.mid) -- (n0.mid);
						\draw[thick] (n5.mid) -- (n2.mid);
						\draw[thick] (n5.mid) -- (n1.mid);
						\draw[thick] (n5.mid) -- (n0.mid);
						\draw[thick] (n4.mid) -- (n1.mid);
						\draw[thick] (n3.mid) -- (n0.mid);
						\draw[thick] (n2.mid) -- (n0.mid);
						\draw[thick] (n1.mid) -- (n0.mid);
						\end{pgfonlayer}
					\end{tikzpicture}}	
				\end{figure}	
			\endminipage}
			\fbox{\minipage{0.225\textwidth}
				\begin{figure}[H]
					\resizebox{1.0\textwidth}{!}{\begin{tikzpicture}[
						roundnode/.style={circle, draw=black, fill=white, thick, minimum size=7mm},
						]
						%Nodes
						\node           (label) 	at	(1,4) 		{Filtración 6 (t=0.4)};
						\node[roundnode]      (n0) 	at 	(4,0)           {0};
						\node[roundnode]      (n1) 	at 	(2,2)           {1};
						\node[roundnode]      (n2) 	at 	(2,0)           {2};
						\node[roundnode]      (n3)      at 	(2,-2)		{3};
						\node[roundnode]      (n4)      at 	(0,3)		{4};
						\node[roundnode]      (n5)      at 	(0,1)		{5};
						\node[roundnode]      (n6) 	at 	(0,-1)           {6};
						\node[roundnode]      (n7) 	at 	(0,-3)           {7};
						\node[roundnode]      (n8) 	at 	(-2,2)           {8};
						\node[roundnode]      (n9) 	at 	(-2,0)           {9};
						\node[roundnode]      (n10) 	at 	(-2,-2)           {10};
						\node 		(label2) 	at 	(1,-4) 		{$\beta_{0}=3,\beta_{1}=0$};
						
						\begin{pgfonlayer}{background}
						%Lines
						\fill[fill=black!20,opacity=1] (n9.mid) to[bend left] (n1.mid) -- (n5.mid) -- cycle;
						\fill[fill=black!20,opacity=1] (n9.mid) to[bend right] (n3.mid) -- (n6.mid) -- cycle;
						\fill[fill=black!20,opacity=1] (n9.mid) -- (n2.mid) -- (n6.mid) -- cycle;
						\fill[fill=black!20,opacity=1] (n6.mid) -- (n0.mid) -- (n3.mid) -- cycle;
						
						\draw[thick] (n10.mid) -- (n7.mid);
						\draw[thick] (n9.mid) -- (n6.mid);
						\draw[thick] (n9.mid) -- (n5.mid);
						\draw[thick] (n9.mid) -- (n3.mid);
						\draw[thick] (n9.mid) -- (n2.mid);
						\draw[thick] (n9.mid) -- (n1.mid);
						\draw[thick] (n8.mid) -- (n4.mid);
						\draw[thick] (n6.mid) -- (n3.mid);
						\draw[thick] (n6.mid) -- (n2.mid);
						\draw[thick] (n6.mid) -- (n0.mid);
						\draw[thick] (n5.mid) -- (n1.mid);
						\draw[thick] (n3.mid) -- (n0.mid);

						\end{pgfonlayer}
					\end{tikzpicture}}	
				\end{figure}	
				\begin{figure}[H]
					\resizebox{1.0\textwidth}{!}{\begin{tikzpicture}[
						roundnode/.style={circle, draw=black, thick, fill=white, minimum size=7mm},
						]
						%Nodes
						\node           (label) 	at	(1,4) 		{Filtración 10 (t=0.0)};
						\node[roundnode]      (n0) 	at 	(4,0)           {0};
						\node[roundnode]      (n1) 	at 	(2,2)           {1};
						\node[roundnode]      (n2) 	at 	(2,0)           {2};
						\node[roundnode]      (n3)      at 	(2,-2)		{3};
						\node[roundnode]      (n4)      at 	(0,3)		{4};
						\node[roundnode]      (n5)      at 	(0,1)		{5};
						\node[roundnode]      (n6) 	at 	(0,-1)           {6};
						\node[roundnode]      (n7) 	at 	(0,-3)           {7};
						\node[roundnode]      (n8) 	at 	(-2,2)           {8};
						\node[roundnode]      (n9) 	at 	(-2,0)           {9};
						\node[roundnode]      (n10) 	at 	(-2,-2)           {10};
						\node 		(label2) 	at 	(1,-4) 		{$\beta_{0}=1,\beta_{1}=0$};
						
						\begin{pgfonlayer}{background}							
						\fill[fill=black!20,opacity=1] (n10.mid) -- (n7.mid) -- (n3.mid) -- cycle;
						\fill[fill=black!20,opacity=1] (n10.mid) -- (n6.mid) -- (n3.mid) -- cycle;
						\fill[fill=black!20,opacity=1] (n10.mid) to[bend left] (n2.mid) -- (n6.mid) -- cycle;
						\fill[fill=black!20,opacity=1] (n9.mid) to[bend left] (n1.mid) -- (n5.mid) -- cycle;
						\fill[fill=black!20,opacity=1] (n9.mid) to[bend right] (n3.mid) -- (n6.mid) -- cycle;
						\fill[fill=black!20,opacity=1] (n9.mid) -- (n2.mid) -- (n6.mid) -- cycle;
						\fill[fill=black!20,opacity=1] (n9.mid) -- (n2.mid) -- (n5.mid) -- cycle;
						\fill[fill=black!20,opacity=1] (n8.mid) -- (n1.mid) -- (n5.mid) -- cycle;
						\fill[fill=black!20,opacity=1] (n8.mid) to[bend right] (n2.mid) -- (n5.mid) -- cycle;
						\fill[fill=black!20,opacity=1] (n8.mid) -- (n1.mid) -- (n4.mid) -- cycle;
						\fill[fill=black!20,opacity=1] (n7.mid) to[bend right] (n0.mid) -- (n3.mid) -- cycle;
						\fill[fill=black!20,opacity=1] (n6.mid) -- (n0.mid) -- (n3.mid) -- cycle;
						\fill[fill=black!20,opacity=1] (n6.mid) -- (n0.mid) -- (n2.mid) -- cycle;
						\fill[fill=black!20,opacity=1] (n5.mid) -- (n0.mid) -- (n1.mid) -- cycle;
						\fill[fill=black!20,opacity=1] (n5.mid) -- (n0.mid) -- (n2.mid) -- cycle;
						\fill[fill=black!20,opacity=1] (n4.mid) to[bend left] (n0.mid) -- (n1.mid) -- cycle;
						
						\draw[thick] (n10.mid) -- (n7.mid);
						\draw[thick] (n10.mid) -- (n6.mid);
						\draw[thick] (n10.mid) -- (n3.mid);
						\draw[thick] (n10.mid) -- (n0.mid);
						\draw[thick] (n9.mid) -- (n6.mid);
						\draw[thick] (n9.mid) -- (n5.mid);
						\draw[thick] (n9.mid) -- (n2.mid);
						\draw[thick] (n8.mid) -- (n5.mid);
						\draw[thick] (n8.mid) -- (n4.mid);
						\draw[thick] (n8.mid) -- (n1.mid);
						\draw[thick] (n8.mid) -- (n0.mid);
						\draw[thick] (n7.mid) -- (n3.mid);
						\draw[thick] (n6.mid) -- (n3.mid);
						\draw[thick] (n6.mid) -- (n2.mid);
						\draw[thick] (n6.mid) -- (n0.mid);
						\draw[thick] (n5.mid) -- (n2.mid);
						\draw[thick] (n5.mid) -- (n1.mid);
						\draw[thick] (n5.mid) -- (n0.mid);
						\draw[thick] (n4.mid) -- (n1.mid);
						\draw[thick] (n3.mid) -- (n0.mid);
						\draw[thick] (n2.mid) -- (n0.mid);
						\draw[thick] (n1.mid) -- (n0.mid);
						\end{pgfonlayer}
					\end{tikzpicture}}
				\end{figure}	
			\endminipage}
					\caption{Pasos en la filtración del 
					complejo simplicial asociado a la red 
					y números de Betti correspondientes.}
					\label{fig:CompSimpGlobal}
		\end{figure}
		Para una mayor claridad, se ha omitido el dibujo de algunos 
		símplices. No obstante, damos una lista completa de los 
		complejos simpliciales de cada filtración:
		\begin{itemize}
			
			\item Filtración 0 (t=1.0):
				\begin{itemize}
					\item {\Large $\nu$}$^{1.0}_{0}=\{\{10\},\{9\},\{8\},\{7\},\{6\},\{5\},\{4\},\{3\},\{2\},\{1\},\{0\}\}$
					\item {\Large $\nu$}$^{1.0}_{1}=\{\{10, 7\},\{8, 4\}\}$
				\end{itemize}
				A partir de ahora omitiremos 
				{\Large $\nu$}$_{0}$ pues se mantiene igual 
				para todos los pasos en la filtración.
			\item Filtración 3 (t=0.7):
				\begin{itemize}
					\item {\Large $\nu$}$^{0.7}_{1}=\{\{10, 7\},\{9, 6\},\{9, 5\},\{8, 4\},\{6, 3\},\{6, 2\},\{5, 1\}\}$
				\end{itemize}
			\item Filtración 4 (t=0.6):
				\begin{itemize}
					\item {\Large $\nu$}$^{0.6}_{1}=\{\{10, 7\},\{9, 6\},\{9, 5\},\{9, 1\},\{8, 4\},\{6, 3\},\{6, 2\},\{5, 1\},\{3, 0\}\}$
					\item {\Large $\nu$}$^{0.6}_{2}=\{\{9, 5, 1\}\}$
				\end{itemize}
			\item Filtración 6 (t=0.4):
				\begin{itemize}
					\item {\Large $\nu$}$^{0.4}_{1}=\{\{10, 7\},\{9, 6\},\{9, 5\},\{9, 3\},\{9, 2\},\{9, 1\},\{8, 4\},\{6, 3\},\{6, 2\},
						\{6, 0\},$ \\$\{5, 1\},\{3, 0\}\}$
					\item {\Large $\nu$}$^{0.4}_{2}=\{\{9, 6, 3\},\{9, 6, 2\},\{9, 5, 1\},\{6, 3, 0\}\}$
				\end{itemize}
			\item Filtración 7 (t=0.3):
				\begin{itemize}
					\item {\Large $\nu$}$^{0.3}_{1}=\{\{10, 7\},\{10, 6\},\{10, 3\},\{9, 6\},\{9, 5\},\{9, 3\},\{9, 2\},\{9, 1\},\{8, 4\},$\\ 
							$\{7, 3\},\{6, 3\},\{6, 2\},\{6, 0\},\{5, 1\},\{3, 0\},\{1, 0\}\}$
					\item {\Large $\nu$}$^{0.3}_{2}=\{\{10, 7, 3\},\{10, 6, 3\},\{9, 6, 3\},\{9, 6, 2\},\{9, 5, 1\},\{6, 3, 0\}\}$
				\end{itemize}
			\item Filtración 8 (t=0.2):
				\begin{itemize}
					\item {\Large $\nu$}$^{0.2}_{1}=\{\{10, 7\},\{10, 6\},\{10, 3\},\{10, 2\},\{9, 6\},\{9, 5\},\{9, 3\},\{9, 2\},\{9, 1\},$\\
							$\{9, 0\},\{8, 5\},\{8, 4\},\{8, 1\},\{7, 3\},\{6, 3\},\{6, 2\},\{6, 0\},\{5, 2\},\{5, 1\},\{5, 0\},$\\
							$\{4, 1\},\{3, 0\},\{1, 0\}\}$
					\item {\Large $\nu$}$^{0.2}_{2}=\{\{10, 7, 3\},\{10, 6, 3\},\{10, 6, 2\},\{9, 6, 3\},\{9, 6, 2\},\{9, 6, 0\},\{9, 5, 2\},$\\
							$\{9, 5, 1\},\{9, 5, 0\},\{9, 3, 0\},\{9, 1, 0\},\{8, 5, 1\},\{8, 4, 1\},\{6, 3, 0\},\{5, 1, 0\}\}$
					\item {\Large $\nu$}$^{0.2}_{3}=\{\{9, 6, 3, 0\},\{9, 5, 1, 0\}\}$
				\end{itemize}
			\item Filtración 9 (t=0.1):
				\begin{itemize}
					\item {\Large $\nu$}$^{0.1}_{1}=\{\{10, 7\},\{10, 6\},\{10, 3\},\{10, 2\},\{10, 0\},\{9, 6\},\{9, 5\},\{9, 3\},\{9, 2\},$\\
							$\{9, 1\},\{9, 0\},\{8, 5\},\{8, 4\},\{8, 1\},\{7, 3\},\{7, 0\},\{6, 3\},\{6, 2\},\{6, 0\},\{5, 2\},$\\
						$\{5, 1\},\{5, 0\},\{4, 1\},\{3, 0\},\{2, 0\},\{1, 0\}\}$
					\item {\Large $\nu$}$^{0.1}_{2}=\{\{10, 7, 3\},\{10, 7, 0\},\{10, 6, 3\},\{10, 6, 2\},\{10, 6, 0\},\{10, 3, 0\},$\\
							$\{10, 2, 0\},\{9, 6, 3\},\{9, 6, 2\},\{9, 6, 0\},\{9, 5, 2\},\{9, 5, 1\},\{9, 5, 0\},\{9, 3, 0\},$\\
							$\{9, 2, 0\},\{9, 1, 0\},\{8, 5, 1\},\{8, 4, 1\},\{7, 3, 0\},\{6, 3, 0\},\{6, 2, 0\},\{5, 2, 0\},$\\
						$\{5, 1, 0\}\}$
					\item {\Large $\nu$}$^{0.1}_{3}=\{\{10, 7, 3, 0\},\{10, 6, 3, 0\},\{10, 6, 2, 0\},\{9, 6, 3, 0\},\{9, 6, 2, 0\},$\\
						$\{9, 5, 2, 0\},\{9, 5, 1, 0\}\}$
				\end{itemize}
			\item Filtración 10 (t=0.0):
				\begin{itemize}
					\item {\Large $\nu$}$^{0.0}_{1}=\{\{10, 7\},\{10, 6\},\{10, 3\},\{10, 2\},\{10, 0\},\{9, 6\},\{9, 5\},\{9, 3\},\{9, 2\},$\\
						$\{9, 1\},\{9, 0\},\{8, 5\},\{8, 4\},\{8, 2\},\{8, 1\},\{8, 0\},\{7, 3\},\{7, 0\},\{6, 3\},\{6, 2\},$\\
					$\{6, 0\},\{5, 2\},\{5, 1\},\{5, 0\},\{4, 1\},\{4, 0\},\{3, 0\},\{2, 0\},\{1, 0\}\}$
					\item {\Large $\nu$}$^{0.0}_{2}=\{\{10, 7, 3\},\{10, 7, 0\},\{10, 6, 3\},\{10, 6, 2\},\{10, 6, 0\},\{10, 3, 0\},$\\
							$\{10, 2, 0\},\{9, 6, 3\},\{9, 6, 2\},\{9, 6, 0\},\{9, 5, 2\},\{9, 5, 1\},\{9, 5, 0\},\{9, 3, 0\},$\\
							$\{9, 2, 0\},\{9, 1, 0\},\{8, 5, 2\},\{8, 5, 1\},\{8, 5, 0\},\{8, 4, 1\},\{8, 4, 0\},\{8, 2, 0\},$\\
						$\{8, 1, 0\},\{7, 3, 0\},\{6, 3, 0\},\{6, 2, 0\},\{5, 2, 0\},\{5, 1, 0\},\{4, 1, 0\}\}$
					\item {\Large $\nu$}$^{0.0}_{3}=\{\{10, 7, 3, 0\},\{10, 6, 3, 0\},\{10, 6, 2, 0\},\{9, 6, 3, 0\},\{9, 6, 2, 0\},$\\
						$\{9, 5, 2, 0\},\{9, 5, 1, 0\},\{8, 5, 2, 0\},\{8, 5, 1, 0\},\{8, 4, 1, 0\}\}$
				\end{itemize}
		\end{itemize}
	Para el cálculo de las anteriores listas se han empleado una serie de 
	algoritmos y programas que detallaremos en la 
	\autoref{sec:local+prog}.

	Finalmente, los diagramas de barras y persistencia asociados al
	complejo simplicial filtrado representado en 
	la \autoref{fig:CompSimpGlobal} pueden verse en 
	la \autoref{fig:ejemGlobalDiag}.
	
	\begin{figure}[!htbp]
		\minipage{0.5\textwidth}
			\begin{figure}[H]
				\resizebox{1.0\textwidth}{!}{\includegraphics{Images/BarrasGlobalEjAutor.png}}
			\end{figure}
		\endminipage
		\minipage{0.5\textwidth}
			\begin{figure}[H]
				\resizebox{1.0\textwidth}{!}{\includegraphics{Images/PerisistenciaGlobalEjAutor.png}}
			\end{figure}
		\endminipage
		\caption{Diagramas de barras y persistencia asociados a la red 
		\ref{fig:ejemGlobal}.}
		\label{fig:ejemGlobalDiag}
	\end{figure}
	
	Observamos que estos diagramas son ligeramente distintos que los 
	presentados en el capítulo de preliminares. Esto es debido a que para
	el cálculo de éstos se ha empleado una librería que los produce 
	automáticamente. Veremos esto en más detalle en la 
	\autoref{sec:local+prog}.

	Si observamos el diagrama izquierdo (diagrama de barras), vemos que en 
	la filtración 0 nacen 9 clases de equivalencia (componentes conexas) y
	que a lo largo de las siguientes filtraciones van muriendo hasta que 
	sólo queda una que persiste. Podemos seguir esta evolución si nos 
	vamos fijando en el número $\beta_{0}$ en  
	la \autoref{fig:CompSimpGlobal}. También podemos observar que en la 
	filtración 7 nace una clase de equivalencia (agujero $2$-dimensional) 
	y muere en la filtración 8. La evolución de esta clase la podemos 
	seguir viendo la \autoref{fig:CompSimpGlobal} y fijándonos en el 
	número $\beta_{1}$.

	Observando ahora el diagrama derecho (diagrama de persistencia) vemos 
	la misma información pero con otra representación. En este caso se 
	aprecian 6 puntos rojos que nacen en la filtración 0 y van muriendo a 
	lo largo de las siguientes filtraciones, excepto uno que persiste. 
	Estos puntos se corresponden 
	con las componentes conexas de la filtración. A diferencia de los 
	diagramas presentados en el capítulo de preliminares, en este caso, 
	cuando varias clases nacen y mueren en la misma filtración se 
	superponen sus puntos correspondientes en el diagrama dando lugar a 
	puntos más nítidos en el diagrama. También observamos un único punto 
	azul que se corresponde con la clase de equivalencia que da lugar a un
	agujero $2$-dimensional, esta clase nace en la filtración 7 y muere en
	la filtración 8.
	\end{ejem}

	En el ejemplo anterior, podemos apreciar que, aunque esta 
	interpretación sea más coherente con la teoría desarrollada, la 
	información que nos aporta sobre la red es un tanto escasa, pues los 
	agujeros $p$-dimensionales colapsan muy rápido. Esto se ve claramente 
	reflejado en la \autoref{fig:ejemGlobalDiag}. Por ello, 
	será conveniente hacer una serie de experimentos con el fin de probar
	el desempeño de esta herramienta.

	\section{Interpretación local. Algoritmos y programas}
	\label{sec:local+prog}
	En esta sección veremos cómo funciona la interpretación que propone 
	el autor (que se corresponde con la que hemos bautizado como local) 
	mediante su desempeño en el mismo ejemplo que hemos visto en la 
	sección anterior. Además, aprovecharemos para describir más 
	detalladamente los algoritmos y programas que hemos empleado para el 
	cálculo automático de la homología persistente en este ejemplo. 
	Siguiendo esta interpretación, hemos realizado el cálculo de los 
	símplices mediante estos pasos:

	\begin{enumerate}[label=\arabic*)]
		\item Se calculan recursivamente todos los posibles caminos 
			para un vértice de origen. 
		\item Se calcula la importancia de cada camino multiplicando 
			la importancia entre cada par vértices, sin tener en 
			cuenta si existe otro camino que comparta vértices.
		\item Al igual que en la interpretación global, se descartan 
			aquellos caminos que no superan el umbral fijado. 

		\item Se calculan todos los subsímplices asociados a cada 
			camino.
	\end{enumerate}

	A continuación veamos el ejemplo tal y cómo lo propone el autor en 
	\cite{Articulo-Watanabe}:
	
	\begin{ejem}
		Supondremos la red que puede verse en 
		la \autoref{fig:ejemGlobal}. Como en el ejemplo 
		anterior ya hemos visto una representación de la misma y su 
		matriz asociada, esta vez los omitimos.

		Vamos directamente con el cálculo del complejo simplicial 
		filtrado asociado. Al igual que antes, vamos a ilustrar unos 
		cuantos pasos en la filtración con los correspondientes 
		números de Betti asociados. Tras esto, mostraremos los 
		complejos simpliciales asociados a cada filtración. La 
		ilustración de los pasos en la filtración puede verse en 
		la \autoref{fig:CompSimpLocal}.

		\begin{figure}[!htbp]
			\fbox{\minipage{0.225\textwidth}
				\begin{figure}[H]
					\resizebox{1.0\textwidth}{!}{\begin{tikzpicture}[
						roundnode/.style={circle, draw=black, thick, fill=white, minimum size=7mm},
						]
						%Nodes
						\node           (label) 	at	(1,4) 		{Filtración 0 (t=1.0)};
						\node[roundnode]      (n0) 	at 	(4,0)           {0};
						\node[roundnode]      (n1) 	at 	(2,2)           {1};
						\node[roundnode]      (n2) 	at 	(2,0)           {2};
						\node[roundnode]      (n3)      at 	(2,-2)		{3};
						\node[roundnode]      (n4)      at 	(0,3)		{4};
						\node[roundnode]      (n5)      at 	(0,1)		{5};
						\node[roundnode]      (n6) 	at 	(0,-1)           {6};
						\node[roundnode]      (n7) 	at 	(0,-3)           {7};
						\node[roundnode]      (n8) 	at 	(-2,2)           {8};
						\node[roundnode]      (n9) 	at 	(-2,0)           {9};
						\node[roundnode]      (n10) 	at 	(-2,-2)           {10};
						\node 		(label2) 	at 	(1,-4) 		{$\beta_{0}=9,\beta_{1}=0$};
						
						\begin{pgfonlayer}{background}
						%Lines
						\draw[thick] (n10.mid) -- (n7.mid);
						\draw[thick] (n8.mid) -- (n4.mid);
						\end{pgfonlayer}
					\end{tikzpicture}}	
				\end{figure}	
				\begin{figure}[H]
					\resizebox{1.0\textwidth}{!}{\begin{tikzpicture}[
						roundnode/.style={circle, draw=black, thick, fill=white, minimum size=7mm},
						]
						%Nodes
						\node           (label) 	at	(1,4) 		{Filtración 7 (t=0.3)};
						\node[roundnode]      (n0) 	at 	(4,0)           {0};
						\node[roundnode]      (n1) 	at 	(2,2)           {1};
						\node[roundnode]      (n2) 	at 	(2,0)           {2};
						\node[roundnode]      (n3)      at 	(2,-2)		{3};
						\node[roundnode]      (n4)      at 	(0,3)		{4};
						\node[roundnode]      (n5)      at 	(0,1)		{5};
						\node[roundnode]      (n6) 	at 	(0,-1)           {6};
						\node[roundnode]      (n7) 	at 	(0,-3)           {7};
						\node[roundnode]      (n8) 	at 	(-2,2)           {8};
						\node[roundnode]      (n9) 	at 	(-2,0)           {9};
						\node[roundnode]      (n10) 	at 	(-2,-2)           {10};
						\node 		(label2) 	at 	(1,-4) 		{$\beta_{0}=2,\beta_{1}=2$};
						
						\begin{pgfonlayer}{background}							
						\fill[fill=black!20,opacity=1] (n10.mid) -- (n7.mid) -- (n3.mid) -- cycle;
						\fill[fill=black!20,opacity=1] (n9.mid) to[bend left] (n1.mid) -- (n5.mid) -- cycle;
						\fill[fill=black!20,opacity=1] (n9.mid) to[bend right] (n3.mid) -- (n6.mid) -- cycle;
						\fill[fill=black!20,opacity=1] (n9.mid) -- (n2.mid) -- (n6.mid) -- cycle;
						\fill[fill=black!20,opacity=1] (n6.mid) -- (n0.mid) -- (n3.mid) -- cycle;
						
						\draw[thick] (n10.mid) -- (n7.mid);
						\draw[thick] (n10.mid) -- (n6.mid);
						\draw[thick] (n10.mid) -- (n3.mid);
						\draw[thick] (n9.mid) -- (n6.mid);
						\draw[thick] (n9.mid) -- (n5.mid);
						\draw[thick] (n9.mid) -- (n3.mid);
						\draw[thick] (n9.mid) -- (n2.mid);
						\draw[thick] (n9.mid) -- (n1.mid);
						\draw[thick] (n8.mid) -- (n4.mid);
						\draw[thick] (n7.mid) -- (n3.mid);
						\draw[thick] (n6.mid) -- (n3.mid);
						\draw[thick] (n6.mid) -- (n2.mid);
						\draw[thick] (n6.mid) -- (n0.mid);
						\draw[thick] (n5.mid) -- (n1.mid);
						\draw[thick] (n3.mid) -- (n0.mid);
						\draw[thick] (n1.mid) -- (n0.mid);
					

						\end{pgfonlayer}
					\end{tikzpicture}}	
				\end{figure}
			\endminipage}
			\fbox{\minipage{0.225\textwidth}
				\begin{figure}[H]
					\resizebox{1.0\textwidth}{!}{\begin{tikzpicture}[
						roundnode/.style={circle, draw=black, thick, fill=white, minimum size=7mm},
						]
						%Nodes
						\node           (label) 	at	(1,4) 		{Filtración 3 (t=0.7)};
						\node[roundnode]      (n0) 	at 	(4,0)           {0};
						\node[roundnode]      (n1) 	at 	(2,2)           {1};
						\node[roundnode]      (n2) 	at 	(2,0)           {2};
						\node[roundnode]      (n3)      at 	(2,-2)		{3};
						\node[roundnode]      (n4)      at 	(0,3)		{4};
						\node[roundnode]      (n5)      at 	(0,1)		{5};
						\node[roundnode]      (n6) 	at 	(0,-1)           {6};
						\node[roundnode]      (n7) 	at 	(0,-3)           {7};
						\node[roundnode]      (n8) 	at 	(-2,2)           {8};
						\node[roundnode]      (n9) 	at 	(-2,0)           {9};
						\node[roundnode]      (n10) 	at 	(-2,-2)           {10};
						\node 		(label2) 	at 	(1,-4) 		{$\beta_{0}=4,\beta_{1}=0$};
						
						\begin{pgfonlayer}{background}
						%Lines
						\draw[thick] (n10.mid) -- (n7.mid);
						\draw[thick] (n8.mid) -- (n4.mid);
						\draw[thick] (n9.mid) -- (n6.mid);
						\draw[thick] (n9.mid) -- (n5.mid);
						\draw[thick] (n6.mid) -- (n3.mid);
						\draw[thick] (n6.mid) -- (n2.mid);
						\draw[thick] (n5.mid) -- (n1.mid);
						\end{pgfonlayer}
					\end{tikzpicture}}	
				\end{figure}	
				\begin{figure}[H]
					\resizebox{1.0\textwidth}{!}{\begin{tikzpicture}[
						roundnode/.style={circle, draw=black, thick, fill=white, minimum size=7mm},
						]
						%Nodes
						\node           (label) 	at	(1,4) 		{Filtración 8 (t=0.2)};
						\node[roundnode]      (n0) 	at 	(4,0)           {0};
						\node[roundnode]      (n1) 	at 	(2,2)           {1};
						\node[roundnode]      (n2) 	at 	(2,0)           {2};
						\node[roundnode]      (n3)      at 	(2,-2)		{3};
						\node[roundnode]      (n4)      at 	(0,3)		{4};
						\node[roundnode]      (n5)      at 	(0,1)		{5};
						\node[roundnode]      (n6) 	at 	(0,-1)           {6};
						\node[roundnode]      (n7) 	at 	(0,-3)           {7};
						\node[roundnode]      (n8) 	at 	(-2,2)           {8};
						\node[roundnode]      (n9) 	at 	(-2,0)           {9};
						\node[roundnode]      (n10) 	at 	(-2,-2)           {10};
						\node 		(label2) 	at 	(1,-4) 		{$\beta_{0}=1,\beta_{1}=3$};
						
						\begin{pgfonlayer}{background}							
						\fill[fill=black!20,opacity=1] (n10.mid) -- (n7.mid) -- (n3.mid) -- cycle;
						\fill[fill=black!20,opacity=1] (n10.mid) -- (n6.mid) -- (n3.mid) -- cycle;
						\fill[fill=black!20,opacity=1] (n10.mid) to[bend left] (n2.mid) -- (n6.mid) -- cycle;
						\fill[fill=black!20,opacity=1] (n9.mid) to[bend left] (n1.mid) -- (n5.mid) -- cycle;
						\fill[fill=black!20,opacity=1] (n9.mid) to[bend right] (n3.mid) -- (n6.mid) -- cycle;
						\fill[fill=black!20,opacity=1] (n9.mid) -- (n2.mid) -- (n6.mid) -- cycle;
						\fill[fill=black!20,opacity=1] (n9.mid) -- (n0.mid) -- (n6.mid) -- cycle;
						\fill[fill=black!20,opacity=1] (n9.mid) -- (n0.mid) -- (n3.mid) -- cycle;
						\fill[fill=black!20,opacity=1] (n8.mid) -- (n1.mid) -- (n4.mid) -- cycle;
						\fill[fill=black!20,opacity=1] (n6.mid) -- (n0.mid) -- (n3.mid) -- cycle;
						\fill[fill=black!20,opacity=1] (n5.mid) -- (n0.mid) -- (n1.mid) -- cycle;
						
						\draw[thick] (n10.mid) -- (n7.mid);
						\draw[thick] (n10.mid) -- (n6.mid);
						\draw[thick] (n10.mid) -- (n3.mid);
						\draw[thick] (n10.mid) -- (n2.mid);
						\draw[thick] (n9.mid) -- (n6.mid);
						\draw[thick] (n9.mid) -- (n5.mid);
						\draw[thick] (n9.mid) -- (n3.mid);
						\draw[thick] (n9.mid) -- (n2.mid);
						\draw[thick] (n9.mid) -- (n1.mid);
						\draw[thick] (n9.mid) -- (n0.mid);
						\draw[thick] (n8.mid) -- (n5.mid);
						\draw[thick] (n8.mid) -- (n4.mid);
						\draw[thick] (n8.mid) -- (n1.mid);
						\draw[thick] (n7.mid) -- (n3.mid);
						\draw[thick] (n6.mid) -- (n3.mid);
						\draw[thick] (n6.mid) -- (n2.mid);
						\draw[thick] (n6.mid) -- (n0.mid);
						\draw[thick] (n5.mid) -- (n2.mid);
						\draw[thick] (n5.mid) -- (n1.mid);
						\draw[thick] (n5.mid) -- (n0.mid);
						\draw[thick] (n4.mid) -- (n1.mid);
						\draw[thick] (n3.mid) -- (n0.mid);
						\draw[thick] (n1.mid) -- (n0.mid);
						\end{pgfonlayer}
					\end{tikzpicture}}	
				\end{figure}	
			\endminipage}
			\fbox{\minipage{0.225\textwidth}
				\begin{figure}[H]
					\resizebox{1.0\textwidth}{!}{\begin{tikzpicture}[
						roundnode/.style={circle, draw=black, thick, fill=white, minimum size=7mm},
						]
						%Nodes
						\node           (label) 	at	(1,4) 		{Filtración 4 (t=0.6)};
						\node[roundnode]      (n0) 	at 	(4,0)           {0};
						\node[roundnode]      (n1) 	at 	(2,2)           {1};
						\node[roundnode]      (n2) 	at 	(2,0)           {2};
						\node[roundnode]      (n3)      at 	(2,-2)		{3};
						\node[roundnode]      (n4)      at 	(0,3)		{4};
						\node[roundnode]      (n5)      at 	(0,1)		{5};
						\node[roundnode]      (n6) 	at 	(0,-1)           {6};
						\node[roundnode]      (n7) 	at 	(0,-3)           {7};
						\node[roundnode]      (n8) 	at 	(-2,2)           {8};
						\node[roundnode]      (n9) 	at 	(-2,0)           {9};
						\node[roundnode]      (n10) 	at 	(-2,-2)           {10};
						\node 		(label2) 	at 	(1,-4) 		{$\beta_{0}=3,\beta_{1}=0$};
						
						\begin{pgfonlayer}{background}
						%Lines
						\fill[fill=black!20,opacity=1] (n9.mid) to[bend right] (n1.mid) -- (n5.mid) -- cycle;
						
						\draw[thick] (n10.mid) -- (n7.mid);
						\draw[thick] (n8.mid) -- (n4.mid);
						\draw[thick] (n9.mid) -- (n6.mid);
						\draw[thick] (n9.mid) -- (n5.mid);
						\draw[thick] (n6.mid) -- (n3.mid);
						\draw[thick] (n6.mid) -- (n2.mid);
						\draw[thick] (n5.mid) -- (n1.mid);
						\draw[thick] (n3.mid) -- (n0.mid);

						
						\end{pgfonlayer}
					\end{tikzpicture}}	
				\end{figure}	
				\begin{figure}[H]
					\resizebox{1.0\textwidth}{!}{\begin{tikzpicture}[
						roundnode/.style={circle, draw=black, thick, fill=white, minimum size=7mm},
						]
						%Nodes
						\node           (label) 	at	(1,4) 		{Filtración 9 (t=0.1)};
						\node[roundnode]      (n0) 	at 	(4,0)           {0};
						\node[roundnode]      (n1) 	at 	(2,2)           {1};
						\node[roundnode]      (n2) 	at 	(2,0)           {2};
						\node[roundnode]      (n3)      at 	(2,-2)		{3};
						\node[roundnode]      (n4)      at 	(0,3)		{4};
						\node[roundnode]      (n5)      at 	(0,1)		{5};
						\node[roundnode]      (n6) 	at 	(0,-1)           {6};
						\node[roundnode]      (n7) 	at 	(0,-3)           {7};
						\node[roundnode]      (n8) 	at 	(-2,2)           {8};
						\node[roundnode]      (n9) 	at 	(-2,0)           {9};
						\node[roundnode]      (n10) 	at 	(-2,-2)           {10};
						\node 		(label2) 	at 	(1,-4) 		{$\beta_{0}=1,\beta_{1}=1$};
						
						\begin{pgfonlayer}{background}							
						\fill[fill=black!20,opacity=1] (n10.mid) -- (n7.mid) -- (n3.mid) -- cycle;
						\fill[fill=black!20,opacity=1] (n10.mid) -- (n6.mid) -- (n3.mid) -- cycle;
						\fill[fill=black!20,opacity=1] (n10.mid) to[bend left] (n2.mid) -- (n6.mid) -- cycle;
						\fill[fill=black!20,opacity=1] (n9.mid) to[bend left] (n1.mid) -- (n5.mid) -- cycle;
						\fill[fill=black!20,opacity=1] (n9.mid) to[bend right] (n3.mid) -- (n6.mid) -- cycle;
						\fill[fill=black!20,opacity=1] (n9.mid) -- (n2.mid) -- (n6.mid) -- cycle;
						\fill[fill=black!20,opacity=1] (n9.mid) -- (n2.mid) -- (n5.mid) -- cycle;
						\fill[fill=black!20,opacity=1] (n8.mid) -- (n1.mid) -- (n5.mid) -- cycle;
						\fill[fill=black!20,opacity=1] (n8.mid) -- (n1.mid) -- (n4.mid) -- cycle;
						\fill[fill=black!20,opacity=1] (n7.mid) to[bend right] (n0.mid) -- (n3.mid) -- cycle;
						\fill[fill=black!20,opacity=1] (n6.mid) -- (n0.mid) -- (n3.mid) -- cycle;
						\fill[fill=black!20,opacity=1] (n6.mid) -- (n0.mid) -- (n2.mid) -- cycle;
						\fill[fill=black!20,opacity=1] (n5.mid) -- (n0.mid) -- (n1.mid) -- cycle;
						\fill[fill=black!20,opacity=1] (n5.mid) -- (n0.mid) -- (n2.mid) -- cycle;
						
						\draw[thick] (n10.mid) -- (n7.mid);
						\draw[thick] (n10.mid) -- (n6.mid);
						\draw[thick] (n10.mid) -- (n3.mid);
						\draw[thick] (n10.mid) -- (n0.mid);
						\draw[thick] (n9.mid) -- (n6.mid);
						\draw[thick] (n9.mid) -- (n5.mid);
						\draw[thick] (n9.mid) -- (n2.mid);
						\draw[thick] (n8.mid) -- (n5.mid);
						\draw[thick] (n8.mid) -- (n4.mid);
						\draw[thick] (n8.mid) -- (n1.mid);
						\draw[thick] (n7.mid) -- (n3.mid);
						\draw[thick] (n6.mid) -- (n3.mid);
						\draw[thick] (n6.mid) -- (n2.mid);
						\draw[thick] (n6.mid) -- (n0.mid);
						\draw[thick] (n5.mid) -- (n2.mid);
						\draw[thick] (n5.mid) -- (n1.mid);
						\draw[thick] (n5.mid) -- (n0.mid);
						\draw[thick] (n4.mid) -- (n1.mid);
						\draw[thick] (n3.mid) -- (n0.mid);
						\draw[thick] (n2.mid) -- (n0.mid);
						\draw[thick] (n1.mid) -- (n0.mid);
						\end{pgfonlayer}
					\end{tikzpicture}}	
				\end{figure}	
			\endminipage}
			\fbox{\minipage{0.225\textwidth}
				\begin{figure}[H]
					\resizebox{1.0\textwidth}{!}{\begin{tikzpicture}[
						roundnode/.style={circle, draw=black, fill=white, thick, minimum size=7mm},
						]
						%Nodes
						\node           (label) 	at	(1,4) 		{Filtración 6 (t=0.4)};
						\node[roundnode]      (n0) 	at 	(4,0)           {0};
						\node[roundnode]      (n1) 	at 	(2,2)           {1};
						\node[roundnode]      (n2) 	at 	(2,0)           {2};
						\node[roundnode]      (n3)      at 	(2,-2)		{3};
						\node[roundnode]      (n4)      at 	(0,3)		{4};
						\node[roundnode]      (n5)      at 	(0,1)		{5};
						\node[roundnode]      (n6) 	at 	(0,-1)           {6};
						\node[roundnode]      (n7) 	at 	(0,-3)           {7};
						\node[roundnode]      (n8) 	at 	(-2,2)           {8};
						\node[roundnode]      (n9) 	at 	(-2,0)           {9};
						\node[roundnode]      (n10) 	at 	(-2,-2)           {10};
						\node 		(label2) 	at 	(1,-4) 		{$\beta_{0}=3,\beta_{1}=0$};
						
						\begin{pgfonlayer}{background}
						%Lines
						\fill[fill=black!20,opacity=1] (n9.mid) to[bend left] (n1.mid) -- (n5.mid) -- cycle;
						\fill[fill=black!20,opacity=1] (n9.mid) to[bend right] (n3.mid) -- (n6.mid) -- cycle;
						\fill[fill=black!20,opacity=1] (n9.mid) -- (n2.mid) -- (n6.mid) -- cycle;
						\fill[fill=black!20,opacity=1] (n6.mid) -- (n0.mid) -- (n3.mid) -- cycle;
						
						\draw[thick] (n10.mid) -- (n7.mid);
						\draw[thick] (n9.mid) -- (n6.mid);
						\draw[thick] (n9.mid) -- (n5.mid);
						\draw[thick] (n9.mid) -- (n3.mid);
						\draw[thick] (n9.mid) -- (n2.mid);
						\draw[thick] (n9.mid) -- (n1.mid);
						\draw[thick] (n8.mid) -- (n4.mid);
						\draw[thick] (n6.mid) -- (n3.mid);
						\draw[thick] (n6.mid) -- (n2.mid);
						\draw[thick] (n6.mid) -- (n0.mid);
						\draw[thick] (n5.mid) -- (n1.mid);
						\draw[thick] (n3.mid) -- (n0.mid);

						\end{pgfonlayer}
					\end{tikzpicture}}	
				\end{figure}	
				\begin{figure}[H]
					\resizebox{1.0\textwidth}{!}{\begin{tikzpicture}[
						roundnode/.style={circle, draw=black, thick, fill=white, minimum size=7mm},
						]
						%Nodes
						\node           (label) 	at	(1,4) 		{Filtración 10 (t=0.0)};
						\node[roundnode]      (n0) 	at 	(4,0)           {0};
						\node[roundnode]      (n1) 	at 	(2,2)           {1};
						\node[roundnode]      (n2) 	at 	(2,0)           {2};
						\node[roundnode]      (n3)      at 	(2,-2)		{3};
						\node[roundnode]      (n4)      at 	(0,3)		{4};
						\node[roundnode]      (n5)      at 	(0,1)		{5};
						\node[roundnode]      (n6) 	at 	(0,-1)           {6};
						\node[roundnode]      (n7) 	at 	(0,-3)           {7};
						\node[roundnode]      (n8) 	at 	(-2,2)           {8};
						\node[roundnode]      (n9) 	at 	(-2,0)           {9};
						\node[roundnode]      (n10) 	at 	(-2,-2)           {10};
						\node 		(label2) 	at 	(1,-4) 		{$\beta_{0}=1,\beta_{1}=0$};
						
						\begin{pgfonlayer}{background}							
						\fill[fill=black!20,opacity=1] (n10.mid) -- (n7.mid) -- (n3.mid) -- cycle;
						\fill[fill=black!20,opacity=1] (n10.mid) -- (n6.mid) -- (n3.mid) -- cycle;
						\fill[fill=black!20,opacity=1] (n10.mid) to[bend left] (n2.mid) -- (n6.mid) -- cycle;
						\fill[fill=black!20,opacity=1] (n9.mid) to[bend left] (n1.mid) -- (n5.mid) -- cycle;
						\fill[fill=black!20,opacity=1] (n9.mid) to[bend right] (n3.mid) -- (n6.mid) -- cycle;
						\fill[fill=black!20,opacity=1] (n9.mid) -- (n2.mid) -- (n6.mid) -- cycle;
						\fill[fill=black!20,opacity=1] (n9.mid) -- (n2.mid) -- (n5.mid) -- cycle;
						\fill[fill=black!20,opacity=1] (n8.mid) -- (n1.mid) -- (n5.mid) -- cycle;
						\fill[fill=black!20,opacity=1] (n8.mid) to[bend right] (n2.mid) -- (n5.mid) -- cycle;
						\fill[fill=black!20,opacity=1] (n8.mid) -- (n1.mid) -- (n4.mid) -- cycle;
						\fill[fill=black!20,opacity=1] (n7.mid) to[bend right] (n0.mid) -- (n3.mid) -- cycle;
						\fill[fill=black!20,opacity=1] (n6.mid) -- (n0.mid) -- (n3.mid) -- cycle;
						\fill[fill=black!20,opacity=1] (n6.mid) -- (n0.mid) -- (n2.mid) -- cycle;
						\fill[fill=black!20,opacity=1] (n5.mid) -- (n0.mid) -- (n1.mid) -- cycle;
						\fill[fill=black!20,opacity=1] (n5.mid) -- (n0.mid) -- (n2.mid) -- cycle;
						\fill[fill=black!20,opacity=1] (n4.mid) to[bend left] (n0.mid) -- (n1.mid) -- cycle;
						
						\draw[thick] (n10.mid) -- (n7.mid);
						\draw[thick] (n10.mid) -- (n6.mid);
						\draw[thick] (n10.mid) -- (n3.mid);
						\draw[thick] (n10.mid) -- (n0.mid);
						\draw[thick] (n9.mid) -- (n6.mid);
						\draw[thick] (n9.mid) -- (n5.mid);
						\draw[thick] (n9.mid) -- (n2.mid);
						\draw[thick] (n8.mid) -- (n5.mid);
						\draw[thick] (n8.mid) -- (n4.mid);
						\draw[thick] (n8.mid) -- (n1.mid);
						\draw[thick] (n8.mid) -- (n0.mid);
						\draw[thick] (n7.mid) -- (n3.mid);
						\draw[thick] (n6.mid) -- (n3.mid);
						\draw[thick] (n6.mid) -- (n2.mid);
						\draw[thick] (n6.mid) -- (n0.mid);
						\draw[thick] (n5.mid) -- (n2.mid);
						\draw[thick] (n5.mid) -- (n1.mid);
						\draw[thick] (n5.mid) -- (n0.mid);
						\draw[thick] (n4.mid) -- (n1.mid);
						\draw[thick] (n3.mid) -- (n0.mid);
						\draw[thick] (n2.mid) -- (n0.mid);
						\draw[thick] (n1.mid) -- (n0.mid);
						\end{pgfonlayer}
					\end{tikzpicture}}
				\end{figure}	
			\endminipage}
				\caption{Pasos en la filtración del 
					complejo simplicial asociado a la red 
					y números de Betti correspondientes.}
				\label{fig:CompSimpLocal}
		\end{figure}
		Para una mayor claridad, se ha omitido el dibujo de algunos 
		símplices. Al igual que antes, damos una lista completa de los 
		complejos simpliciales de cada filtración:
		\begin{itemize}
			
			\item Filtración 0 (t=1.0):
				\begin{itemize}
					\item {\Large $\nu$}$^{1.0}_{0}=\{\{10\},\{9\},\{8\},\{7\},\{6\},\{5\},\{4\},\{3\},\{2\},\{1\},\{0\}\}$
					\item {\Large $\nu$}$^{1.0}_{1}=\{\{10, 7\},\{8, 4\}\}$
				\end{itemize}
				A partir de ahora omitiremos 
				{\Large $\nu$}$_{0}$ pues se mantiene igual 
				para todos los pasos en la filtración.
			\item Filtración 3 (t=0.7):
				\begin{itemize}
					\item {\Large $\nu$}$^{0.7}_{1}=\{\{10, 7\},\{9, 6\},\{9, 5\},\{8, 4\},\{6, 3\},\{6, 2\},\{5, 1\}\}$
				\end{itemize}
			\item Filtración 4 (t=0.6):
				\begin{itemize}
					\item {\Large $\nu$}$^{0.6}_{1}=\{\{10, 7\},\{9, 6\},\{9, 5\},\{9, 1\},\{8, 4\},\{6, 3\},\{6, 2\},\{5, 1\},\{3, 0\}\}$
					\item {\Large $\nu$}$^{0.6}_{2}=\{\{9, 5, 1\}\}$
				\end{itemize}
			\item Filtración 6 (t=0.4):
				\begin{itemize}
					\item {\Large $\nu$}$^{0.4}_{1}=\{\{10, 7\},\{9, 6\},\{9, 5\},\{9, 3\},\{9, 2\},\{9, 1\},\{8, 4\},\{6, 3\},\{6, 2\},
						\{6, 0\},$ \\$\{5, 1\},\{3, 0\}\}$
					\item {\Large $\nu$}$^{0.4}_{2}=\{\{9, 6, 3\},\{9, 6, 2\},\{9, 5, 1\},\{6, 3, 0\}\}$
				\end{itemize}
			\item Filtración 7 (t=0.3):
				\begin{itemize}
					\item {\Large $\nu$}$^{0.3}_{1}=\{\{10, 7\},\{10, 6\},\{10, 3\},\{9, 6\},\{9, 5\},\{9, 3\},\{9, 2\},\{9, 1\},\{8, 4\},$\\ 
							$\{7, 3\},\{6, 3\},\{6, 2\},\{6, 0\},\{5, 1\},\{3, 0\},\{1, 0\}\}$
					\item {\Large $\nu$}$^{0.3}_{2}=\{\{10, 7, 3\},\{9, 6, 3\},\{9, 6, 2\},\{9, 5, 1\},\{6, 3, 0\}\}$
				\end{itemize}
			\item Filtración 8 (t=0.2):
				\begin{itemize}
					\item {\Large $\nu$}$^{0.2}_{1}=\{\{10, 7\},\{10, 6\},\{10, 3\},\{10, 2\},\{9, 6\},\{9, 5\},\{9, 3\},\{9, 2\},\{9, 1\},$\\
							$\{9, 0\},\{8, 5\},\{8, 4\},\{8, 1\},\{7, 3\},\{6, 3\},\{6, 2\},\{6, 0\},\{5, 2\},\{5, 1\},\{5, 0\},$\\
							$\{4, 1\},\{3, 0\},\{1, 0\}\}$
					\item {\Large $\nu$}$^{0.2}_{2}=\{\{10, 7, 3\},\{10, 6, 3\},\{10, 6, 2\},\{9, 6, 3\},\{9, 6, 2\},\{9, 6, 0\},
							\{9, 5, 1\},$\\$\{9, 3, 0\},\{8, 4, 1\},\{6, 3, 0\},\{5, 1, 0\}\}$
					\item {\Large $\nu$}$^{0.2}_{3}=\{\{9, 6, 3, 0\}\}$
				\end{itemize}
			\item Filtración 9 (t=0.1):
				\begin{itemize}
					\item {\Large $\nu$}$^{0.1}_{1}=\{\{10, 7\},\{10, 6\},\{10, 3\},\{10, 2\},\{10, 0\},\{9, 6\},\{9, 5\},\{9, 3\},\{9, 2\},$\\
							$\{9, 1\},\{9, 0\},\{8, 5\},\{8, 4\},\{8, 1\},\{7, 3\},\{7, 0\},\{6, 3\},\{6, 2\},\{6, 0\},\{5, 2\},$\\
						$\{5, 1\},\{5, 0\},\{4, 1\},\{3, 0\},\{2, 0\},\{1, 0\}\}$
					\item {\Large $\nu$}$^{0.1}_{2}=\{\{10, 7, 3\},\{10, 7, 0\},\{10, 6, 3\},\{10, 6, 2\},\{10, 6, 0\},\{10, 3, 0\},\{9, 6, 3\},$
							\\$\{9, 6, 2\},\{9, 6, 0\},\{9, 5, 2\},\{9, 5, 1\},\{9, 5, 0\},\{9, 3, 0\},\{9, 1, 0\},\{8, 5, 1\},$
						\\$\{8, 4, 1\},\{7, 3, 0\},\{6, 3, 0\},\{5, 1, 0\}\}$
					\item {\Large $\nu$}$^{0.1}_{3}=\{\{10, 7, 3, 0\},\{10, 6, 3, 0\},\{9, 6, 3, 0\},\{9, 5, 1, 0\}\}$
				\end{itemize}
			\item Filtración 10 (t=0.0):
				\begin{itemize}
					\item {\Large $\nu$}$^{0.0}_{1}=\{\{10, 7\},\{10, 6\},\{10, 3\},\{10, 2\},\{10, 0\},\{9, 6\},\{9, 5\},\{9, 3\},\{9, 2\},$
							\\$\{9, 1\},\{9, 0\},\{8, 5\},\{8, 4\},\{8, 2\},\{8, 1\},\{8, 0\},\{7, 3\},\{7, 0\},\{6, 3\},\{6, 2\},$
						\\$\{6, 0\},\{5, 2\},\{5, 1\},\{5, 0\},\{4, 1\},\{4, 0\},\{3, 0\},\{2, 0\},\{1, 0\}\}$
					\item {\Large $\nu$}$^{0.0}_{2}=\{\{10, 7, 3\},\{10, 7, 0\},\{10, 6, 3\},\{10, 6, 2\},\{10, 6, 0\},\{10, 3, 0\},$
						\\$\{10, 2, 0\},\{9, 6, 3\},\{9, 6, 2\},\{9, 6, 0\},\{9, 5, 2\},\{9, 5, 1\},\{9, 5, 0\},\{9, 3, 0\},$
					\\$\{9, 2, 0\},\{9, 1, 0\},\{8, 5, 2\},\{8, 5, 1\},\{8, 5, 0\},\{8, 4, 1\},\{8, 4, 0\},\{8, 2, 0\},$
				\\$\{8, 1, 0\},\{7, 3, 0\},\{6, 3, 0\},\{6, 2, 0\},\{5, 2, 0\},\{5, 1, 0\},\{4, 1, 0\}\}$
					\item {\Large $\nu$}$^{0.0}_{3}=\{\{10, 7, 3, 0\},\{10, 6, 3, 0\},\{10, 6, 2, 0\},\{9, 6, 3, 0\},\{9, 6, 2, 0\},\{9, 5, 2, 0\},$
						\\$\{9, 5, 1, 0\},\{8, 5, 2, 0\},\{8, 5, 1, 0\},\{8, 4, 1, 0\}\}$
				\end{itemize}
		\end{itemize}

	Finalmente, podemos ver los diagramas de barras y persistencia 
	asociados al complejo simplicial filtrado representado en 
	la \autoref{fig:CompSimpLocal} en la \autoref{fig:ejemLocalDiag}.
	
	\begin{figure}[!htbp]
		\minipage{0.5\textwidth}
			\begin{figure}[H]
				\resizebox{1.0\textwidth}{!}{\includegraphics{Images/BarrasLocalEjAutor.png}}
			\end{figure}
		\endminipage
		\minipage{0.5\textwidth}
			\begin{figure}[H]
				\resizebox{1.0\textwidth}{!}{\includegraphics{Images/PerisistenciaLocalEjAutor.png}}
			\end{figure}
		\endminipage
		\caption{Diagramas de barras y persistencia asociados a la red 
		\ref{fig:ejemGlobal}.}
		\label{fig:ejemLocalDiag}
	\end{figure}
	
	Nótese que en esta interpretación, a diferencia de la anterior, 
	tenemos agujeros $p$-dimensionales ocultos, como el que aparece en 
	la \autoref{fig:CompSimpLocal} en la filtración 9.

	Observamos que, en este caso, los diagramas obtenidos son muy 
	diferentes a los de la sección anterior. Si bien obtenemos las mismas 
	clases de equivalencia de la homología de grado 0 (componentes 
	conexas), vemos que hemos obtenido un resultado muy distinto para las
	clases de equivalencia de la homología de grado 1 (agujeros 
	$2$-dimensionales). En este caso hemos obtenido 5 clases de 
	equivalencia mientras que en el caso anterior sólo teníamos 1. El 
	hecho de que tengamos las mismas clases de equivalencia de la 
	homología de grado 0 se debe a que, como ya hemos comentado, el 
	cálculo de las aristas en esta interpretación se realiza siguiendo el
	razonamiento global, lo que resulta un tanto incoherente con el 
	cálculo del resto de los $p$-símplices. 

	Cabe reseñar que este ejemplo reproduce fielmente el propuesto por el 
	autor en \cite{Articulo-Watanabe}, aunque habiendo solucionado alguna 
	errata presente en los dibujos del autor.
	\end{ejem}

	Del ejemplo anterior el autor extrae unas conclusiones muy 
	importantes. Por una parte, observa que si las neuronas de entrada se 
	conectan directamente a las de salida, el conocimiento de la red será 
	<<pobre>> ya que será equivalente a la detección de patrones. Por otra 
	parte, el incremento del número de Betti $\beta_{1}$ indica que la red 
	determina la neurona de llegada por combinación de las neuronas de 
	salida. De este modo, supone que el aumento de $\beta_{1}$ refleja la 
	complejidad del conocimiento adquirido por la red. Por lo tanto, 
	mediante la homología persistente es posible medir la complejidad del 
	conocimiento adquirido por la red.

	Este supuesto puede tener especial sentido en el caso de las redes 
	neuronales convolucionales (CNN). Aunque una descripción detallada de 
	las mismas queda fuera del alcance del presente trabajo, en estas 
	redes la información se presenta de manera jerarquizada, es decir, las
	neuronas de las capas iniciales detectan patrones simples como líneas
	horizontales y verticales, y a medida que se avanza en las capas, 
	estos patrones se combinan para representar objetos más complejos, por 
	ejemplo una rueda de un coche. Estas redes son muy utilizadas en el 
	ámbito de la visión por computador.

	Habiendo presentado detalladamente el ejemplo que propone el autor, 
	vamos a describir todos los algoritmos y programas que hemos empleado
	para el cálculo automático de los complejos simpliciales asociados, 
	los números de Betti y los diagramas correspondientes.

	Para el desarrollo de los algoritmos y programas se ha empleado el 
	lenguaje \emph{Python} (\cite{10.5555/1593511}) y el programa 
	\emph{Neovim} como entorno de desarrollo. Hemos elegido Python como 
	lenguaje de programación ya que es un lenguaje de alto nivel (lo que 
	lo hace bastante sencillo) y el código producido es de fácil lectura. 
	Este lenguaje es de código abierto y es muy empleado en el campo de la 
	inteligencia artificial. No obstante, este lenguaje tiene el 
	inconveniente de ser poco eficiente ya que se trata de un lenguaje 
	interpretado. Para solucionarlo, hemos empleado librerías que ofrecen 
	una interfaz en Python pero que están programadas en lenguajes más 
	eficientes como C++. Estas librerías son \emph{GUDHI} 
	(\cite{gudhi:urm}) para el cálculo de los diagramas, y \emph{Dionysus} 
	(\cite{morozov_2021}) para el cálculo de los números de Betti.

	\begin{remark}
	Todo el código desarrollado para este trabajo se encuentra disponible 
	en: \url{https://github.com/joros244/TFGMath2022.git}
	\end{remark}

	En cuanto a los algoritmos empleados para el cálculo de los complejos
	simpliciales, nos hemos basado en los algoritmos propuestos en 
	\cite{Algoritmos-Sedgewick} y los hemos adaptado a nuestras 
	necesidades particulares. Veamos tales algoritmos en detalle

	En primer lugar, se define un objeto grafo dirigido acíclico (GDA). 
	Este objeto representa un grafo dirigido acíclico mediante una lista 
	de adyacencia. Tras esto, se define un objeto grafo dirigido acíclico 
	con pesos (GDAP). Este objeto consta de un objeto grafo dirigido 
	acíclico y de una matriz de pesos asociada, esta matriz es la original 
	dada por la red neuronal y define completamente las aristas del objeto 
	grafo asociado. La definición del objeto la podemos ver en 
	el \autoref{cod:DefGDAP}.

	\begin{lstlisting}[language=Python, float=htbp, label=cod:DefGDAP, caption= 
	Definición del objeto GDAP.]
	#Un GDAP es un GDA con una matriz asociada. La matriz determina las 
	#aristas y los vértices del GDA asociado.
    	
	def __init__(self, M):
        self.G = GDA.Grafo_dirigido_aciclico(len(M))
        self.M = M
        for i in range(0,len(self.M)):
            for j in range(i):
                if( self.M[i][j] > 0 ):
                    self.G.crea_arista(i,j)
		    
	\end{lstlisting}

	Ahora debemos distinguir entre los algoritmos empleados para el 
	cálculo local y global. Empezaremos con éste último.

	Para el cálculo global hemos empleado la función que podemos ver en 
	el \autoref{cod:DefSMG}.

	\begin{lstlisting}[language=Python,float=htbp,caption=Función para el 
	cálculo de símplices maximales (global)., label=cod:DefSMG]
	#Devuelve una lista de los símplices maximales de cada vértice si al 
	#menos uno de ellos pasa el filtro t. Recorre la lista de vértices 
	#adyacentes a uno dado y en caso de que no haya adyacencia directa, 
	#calcula los caminos indirectos y los añade. Si la adyacencia directa 
	#pasa el filtro, la añade.
	def __calcula_simplices_global(self,t)->List[List[int]]:
        simp=[]
        #Necesario para recorrer los adyacentes. OPTIMIZACIÓN
        P=self.filtracion(t)
        T=P.clausura_transitiva()
        for i in range(len(self.M)):
            for j in T.G.adj[i]:
                if(self.M[i][j]==0):
                    p=[]
                    self.busca_pesos(i,j,p)
                    if(len(p)>0):
                        m=max([v[0] for v in p])
                        if(m>=t):
                            l = [k[1] for k in p]
                            simp.extend(l)
                            simp.append([i,j])
                elif (self.M[i][j]>=t):
                    simp.append([i,j])
        return simp
	\end{lstlisting}

	Aquí la función <<filtración>> es común a ambas 
	interpretaciones y su funcionamiento es el siguiente: recorre la 
	matriz asociada al grafo, si la entrada $a_{ij}$ supera el umbral $t$ 
	pasa a la siguiente, en caso contrario la sustituye por 0; tras esto 
	devuelve el GDAP asociado a la nueva matriz. La función 
	<<clausura transitiva>> calcula la clausura transitiva de un GDA y 
	la devuelve. La función <<busca pesos>>, común a ambas 
	interpretaciones, es especialmente importante ya 
	que hace una búsqueda en profundidad sobre la matriz para calcular el
	peso apropiado. Podemos ver su definición en el \autoref{cod:DefBP}.

	\begin{lstlisting}[language=Python, float=htbp, caption= Función de 
	búsqueda de pesos., label=cod:DefBP]
	#Almacena en p una lista de tuplas donde el primer argumento es el 
	#posible peso entre origen(o) y destino(d); y el segundo argumento es 
	#el camino que produce dicho peso. 
	def busca_pesos(self, o, d, p=[], q=1, c=[]):
        #BÚSQUEDA EN PROFUNDIDAD
        if(o==d):
            c.append(o)
            p.append((q.__round__(4),c))
            q=1
            c=[]
        else:   
            for j in range(d,o):
                if(self.M[o][j]!=0):
        #NOTA: Podríamos optimizarlo para que si q<t pase 
		    #al siguente vértice. SÓLO VÁLIDO PARA V.LOCAL
                    self.busca_pesos(j,d,p,q*self.M[o][j],c+[o])
	\end{lstlisting}

	Notemos que este algoritmo es el clave dentro de toda esta 
	implementación, y además, es recursivo, lo que lo hace bastante 
	ineficiente. Esta observación nos lleva a 
	cuestionar la aplicabilidad de estos algoritmos a casos reales en los 
	que el número de neuronas de la red es muy grande. No obstante, para 
	el presente trabajo nos ha resultado suficiente con esta 
	implementación.

	Finalizamos la implementación de la interpretación global con una 
	última función que nos devuelve la lista de los símplices del complejo
	simplicial. Podemos ver esta función en el \autoref{cod:DefSG}.

	\begin{lstlisting}[language=Python, float=htbp, caption= Función para 
	el cálculo de todos los símplices (global)., label=cod:DefSG]
	#Devuelve un lista con el complejo simplicial asociado al GDAP. 
	#V.GLOBAL
	def get_simplex_global(self, t : float)->List[List[int]]:
        simplices=self.__calcula_simplices_global(t)
        
        #En este caso hay que añadir los vértices
        simplices.extend([[i] for i in range(len(self.M))])
        
        #Ahora añadimos los subsímplices de los calculados
        for s in simplices:
            for L in range(3,len(s)+1):
                for sub in it.combinations(s,L):
                    if(list(sub) not in simplices):
                       simplices.append(list(sub))
        
        #Limpiamos aquellos símplices con aristas irreales
        tr=[s for s in simplices if len(s)>2]
        for e in tr:
            if(any([list(sub) not in simplices for sub in 
                it.combinations(e,len(e)-1)])):
                simplices.remove(e)

        return simplices
	\end{lstlisting}

	Veamos ahora la implementación de los algoritmos asociados a la 
	interpretación local. Como ya hemos descrito los algoritmos comunes a 
	ambas interpretaciones, en este caso sólo son necesarias dos 
	funciones. La primera de estas funciones calcula los símplices 
	maximales y su definición la podemos ver en el \autoref{cod:DefSML}. 
	La 
	definición de la segunda de estas funciones, responsable del cálculo 
	de la lista de los símplices, puede verse en el \autoref{cod:DefSL}.

	\begin{lstlisting}[language=Python, float=htbp, caption=Función para 
	el cálculo de símplices maximales (local)., label=cod:DefSML]
	#Devuelve una lista de los símplices maximales de cada vértice que 
	#pasan el filtro t. Lo que hace es recorrer la lista de vértices 
	#adyacentes y en caso de que no haya adyacencia directa, calcula los 
	#caminos indirectos.
	def __calcula_simplices(self,t)->List[List[int]]:
        simp=[]
        #Necesario para recorrer los adyacentes. OPTIMIZACIÓN
        P=self.filtracion(t)
        T=P.clausura_transitiva()
        for i in range(len(self.M)):
            for j in T.G.adj[i]:
                if(self.M[i][j]==0):
                    p=[]
                    self.busca_pesos(i,j,p)
                    if(len(p)>0):
                        l = [k[1] for k in p if k[0]>=t]
                        simp.extend(l)
                elif (self.M[i][j]>=t):
                    simp.append([i,j])
        return simp
	\end{lstlisting}

	\begin{lstlisting}[language=Python,float=htbp, caption= Función para 
	el cálculo de todos los símplices (local)., label=cod:DefSL]
	#Devuelve un lista con el complejo simplicial asociado al GDAP. 
	#V.LOCAL
	def get_simplex_local(self, t : float)->List[List[int]]:
        simplices=self.__calcula_simplices(t)
        
        simplices.extend([[i] for i in range(len(self.M))])
        
        #Hemos calculado símplices "maximales", nos faltan sus 
        #subsímplices
        for s in simplices:
            for L in range(2,len(s)+1):
                for sub in it.combinations(s,L):
                    if(list(sub) not in simplices):
                       simplices.append(list(sub))
        return simplices
	\end{lstlisting}

	Como ya hemos comentado, los algoritmos aquí expuestos son bastante 
	ineficientes al ser recursivos. Aunque hemos tratado de optimizarlos 
	en la medida de lo posible, no hemos conseguido una mejora 
	significativa en el tiempo de ejecución. Aunque para los ejemplos del 
	presente trabajo nos ha bastado con esta implementación, se queda como
	trabajo futuro la optimización de la presente implementación.
	
	\section{Ejemplos}
	
	En las secciones anteriores hemos reproducido fielmente los
	resultados que presenta el autor en \cite{Articulo-Watanabe}. En esta 
	sección vamos a proponer una serie de ejemplos para observar más 
	detalladamente el funcionamiento de la homología persistente en casos 
	concretos. 

	Estos ejemplos van a estar centrados en comprobar cómo responde la 
	homología persistente ante un cambio en la distribución de los pesos 
	de la red. Es decir, vamos a ver un ejemplo con una distribución 
	homogénea de los pesos, un ejemplo con una distribución extrema, y un 
	ejemplo con una distribución intermedia. El objetivo de estos ejemplos 
	es desarrollar una mejor intuición en el uso de la herramienta que 
	hemos presentado en este trabajo, así como validar las hipótesis que
	se presentan en \cite{Articulo-Watanabe}. Adicionalmente, para un 
	mayor interés de estos ejemplos, vamos a incluir los cálculos para 
	las dos interpretaciones que hemos visto en este capítulo. 

	Para que la comparativa entre las distribuciones tenga sentido vamos 
	a considerar fija la arquitectura de la red en los tres ejemplos. La 
	arquitectura escogida puede verse en la \autoref{fig:arqEjs}.

	\begin{figure}[!htbp]
		\centering
		\begin{tikzpicture}[
			roundnode/.style={circle, draw=black, thick, fill=white, minimum size=7mm},
			]
			%Nodes
			\node[roundnode]      (n0) 	at 	(4,1)           {0};
			\node[roundnode]      (n1) 	at 	(4,-1)          {1};
			\node[roundnode]      (n2) 	at 	(2,2)           {2};
			\node[roundnode]      (n3) 	at 	(2,0)           {3};
			\node[roundnode]      (n4)      at 	(2,-2)		{4};
			\node[roundnode]      (n5)      at 	(0,3)		{5};
			\node[roundnode]      (n6)      at 	(0,1)		{6};
			\node[roundnode]      (n7) 	at 	(0,-1)           {7};
			\node[roundnode]      (n8) 	at 	(0,-3)           {8};
			\node[roundnode]      (n9) 	at 	(-2,2)           {9};
			\node[roundnode]      (n10) 	at 	(-2,0)           {10};
			\node[roundnode]      (n11) 	at 	(-2,-2)           {11};
			
			\begin{pgfonlayer}{background}
			%Lines
			\draw[thick,->] (n9.mid) -- (n5.west);
			\draw[thick,->] (n9.mid) -- (n6.west);
			\draw[thick,->] (n10.mid) --  (n6.west);
			\draw[thick,->] (n10.mid) --  (n7.west);
			\draw[thick,->] (n11.mid) --  (n7.west);
			\draw[thick,->] (n11.mid) --  (n8.west);
			\draw[thick,->] (n5.mid) -- (n2.west);
			\draw[thick,->] (n6.mid) -- (n2.west);
			\draw[thick,->] (n6.mid) --  (n3.west);
			\draw[thick,->] (n7.mid) --  (n3.west);
			\draw[thick,->] (n7.mid) --  (n4.west);
			\draw[thick,->] (n8.mid) --  (n4.west);
			\draw[thick,->] (n2.mid) --  (n0.west);
			\draw[thick,->] (n3.mid) --  (n0.west);
			\draw[thick,->] (n3.mid) --  (n1.west);
			\draw[thick,->] (n4.mid) --  (n1.west);
			\end{pgfonlayer}
		\end{tikzpicture}
			\caption{Representación de una red neuronal de 12 
			neuronas y 2 capas.}
			\label{fig:arqEjs}
	\end{figure}
	
	Aunque una arquitectura de mayor tamaño pudiera resultar más 
	enriquecedora en 
	cuanto a las conclusiones obtenidas de su estudio, motivados por la 
	preocupación por la eficiencia de los algoritmos presentados en la 
	sección anterior, hemos decidido estudiar esta arquitectura más 
	reducida. No obstante, esta decisión no es demasiado restrictiva ya 
	que los resultados que obtengamos podrán extrapolarse a redes más 
	grandes. Para una mayor legibilidad de los ejemplos hemos decidido 
	omitir la lista de los símplices de cada filtración que incluíamos en 
	ejemplos anteriores. Para acceder a las listas de los símplices véase 
	\url{https://github.com/joros244/TFGMath2022.git}.

	Vamos a comenzar con el ejemplo que presenta una distribución 
	homogénea de los pesos.

	\begin{ejem}\textbf{Distribución homogénea.}
	En un primer lugar, supondremos que la red estudiada posee una 
	distribución ciertamente homogénea de los pesos. Esta distribución da 
	lugar la representación de la red, con las importancias entre neuronas
	ya calculadas, que puede verse en la \autoref{fig:distHomR}.
	
	\begin{figure}[!htbp]
			\centering
			\begin{tikzpicture}[
				roundnode/.style={circle, draw=black, thick, fill=white, minimum size=7mm},
				]
				%Nodes
				\node[roundnode]      (n0) 	at 	(4,1)           {0};
				\node[roundnode]      (n1) 	at 	(4,-1)          {1};
				\node[roundnode]      (n2) 	at 	(2,2)           {2};
				\node[roundnode]      (n3) 	at 	(2,0)           {3};
				\node[roundnode]      (n4)      at 	(2,-2)		{4};
				\node[roundnode]      (n5)      at 	(0,3)		{5};
				\node[roundnode]      (n6)      at 	(0,1)		{6};
				\node[roundnode]      (n7) 	at 	(0,-1)           {7};
				\node[roundnode]      (n8) 	at 	(0,-3)           {8};
				\node[roundnode]      (n9) 	at 	(-2,2)           {9};
				\node[roundnode]      (n10) 	at 	(-2,0)           {10};
				\node[roundnode]      (n11) 	at 	(-2,-2)           {11};
				
				\begin{pgfonlayer}{background}
				%Lines
				\draw[thick,->] (n9.mid) -- node[above,sloped] {1.0} (n5.west);
				\draw[thick,->] (n9.mid) -- node[above,sloped] {0.5} (n6.west);
				\draw[thick,->] (n10.mid) -- node[above,sloped] {0.5} (n6.west);
				\draw[thick,->] (n10.mid) -- node[above,sloped] {0.55} (n7.west);
				\draw[thick,->] (n11.mid) -- node[above,sloped] {0.45} (n7.west);
				\draw[thick,->] (n11.mid) -- node[above,sloped] {1.0} (n8.west);
				\draw[thick,->] (n5.mid) -- node[above,sloped] {0.4} (n2.west);
				\draw[thick,->] (n6.mid) -- node[above,sloped] {0.6} (n2.west);
				\draw[thick,->] (n6.mid) -- node[above,sloped] {0.58} (n3.west);
				\draw[thick,->] (n7.mid) -- node[above,sloped] {0.42} (n3.west);
				\draw[thick,->] (n7.mid) -- node[above,sloped] {0.56} (n4.west);
				\draw[thick,->] (n8.mid) -- node[above,sloped] {0.44} (n4.west);
				\draw[thick,->] (n2.mid) -- node[above,sloped] {0.57} (n0.west);
				\draw[thick,->] (n3.mid) -- node[above,sloped] {0.43} (n0.west);
				\draw[thick,->] (n3.mid) -- node[above,sloped] {0.59} (n1.west);
				\draw[thick,->] (n4.mid) -- node[above,sloped] {0.41} (n1.west);
				\end{pgfonlayer}
			\end{tikzpicture}
			\caption{Representación de la red \ref{fig:arqEjs} con 
			una distribución homogénea de las importancias entre 
			neuronas.}
			\label{fig:distHomR}
	\end{figure}

	Procedemos ahora con el cálculo del complejo simplicial asociado. Al 
	igual que en los ejemplos que hemos presentado anteriormente, vamos a 
	ilustrar unos cuantos pasos en la filtración con los correspondientes 
	números de Betti asociados. Esta ilustración puede verse en 
	la \autoref{fig:distHomRSL} y en la \autoref{fig:distHomRSG}. También 
	añadimos los diagramas de 
	barras y de persistencia que pueden verse en la 
	\autoref{fig:distHomRDL} 
	y en la \autoref{fig:distHomRDG}.

	\begin{figure}[!htbp]
		\fbox{\minipage{0.225\textwidth}
			\begin{figure}[H]
				\resizebox{1.0\textwidth}{!}{\begin{tikzpicture}[
					roundnode/.style={circle, draw=black, thick, fill=white, minimum size=7mm},
					]
					%Nodes
					\node           (label) 	at	(1,4) 		{Filtración 0 (t=1.0)};
					\node[roundnode]      (n0) 	at 	(4,1)           {0};
					\node[roundnode]      (n1) 	at 	(4,-1)          {1};
					\node[roundnode]      (n2) 	at 	(2,2)           {2};
					\node[roundnode]      (n3) 	at 	(2,0)           {3};
					\node[roundnode]      (n4)      at 	(2,-2)		{4};
					\node[roundnode]      (n5)      at 	(0,3)		{5};
					\node[roundnode]      (n6)      at 	(0,1)		{6};
					\node[roundnode]      (n7) 	at 	(0,-1)           {7};
					\node[roundnode]      (n8) 	at 	(0,-3)           {8};
					\node[roundnode]      (n9) 	at 	(-2,2)           {9};
					\node[roundnode]      (n10) 	at 	(-2,0)           {10};
					\node[roundnode]      (n11) 	at 	(-2,-2)           {11};
					\node 		(label2) 	at 	(1,-4) 		{$\beta_{0}=10,\beta_{1}=0$};
					
					\begin{pgfonlayer}{background}
					%Lines
					\draw[thick] (n11.mid) --  (n8);
					\draw[thick] (n9.mid) --  (n5);
					\end{pgfonlayer}
				\end{tikzpicture}}	
			\end{figure}	
			\begin{figure}[H]
				\resizebox{1.0\textwidth}{!}{\begin{tikzpicture}[
					roundnode/.style={circle, draw=black, thick, fill=white, minimum size=7mm},
					]
					%Nodes
					\node           (label) 	at	(1,4) 		{Filtración 7 (t=0.3)};
					\node[roundnode]      (n0) 	at 	(4,1)           {0};
					\node[roundnode]      (n1) 	at 	(4,-1)          {1};
					\node[roundnode]      (n2) 	at 	(2,2)           {2};
					\node[roundnode]      (n3) 	at 	(2,0)           {3};
					\node[roundnode]      (n4)      at 	(2,-2)		{4};
					\node[roundnode]      (n5)      at 	(0,3)		{5};
					\node[roundnode]      (n6)      at 	(0,1)		{6};
					\node[roundnode]      (n7) 	at 	(0,-1)           {7};
					\node[roundnode]      (n8) 	at 	(0,-3)           {8};
					\node[roundnode]      (n9) 	at 	(-2,2)           {9};
					\node[roundnode]      (n10) 	at 	(-2,0)           {10};
					\node[roundnode]      (n11) 	at 	(-2,-2)           {11};
					\node 		(label2) 	at 	(1,-4) 		{$\beta_{0}=1,\beta_{1}=4$};
					
					\begin{pgfonlayer}{background}							
				

					\fill[fill=black!20,opacity=1] (n11.mid) -- (n8.mid) -- (n4.mid) -- cycle;
					\fill[fill=black!20,opacity=1] (n10.mid) to[bend right] (n4.mid) -- (n7.mid) -- cycle;
					\fill[fill=black!20,opacity=1] (n10.mid) to[bend left] (n2.mid) -- (n6.mid) -- cycle;
					\fill[fill=black!20,opacity=1] (n9.mid) -- (n6.mid) -- (n2.mid) -- cycle;
					\fill[fill=black!20,opacity=1] (n9.mid) -- (n5.mid) -- (n2.mid) -- cycle;
					\fill[fill=black!20,opacity=1] (n6.mid) to[bend left] (n1.mid) -- (n3.mid) -- cycle;
					\fill[fill=black!20,opacity=1] (n6.mid) -- (n2.mid) -- (n0.mid) -- cycle;
					
					\draw[thick] (n11.mid) --  (n8);
					\draw[thick] (n11.mid) --  (n7);
					\draw[thick] (n11.mid) --  (n4.mid);
					\draw[thick] (n10.mid) --  (n7);
					\draw[thick] (n10.mid) --  (n6);
					\draw[thick] (n9.mid) --  (n6);
					\draw[thick] (n9.mid) --  (n5);
					\draw[thick] (n9.mid) --  (n2.mid);
					\draw[thick] (n8.mid) --  (n4);
					\draw[thick] (n7.mid) --  (n4);
					\draw[thick] (n7.mid) --  (n3);
					\draw[thick] (n6.mid) --  (n3);
					\draw[thick] (n6.mid) --  (n2);
					\draw[thick] (n6.mid) --  (n0);
					\draw[thick] (n5.mid) --  (n2);
					\draw[thick] (n4.mid) --  (n1);
					\draw[thick] (n3.mid) --  (n1);
					\draw[thick] (n3.mid) --  (n0);
					\draw[thick] (n2.mid) --  (n0);
					\end{pgfonlayer}
				\end{tikzpicture}}	
			\end{figure}	
		\endminipage}
		\fbox{\minipage{0.225\textwidth}
			\begin{figure}[H]
				\resizebox{1.0\textwidth}{!}{\begin{tikzpicture}[
					roundnode/.style={circle, draw=black, thick, fill=white, minimum size=7mm},
					]
					%Nodes
					\node           (label) 	at	(1,4) 		{Filtración 4 (t=0.6)};
					\node[roundnode]      (n0) 	at 	(4,1)           {0};
					\node[roundnode]      (n1) 	at 	(4,-1)          {1};
					\node[roundnode]      (n2) 	at 	(2,2)           {2};
					\node[roundnode]      (n3) 	at 	(2,0)           {3};
					\node[roundnode]      (n4)      at 	(2,-2)		{4};
					\node[roundnode]      (n5)      at 	(0,3)		{5};
					\node[roundnode]      (n6)      at 	(0,1)		{6};
					\node[roundnode]      (n7) 	at 	(0,-1)           {7};
					\node[roundnode]      (n8) 	at 	(0,-3)           {8};
					\node[roundnode]      (n9) 	at 	(-2,2)           {9};
					\node[roundnode]      (n10) 	at 	(-2,0)           {10};
					\node[roundnode]      (n11) 	at 	(-2,-2)           {11};
					\node 		(label2) 	at 	(1,-4) 		{$\beta_{0}=9,\beta_{1}=0$};
					
					\begin{pgfonlayer}{background}
					%Lines
					\draw[thick] (n11.mid) --  (n8);
					\draw[thick] (n9.mid) --  (n5);
					\draw[thick] (n6.mid) --  (n2);
					\end{pgfonlayer}
				\end{tikzpicture}}	
			\end{figure}	
			\begin{figure}[H]
				\resizebox{1.0\textwidth}{!}{\begin{tikzpicture}[
					roundnode/.style={circle, draw=black, thick, fill=white, minimum size=7mm},
					]
					%Nodes
					\node           (label) 	at	(1,4) 		{Filtración 8 (t=0.2)};
					\node[roundnode]      (n0) 	at 	(4,1)           {0};
					\node[roundnode]      (n1) 	at 	(4,-1)          {1};
					\node[roundnode]      (n2) 	at 	(2,2)           {2};
					\node[roundnode]      (n3) 	at 	(2,0)           {3};
					\node[roundnode]      (n4)      at 	(2,-2)		{4};
					\node[roundnode]      (n5)      at 	(0,3)		{5};
					\node[roundnode]      (n6)      at 	(0,1)		{6};
					\node[roundnode]      (n7) 	at 	(0,-1)           {7};
					\node[roundnode]      (n8) 	at 	(0,-3)           {8};
					\node[roundnode]      (n9) 	at 	(-2,2)           {9};
					\node[roundnode]      (n10) 	at 	(-2,0)           {10};
					\node[roundnode]      (n11) 	at 	(-2,-2)           {11};
					\node 		(label2) 	at 	(1,-4) 		{$\beta_{0}=1,\beta_{1}=0$};
					
					\begin{pgfonlayer}{background}							
				

					\fill[fill=black!20,opacity=1] (n11.mid) -- (n8.mid) -- (n4.mid) -- cycle;
					\fill[fill=black!20,opacity=1] (n11.mid) -- (n7.mid) -- (n4.mid) -- cycle;
					\fill[fill=black!20,opacity=1] (n10.mid) to[bend right] (n4.mid) -- (n7.mid) -- cycle;
					\fill[fill=black!20,opacity=1] (n10.mid) -- (n7.mid) -- (n3.mid) -- cycle;
					\fill[fill=black!20,opacity=1] (n10.mid) -- (n6.mid) -- (n3.mid) -- cycle;
					\fill[fill=black!20,opacity=1] (n10.mid) to[bend left] (n2.mid) -- (n6.mid) -- cycle;
					\fill[fill=black!20,opacity=1] (n9.mid) to[bend right] (n3.mid) -- (n6.mid) -- cycle;
					\fill[fill=black!20,opacity=1] (n9.mid) -- (n6.mid) -- (n2.mid) -- cycle;
					\fill[fill=black!20,opacity=1] (n9.mid) -- (n5.mid) -- (n2.mid) -- cycle;
					\fill[fill=black!20,opacity=1] (n7.mid) -- (n4.mid) -- (n1.mid) -- cycle;
					\fill[fill=black!20,opacity=1] (n7.mid) -- (n3.mid) -- (n1.mid) -- cycle;
					\fill[fill=black!20,opacity=1] (n6.mid) to[bend left] (n1.mid) -- (n3.mid) -- cycle;
					\fill[fill=black!20,opacity=1] (n6.mid) -- (n3.mid) -- (n0.mid) -- cycle;
					\fill[fill=black!20,opacity=1] (n6.mid) -- (n2.mid) -- (n0.mid) -- cycle;
					\fill[fill=black!20,opacity=1] (n5.mid) to[bend left] (n0.mid) -- (n2.mid) -- cycle;
					
					\draw[thick] (n11.mid) --  (n8.mid);
					\draw[thick] (n11.mid) --  (n7.mid);
					\draw[thick] (n11.mid) --  (n4.mid);
					\draw[thick] (n10.mid) --  (n7.mid);
					\draw[thick] (n10.mid) --  (n6.mid);
					\draw[thick] (n10.mid) --  (n3.mid);
					\draw[thick] (n9.mid) --  (n6.mid);
					\draw[thick] (n9.mid) --  (n5.mid);
					\draw[thick] (n9.mid) --  (n2.mid);
					\draw[thick] (n8.mid) --  (n4.mid);
					\draw[thick] (n7.mid) --  (n4.mid);
					\draw[thick] (n7.mid) --  (n3.mid);
					\draw[thick] (n7.mid) --  (n1.mid);
					\draw[thick] (n6.mid) --  (n3.mid);
					\draw[thick] (n6.mid) --  (n2.mid);
					\draw[thick] (n6.mid) --  (n0.mid);
					\draw[thick] (n5.mid) --  (n2.mid);
					\draw[thick] (n4.mid) --  (n1.mid);
					\draw[thick] (n3.mid) --  (n1.mid);
					\draw[thick] (n3.mid) --  (n0.mid);
					\draw[thick] (n2.mid) --  (n0.mid);
					\end{pgfonlayer}
				\end{tikzpicture}}	
			\end{figure}	
		\endminipage}
		\fbox{\minipage{0.225\textwidth}
			\begin{figure}[H]
				\resizebox{1.0\textwidth}{!}{\begin{tikzpicture}[
					roundnode/.style={circle, draw=black, thick, fill=white, minimum size=7mm},
					]
					%Nodes
					\node           (label) 	at	(1,4) 		{Filtración 5 (t=0.5)};
					\node[roundnode]      (n0) 	at 	(4,1)           {0};
					\node[roundnode]      (n1) 	at 	(4,-1)          {1};
					\node[roundnode]      (n2) 	at 	(2,2)           {2};
					\node[roundnode]      (n3) 	at 	(2,0)           {3};
					\node[roundnode]      (n4)      at 	(2,-2)		{4};
					\node[roundnode]      (n5)      at 	(0,3)		{5};
					\node[roundnode]      (n6)      at 	(0,1)		{6};
					\node[roundnode]      (n7) 	at 	(0,-1)           {7};
					\node[roundnode]      (n8) 	at 	(0,-3)           {8};
					\node[roundnode]      (n9) 	at 	(-2,2)           {9};
					\node[roundnode]      (n10) 	at 	(-2,0)           {10};
					\node[roundnode]      (n11) 	at 	(-2,-2)           {11};
					\node 		(label2) 	at 	(1,-4) 		{$\beta_{0}=2,\beta_{1}=0$};
					
					\begin{pgfonlayer}{background}
					%Lines
					\draw[thick] (n11.mid) --  (n8);
					\draw[thick] (n9.mid) --  (n5);
					\draw[thick] (n6.mid) --  (n2);
					\draw[thick] (n10.mid) --  (n7);
					\draw[thick] (n10.mid) --  (n6);
					\draw[thick] (n9.mid) --  (n6);
					\draw[thick] (n7.mid) --  (n4);
					\draw[thick] (n6.mid) --  (n3);
					\draw[thick] (n3.mid) --  (n1);
					\draw[thick] (n2.mid) --  (n0);
					\end{pgfonlayer}
				\end{tikzpicture}}	
			\end{figure}	
			\begin{figure}[H]
				\resizebox{1.0\textwidth}{!}{\begin{tikzpicture}[
					roundnode/.style={circle, draw=black, thick, fill=white, minimum size=7mm},
					]
					%Nodes
					\node           (label) 	at	(1,4) 		{Filtración 9 (t=0.1)};
					\node[roundnode]      (n0) 	at 	(4,1)           {0};
					\node[roundnode]      (n1) 	at 	(4,-1)          {1};
					\node[roundnode]      (n2) 	at 	(2,2)           {2};
					\node[roundnode]      (n3) 	at 	(2,0)           {3};
					\node[roundnode]      (n4)      at 	(2,-2)		{4};
					\node[roundnode]      (n5)      at 	(0,3)		{5};
					\node[roundnode]      (n6)      at 	(0,1)		{6};
					\node[roundnode]      (n7) 	at 	(0,-1)           {7};
					\node[roundnode]      (n8) 	at 	(0,-3)           {8};
					\node[roundnode]      (n9) 	at 	(-2,2)           {9};
					\node[roundnode]      (n10) 	at 	(-2,0)           {10};
					\node[roundnode]      (n11) 	at 	(-2,-2)           {11};
					\node 		(label2) 	at 	(1,-4) 		{$\beta_{0}=1,\beta_{1}=0$};
					
					\begin{pgfonlayer}{background}							
				

					\fill[fill=black!20,opacity=1] (n11.mid) -- (n8.mid) -- (n4.mid) -- cycle;
					\fill[fill=black!20,opacity=1] (n11.mid) -- (n7.mid) -- (n4.mid) -- cycle;
					\fill[fill=black!20,opacity=1] (n11.mid) to[bend left] (n3.mid) -- (n7.mid) -- cycle;
					\fill[fill=black!20,opacity=1] (n10.mid) to[bend right] (n4.mid) -- (n7.mid) -- cycle;
					\fill[fill=black!20,opacity=1] (n10.mid) -- (n7.mid) -- (n3.mid) -- cycle;
					\fill[fill=black!20,opacity=1] (n10.mid) -- (n6.mid) -- (n3.mid) -- cycle;
					\fill[fill=black!20,opacity=1] (n10.mid) to[bend left] (n2.mid) -- (n6.mid) -- cycle;
					\fill[fill=black!20,opacity=1] (n9.mid) to[bend right] (n3.mid) -- (n6.mid) -- cycle;
					\fill[fill=black!20,opacity=1] (n9.mid) -- (n6.mid) -- (n2.mid) -- cycle;
					\fill[fill=black!20,opacity=1] (n9.mid) -- (n5.mid) -- (n2.mid) -- cycle;
					\fill[fill=black!20,opacity=1] (n8.mid) to[bend right] (n1.mid) -- (n4.mid) -- cycle;
					\fill[fill=black!20,opacity=1] (n7.mid) -- (n4.mid) -- (n1.mid) -- cycle;
					\fill[fill=black!20,opacity=1] (n7.mid) -- (n3.mid) -- (n1.mid) -- cycle;
					\fill[fill=black!20,opacity=1] (n7.mid) to[bend right] (n0.mid) -- (n3.mid) -- cycle;
					\fill[fill=black!20,opacity=1] (n6.mid) to[bend left] (n1.mid) -- (n3.mid) -- cycle;
					\fill[fill=black!20,opacity=1] (n6.mid) -- (n3.mid) -- (n0.mid) -- cycle;
					\fill[fill=black!20,opacity=1] (n6.mid) -- (n2.mid) -- (n0.mid) -- cycle;
					\fill[fill=black!20,opacity=1] (n5.mid) to[bend left] (n0.mid) -- (n2.mid) -- cycle;
					
					\draw[thick] (n11.mid) --  (n8.mid);
					\draw[thick] (n11.mid) --  (n7.mid);
					\draw[thick] (n11.mid) --  (n4.mid);
					\draw[thick] (n10.mid) --  (n7.mid);
					\draw[thick] (n10.mid) --  (n6.mid);
					\draw[thick] (n10.mid) --  (n3.mid);
					\draw[thick] (n9.mid) --  (n6.mid);
					\draw[thick] (n9.mid) --  (n5.mid);
					\draw[thick] (n9.mid) --  (n2.mid);
					\draw[thick] (n8.mid) --  (n4.mid);
					\draw[thick] (n7.mid) --  (n4.mid);
					\draw[thick] (n7.mid) --  (n3.mid);
					\draw[thick] (n7.mid) --  (n1.mid);
					\draw[thick] (n6.mid) --  (n3.mid);
					\draw[thick] (n6.mid) --  (n2.mid);
					\draw[thick] (n6.mid) --  (n0.mid);
					\draw[thick] (n5.mid) --  (n2.mid);
					\draw[thick] (n4.mid) --  (n1.mid);
					\draw[thick] (n3.mid) --  (n1.mid);
					\draw[thick] (n3.mid) --  (n0.mid);
					\draw[thick] (n2.mid) --  (n0.mid);
					\end{pgfonlayer}
				\end{tikzpicture}}	
			\end{figure}	
		\endminipage}
		\fbox{\minipage{0.225\textwidth}
			\begin{figure}[H]
				\resizebox{1.0\textwidth}{!}{\begin{tikzpicture}[
					roundnode/.style={circle, draw=black, fill=white, thick, minimum size=7mm},
					]
					%Nodes
					\node           (label) 	at	(1,4) 		{Filtración 6 (t=0.4)};
					\node[roundnode]      (n0) 	at 	(4,1)           {0};
					\node[roundnode]      (n1) 	at 	(4,-1)          {1};
					\node[roundnode]      (n2) 	at 	(2,2)           {2};
					\node[roundnode]      (n3) 	at 	(2,0)           {3};
					\node[roundnode]      (n4)      at 	(2,-2)		{4};
					\node[roundnode]      (n5)      at 	(0,3)		{5};
					\node[roundnode]      (n6)      at 	(0,1)		{6};
					\node[roundnode]      (n7) 	at 	(0,-1)           {7};
					\node[roundnode]      (n8) 	at 	(0,-3)           {8};
					\node[roundnode]      (n9) 	at 	(-2,2)           {9};
					\node[roundnode]      (n10) 	at 	(-2,0)           {10};
					\node[roundnode]      (n11) 	at 	(-2,-2)           {11};
					\node 		(label2) 	at 	(1,-4) 		{$\beta_{0}=1,\beta_{1}=5$};
					
					\begin{pgfonlayer}{background}
					\draw[thick] (n11.mid) --  (n7);
					\draw[thick] (n10.mid) --  (n7);
					\draw[thick] (n10.mid) --  (n6);
					\draw[thick] (n9.mid) --  (n6);
					\draw[thick] (n7.mid) --  (n4);
					\draw[thick] (n7.mid) --  (n3);
					\draw[thick] (n6.mid) --  (n3);
					\draw[thick] (n6.mid) --  (n2);
					\draw[thick] (n4.mid) --  (n1);
					\draw[thick] (n3.mid) --  (n1);
					\draw[thick] (n3.mid) --  (n0);
					\draw[thick] (n2.mid) --  (n0);
				

					\filldraw[fill=black!20,opacity=1] (n11.mid) -- (n8.mid) -- (n4.mid) -- cycle;
					\filldraw[fill=black!20,opacity=1] (n9.mid) -- (n5.mid) -- (n2.mid) -- cycle;
					
					\end{pgfonlayer}
				\end{tikzpicture}}	
			\end{figure}	
			\begin{figure}[H]
				\resizebox{1.0\textwidth}{!}{\begin{tikzpicture}[
					roundnode/.style={circle, draw=black, thick, fill=white, minimum size=7mm},
					]
					%Nodes
					\node           (label) 	at	(1,4) 		{Filtración 10 (t=0.0)};
					\node[roundnode]      (n0) 	at 	(4,1)           {0};
					\node[roundnode]      (n1) 	at 	(4,-1)          {1};
					\node[roundnode]      (n2) 	at 	(2,2)           {2};
					\node[roundnode]      (n3) 	at 	(2,0)           {3};
					\node[roundnode]      (n4)      at 	(2,-2)		{4};
					\node[roundnode]      (n5)      at 	(0,3)		{5};
					\node[roundnode]      (n6)      at 	(0,1)		{6};
					\node[roundnode]      (n7) 	at 	(0,-1)           {7};
					\node[roundnode]      (n8) 	at 	(0,-3)           {8};
					\node[roundnode]      (n9) 	at 	(-2,2)           {9};
					\node[roundnode]      (n10) 	at 	(-2,0)           {10};
					\node[roundnode]      (n11) 	at 	(-2,-2)           {11};
					\node 		(label2) 	at 	(1,-4) 		{$\beta_{0}=1,\beta_{1}=0$};
					
					\begin{pgfonlayer}{background}							
				

					\fill[fill=black!20,opacity=1] (n11.mid) -- (n8.mid) -- (n4.mid) -- cycle;
					\fill[fill=black!20,opacity=1] (n11.mid) -- (n7.mid) -- (n4.mid) -- cycle;
					\fill[fill=black!20,opacity=1] (n11.mid) to[bend left] (n3.mid) -- (n7.mid) -- cycle;
					\fill[fill=black!20,opacity=1] (n10.mid) to[bend right] (n4.mid) -- (n7.mid) -- cycle;
					\fill[fill=black!20,opacity=1] (n10.mid) -- (n7.mid) -- (n3.mid) -- cycle;
					\fill[fill=black!20,opacity=1] (n10.mid) -- (n6.mid) -- (n3.mid) -- cycle;
					\fill[fill=black!20,opacity=1] (n10.mid) to[bend left] (n2.mid) -- (n6.mid) -- cycle;
					\fill[fill=black!20,opacity=1] (n9.mid) to[bend right] (n3.mid) -- (n6.mid) -- cycle;
					\fill[fill=black!20,opacity=1] (n9.mid) -- (n6.mid) -- (n2.mid) -- cycle;
					\fill[fill=black!20,opacity=1] (n9.mid) -- (n5.mid) -- (n2.mid) -- cycle;
					\fill[fill=black!20,opacity=1] (n8.mid) to[bend right] (n1.mid) -- (n4.mid) -- cycle;
					\fill[fill=black!20,opacity=1] (n7.mid) -- (n4.mid) -- (n1.mid) -- cycle;
					\fill[fill=black!20,opacity=1] (n7.mid) -- (n3.mid) -- (n1.mid) -- cycle;
					\fill[fill=black!20,opacity=1] (n7.mid) to[bend right] (n0.mid) -- (n3.mid) -- cycle;
					\fill[fill=black!20,opacity=1] (n6.mid) to[bend left] (n1.mid) -- (n3.mid) -- cycle;
					\fill[fill=black!20,opacity=1] (n6.mid) -- (n3.mid) -- (n0.mid) -- cycle;
					\fill[fill=black!20,opacity=1] (n6.mid) -- (n2.mid) -- (n0.mid) -- cycle;
					\fill[fill=black!20,opacity=1] (n5.mid) to[bend left] (n0.mid) -- (n2.mid) -- cycle;
					
					\draw[thick] (n11.mid) --  (n8.mid);
					\draw[thick] (n11.mid) --  (n7.mid);
					\draw[thick] (n11.mid) --  (n4.mid);
					\draw[thick] (n10.mid) --  (n7.mid);
					\draw[thick] (n10.mid) --  (n6.mid);
					\draw[thick] (n10.mid) --  (n3.mid);
					\draw[thick] (n9.mid) --  (n6.mid);
					\draw[thick] (n9.mid) --  (n5.mid);
					\draw[thick] (n9.mid) --  (n2.mid);
					\draw[thick] (n8.mid) --  (n4.mid);
					\draw[thick] (n7.mid) --  (n4.mid);
					\draw[thick] (n7.mid) --  (n3.mid);
					\draw[thick] (n7.mid) --  (n1.mid);
					\draw[thick] (n6.mid) --  (n3.mid);
					\draw[thick] (n6.mid) --  (n2.mid);
					\draw[thick] (n6.mid) --  (n0.mid);
					\draw[thick] (n5.mid) --  (n2.mid);
					\draw[thick] (n4.mid) --  (n1.mid);
					\draw[thick] (n3.mid) --  (n1.mid);
					\draw[thick] (n3.mid) --  (n0.mid);
					\draw[thick] (n2.mid) --  (n0.mid);
					\end{pgfonlayer}
				\end{tikzpicture}}	
			\end{figure}	
		\endminipage}
		\caption{Pasos en la filtración del complejo simplicial 
		asociado a \ref{fig:distHomR} y números de Betti 
		correspondientes (interpretación local).}
		\label{fig:distHomRSL}
	\end{figure}
	\begin{figure}[!htbp]
		\minipage{0.5\textwidth}
			\begin{figure}[H]
				\resizebox{1.0\textwidth}{!}{\includegraphics{Images/DiagramaBarrasEj7LOCAL.png}}
			\end{figure}
		\endminipage
		\minipage{0.5\textwidth}
			\begin{figure}[H]
				\resizebox{1.0\textwidth}{!}{\includegraphics{Images/DiagramaPersistenciaEj7LOCAL.png}}
			\end{figure}
		\endminipage
		\caption{Diagrama de barras (izquierda) y diagrama de 
		persistencia (derecha) asociados a la filtración ilustrada en 
		\ref{fig:distHomRSL} (interpretación local).}
		\label{fig:distHomRDL}
	\end{figure}
	\begin{figure}[!htbp]
		\fbox{\minipage{0.225\textwidth}
			\begin{figure}[H]
				\resizebox{1.0\textwidth}{!}{\begin{tikzpicture}[
					roundnode/.style={circle, draw=black, thick, fill=white, minimum size=7mm},
					]
					%Nodes
					\node           (label) 	at	(1,4) 		{Filtración 0 (t=1.0)};
					\node[roundnode]      (n0) 	at 	(4,1)           {0};
					\node[roundnode]      (n1) 	at 	(4,-1)          {1};
					\node[roundnode]      (n2) 	at 	(2,2)           {2};
					\node[roundnode]      (n3) 	at 	(2,0)           {3};
					\node[roundnode]      (n4)      at 	(2,-2)		{4};
					\node[roundnode]      (n5)      at 	(0,3)		{5};
					\node[roundnode]      (n6)      at 	(0,1)		{6};
					\node[roundnode]      (n7) 	at 	(0,-1)           {7};
					\node[roundnode]      (n8) 	at 	(0,-3)           {8};
					\node[roundnode]      (n9) 	at 	(-2,2)           {9};
					\node[roundnode]      (n10) 	at 	(-2,0)           {10};
					\node[roundnode]      (n11) 	at 	(-2,-2)           {11};
					\node 		(label2) 	at 	(1,-4) 		{$\beta_{0}=10,\beta_{1}=0$};
					
					\begin{pgfonlayer}{background}
					%Lines
					\draw[thick] (n11.mid) --  (n8);
					\draw[thick] (n9.mid) --  (n5);
					\end{pgfonlayer}
				\end{tikzpicture}}	
			\end{figure}	
			\begin{figure}[H]
				\resizebox{1.0\textwidth}{!}{\begin{tikzpicture}[
					roundnode/.style={circle, draw=black, thick, fill=white, minimum size=7mm},
					]
					%Nodes
					\node           (label) 	at	(1,4) 		{Filtración 7 (t=0.3)};
					\node[roundnode]      (n0) 	at 	(4,1)           {0};
					\node[roundnode]      (n1) 	at 	(4,-1)          {1};
					\node[roundnode]      (n2) 	at 	(2,2)           {2};
					\node[roundnode]      (n3) 	at 	(2,0)           {3};
					\node[roundnode]      (n4)      at 	(2,-2)		{4};
					\node[roundnode]      (n5)      at 	(0,3)		{5};
					\node[roundnode]      (n6)      at 	(0,1)		{6};
					\node[roundnode]      (n7) 	at 	(0,-1)           {7};
					\node[roundnode]      (n8) 	at 	(0,-3)           {8};
					\node[roundnode]      (n9) 	at 	(-2,2)           {9};
					\node[roundnode]      (n10) 	at 	(-2,0)           {10};
					\node[roundnode]      (n11) 	at 	(-2,-2)           {11};
					\node 		(label2) 	at 	(1,-4) 		{$\beta_{0}=1,\beta_{1}=2$};
					
					\begin{pgfonlayer}{background}							
				

					\fill[fill=black!20,opacity=1] (n11.mid) -- (n8.mid) -- (n4.mid) -- cycle;
					\fill[fill=black!20,opacity=1] (n11.mid) -- (n7.mid) -- (n4.mid) -- cycle;
					\fill[fill=black!20,opacity=1] (n10.mid) to[bend right] (n4.mid) -- (n7.mid) -- cycle;
					\fill[fill=black!20,opacity=1] (n10.mid) to[bend left] (n2.mid) -- (n6.mid) -- cycle;
					\fill[fill=black!20,opacity=1] (n9.mid) -- (n6.mid) -- (n2.mid) -- cycle;
					\fill[fill=black!20,opacity=1] (n9.mid) -- (n5.mid) -- (n2.mid) -- cycle;
					\fill[fill=black!20,opacity=1] (n6.mid) to[bend left] (n1.mid) -- (n3.mid) -- cycle;
					\fill[fill=black!20,opacity=1] (n6.mid) -- (n3.mid) -- (n0.mid) -- cycle;
					\fill[fill=black!20,opacity=1] (n6.mid) -- (n2.mid) -- (n0.mid) -- cycle;
					
					\draw[thick] (n11.mid) --  (n8);
					\draw[thick] (n11.mid) --  (n7);
					\draw[thick] (n11.mid) --  (n4.mid);
					\draw[thick] (n10.mid) --  (n7);
					\draw[thick] (n10.mid) --  (n6);
					\draw[thick] (n9.mid) --  (n6);
					\draw[thick] (n9.mid) --  (n5);
					\draw[thick] (n9.mid) --  (n2.mid);
					\draw[thick] (n8.mid) --  (n4);
					\draw[thick] (n7.mid) --  (n4);
					\draw[thick] (n7.mid) --  (n3);
					\draw[thick] (n6.mid) --  (n3);
					\draw[thick] (n6.mid) --  (n2);
					\draw[thick] (n6.mid) --  (n0);
					\draw[thick] (n5.mid) --  (n2);
					\draw[thick] (n4.mid) --  (n1);
					\draw[thick] (n3.mid) --  (n1);
					\draw[thick] (n3.mid) --  (n0);
					\draw[thick] (n2.mid) --  (n0);
					\end{pgfonlayer}
				\end{tikzpicture}}	
			\end{figure}	
		\endminipage}
		\fbox{\minipage{0.225\textwidth}
			\begin{figure}[H]
				\resizebox{1.0\textwidth}{!}{\begin{tikzpicture}[
					roundnode/.style={circle, draw=black, thick, fill=white, minimum size=7mm},
					]
					%Nodes
					\node           (label) 	at	(1,4) 		{Filtración 4 (t=0.6)};
					\node[roundnode]      (n0) 	at 	(4,1)           {0};
					\node[roundnode]      (n1) 	at 	(4,-1)          {1};
					\node[roundnode]      (n2) 	at 	(2,2)           {2};
					\node[roundnode]      (n3) 	at 	(2,0)           {3};
					\node[roundnode]      (n4)      at 	(2,-2)		{4};
					\node[roundnode]      (n5)      at 	(0,3)		{5};
					\node[roundnode]      (n6)      at 	(0,1)		{6};
					\node[roundnode]      (n7) 	at 	(0,-1)           {7};
					\node[roundnode]      (n8) 	at 	(0,-3)           {8};
					\node[roundnode]      (n9) 	at 	(-2,2)           {9};
					\node[roundnode]      (n10) 	at 	(-2,0)           {10};
					\node[roundnode]      (n11) 	at 	(-2,-2)           {11};
					\node 		(label2) 	at 	(1,-4) 		{$\beta_{0}=9,\beta_{1}=0$};
					
					\begin{pgfonlayer}{background}
					%Lines
					\draw[thick] (n11.mid) --  (n8);
					\draw[thick] (n9.mid) --  (n5);
					\draw[thick] (n6.mid) --  (n2);
					\end{pgfonlayer}
				\end{tikzpicture}}	
			\end{figure}	
			\begin{figure}[H]
				\resizebox{1.0\textwidth}{!}{\begin{tikzpicture}[
					roundnode/.style={circle, draw=black, thick, fill=white, minimum size=7mm},
					]
					%Nodes
					\node           (label) 	at	(1,4) 		{Filtración 8 (t=0.2)};
					\node[roundnode]      (n0) 	at 	(4,1)           {0};
					\node[roundnode]      (n1) 	at 	(4,-1)          {1};
					\node[roundnode]      (n2) 	at 	(2,2)           {2};
					\node[roundnode]      (n3) 	at 	(2,0)           {3};
					\node[roundnode]      (n4)      at 	(2,-2)		{4};
					\node[roundnode]      (n5)      at 	(0,3)		{5};
					\node[roundnode]      (n6)      at 	(0,1)		{6};
					\node[roundnode]      (n7) 	at 	(0,-1)           {7};
					\node[roundnode]      (n8) 	at 	(0,-3)           {8};
					\node[roundnode]      (n9) 	at 	(-2,2)           {9};
					\node[roundnode]      (n10) 	at 	(-2,0)           {10};
					\node[roundnode]      (n11) 	at 	(-2,-2)           {11};
					\node 		(label2) 	at 	(1,-4) 		{$\beta_{0}=1,\beta_{1}=0$};
					
					\begin{pgfonlayer}{background}							
				

					\fill[fill=black!20,opacity=1] (n11.mid) -- (n8.mid) -- (n4.mid) -- cycle;
					\fill[fill=black!20,opacity=1] (n11.mid) -- (n7.mid) -- (n4.mid) -- cycle;
					\fill[fill=black!20,opacity=1] (n10.mid) to[bend right] (n4.mid) -- (n7.mid) -- cycle;
					\fill[fill=black!20,opacity=1] (n10.mid) -- (n7.mid) -- (n3.mid) -- cycle;
					\fill[fill=black!20,opacity=1] (n10.mid) -- (n6.mid) -- (n3.mid) -- cycle;
					\fill[fill=black!20,opacity=1] (n10.mid) to[bend left] (n2.mid) -- (n6.mid) -- cycle;
					\fill[fill=black!20,opacity=1] (n9.mid) to[bend right] (n3.mid) -- (n6.mid) -- cycle;
					\fill[fill=black!20,opacity=1] (n9.mid) -- (n6.mid) -- (n2.mid) -- cycle;
					\fill[fill=black!20,opacity=1] (n9.mid) -- (n5.mid) -- (n2.mid) -- cycle;
					\fill[fill=black!20,opacity=1] (n7.mid) -- (n4.mid) -- (n1.mid) -- cycle;
					\fill[fill=black!20,opacity=1] (n7.mid) -- (n3.mid) -- (n1.mid) -- cycle;
					\fill[fill=black!20,opacity=1] (n6.mid) to[bend left] (n1.mid) -- (n3.mid) -- cycle;
					\fill[fill=black!20,opacity=1] (n6.mid) -- (n3.mid) -- (n0.mid) -- cycle;
					\fill[fill=black!20,opacity=1] (n6.mid) -- (n2.mid) -- (n0.mid) -- cycle;
					\fill[fill=black!20,opacity=1] (n5.mid) to[bend left] (n0.mid) -- (n2.mid) -- cycle;
					
					\draw[thick] (n11.mid) --  (n8.mid);
					\draw[thick] (n11.mid) --  (n7.mid);
					\draw[thick] (n11.mid) --  (n4.mid);
					\draw[thick] (n10.mid) --  (n7.mid);
					\draw[thick] (n10.mid) --  (n6.mid);
					\draw[thick] (n10.mid) --  (n3.mid);
					\draw[thick] (n9.mid) --  (n6.mid);
					\draw[thick] (n9.mid) --  (n5.mid);
					\draw[thick] (n9.mid) --  (n2.mid);
					\draw[thick] (n8.mid) --  (n4.mid);
					\draw[thick] (n7.mid) --  (n4.mid);
					\draw[thick] (n7.mid) --  (n3.mid);
					\draw[thick] (n7.mid) --  (n1.mid);
					\draw[thick] (n6.mid) --  (n3.mid);
					\draw[thick] (n6.mid) --  (n2.mid);
					\draw[thick] (n6.mid) --  (n0.mid);
					\draw[thick] (n5.mid) --  (n2.mid);
					\draw[thick] (n4.mid) --  (n1.mid);
					\draw[thick] (n3.mid) --  (n1.mid);
					\draw[thick] (n3.mid) --  (n0.mid);
					\draw[thick] (n2.mid) --  (n0.mid);
					\end{pgfonlayer}
				\end{tikzpicture}}	
			\end{figure}	
		\endminipage}
		\fbox{\minipage{0.225\textwidth}
			\begin{figure}[H]
				\resizebox{1.0\textwidth}{!}{\begin{tikzpicture}[
					roundnode/.style={circle, draw=black, thick, fill=white, minimum size=7mm},
					]
					%Nodes
					\node           (label) 	at	(1,4) 		{Filtración 5 (t=0.5)};
					\node[roundnode]      (n0) 	at 	(4,1)           {0};
					\node[roundnode]      (n1) 	at 	(4,-1)          {1};
					\node[roundnode]      (n2) 	at 	(2,2)           {2};
					\node[roundnode]      (n3) 	at 	(2,0)           {3};
					\node[roundnode]      (n4)      at 	(2,-2)		{4};
					\node[roundnode]      (n5)      at 	(0,3)		{5};
					\node[roundnode]      (n6)      at 	(0,1)		{6};
					\node[roundnode]      (n7) 	at 	(0,-1)           {7};
					\node[roundnode]      (n8) 	at 	(0,-3)           {8};
					\node[roundnode]      (n9) 	at 	(-2,2)           {9};
					\node[roundnode]      (n10) 	at 	(-2,0)           {10};
					\node[roundnode]      (n11) 	at 	(-2,-2)           {11};
					\node 		(label2) 	at 	(1,-4) 		{$\beta_{0}=2,\beta_{1}=0$};
					
					\begin{pgfonlayer}{background}
					%Lines
					\draw[thick] (n11.mid) --  (n8);
					\draw[thick] (n9.mid) --  (n5);
					\draw[thick] (n6.mid) --  (n2);
					\draw[thick] (n10.mid) --  (n7);
					\draw[thick] (n10.mid) --  (n6);
					\draw[thick] (n9.mid) --  (n6);
					\draw[thick] (n7.mid) --  (n4);
					\draw[thick] (n6.mid) --  (n3);
					\draw[thick] (n3.mid) --  (n1);
					\draw[thick] (n2.mid) --  (n0);
					\end{pgfonlayer}
				\end{tikzpicture}}	
			\end{figure}	
			\begin{figure}[H]
				\resizebox{1.0\textwidth}{!}{\begin{tikzpicture}[
					roundnode/.style={circle, draw=black, thick, fill=white, minimum size=7mm},
					]
					%Nodes
					\node           (label) 	at	(1,4) 		{Filtración 9 (t=0.1)};
					\node[roundnode]      (n0) 	at 	(4,1)           {0};
					\node[roundnode]      (n1) 	at 	(4,-1)          {1};
					\node[roundnode]      (n2) 	at 	(2,2)           {2};
					\node[roundnode]      (n3) 	at 	(2,0)           {3};
					\node[roundnode]      (n4)      at 	(2,-2)		{4};
					\node[roundnode]      (n5)      at 	(0,3)		{5};
					\node[roundnode]      (n6)      at 	(0,1)		{6};
					\node[roundnode]      (n7) 	at 	(0,-1)           {7};
					\node[roundnode]      (n8) 	at 	(0,-3)           {8};
					\node[roundnode]      (n9) 	at 	(-2,2)           {9};
					\node[roundnode]      (n10) 	at 	(-2,0)           {10};
					\node[roundnode]      (n11) 	at 	(-2,-2)           {11};
					\node 		(label2) 	at 	(1,-4) 		{$\beta_{0}=1,\beta_{1}=0$};
					
					\begin{pgfonlayer}{background}							
				

					\fill[fill=black!20,opacity=1] (n11.mid) -- (n8.mid) -- (n4.mid) -- cycle;
					\fill[fill=black!20,opacity=1] (n11.mid) -- (n7.mid) -- (n4.mid) -- cycle;
					\fill[fill=black!20,opacity=1] (n11.mid) to[bend left] (n3.mid) -- (n7.mid) -- cycle;
					\fill[fill=black!20,opacity=1] (n10.mid) to[bend right] (n4.mid) -- (n7.mid) -- cycle;
					\fill[fill=black!20,opacity=1] (n10.mid) -- (n7.mid) -- (n3.mid) -- cycle;
					\fill[fill=black!20,opacity=1] (n10.mid) -- (n6.mid) -- (n3.mid) -- cycle;
					\fill[fill=black!20,opacity=1] (n10.mid) to[bend left] (n2.mid) -- (n6.mid) -- cycle;
					\fill[fill=black!20,opacity=1] (n9.mid) to[bend right] (n3.mid) -- (n6.mid) -- cycle;
					\fill[fill=black!20,opacity=1] (n9.mid) -- (n6.mid) -- (n2.mid) -- cycle;
					\fill[fill=black!20,opacity=1] (n9.mid) -- (n5.mid) -- (n2.mid) -- cycle;
					\fill[fill=black!20,opacity=1] (n8.mid) to[bend right] (n1.mid) -- (n4.mid) -- cycle;
					\fill[fill=black!20,opacity=1] (n7.mid) -- (n4.mid) -- (n1.mid) -- cycle;
					\fill[fill=black!20,opacity=1] (n7.mid) -- (n3.mid) -- (n1.mid) -- cycle;
					\fill[fill=black!20,opacity=1] (n7.mid) to[bend right] (n0.mid) -- (n3.mid) -- cycle;
					\fill[fill=black!20,opacity=1] (n6.mid) to[bend left] (n1.mid) -- (n3.mid) -- cycle;
					\fill[fill=black!20,opacity=1] (n6.mid) -- (n3.mid) -- (n0.mid) -- cycle;
					\fill[fill=black!20,opacity=1] (n6.mid) -- (n2.mid) -- (n0.mid) -- cycle;
					\fill[fill=black!20,opacity=1] (n5.mid) to[bend left] (n0.mid) -- (n2.mid) -- cycle;
					
					\draw[thick] (n11.mid) --  (n8.mid);
					\draw[thick] (n11.mid) --  (n7.mid);
					\draw[thick] (n11.mid) --  (n4.mid);
					\draw[thick] (n10.mid) --  (n7.mid);
					\draw[thick] (n10.mid) --  (n6.mid);
					\draw[thick] (n10.mid) --  (n3.mid);
					\draw[thick] (n9.mid) --  (n6.mid);
					\draw[thick] (n9.mid) --  (n5.mid);
					\draw[thick] (n9.mid) --  (n2.mid);
					\draw[thick] (n8.mid) --  (n4.mid);
					\draw[thick] (n7.mid) --  (n4.mid);
					\draw[thick] (n7.mid) --  (n3.mid);
					\draw[thick] (n7.mid) --  (n1.mid);
					\draw[thick] (n6.mid) --  (n3.mid);
					\draw[thick] (n6.mid) --  (n2.mid);
					\draw[thick] (n6.mid) --  (n0.mid);
					\draw[thick] (n5.mid) --  (n2.mid);
					\draw[thick] (n4.mid) --  (n1.mid);
					\draw[thick] (n3.mid) --  (n1.mid);
					\draw[thick] (n3.mid) --  (n0.mid);
					\draw[thick] (n2.mid) --  (n0.mid);
					\end{pgfonlayer}
				\end{tikzpicture}}	
			\end{figure}	
		\endminipage}
		\fbox{\minipage{0.225\textwidth}
			\begin{figure}[H]
				\resizebox{1.0\textwidth}{!}{\begin{tikzpicture}[
					roundnode/.style={circle, draw=black, fill=white, thick, minimum size=7mm},
					]
					%Nodes
					\node           (label) 	at	(1,4) 		{Filtración 6 (t=0.4)};
					\node[roundnode]      (n0) 	at 	(4,1)           {0};
					\node[roundnode]      (n1) 	at 	(4,-1)          {1};
					\node[roundnode]      (n2) 	at 	(2,2)           {2};
					\node[roundnode]      (n3) 	at 	(2,0)           {3};
					\node[roundnode]      (n4)      at 	(2,-2)		{4};
					\node[roundnode]      (n5)      at 	(0,3)		{5};
					\node[roundnode]      (n6)      at 	(0,1)		{6};
					\node[roundnode]      (n7) 	at 	(0,-1)           {7};
					\node[roundnode]      (n8) 	at 	(0,-3)           {8};
					\node[roundnode]      (n9) 	at 	(-2,2)           {9};
					\node[roundnode]      (n10) 	at 	(-2,0)           {10};
					\node[roundnode]      (n11) 	at 	(-2,-2)           {11};
					\node 		(label2) 	at 	(1,-4) 		{$\beta_{0}=1,\beta_{1}=3$};
					
					\begin{pgfonlayer}{background}
				
					\fill[fill=black!20,opacity=1] (n11.mid) -- (n8.mid) -- (n4.mid) -- cycle;
					\fill[fill=black!20,opacity=1] (n11.mid) -- (n7.mid) -- (n4.mid) -- cycle;
					\fill[fill=black!20,opacity=1] (n9.mid) -- (n6.mid) -- (n2.mid) -- cycle;
					\fill[fill=black!20,opacity=1] (n9.mid) -- (n5.mid) -- (n2.mid) -- cycle;
					
					\draw[thick] (n11.mid) --  (n8.mid);
					\draw[thick] (n11.mid) --  (n7.mid);
					\draw[thick] (n11.mid) --  (n4.mid);
					\draw[thick] (n10.mid) --  (n7.mid);
					\draw[thick] (n10.mid) --  (n6);
					\draw[thick] (n9.mid) --  (n6.mid);
					\draw[thick] (n9.mid) --  (n5.mid);
					\draw[thick] (n9.mid) --  (n2.mid);
					\draw[thick] (n8.mid) --  (n4);
					\draw[thick] (n7.mid) --  (n4);
					\draw[thick] (n7.mid) --  (n3);
					\draw[thick] (n6.mid) --  (n3);
					\draw[thick] (n6.mid) --  (n2);
					\draw[thick] (n5.mid) --  (n2);
					\draw[thick] (n4.mid) --  (n1);
					\draw[thick] (n3.mid) --  (n1);
					\draw[thick] (n3.mid) --  (n0);
					\draw[thick] (n2.mid) --  (n0);
					
					\end{pgfonlayer}
				\end{tikzpicture}}	
			\end{figure}	
			\begin{figure}[H]
				\resizebox{1.0\textwidth}{!}{\begin{tikzpicture}[
					roundnode/.style={circle, draw=black, thick, fill=white, minimum size=7mm},
					]
					%Nodes
					\node           (label) 	at	(1,4) 		{Filtración 10 (t=0.0)};
					\node[roundnode]      (n0) 	at 	(4,1)           {0};
					\node[roundnode]      (n1) 	at 	(4,-1)          {1};
					\node[roundnode]      (n2) 	at 	(2,2)           {2};
					\node[roundnode]      (n3) 	at 	(2,0)           {3};
					\node[roundnode]      (n4)      at 	(2,-2)		{4};
					\node[roundnode]      (n5)      at 	(0,3)		{5};
					\node[roundnode]      (n6)      at 	(0,1)		{6};
					\node[roundnode]      (n7) 	at 	(0,-1)           {7};
					\node[roundnode]      (n8) 	at 	(0,-3)           {8};
					\node[roundnode]      (n9) 	at 	(-2,2)           {9};
					\node[roundnode]      (n10) 	at 	(-2,0)           {10};
					\node[roundnode]      (n11) 	at 	(-2,-2)           {11};
					\node 		(label2) 	at 	(1,-4) 		{$\beta_{0}=1,\beta_{1}=0$};
					
					\begin{pgfonlayer}{background}							
				

					\fill[fill=black!20,opacity=1] (n11.mid) -- (n8.mid) -- (n4.mid) -- cycle;
					\fill[fill=black!20,opacity=1] (n11.mid) -- (n7.mid) -- (n4.mid) -- cycle;
					\fill[fill=black!20,opacity=1] (n11.mid) to[bend left] (n3.mid) -- (n7.mid) -- cycle;
					\fill[fill=black!20,opacity=1] (n10.mid) to[bend right] (n4.mid) -- (n7.mid) -- cycle;
					\fill[fill=black!20,opacity=1] (n10.mid) -- (n7.mid) -- (n3.mid) -- cycle;
					\fill[fill=black!20,opacity=1] (n10.mid) -- (n6.mid) -- (n3.mid) -- cycle;
					\fill[fill=black!20,opacity=1] (n10.mid) to[bend left] (n2.mid) -- (n6.mid) -- cycle;
					\fill[fill=black!20,opacity=1] (n9.mid) to[bend right] (n3.mid) -- (n6.mid) -- cycle;
					\fill[fill=black!20,opacity=1] (n9.mid) -- (n6.mid) -- (n2.mid) -- cycle;
					\fill[fill=black!20,opacity=1] (n9.mid) -- (n5.mid) -- (n2.mid) -- cycle;
					\fill[fill=black!20,opacity=1] (n8.mid) to[bend right] (n1.mid) -- (n4.mid) -- cycle;
					\fill[fill=black!20,opacity=1] (n7.mid) -- (n4.mid) -- (n1.mid) -- cycle;
					\fill[fill=black!20,opacity=1] (n7.mid) -- (n3.mid) -- (n1.mid) -- cycle;
					\fill[fill=black!20,opacity=1] (n7.mid) to[bend right] (n0.mid) -- (n3.mid) -- cycle;
					\fill[fill=black!20,opacity=1] (n6.mid) to[bend left] (n1.mid) -- (n3.mid) -- cycle;
					\fill[fill=black!20,opacity=1] (n6.mid) -- (n3.mid) -- (n0.mid) -- cycle;
					\fill[fill=black!20,opacity=1] (n6.mid) -- (n2.mid) -- (n0.mid) -- cycle;
					\fill[fill=black!20,opacity=1] (n5.mid) to[bend left] (n0.mid) -- (n2.mid) -- cycle;
					
					\draw[thick] (n11.mid) --  (n8.mid);
					\draw[thick] (n11.mid) --  (n7.mid);
					\draw[thick] (n11.mid) --  (n4.mid);
					\draw[thick] (n10.mid) --  (n7.mid);
					\draw[thick] (n10.mid) --  (n6.mid);
					\draw[thick] (n10.mid) --  (n3.mid);
					\draw[thick] (n9.mid) --  (n6.mid);
					\draw[thick] (n9.mid) --  (n5.mid);
					\draw[thick] (n9.mid) --  (n2.mid);
					\draw[thick] (n8.mid) --  (n4.mid);
					\draw[thick] (n7.mid) --  (n4.mid);
					\draw[thick] (n7.mid) --  (n3.mid);
					\draw[thick] (n7.mid) --  (n1.mid);
					\draw[thick] (n6.mid) --  (n3.mid);
					\draw[thick] (n6.mid) --  (n2.mid);
					\draw[thick] (n6.mid) --  (n0.mid);
					\draw[thick] (n5.mid) --  (n2.mid);
					\draw[thick] (n4.mid) --  (n1.mid);
					\draw[thick] (n3.mid) --  (n1.mid);
					\draw[thick] (n3.mid) --  (n0.mid);
					\draw[thick] (n2.mid) --  (n0.mid);
					\end{pgfonlayer}
				\end{tikzpicture}}	
			\end{figure}	
		\endminipage}
		\caption{Pasos en la filtración del complejo simplicial 
		asociado a \ref{fig:distHomR} y números de Betti 
		correspondientes (interpretación global).}
		\label{fig:distHomRSG}
	\end{figure}
	\begin{figure}[!htbp]
		\minipage{0.5\textwidth}
			\begin{figure}[H]
				\resizebox{1.0\textwidth}{!}{\includegraphics{Images/DiagramaBarrasEj7GLOBAL.png}}
			\end{figure}
		\endminipage
		\minipage{0.5\textwidth}
			\begin{figure}[H]
				\resizebox{1.0\textwidth}{!}{\includegraphics{Images/DiagramaPersistenciaEj7GLOBAL.png}}
			\end{figure}
		\endminipage
		\caption{Diagrama de barras (izquierda) y diagrama de 
		persistencia (derecha) asociados a la filtración ilustrada en 
		\ref{fig:distHomRSG} (interpretación global).}
		\label{fig:distHomRDG}
	\end{figure}

	Si observamos detenidamente los pasos en la filtración del complejo 
	simplicial en la \autoref{fig:distHomRSL} y en la 
	\autoref{fig:distHomRSG}, 
	apreciamos que la diferencia entre ambas interpretaciones se produce 
	en las filtraciones 6 y 7, donde los números de Betti $\beta_{1}$ 
	valen 4 en la interpretación local, y 2 en la interpretación global. 
	También podemos apreciar esta distinción en los diagramas de barras 
	correspondientes que podemos encontrar en la 
	\autoref{fig:distHomRDL} y 
	la \autoref{fig:distHomRDG}. Como ya habíamos notado anteriormente, 
	esta 
	disparidad nos muestra que la información recogida por la 
	interpretación global es escasa, lo que nos induce a pensar que esta 
	interpretación, a pesar de su solidez teórica, proporciona menos 
	información. No 
	obstante, antes de tomar cualquier decisión en este aspecto, debemos 
	ver su desempeño ante los cambios en las distribuciones de los 
	siguientes ejemplos. 
	\end{ejem}
	
	Veamos ahora el ejemplo correspondiente a una distribución intermedia
	de los pesos de la red.

	\begin{ejem}\textbf{Distribución intermedia.}
	Supondremos que la red estudiada posee una 
	distribución heterogénea de los pesos. Esta distribución, aunque 
	heterogénea, no presentará desigualdades muy significativas entre los 
	pesos de la red. Esta distribución dará lugar a la representación de 
	la red, con las importancias entre neuronas ya calculadas, que puede 
	verse en la \autoref{fig:distIntR}.
	
	\begin{figure}[!htbp]
			\centering
			\begin{tikzpicture}[
				roundnode/.style={circle, draw=black, thick, fill=white, minimum size=7mm},
				]
				%Nodes
				\node[roundnode]      (n0) 	at 	(4,1)           {0};
				\node[roundnode]      (n1) 	at 	(4,-1)          {1};
				\node[roundnode]      (n2) 	at 	(2,2)           {2};
				\node[roundnode]      (n3) 	at 	(2,0)           {3};
				\node[roundnode]      (n4)      at 	(2,-2)		{4};
				\node[roundnode]      (n5)      at 	(0,3)		{5};
				\node[roundnode]      (n6)      at 	(0,1)		{6};
				\node[roundnode]      (n7) 	at 	(0,-1)           {7};
				\node[roundnode]      (n8) 	at 	(0,-3)           {8};
				\node[roundnode]      (n9) 	at 	(-2,2)           {9};
				\node[roundnode]      (n10) 	at 	(-2,0)           {10};
				\node[roundnode]      (n11) 	at 	(-2,-2)           {11};
				
				\begin{pgfonlayer}{background}
				%Lines
				\draw[thick,->] (n9.mid) -- node[above,sloped] {1.0} (n5.west);
				\draw[thick,->] (n9.mid) -- node[above,sloped] {0.65} (n6.west);
				\draw[thick,->] (n10.mid) -- node[above,sloped] {0.35} (n6.west);
				\draw[thick,->] (n10.mid) -- node[above,sloped] {0.7} (n7.west);
				\draw[thick,->] (n11.mid) -- node[above,sloped] {0.3} (n7.west);
				\draw[thick,->] (n11.mid) -- node[above,sloped] {1.0} (n8.west);
				\draw[thick,->] (n5.mid) -- node[above,sloped] {0.4} (n2.west);
				\draw[thick,->] (n6.mid) -- node[above,sloped] {0.6} (n2.west);
				\draw[thick,->] (n6.mid) -- node[above,sloped] {0.72} (n3.west);
				\draw[thick,->] (n7.mid) -- node[above,sloped] {0.28} (n3.west);
				\draw[thick,->] (n7.mid) -- node[above,sloped] {0.32} (n4.west);
				\draw[thick,->] (n8.mid) -- node[above,sloped] {0.68} (n4.west);
				\draw[thick,->] (n2.mid) -- node[above,sloped] {0.25} (n0.west);
				\draw[thick,->] (n3.mid) -- node[above,sloped] {0.75} (n0.west);
				\draw[thick,->] (n3.mid) -- node[above,sloped] {0.63} (n1.west);
				\draw[thick,->] (n4.mid) -- node[above,sloped] {0.37} (n1.west);
				\end{pgfonlayer}
			\end{tikzpicture}
			\caption{Representación de la red \ref{fig:arqEjs} con 
			una distribución intermedia de las importancias entre 
			neuronas.}
			\label{fig:distIntR}
	\end{figure}

	Procedemos ahora con el cálculo del complejo simplicial asociado. Como 
	ya hemos mencionado anteriormente, vamos a ilustrar unos cuantos pasos 
	en la filtración con los correspondientes 
	números de Betti asociados. Esta ilustración puede verse en 
	la \autoref{fig:distIntRSL} y en la \autoref{fig:distIntRSG}. También 
	añadimos los diagramas de 
	barras y de persistencia asociados que pueden verse en 
	la \autoref{fig:distIntRDL} y en la \autoref{fig:distIntRDG}.

	\begin{figure}[!htbp]
		\fbox{\minipage{0.225\textwidth}
			\begin{figure}[H]
				\resizebox{1.0\textwidth}{!}{\begin{tikzpicture}[
					roundnode/.style={circle, draw=black, thick, fill=white, minimum size=7mm},
					]
					%Nodes
					\node           (label) 	at	(1,4) 		{Filtración 0 (t=1.0)};
					\node[roundnode]      (n0) 	at 	(4,1)           {0};
					\node[roundnode]      (n1) 	at 	(4,-1)          {1};
					\node[roundnode]      (n2) 	at 	(2,2)           {2};
					\node[roundnode]      (n3) 	at 	(2,0)           {3};
					\node[roundnode]      (n4)      at 	(2,-2)		{4};
					\node[roundnode]      (n5)      at 	(0,3)		{5};
					\node[roundnode]      (n6)      at 	(0,1)		{6};
					\node[roundnode]      (n7) 	at 	(0,-1)           {7};
					\node[roundnode]      (n8) 	at 	(0,-3)           {8};
					\node[roundnode]      (n9) 	at 	(-2,2)           {9};
					\node[roundnode]      (n10) 	at 	(-2,0)           {10};
					\node[roundnode]      (n11) 	at 	(-2,-2)           {11};
					\node 		(label2) 	at 	(1,-4) 		{$\beta_{0}=10,\beta_{1}=0$};
					
					\begin{pgfonlayer}{background}
					%Lines
					\draw[thick] (n11.mid) --  (n8);
					\draw[thick] (n9.mid) --  (n5);
					\end{pgfonlayer}
				\end{tikzpicture}}	
			\end{figure}	
			\begin{figure}[H]
				\resizebox{1.0\textwidth}{!}{\begin{tikzpicture}[
					roundnode/.style={circle, draw=black, thick, fill=white, minimum size=7mm},
					]
					%Nodes
					\node           (label) 	at	(1,4) 		{Filtración 7 (t=0.3)};
					\node[roundnode]      (n0) 	at 	(4,1)           {0};
					\node[roundnode]      (n1) 	at 	(4,-1)          {1};
					\node[roundnode]      (n2) 	at 	(2,2)           {2};
					\node[roundnode]      (n3) 	at 	(2,0)           {3};
					\node[roundnode]      (n4)      at 	(2,-2)		{4};
					\node[roundnode]      (n5)      at 	(0,3)		{5};
					\node[roundnode]      (n6)      at 	(0,1)		{6};
					\node[roundnode]      (n7) 	at 	(0,-1)           {7};
					\node[roundnode]      (n8) 	at 	(0,-3)           {8};
					\node[roundnode]      (n9) 	at 	(-2,2)           {9};
					\node[roundnode]      (n10) 	at 	(-2,0)           {10};
					\node[roundnode]      (n11) 	at 	(-2,-2)           {11};
					\node 		(label2) 	at 	(1,-4) 		{$\beta_{0}=1,\beta_{1}=2$};
					
					\begin{pgfonlayer}{background}
					
					\fill[fill=black!20,opacity=1] (n11.mid) -- (n8.mid) -- (n4.mid) -- cycle;
					\fill[fill=black!20,opacity=1] (n9.mid) to[bend right] (n3.mid) -- (n6.mid) -- cycle;
					\fill[fill=black!20,opacity=1] (n9.mid) -- (n6.mid) -- (n2.mid) -- cycle;
					\fill[fill=black!20,opacity=1] (n9.mid) -- (n6.mid) -- (n0.mid) -- cycle;
					\fill[fill=black!20,opacity=1] (n9.mid) -- (n5.mid) -- (n2.mid) -- cycle;
					\fill[fill=black!20,opacity=1] (n6.mid) to[bend left] (n1.mid) -- (n3.mid) -- cycle;
					\fill[fill=black!20,opacity=1] (n6.mid) -- (n3.mid) -- (n0.mid) -- cycle;
					%Lines
					\draw[thick] (n11.mid) --  (n8.mid);
					\draw[thick] (n11.mid) --  (n7.mid);
					\draw[thick] (n11.mid) --  (n4.mid);
					\draw[thick] (n10.mid) --  (n7.mid);
					\draw[thick] (n10.mid) --  (n6.mid);
					\draw[thick] (n9.mid) --  (n6.mid);
					\draw[thick] (n9.mid) --  (n5.mid);
					\draw[thick] (n9.mid) --  (n2.mid);
					\draw[thick] (n9.mid) --  (n0.mid);
					\draw[thick] (n8.mid) --  (n4.mid);
					\draw[thick] (n7.mid) --  (n4.mid);
					\draw[thick] (n6.mid) --  (n3.mid);
					\draw[thick] (n6.mid) --  (n2.mid);
					\draw[thick] (n6.mid) --  (n0.mid);
					\draw[thick] (n5.mid) --  (n2.mid);
					\draw[thick] (n4.mid) --  (n1.mid);
					\draw[thick] (n3.mid) --  (n1.mid);
					\draw[thick] (n3.mid) --  (n0.mid);
					\end{pgfonlayer}
				\end{tikzpicture}}	
			\end{figure}	
		\endminipage}
		\fbox{\minipage{0.225\textwidth}
			\begin{figure}[H]
				\resizebox{1.0\textwidth}{!}{\begin{tikzpicture}[
					roundnode/.style={circle, draw=black, thick, fill=white, minimum size=7mm},
					]
					%Nodes
					\node           (label) 	at	(1,4) 		{Filtración 4 (t=0.6)};
					\node[roundnode]      (n0) 	at 	(4,1)           {0};
					\node[roundnode]      (n1) 	at 	(4,-1)          {1};
					\node[roundnode]      (n2) 	at 	(2,2)           {2};
					\node[roundnode]      (n3) 	at 	(2,0)           {3};
					\node[roundnode]      (n4)      at 	(2,-2)		{4};
					\node[roundnode]      (n5)      at 	(0,3)		{5};
					\node[roundnode]      (n6)      at 	(0,1)		{6};
					\node[roundnode]      (n7) 	at 	(0,-1)           {7};
					\node[roundnode]      (n8) 	at 	(0,-3)           {8};
					\node[roundnode]      (n9) 	at 	(-2,2)           {9};
					\node[roundnode]      (n10) 	at 	(-2,0)           {10};
					\node[roundnode]      (n11) 	at 	(-2,-2)           {11};
					\node 		(label2) 	at 	(1,-4) 		{$\beta_{0}=3,\beta_{1}=0$};
					
					\begin{pgfonlayer}{background}
					
					\fill[fill=black!20,opacity=1] (n11.mid) -- (n8.mid) -- (n4.mid) -- cycle;
					%Lines
					\draw[thick] (n11.mid) --  (n8.mid);
					\draw[thick] (n11.mid) --  (n4.mid);
					\draw[thick] (n10.mid) --  (n7.mid);
					\draw[thick] (n9.mid) --  (n6.mid);
					\draw[thick] (n9.mid) --  (n5.mid);
					\draw[thick] (n8.mid) --  (n4.mid);
					\draw[thick] (n6.mid) --  (n3.mid);
					\draw[thick] (n6.mid) --  (n2.mid);
					\draw[thick] (n3.mid) --  (n1.mid);
					\draw[thick] (n3.mid) --  (n0.mid);
					\end{pgfonlayer}
				\end{tikzpicture}}	
			\end{figure}	
			\begin{figure}[H]
				\resizebox{1.0\textwidth}{!}{\begin{tikzpicture}[
					roundnode/.style={circle, draw=black, thick, fill=white, minimum size=7mm},
					]
					%Nodes
					\node           (label) 	at	(1,4) 		{Filtración 8 (t=0.2)};
					\node[roundnode]      (n0) 	at 	(4,1)           {0};
					\node[roundnode]      (n1) 	at 	(4,-1)          {1};
					\node[roundnode]      (n2) 	at 	(2,2)           {2};
					\node[roundnode]      (n3) 	at 	(2,0)           {3};
					\node[roundnode]      (n4)      at 	(2,-2)		{4};
					\node[roundnode]      (n5)      at 	(0,3)		{5};
					\node[roundnode]      (n6)      at 	(0,1)		{6};
					\node[roundnode]      (n7) 	at 	(0,-1)           {7};
					\node[roundnode]      (n8) 	at 	(0,-3)           {8};
					\node[roundnode]      (n9) 	at 	(-2,2)           {9};
					\node[roundnode]      (n10) 	at 	(-2,0)           {10};
					\node[roundnode]      (n11) 	at 	(-2,-2)           {11};
					\node 		(label2) 	at 	(1,-4) 		{$\beta_{0}=1,\beta_{1}=4$};
					
					\begin{pgfonlayer}{background}
					
					\fill[fill=black!20,opacity=1] (n11.mid) -- (n8.mid) -- (n4.mid) -- cycle;
					\fill[fill=black!20,opacity=1] (n11.mid) -- (n8.mid) -- (n1.mid) -- cycle;
					\fill[fill=black!20,opacity=1] (n10.mid) to[bend right] (n4.mid) -- (n7.mid) -- cycle;
					\fill[fill=black!20,opacity=1] (n10.mid) -- (n6.mid) -- (n3.mid) -- cycle;
					\fill[fill=black!20,opacity=1] (n10.mid) to[bend left] (n2.mid) -- (n6.mid) -- cycle;
					\fill[fill=black!20,opacity=1] (n9.mid) to[bend right] (n3.mid) -- (n6.mid) -- cycle;
					\fill[fill=black!20,opacity=1] (n9.mid) -- (n6.mid) -- (n2.mid) -- cycle;
					\fill[fill=black!20,opacity=1] (n9.mid) -- (n6.mid) -- (n0.mid) -- cycle;
					\fill[fill=black!20,opacity=1] (n9.mid) -- (n5.mid) -- (n2.mid) -- cycle;
					\fill[fill=black!20,opacity=1] (n8.mid) to[bend right] (n1.mid) -- (n4.mid) -- cycle;
					\fill[fill=black!20,opacity=1] (n7.mid) to[bend right] (n0.mid) -- (n3.mid) -- cycle;
					\fill[fill=black!20,opacity=1] (n6.mid) to[bend left] (n1.mid) -- (n3.mid) -- cycle;
					\fill[fill=black!20,opacity=1] (n6.mid) -- (n3.mid) -- (n0.mid) -- cycle;
					%Lines
					\draw[thick] (n11.mid) --  (n8.mid);
					\draw[thick] (n11.mid) --  (n7.mid);
					\draw[thick] (n11.mid) --  (n4.mid);
					\draw[thick] (n11.mid) --  (n1.mid);
					\draw[thick] (n10.mid) --  (n7.mid);
					\draw[thick] (n10.mid) --  (n6.mid);
					\draw[thick] (n10.mid) --  (n3.mid);
					\draw[thick] (n9.mid) --  (n6.mid);
					\draw[thick] (n9.mid) --  (n5.mid);
					\draw[thick] (n9.mid) --  (n2.mid);
					\draw[thick] (n9.mid) --  (n0.mid);
					\draw[thick] (n8.mid) --  (n4.mid);
					\draw[thick] (n7.mid) --  (n4.mid);
					\draw[thick] (n7.mid) --  (n3.mid);
					\draw[thick] (n6.mid) --  (n3.mid);
					\draw[thick] (n6.mid) --  (n2.mid);
					\draw[thick] (n6.mid) --  (n0.mid);
					\draw[thick] (n5.mid) --  (n2.mid);
					\draw[thick] (n4.mid) --  (n1.mid);
					\draw[thick] (n3.mid) --  (n1.mid);
					\draw[thick] (n3.mid) --  (n0.mid);
					\draw[thick] (n2.mid) --  (n0.mid);
					\end{pgfonlayer}
				\end{tikzpicture}}	
			\end{figure}	
		\endminipage}
		\fbox{\minipage{0.225\textwidth}
			\begin{figure}[H]
				\resizebox{1.0\textwidth}{!}{\begin{tikzpicture}[
					roundnode/.style={circle, draw=black, thick, fill=white, minimum size=7mm},
					]
					%Nodes
					\node           (label) 	at	(1,4) 		{Filtración 5 (t=0.5)};
					\node[roundnode]      (n0) 	at 	(4,1)           {0};
					\node[roundnode]      (n1) 	at 	(4,-1)          {1};
					\node[roundnode]      (n2) 	at 	(2,2)           {2};
					\node[roundnode]      (n3) 	at 	(2,0)           {3};
					\node[roundnode]      (n4)      at 	(2,-2)		{4};
					\node[roundnode]      (n5)      at 	(0,3)		{5};
					\node[roundnode]      (n6)      at 	(0,1)		{6};
					\node[roundnode]      (n7) 	at 	(0,-1)           {7};
					\node[roundnode]      (n8) 	at 	(0,-3)           {8};
					\node[roundnode]      (n9) 	at 	(-2,2)           {9};
					\node[roundnode]      (n10) 	at 	(-2,0)           {10};
					\node[roundnode]      (n11) 	at 	(-2,-2)           {11};
					\node 		(label2) 	at 	(1,-4) 		{$\beta_{0}=3,\beta_{1}=0$};
					
					\begin{pgfonlayer}{background}
					
					\fill[fill=black!20,opacity=1] (n11.mid) -- (n8.mid) -- (n4.mid) -- cycle;
					\fill[fill=black!20,opacity=1] (n6.mid) -- (n3.mid) -- (n0.mid) -- cycle;
					%Lines
					\draw[thick] (n11.mid) --  (n8.mid);
					\draw[thick] (n11.mid) --  (n4.mid);
					\draw[thick] (n10.mid) --  (n7.mid);
					\draw[thick] (n9.mid) --  (n6.mid);
					\draw[thick] (n9.mid) --  (n5.mid);
					\draw[thick] (n8.mid) --  (n4.mid);
					\draw[thick] (n6.mid) --  (n3.mid);
					\draw[thick] (n6.mid) --  (n2.mid);
					\draw[thick] (n6.mid) --  (n0.mid);
					\draw[thick] (n3.mid) --  (n1.mid);
					\draw[thick] (n3.mid) --  (n0.mid);
					\end{pgfonlayer}
				\end{tikzpicture}}	
			\end{figure}	
			\begin{figure}[H]
				\resizebox{1.0\textwidth}{!}{\begin{tikzpicture}[
					roundnode/.style={circle, draw=black, thick, fill=white, minimum size=7mm},
					]
					%Nodes
					\node           (label) 	at	(1,4) 		{Filtración 9 (t=0.1)};
					\node[roundnode]      (n0) 	at 	(4,1)           {0};
					\node[roundnode]      (n1) 	at 	(4,-1)          {1};
					\node[roundnode]      (n2) 	at 	(2,2)           {2};
					\node[roundnode]      (n3) 	at 	(2,0)           {3};
					\node[roundnode]      (n4)      at 	(2,-2)		{4};
					\node[roundnode]      (n5)      at 	(0,3)		{5};
					\node[roundnode]      (n6)      at 	(0,1)		{6};
					\node[roundnode]      (n7) 	at 	(0,-1)           {7};
					\node[roundnode]      (n8) 	at 	(0,-3)           {8};
					\node[roundnode]      (n9) 	at 	(-2,2)           {9};
					\node[roundnode]      (n10) 	at 	(-2,0)           {10};
					\node[roundnode]      (n11) 	at 	(-2,-2)           {11};
					\node 		(label2) 	at 	(1,-4) 		{$\beta_{0}=1,\beta_{1}=1,$\hl{$\beta_{2}=1$}};
					
					\begin{pgfonlayer}{background}
					
					\fill[fill=black!20,opacity=1] (n11.mid) -- (n8.mid) -- (n4.mid) -- cycle;
					\fill[fill=black!20,opacity=1] (n11.mid) -- (n8.mid) -- (n1.mid) -- cycle;
					\fill[fill=black!20,opacity=1] (n10.mid) to[bend right] (n4.mid) -- (n7.mid) -- cycle;
					\fill[fill=black!20,opacity=1] (n10.mid) -- (n7.mid) -- (n3.mid) -- cycle;
					\fill[fill=black!20,opacity=1] (n10.mid) -- (n7.mid) -- (n1.mid) -- cycle;
					\fill[fill=black!20,opacity=1] (n10.mid) -- (n6.mid) -- (n3.mid) -- cycle;
					\fill[fill=black!20,opacity=1] (n10.mid) to[bend left] (n2.mid) -- (n6.mid) -- cycle;
					\fill[fill=black!20,opacity=1] (n9.mid) to[bend right] (n3.mid) -- (n6.mid) -- cycle;
					\fill[fill=black!20,opacity=1] (n9.mid) -- (n6.mid) -- (n2.mid) -- cycle;
					\fill[fill=black!20,opacity=1] (n9.mid) -- (n6.mid) -- (n0.mid) -- cycle;
					\fill[fill=black!20,opacity=1] (n9.mid) -- (n5.mid) -- (n2.mid) -- cycle;
					\fill[fill=black!20,opacity=1] (n9.mid) -- (n5.mid) -- (n0.mid) -- cycle;
					\fill[fill=black!20,opacity=1] (n8.mid) to[bend right] (n1.mid) -- (n4.mid) -- cycle;
					\fill[fill=black!20,opacity=1] (n7.mid) to[bend right] (n0.mid) -- (n3.mid) -- cycle;
					\fill[fill=black!20,opacity=1] (n7.mid) -- (n4.mid) -- (n1.mid) -- cycle;
					\fill[fill=black!20,opacity=1] (n7.mid) -- (n3.mid) -- (n1.mid) -- cycle;
					\fill[fill=black!20,opacity=1] (n6.mid) to[bend left] (n1.mid) -- (n3.mid) -- cycle;
					\fill[fill=black!20,opacity=1] (n6.mid) -- (n3.mid) -- (n0.mid) -- cycle;
					\fill[fill=black!20,opacity=1] (n5.mid) to[bend left] (n0.mid) -- (n2.mid) -- cycle;
					%Lines
					\draw[thick] (n11.mid) --  (n8.mid);
					\draw[thick] (n11.mid) --  (n7.mid);
					\draw[thick] (n11.mid) --  (n4.mid);
					\draw[thick] (n11.mid) --  (n1.mid);
					\draw[thick] (n10.mid) --  (n7.mid);
					\draw[thick] (n10.mid) --  (n6.mid);
					\draw[thick] (n10.mid) --  (n3.mid);
					\draw[thick] (n10.mid) --  (n1.mid);
					\draw[thick] (n10.mid) --  (n0.mid);
					\draw[thick] (n9.mid) --  (n6.mid);
					\draw[thick] (n9.mid) --  (n5.mid);
					\draw[thick] (n9.mid) --  (n2.mid);
					\draw[thick] (n9.mid) --  (n0.mid);
					\draw[thick] (n8.mid) --  (n4.mid);
					\draw[thick] (n7.mid) --  (n4.mid);
					\draw[thick] (n7.mid) --  (n3.mid);
					\draw[thick] (n7.mid) --  (n1.mid);
					\draw[thick] (n6.mid) --  (n3.mid);
					\draw[thick] (n6.mid) --  (n2.mid);
					\draw[thick] (n6.mid) --  (n0.mid);
					\draw[thick] (n5.mid) --  (n2.mid);
					\draw[thick] (n4.mid) --  (n1.mid);
					\draw[thick] (n3.mid) --  (n1.mid);
					\draw[thick] (n3.mid) --  (n0.mid);
					\draw[thick] (n2.mid) --  (n0.mid);
					\end{pgfonlayer}
				\end{tikzpicture}}	
			\end{figure}	
	\endminipage}
		\fbox{\minipage{0.225\textwidth}
			\begin{figure}[H]
				\resizebox{1.0\textwidth}{!}{\begin{tikzpicture}[
					roundnode/.style={circle, draw=black, fill=white, thick, minimum size=7mm},
					]
					%Nodes
					\node           (label) 	at	(1,4) 		{Filtración 6 (t=0.4)};
					\node[roundnode]      (n0) 	at 	(4,1)           {0};
					\node[roundnode]      (n1) 	at 	(4,-1)          {1};
					\node[roundnode]      (n2) 	at 	(2,2)           {2};
					\node[roundnode]      (n3) 	at 	(2,0)           {3};
					\node[roundnode]      (n4)      at 	(2,-2)		{4};
					\node[roundnode]      (n5)      at 	(0,3)		{5};
					\node[roundnode]      (n6)      at 	(0,1)		{6};
					\node[roundnode]      (n7) 	at 	(0,-1)           {7};
					\node[roundnode]      (n8) 	at 	(0,-3)           {8};
					\node[roundnode]      (n9) 	at 	(-2,2)           {9};
					\node[roundnode]      (n10) 	at 	(-2,0)           {10};
					\node[roundnode]      (n11) 	at 	(-2,-2)           {11};
					\node 		(label2) 	at 	(1,-4) 		{$\beta_{0}=3,\beta_{1}=1$};
					
					\begin{pgfonlayer}{background}
					
					\fill[fill=black!20,opacity=1] (n11.mid) -- (n8.mid) -- (n4.mid) -- cycle;
					\fill[fill=black!20,opacity=1] (n9.mid) to[bend right] (n3.mid) -- (n6.mid) -- cycle;
					\fill[fill=black!20,opacity=1] (n9.mid) -- (n5.mid) -- (n2.mid) -- cycle;
					\fill[fill=black!20,opacity=1] (n6.mid) to[bend left] (n1.mid) -- (n3.mid) -- cycle;
					\fill[fill=black!20,opacity=1] (n6.mid) -- (n3.mid) -- (n0.mid) -- cycle;
					%Lines
					\draw[thick] (n11.mid) --  (n8.mid);
					\draw[thick] (n11.mid) --  (n4.mid);
					\draw[thick] (n10.mid) --  (n7.mid);
					\draw[thick] (n9.mid) --  (n6.mid);
					\draw[thick] (n9.mid) --  (n5.mid);
					\draw[thick] (n9.mid) --  (n2.mid);
					\draw[thick] (n8.mid) --  (n4.mid);
					\draw[thick] (n6.mid) --  (n3.mid);
					\draw[thick] (n6.mid) --  (n2.mid);
					\draw[thick] (n6.mid) --  (n0.mid);
					\draw[thick] (n5.mid) --  (n2.mid);
					\draw[thick] (n3.mid) --  (n1.mid);
					\draw[thick] (n3.mid) --  (n0.mid);
					\end{pgfonlayer}
				\end{tikzpicture}}	
			\end{figure}	
			\begin{figure}[H]
				\resizebox{1.0\textwidth}{!}{\begin{tikzpicture}[
					roundnode/.style={circle, draw=black, thick, fill=white, minimum size=7mm},
					]
					%Nodes
					\node           (label) 	at	(1,4) 		{Filtración 10 (t=0.0)};
					\node[roundnode]      (n0) 	at 	(4,1)           {0};
					\node[roundnode]      (n1) 	at 	(4,-1)          {1};
					\node[roundnode]      (n2) 	at 	(2,2)           {2};
					\node[roundnode]      (n3) 	at 	(2,0)           {3};
					\node[roundnode]      (n4)      at 	(2,-2)		{4};
					\node[roundnode]      (n5)      at 	(0,3)		{5};
					\node[roundnode]      (n6)      at 	(0,1)		{6};
					\node[roundnode]      (n7) 	at 	(0,-1)           {7};
					\node[roundnode]      (n8) 	at 	(0,-3)           {8};
					\node[roundnode]      (n9) 	at 	(-2,2)           {9};
					\node[roundnode]      (n10) 	at 	(-2,0)           {10};
					\node[roundnode]      (n11) 	at 	(-2,-2)           {11};
					\node 		(label2) 	at 	(1,-4) 		{$\beta_{0}=1,\beta_{1}=0$};
					
					\begin{pgfonlayer}{background}
					
					\fill[fill=black!20,opacity=1] (n11.mid) -- (n8.mid) -- (n4.mid) -- cycle;
					\fill[fill=black!20,opacity=1] (n11.mid) -- (n8.mid) -- (n1.mid) -- cycle;
					\fill[fill=black!20,opacity=1] (n11.mid) -- (n7.mid) -- (n4.mid) -- cycle;
					\fill[fill=black!20,opacity=1] (n11.mid) to[bend left] (n3.mid) -- (n7.mid) -- cycle;
					\fill[fill=black!20,opacity=1] (n10.mid) to[bend right] (n4.mid) -- (n7.mid) -- cycle;
					\fill[fill=black!20,opacity=1] (n10.mid) -- (n7.mid) -- (n3.mid) -- cycle;
					\fill[fill=black!20,opacity=1] (n10.mid) -- (n7.mid) -- (n1.mid) -- cycle;
					\fill[fill=black!20,opacity=1] (n10.mid) -- (n6.mid) -- (n3.mid) -- cycle;
					\fill[fill=black!20,opacity=1] (n10.mid) to[bend left] (n2.mid) -- (n6.mid) -- cycle;
					\fill[fill=black!20,opacity=1] (n9.mid) to[bend right] (n3.mid) -- (n6.mid) -- cycle;
					\fill[fill=black!20,opacity=1] (n9.mid) -- (n6.mid) -- (n2.mid) -- cycle;
					\fill[fill=black!20,opacity=1] (n9.mid) -- (n6.mid) -- (n0.mid) -- cycle;
					\fill[fill=black!20,opacity=1] (n9.mid) -- (n5.mid) -- (n2.mid) -- cycle;
					\fill[fill=black!20,opacity=1] (n9.mid) -- (n5.mid) -- (n0.mid) -- cycle;
					\fill[fill=black!20,opacity=1] (n8.mid) to[bend right] (n1.mid) -- (n4.mid) -- cycle;
					\fill[fill=black!20,opacity=1] (n7.mid) to[bend right] (n0.mid) -- (n3.mid) -- cycle;
					\fill[fill=black!20,opacity=1] (n7.mid) -- (n4.mid) -- (n1.mid) -- cycle;
					\fill[fill=black!20,opacity=1] (n7.mid) -- (n3.mid) -- (n1.mid) -- cycle;
					\fill[fill=black!20,opacity=1] (n6.mid) to[bend left] (n1.mid) -- (n3.mid) -- cycle;
					\fill[fill=black!20,opacity=1] (n6.mid) -- (n3.mid) -- (n0.mid) -- cycle;
					\fill[fill=black!20,opacity=1] (n5.mid) to[bend left] (n0.mid) -- (n2.mid) -- cycle;
					%Lines
					\draw[thick] (n11.mid) --  (n8.mid);
					\draw[thick] (n11.mid) --  (n7.mid);
					\draw[thick] (n11.mid) --  (n4.mid);
					\draw[thick] (n11.mid) --  (n1.mid);
					\draw[thick] (n10.mid) --  (n7.mid);
					\draw[thick] (n10.mid) --  (n6.mid);
					\draw[thick] (n10.mid) --  (n3.mid);
					\draw[thick] (n10.mid) --  (n1.mid);
					\draw[thick] (n10.mid) --  (n0.mid);
					\draw[thick] (n9.mid) --  (n6.mid);
					\draw[thick] (n9.mid) --  (n5.mid);
					\draw[thick] (n9.mid) --  (n2.mid);
					\draw[thick] (n9.mid) --  (n0.mid);
					\draw[thick] (n8.mid) --  (n4.mid);
					\draw[thick] (n7.mid) --  (n4.mid);
					\draw[thick] (n7.mid) --  (n3.mid);
					\draw[thick] (n7.mid) --  (n1.mid);
					\draw[thick] (n6.mid) --  (n3.mid);
					\draw[thick] (n6.mid) --  (n2.mid);
					\draw[thick] (n6.mid) --  (n0.mid);
					\draw[thick] (n5.mid) --  (n2.mid);
					\draw[thick] (n4.mid) --  (n1.mid);
					\draw[thick] (n3.mid) --  (n1.mid);
					\draw[thick] (n3.mid) --  (n0.mid);
					\draw[thick] (n2.mid) --  (n0.mid);
					\end{pgfonlayer}
				\end{tikzpicture}}	
			\end{figure}	
		\endminipage}
		\caption{Pasos en la filtración del complejo simplicial 
		asociado a \ref{fig:distIntR} y números de Betti 
		correspondientes (interpretación local).}
		\label{fig:distIntRSL}
	\end{figure}
	\begin{figure}[!htbp]
		\minipage{0.5\textwidth}
			\begin{figure}[H]
				\resizebox{1.0\textwidth}{!}{\includegraphics{Images/DiagramaBarrasEjINTLOCAL.png}}
			\end{figure}
		\endminipage
		\minipage{0.5\textwidth}
			\begin{figure}[H]
				\resizebox{1.0\textwidth}{!}{\includegraphics{Images/DiagramaPersistenciaEjINTLOCAL.png}}
			\end{figure}
		\endminipage
		\caption{Diagrama de barras (izquierda) y diagrama de 
		persistencia (derecha) asociados a la filtración ilustrada en 
		\ref{fig:distIntRSL} (interpretación local).}
		\label{fig:distIntRDL}
	\end{figure}
	\begin{figure}[!htbp]
		\fbox{\minipage{0.225\textwidth}
			\begin{figure}[H]
				\resizebox{1.0\textwidth}{!}{\begin{tikzpicture}[
					roundnode/.style={circle, draw=black, thick, fill=white, minimum size=7mm},
					]
					%Nodes
					\node           (label) 	at	(1,4) 		{Filtración 0 (t=1.0)};
					\node[roundnode]      (n0) 	at 	(4,1)           {0};
					\node[roundnode]      (n1) 	at 	(4,-1)          {1};
					\node[roundnode]      (n2) 	at 	(2,2)           {2};
					\node[roundnode]      (n3) 	at 	(2,0)           {3};
					\node[roundnode]      (n4)      at 	(2,-2)		{4};
					\node[roundnode]      (n5)      at 	(0,3)		{5};
					\node[roundnode]      (n6)      at 	(0,1)		{6};
					\node[roundnode]      (n7) 	at 	(0,-1)           {7};
					\node[roundnode]      (n8) 	at 	(0,-3)           {8};
					\node[roundnode]      (n9) 	at 	(-2,2)           {9};
					\node[roundnode]      (n10) 	at 	(-2,0)           {10};
					\node[roundnode]      (n11) 	at 	(-2,-2)           {11};
					\node 		(label2) 	at 	(1,-4) 		{$\beta_{0}=10,\beta_{1}=0$};
					
					\begin{pgfonlayer}{background}
					%Lines
					\draw[thick] (n11.mid) --  (n8);
					\draw[thick] (n9.mid) --  (n5);
					\end{pgfonlayer}
				\end{tikzpicture}}	
			\end{figure}	
			\begin{figure}[H]
				\resizebox{1.0\textwidth}{!}{\begin{tikzpicture}[
					roundnode/.style={circle, draw=black, thick, fill=white, minimum size=7mm},
					]
					%Nodes
					\node           (label) 	at	(1,4) 		{Filtración 7 (t=0.3)};
					\node[roundnode]      (n0) 	at 	(4,1)           {0};
					\node[roundnode]      (n1) 	at 	(4,-1)          {1};
					\node[roundnode]      (n2) 	at 	(2,2)           {2};
					\node[roundnode]      (n3) 	at 	(2,0)           {3};
					\node[roundnode]      (n4)      at 	(2,-2)		{4};
					\node[roundnode]      (n5)      at 	(0,3)		{5};
					\node[roundnode]      (n6)      at 	(0,1)		{6};
					\node[roundnode]      (n7) 	at 	(0,-1)           {7};
					\node[roundnode]      (n8) 	at 	(0,-3)           {8};
					\node[roundnode]      (n9) 	at 	(-2,2)           {9};
					\node[roundnode]      (n10) 	at 	(-2,0)           {10};
					\node[roundnode]      (n11) 	at 	(-2,-2)           {11};
					\node 		(label2) 	at 	(1,-4) 		{$\beta_{0}=1,\beta_{1}=1$};
					
					\begin{pgfonlayer}{background}
					
					\fill[fill=black!20,opacity=1] (n11.mid) -- (n8.mid) -- (n4.mid) -- cycle;
					\fill[fill=black!20,opacity=1] (n11.mid) -- (n7.mid) -- (n4.mid) -- cycle;
					\fill[fill=black!20,opacity=1] (n9.mid) to[bend right] (n3.mid) -- (n6.mid) -- cycle;
					\fill[fill=black!20,opacity=1] (n9.mid) -- (n6.mid) -- (n2.mid) -- cycle;
					\fill[fill=black!20,opacity=1] (n9.mid) -- (n6.mid) -- (n0.mid) -- cycle;
					\fill[fill=black!20,opacity=1] (n9.mid) -- (n5.mid) -- (n2.mid) -- cycle;
					\fill[fill=black!20,opacity=1] (n6.mid) to[bend left] (n1.mid) -- (n3.mid) -- cycle;
					\fill[fill=black!20,opacity=1] (n6.mid) -- (n3.mid) -- (n0.mid) -- cycle;
					%Lines
					\draw[thick] (n11.mid) --  (n8.mid);
					\draw[thick] (n11.mid) --  (n7.mid);
					\draw[thick] (n11.mid) --  (n4.mid);
					\draw[thick] (n10.mid) --  (n7.mid);
					\draw[thick] (n10.mid) --  (n6.mid);
					\draw[thick] (n9.mid) --  (n6.mid);
					\draw[thick] (n9.mid) --  (n5.mid);
					\draw[thick] (n9.mid) --  (n2.mid);
					\draw[thick] (n9.mid) --  (n0.mid);
					\draw[thick] (n8.mid) --  (n4.mid);
					\draw[thick] (n7.mid) --  (n4.mid);
					\draw[thick] (n6.mid) --  (n3.mid);
					\draw[thick] (n6.mid) --  (n2.mid);
					\draw[thick] (n6.mid) --  (n0.mid);
					\draw[thick] (n5.mid) --  (n2.mid);
					\draw[thick] (n4.mid) --  (n1.mid);
					\draw[thick] (n3.mid) --  (n1.mid);
					\draw[thick] (n3.mid) --  (n0.mid);
					\end{pgfonlayer}
				\end{tikzpicture}}	
			\end{figure}	
		\endminipage}
		\fbox{\minipage{0.225\textwidth}
			\begin{figure}[H]
				\resizebox{1.0\textwidth}{!}{\begin{tikzpicture}[
					roundnode/.style={circle, draw=black, thick, fill=white, minimum size=7mm},
					]
					%Nodes
					\node           (label) 	at	(1,4) 		{Filtración 4 (t=0.6)};
					\node[roundnode]      (n0) 	at 	(4,1)           {0};
					\node[roundnode]      (n1) 	at 	(4,-1)          {1};
					\node[roundnode]      (n2) 	at 	(2,2)           {2};
					\node[roundnode]      (n3) 	at 	(2,0)           {3};
					\node[roundnode]      (n4)      at 	(2,-2)		{4};
					\node[roundnode]      (n5)      at 	(0,3)		{5};
					\node[roundnode]      (n6)      at 	(0,1)		{6};
					\node[roundnode]      (n7) 	at 	(0,-1)           {7};
					\node[roundnode]      (n8) 	at 	(0,-3)           {8};
					\node[roundnode]      (n9) 	at 	(-2,2)           {9};
					\node[roundnode]      (n10) 	at 	(-2,0)           {10};
					\node[roundnode]      (n11) 	at 	(-2,-2)           {11};
					\node 		(label2) 	at 	(1,-4) 		{$\beta_{0}=3,\beta_{1}=0$};
					
					\begin{pgfonlayer}{background}
					
					\fill[fill=black!20,opacity=1] (n11.mid) -- (n8.mid) -- (n4.mid) -- cycle;
					%Lines
					\draw[thick] (n11.mid) --  (n8.mid);
					\draw[thick] (n11.mid) --  (n4.mid);
					\draw[thick] (n10.mid) --  (n7.mid);
					\draw[thick] (n9.mid) --  (n6.mid);
					\draw[thick] (n9.mid) --  (n5.mid);
					\draw[thick] (n8.mid) --  (n4.mid);
					\draw[thick] (n6.mid) --  (n3.mid);
					\draw[thick] (n6.mid) --  (n2.mid);
					\draw[thick] (n3.mid) --  (n1.mid);
					\draw[thick] (n3.mid) --  (n0.mid);
					\end{pgfonlayer}
				\end{tikzpicture}}	
			\end{figure}	
			\begin{figure}[H]
				\resizebox{1.0\textwidth}{!}{\begin{tikzpicture}[
					roundnode/.style={circle, draw=black, thick, fill=white, minimum size=7mm},
					]
					%Nodes
					\node           (label) 	at	(1,4) 		{Filtración 8 (t=0.2)};
					\node[roundnode]      (n0) 	at 	(4,1)           {0};
					\node[roundnode]      (n1) 	at 	(4,-1)          {1};
					\node[roundnode]      (n2) 	at 	(2,2)           {2};
					\node[roundnode]      (n3) 	at 	(2,0)           {3};
					\node[roundnode]      (n4)      at 	(2,-2)		{4};
					\node[roundnode]      (n5)      at 	(0,3)		{5};
					\node[roundnode]      (n6)      at 	(0,1)		{6};
					\node[roundnode]      (n7) 	at 	(0,-1)           {7};
					\node[roundnode]      (n8) 	at 	(0,-3)           {8};
					\node[roundnode]      (n9) 	at 	(-2,2)           {9};
					\node[roundnode]      (n10) 	at 	(-2,0)           {10};
					\node[roundnode]      (n11) 	at 	(-2,-2)           {11};
					\node 		(label2) 	at 	(1,-4) 		{$\beta_{0}=1,\beta_{1}=1$};
					
					\begin{pgfonlayer}{background}
					
					\fill[fill=black!20,opacity=1] (n11.mid) -- (n8.mid) -- (n4.mid) -- cycle;
					\fill[fill=black!20,opacity=1] (n11.mid) -- (n8.mid) -- (n1.mid) -- cycle;
					\fill[fill=black!20,opacity=1] (n11.mid) -- (n7.mid) -- (n4.mid) -- cycle;
					\fill[fill=black!20,opacity=1] (n10.mid) to[bend right] (n4.mid) -- (n7.mid) -- cycle;
					\fill[fill=black!20,opacity=1] (n10.mid) -- (n7.mid) -- (n3.mid) -- cycle;
					\fill[fill=black!20,opacity=1] (n10.mid) -- (n6.mid) -- (n3.mid) -- cycle;
					\fill[fill=black!20,opacity=1] (n10.mid) to[bend left] (n2.mid) -- (n6.mid) -- cycle;
					\fill[fill=black!20,opacity=1] (n9.mid) to[bend right] (n3.mid) -- (n6.mid) -- cycle;
					\fill[fill=black!20,opacity=1] (n9.mid) -- (n6.mid) -- (n2.mid) -- cycle;
					\fill[fill=black!20,opacity=1] (n9.mid) -- (n6.mid) -- (n0.mid) -- cycle;
					\fill[fill=black!20,opacity=1] (n9.mid) -- (n5.mid) -- (n2.mid) -- cycle;
					\fill[fill=black!20,opacity=1] (n8.mid) to[bend right] (n1.mid) -- (n4.mid) -- cycle;
					\fill[fill=black!20,opacity=1] (n7.mid) to[bend right] (n0.mid) -- (n3.mid) -- cycle;
					\fill[fill=black!20,opacity=1] (n6.mid) to[bend left] (n1.mid) -- (n3.mid) -- cycle;
					\fill[fill=black!20,opacity=1] (n6.mid) -- (n3.mid) -- (n0.mid) -- cycle;
					\fill[fill=black!20,opacity=1] (n6.mid) -- (n2.mid) -- (n0.mid) -- cycle;
					%Lines
					\draw[thick] (n11.mid) --  (n8.mid);
					\draw[thick] (n11.mid) --  (n7.mid);
					\draw[thick] (n11.mid) --  (n4.mid);
					\draw[thick] (n11.mid) --  (n1.mid);
					\draw[thick] (n10.mid) --  (n7.mid);
					\draw[thick] (n10.mid) --  (n6.mid);
					\draw[thick] (n10.mid) --  (n3.mid);
					\draw[thick] (n9.mid) --  (n6.mid);
					\draw[thick] (n9.mid) --  (n5.mid);
					\draw[thick] (n9.mid) --  (n2.mid);
					\draw[thick] (n9.mid) --  (n0.mid);
					\draw[thick] (n8.mid) --  (n4.mid);
					\draw[thick] (n7.mid) --  (n4.mid);
					\draw[thick] (n7.mid) --  (n3.mid);
					\draw[thick] (n6.mid) --  (n3.mid);
					\draw[thick] (n6.mid) --  (n2.mid);
					\draw[thick] (n6.mid) --  (n0.mid);
					\draw[thick] (n5.mid) --  (n2.mid);
					\draw[thick] (n4.mid) --  (n1.mid);
					\draw[thick] (n3.mid) --  (n1.mid);
					\draw[thick] (n3.mid) --  (n0.mid);
					\draw[thick] (n2.mid) --  (n0.mid);
					\end{pgfonlayer}
				\end{tikzpicture}}	
			\end{figure}	
		\endminipage}
		\fbox{\minipage{0.225\textwidth}
			\begin{figure}[H]
				\resizebox{1.0\textwidth}{!}{\begin{tikzpicture}[
					roundnode/.style={circle, draw=black, thick, fill=white, minimum size=7mm},
					]
					%Nodes
					\node           (label) 	at	(1,4) 		{Filtración 5 (t=0.5)};
					\node[roundnode]      (n0) 	at 	(4,1)           {0};
					\node[roundnode]      (n1) 	at 	(4,-1)          {1};
					\node[roundnode]      (n2) 	at 	(2,2)           {2};
					\node[roundnode]      (n3) 	at 	(2,0)           {3};
					\node[roundnode]      (n4)      at 	(2,-2)		{4};
					\node[roundnode]      (n5)      at 	(0,3)		{5};
					\node[roundnode]      (n6)      at 	(0,1)		{6};
					\node[roundnode]      (n7) 	at 	(0,-1)           {7};
					\node[roundnode]      (n8) 	at 	(0,-3)           {8};
					\node[roundnode]      (n9) 	at 	(-2,2)           {9};
					\node[roundnode]      (n10) 	at 	(-2,0)           {10};
					\node[roundnode]      (n11) 	at 	(-2,-2)           {11};
					\node 		(label2) 	at 	(1,-4) 		{$\beta_{0}=3,\beta_{1}=0$};
					
					\begin{pgfonlayer}{background}
					
					\fill[fill=black!20,opacity=1] (n11.mid) -- (n8.mid) -- (n4.mid) -- cycle;
					\fill[fill=black!20,opacity=1] (n6.mid) -- (n3.mid) -- (n0.mid) -- cycle;
					%Lines
					\draw[thick] (n11.mid) --  (n8.mid);
					\draw[thick] (n11.mid) --  (n4.mid);
					\draw[thick] (n10.mid) --  (n7.mid);
					\draw[thick] (n9.mid) --  (n6.mid);
					\draw[thick] (n9.mid) --  (n5.mid);
					\draw[thick] (n8.mid) --  (n4.mid);
					\draw[thick] (n6.mid) --  (n3.mid);
					\draw[thick] (n6.mid) --  (n2.mid);
					\draw[thick] (n6.mid) --  (n0.mid);
					\draw[thick] (n3.mid) --  (n1.mid);
					\draw[thick] (n3.mid) --  (n0.mid);
					\end{pgfonlayer}
				\end{tikzpicture}}	
			\end{figure}	
			\begin{figure}[H]
				\resizebox{1.0\textwidth}{!}{\begin{tikzpicture}[
					roundnode/.style={circle, draw=black, thick, fill=white, minimum size=7mm},
					]
					%Nodes
					\node           (label) 	at	(1,4) 		{Filtración 9 (t=0.1)};
					\node[roundnode]      (n0) 	at 	(4,1)           {0};
					\node[roundnode]      (n1) 	at 	(4,-1)          {1};
					\node[roundnode]      (n2) 	at 	(2,2)           {2};
					\node[roundnode]      (n3) 	at 	(2,0)           {3};
					\node[roundnode]      (n4)      at 	(2,-2)		{4};
					\node[roundnode]      (n5)      at 	(0,3)		{5};
					\node[roundnode]      (n6)      at 	(0,1)		{6};
					\node[roundnode]      (n7) 	at 	(0,-1)           {7};
					\node[roundnode]      (n8) 	at 	(0,-3)           {8};
					\node[roundnode]      (n9) 	at 	(-2,2)           {9};
					\node[roundnode]      (n10) 	at 	(-2,0)           {10};
					\node[roundnode]      (n11) 	at 	(-2,-2)           {11};
					\node 		(label2) 	at 	(1,-4) 		{$\beta_{0}=1,\beta_{1}=0$};
					
					\begin{pgfonlayer}{background}
					
					\fill[fill=black!20,opacity=1] (n11.mid) -- (n8.mid) -- (n4.mid) -- cycle;
					\fill[fill=black!20,opacity=1] (n11.mid) -- (n8.mid) -- (n1.mid) -- cycle;
					\fill[fill=black!20,opacity=1] (n11.mid) -- (n7.mid) -- (n4.mid) -- cycle;
					\fill[fill=black!20,opacity=1] (n10.mid) to[bend right] (n4.mid) -- (n7.mid) -- cycle;
					\fill[fill=black!20,opacity=1] (n10.mid) -- (n7.mid) -- (n3.mid) -- cycle;
					\fill[fill=black!20,opacity=1] (n10.mid) -- (n7.mid) -- (n1.mid) -- cycle;
					\fill[fill=black!20,opacity=1] (n10.mid) -- (n6.mid) -- (n3.mid) -- cycle;
					\fill[fill=black!20,opacity=1] (n10.mid) to[bend left] (n2.mid) -- (n6.mid) -- cycle;
					\fill[fill=black!20,opacity=1] (n9.mid) to[bend right] (n3.mid) -- (n6.mid) -- cycle;
					\fill[fill=black!20,opacity=1] (n9.mid) -- (n6.mid) -- (n2.mid) -- cycle;
					\fill[fill=black!20,opacity=1] (n9.mid) -- (n6.mid) -- (n0.mid) -- cycle;
					\fill[fill=black!20,opacity=1] (n9.mid) -- (n5.mid) -- (n2.mid) -- cycle;
					\fill[fill=black!20,opacity=1] (n9.mid) -- (n5.mid) -- (n0.mid) -- cycle;
					\fill[fill=black!20,opacity=1] (n8.mid) to[bend right] (n1.mid) -- (n4.mid) -- cycle;
					\fill[fill=black!20,opacity=1] (n7.mid) to[bend right] (n0.mid) -- (n3.mid) -- cycle;
					\fill[fill=black!20,opacity=1] (n7.mid) -- (n4.mid) -- (n1.mid) -- cycle;
					\fill[fill=black!20,opacity=1] (n7.mid) -- (n3.mid) -- (n1.mid) -- cycle;
					\fill[fill=black!20,opacity=1] (n6.mid) to[bend left] (n1.mid) -- (n3.mid) -- cycle;
					\fill[fill=black!20,opacity=1] (n6.mid) -- (n3.mid) -- (n0.mid) -- cycle;
					\fill[fill=black!20,opacity=1] (n5.mid) to[bend left] (n0.mid) -- (n2.mid) -- cycle;
					%Lines
					\draw[thick] (n11.mid) --  (n8.mid);
					\draw[thick] (n11.mid) --  (n7.mid);
					\draw[thick] (n11.mid) --  (n4.mid);
					\draw[thick] (n11.mid) --  (n1.mid);
					\draw[thick] (n10.mid) --  (n7.mid);
					\draw[thick] (n10.mid) --  (n6.mid);
					\draw[thick] (n10.mid) --  (n3.mid);
					\draw[thick] (n10.mid) --  (n1.mid);
					\draw[thick] (n10.mid) --  (n0.mid);
					\draw[thick] (n9.mid) --  (n6.mid);
					\draw[thick] (n9.mid) --  (n5.mid);
					\draw[thick] (n9.mid) --  (n2.mid);
					\draw[thick] (n9.mid) --  (n0.mid);
					\draw[thick] (n8.mid) --  (n4.mid);
					\draw[thick] (n7.mid) --  (n4.mid);
					\draw[thick] (n7.mid) --  (n3.mid);
					\draw[thick] (n7.mid) --  (n1.mid);
					\draw[thick] (n6.mid) --  (n3.mid);
					\draw[thick] (n6.mid) --  (n2.mid);
					\draw[thick] (n6.mid) --  (n0.mid);
					\draw[thick] (n5.mid) --  (n2.mid);
					\draw[thick] (n4.mid) --  (n1.mid);
					\draw[thick] (n3.mid) --  (n1.mid);
					\draw[thick] (n3.mid) --  (n0.mid);
					\draw[thick] (n2.mid) --  (n0.mid);
					\end{pgfonlayer}
				\end{tikzpicture}}	
			\end{figure}	
	\endminipage}
		\fbox{\minipage{0.225\textwidth}
			\begin{figure}[H]
				\resizebox{1.0\textwidth}{!}{\begin{tikzpicture}[
					roundnode/.style={circle, draw=black, fill=white, thick, minimum size=7mm},
					]
					%Nodes
					\node           (label) 	at	(1,4) 		{Filtración 6 (t=0.4)};
					\node[roundnode]      (n0) 	at 	(4,1)           {0};
					\node[roundnode]      (n1) 	at 	(4,-1)          {1};
					\node[roundnode]      (n2) 	at 	(2,2)           {2};
					\node[roundnode]      (n3) 	at 	(2,0)           {3};
					\node[roundnode]      (n4)      at 	(2,-2)		{4};
					\node[roundnode]      (n5)      at 	(0,3)		{5};
					\node[roundnode]      (n6)      at 	(0,1)		{6};
					\node[roundnode]      (n7) 	at 	(0,-1)           {7};
					\node[roundnode]      (n8) 	at 	(0,-3)           {8};
					\node[roundnode]      (n9) 	at 	(-2,2)           {9};
					\node[roundnode]      (n10) 	at 	(-2,0)           {10};
					\node[roundnode]      (n11) 	at 	(-2,-2)           {11};
					\node 		(label2) 	at 	(1,-4) 		{$\beta_{0}=3,\beta_{1}=0$};
					
					\begin{pgfonlayer}{background}
					
					\fill[fill=black!20,opacity=1] (n11.mid) -- (n8.mid) -- (n4.mid) -- cycle;
					\fill[fill=black!20,opacity=1] (n9.mid) to[bend right] (n3.mid) -- (n6.mid) -- cycle;
					\fill[fill=black!20,opacity=1] (n9.mid) -- (n6.mid) -- (n2.mid) -- cycle;
					\fill[fill=black!20,opacity=1] (n9.mid) -- (n5.mid) -- (n2.mid) -- cycle;
					\fill[fill=black!20,opacity=1] (n6.mid) to[bend left] (n1.mid) -- (n3.mid) -- cycle;
					\fill[fill=black!20,opacity=1] (n6.mid) -- (n3.mid) -- (n0.mid) -- cycle;
					%Lines
					\draw[thick] (n11.mid) --  (n8.mid);
					\draw[thick] (n11.mid) --  (n4.mid);
					\draw[thick] (n10.mid) --  (n7.mid);
					\draw[thick] (n9.mid) --  (n6.mid);
					\draw[thick] (n9.mid) --  (n5.mid);
					\draw[thick] (n9.mid) --  (n2.mid);
					\draw[thick] (n8.mid) --  (n4.mid);
					\draw[thick] (n6.mid) --  (n3.mid);
					\draw[thick] (n6.mid) --  (n2.mid);
					\draw[thick] (n6.mid) --  (n0.mid);
					\draw[thick] (n5.mid) --  (n2.mid);
					\draw[thick] (n3.mid) --  (n1.mid);
					\draw[thick] (n3.mid) --  (n0.mid);
					\end{pgfonlayer}
				\end{tikzpicture}}	
			\end{figure}	
			\begin{figure}[H]
				\resizebox{1.0\textwidth}{!}{\begin{tikzpicture}[
					roundnode/.style={circle, draw=black, thick, fill=white, minimum size=7mm},
					]
					%Nodes
					\node           (label) 	at	(1,4) 		{Filtración 10 (t=0.0)};
					\node[roundnode]      (n0) 	at 	(4,1)           {0};
					\node[roundnode]      (n1) 	at 	(4,-1)          {1};
					\node[roundnode]      (n2) 	at 	(2,2)           {2};
					\node[roundnode]      (n3) 	at 	(2,0)           {3};
					\node[roundnode]      (n4)      at 	(2,-2)		{4};
					\node[roundnode]      (n5)      at 	(0,3)		{5};
					\node[roundnode]      (n6)      at 	(0,1)		{6};
					\node[roundnode]      (n7) 	at 	(0,-1)           {7};
					\node[roundnode]      (n8) 	at 	(0,-3)           {8};
					\node[roundnode]      (n9) 	at 	(-2,2)           {9};
					\node[roundnode]      (n10) 	at 	(-2,0)           {10};
					\node[roundnode]      (n11) 	at 	(-2,-2)           {11};
					\node 		(label2) 	at 	(1,-4) 		{$\beta_{0}=1,\beta_{1}=0$};
					
					\begin{pgfonlayer}{background}
					
					\fill[fill=black!20,opacity=1] (n11.mid) -- (n8.mid) -- (n4.mid) -- cycle;
					\fill[fill=black!20,opacity=1] (n11.mid) -- (n8.mid) -- (n1.mid) -- cycle;
					\fill[fill=black!20,opacity=1] (n11.mid) -- (n7.mid) -- (n4.mid) -- cycle;
					\fill[fill=black!20,opacity=1] (n11.mid) to[bend left] (n3.mid) -- (n7.mid) -- cycle;
					\fill[fill=black!20,opacity=1] (n10.mid) to[bend right] (n4.mid) -- (n7.mid) -- cycle;
					\fill[fill=black!20,opacity=1] (n10.mid) -- (n7.mid) -- (n3.mid) -- cycle;
					\fill[fill=black!20,opacity=1] (n10.mid) -- (n7.mid) -- (n1.mid) -- cycle;
					\fill[fill=black!20,opacity=1] (n10.mid) -- (n6.mid) -- (n3.mid) -- cycle;
					\fill[fill=black!20,opacity=1] (n10.mid) to[bend left] (n2.mid) -- (n6.mid) -- cycle;
					\fill[fill=black!20,opacity=1] (n9.mid) to[bend right] (n3.mid) -- (n6.mid) -- cycle;
					\fill[fill=black!20,opacity=1] (n9.mid) -- (n6.mid) -- (n2.mid) -- cycle;
					\fill[fill=black!20,opacity=1] (n9.mid) -- (n6.mid) -- (n0.mid) -- cycle;
					\fill[fill=black!20,opacity=1] (n9.mid) -- (n5.mid) -- (n2.mid) -- cycle;
					\fill[fill=black!20,opacity=1] (n9.mid) -- (n5.mid) -- (n0.mid) -- cycle;
					\fill[fill=black!20,opacity=1] (n8.mid) to[bend right] (n1.mid) -- (n4.mid) -- cycle;
					\fill[fill=black!20,opacity=1] (n7.mid) to[bend right] (n0.mid) -- (n3.mid) -- cycle;
					\fill[fill=black!20,opacity=1] (n7.mid) -- (n4.mid) -- (n1.mid) -- cycle;
					\fill[fill=black!20,opacity=1] (n7.mid) -- (n3.mid) -- (n1.mid) -- cycle;
					\fill[fill=black!20,opacity=1] (n6.mid) to[bend left] (n1.mid) -- (n3.mid) -- cycle;
					\fill[fill=black!20,opacity=1] (n6.mid) -- (n3.mid) -- (n0.mid) -- cycle;
					\fill[fill=black!20,opacity=1] (n5.mid) to[bend left] (n0.mid) -- (n2.mid) -- cycle;
					%Lines
					\draw[thick] (n11.mid) --  (n8.mid);
					\draw[thick] (n11.mid) --  (n7.mid);
					\draw[thick] (n11.mid) --  (n4.mid);
					\draw[thick] (n11.mid) --  (n1.mid);
					\draw[thick] (n10.mid) --  (n7.mid);
					\draw[thick] (n10.mid) --  (n6.mid);
					\draw[thick] (n10.mid) --  (n3.mid);
					\draw[thick] (n10.mid) --  (n1.mid);
					\draw[thick] (n10.mid) --  (n0.mid);
					\draw[thick] (n9.mid) --  (n6.mid);
					\draw[thick] (n9.mid) --  (n5.mid);
					\draw[thick] (n9.mid) --  (n2.mid);
					\draw[thick] (n9.mid) --  (n0.mid);
					\draw[thick] (n8.mid) --  (n4.mid);
					\draw[thick] (n7.mid) --  (n4.mid);
					\draw[thick] (n7.mid) --  (n3.mid);
					\draw[thick] (n7.mid) --  (n1.mid);
					\draw[thick] (n6.mid) --  (n3.mid);
					\draw[thick] (n6.mid) --  (n2.mid);
					\draw[thick] (n6.mid) --  (n0.mid);
					\draw[thick] (n5.mid) --  (n2.mid);
					\draw[thick] (n4.mid) --  (n1.mid);
					\draw[thick] (n3.mid) --  (n1.mid);
					\draw[thick] (n3.mid) --  (n0.mid);
					\draw[thick] (n2.mid) --  (n0.mid);
					\end{pgfonlayer}
				\end{tikzpicture}}	
			\end{figure}	
		\endminipage}
		\caption{Pasos en la filtración del complejo simplicial 
		asociado a \ref{fig:distIntR} y números de Betti 
		correspondientes (interpretación global).}
		\label{fig:distIntRSG}
	\end{figure}
	\begin{figure}[!htbp]
		\minipage{0.5\textwidth}
			\begin{figure}[H]
				\resizebox{1.0\textwidth}{!}{\includegraphics{Images/DiagramaBarrasEjINTGLOBAL.png}}
			\end{figure}
		\endminipage
		\minipage{0.5\textwidth}
			\begin{figure}[H]
				\resizebox{1.0\textwidth}{!}{\includegraphics{Images/DiagramaPersistenciaEjINTGLOBAL.png}}
			\end{figure}
		\endminipage
		\caption{Diagrama de barras (izquierda) y diagrama de 
		persistencia (derecha) asociados a la filtración ilustrada en 
		\ref{fig:distIntRSG} (interpretación global).}
		\label{fig:distIntRDG}
	\end{figure}

	Si observamos detenidamente los pasos en la filtración del complejo 
	simplicial en la \autoref{fig:distIntRSL} y en la 
	\autoref{fig:distIntRSG}, 
	apreciamos una clara diferencia entre ambas interpretaciones que se 
	produce en la filtración 6 y se mantiene hasta la última filtración. 
	Esta disparidad es especialmente significativa en la filtración 9: 
	mientras que en la interpretación local tenemos que $\beta_{2}=1$, en 
	la interpretación global tenemos que $\beta_{2}=0$. Es decir, con la 
	interpretación local estamos capturando un vacío de dimensión 2 que
	se nos escapa con la interpretación global. Este caso no nos había 
	aparecido anteriormente y merece la pena detenerse momentáneamente a 
	hacer una reflexión sobre ello.

	Conceptualmente, $\beta_{2}=1$ se ha producido porque existe un agujero
	entre $3$-símplices, es decir, existe un agujero entre tetraedro en 
	la filtración 9 de la la \autoref{fig:distIntRSL}. De manera intuitiva, 
	esto nos dice que la red ha necesitado, equitativamente, la 
	salida de 3 neuronas para determinar la neurona de llegada. Siguiendo 
	el razonamiento hecho en \cite{Articulo-Watanabe}, esto mostraría un 
	aumento en el grado de complejidad del conocimiento adquirido por la 
	red.

	No obstante, en este ejemplo en concreto, dicha clase de homología 
	nace y muere en un breve lapso de filtraciones, lo que nos induce a 
	pensar que simplemente se trata de ruido. Esta consideración es 
	fácilmente observable en el diagrama de persistencia de la 
	\autoref{fig:distIntRDL}, en el que las clases próximas a la diagonal 
	las podemos considerar como ruido.

	Adicionalmente, podemos observar que la interpretación local nos está 
	permitiendo visualizar un número mayor de agujeros de dimensión 1. 
	Esto puede verse en la filtración 8 de la \autoref{fig:distIntRSL} y 
	la \autoref{fig:distIntRSG}.

	Todas estas disparidades quedan todavía mejor reflejadas en la 
	comparativa de la \autoref{fig:distHomRDL} con la 
	\autoref{fig:distIntRDG}.
	\end{ejem}

	Por último, veamos el ejemplo correspondiente a una distribución 
	extrema de los pesos de la red.

	\begin{ejem}\textbf{Distribución extrema.}
	Supondremos que la red estudiada posee una 
	distribución heterogénea de los pesos. Esta distribución, presentará 
	desigualdades muy significativas entre los pesos de la red. Esta 
	distribución dará lugar a la representación de 
	la red, con las importancias entre neuronas ya calculadas, que puede 
	verse en la \autoref{fig:distExtR}.
	
	\begin{figure}[!htbp]
			\centering
			\begin{tikzpicture}[
				roundnode/.style={circle, draw=black, thick, fill=white, minimum size=7mm},
				]
				%Nodes
				\node[roundnode]      (n0) 	at 	(4,1)           {0};
				\node[roundnode]      (n1) 	at 	(4,-1)          {1};
				\node[roundnode]      (n2) 	at 	(2,2)           {2};
				\node[roundnode]      (n3) 	at 	(2,0)           {3};
				\node[roundnode]      (n4)      at 	(2,-2)		{4};
				\node[roundnode]      (n5)      at 	(0,3)		{5};
				\node[roundnode]      (n6)      at 	(0,1)		{6};
				\node[roundnode]      (n7) 	at 	(0,-1)           {7};
				\node[roundnode]      (n8) 	at 	(0,-3)           {8};
				\node[roundnode]      (n9) 	at 	(-2,2)           {9};
				\node[roundnode]      (n10) 	at 	(-2,0)           {10};
				\node[roundnode]      (n11) 	at 	(-2,-2)           {11};
				
				\begin{pgfonlayer}{background}
				%Lines
				\draw[thick,->] (n9.mid) -- node[above,sloped] {1.0} (n5.west);
				\draw[thick,->] (n9.mid) -- node[above,sloped] {0.9} (n6.west);
				\draw[thick,->] (n10.mid) -- node[above,sloped] {0.1} (n6.west);
				\draw[thick,->] (n10.mid) -- node[above,sloped] {0.2} (n7.west);
				\draw[thick,->] (n11.mid) -- node[above,sloped] {0.8} (n7.west);
				\draw[thick,->] (n11.mid) -- node[above,sloped] {1.0} (n8.west);
				\draw[thick,->] (n5.mid) -- node[above,sloped] {0.85} (n2.west);
				\draw[thick,->] (n6.mid) -- node[above,sloped] {0.15} (n2.west);
				\draw[thick,->] (n6.mid) -- node[above,sloped] {0.08} (n3.west);
				\draw[thick,->] (n7.mid) -- node[above,sloped] {0.92} (n3.west);
				\draw[thick,->] (n7.mid) -- node[above,sloped] {0.18} (n4.west);
				\draw[thick,->] (n8.mid) -- node[above,sloped] {0.82} (n4.west);
				\draw[thick,->] (n2.mid) -- node[above,sloped] {0.16} (n0.west);
				\draw[thick,->] (n3.mid) -- node[above,sloped] {0.84} (n0.west);
				\draw[thick,->] (n3.mid) -- node[above,sloped] {0.95} (n1.west);
				\draw[thick,->] (n4.mid) -- node[above,sloped] {0.05} (n1.west);
				\end{pgfonlayer}
			\end{tikzpicture}
			\caption{Representación de la red \ref{fig:arqEjs} con 
			una distribución extrema de las importancias entre 
			neuronas.}
			\label{fig:distExtR}
	\end{figure}

	Procedemos ahora con el cálculo del complejo simplicial asociado. Como 
	ya hemos mencionado anteriormente, vamos a ilustrar unos cuantos pasos 
	en la filtración con los correspondientes 
	números de Betti asociados. Esta ilustración puede verse en 
	la \autoref{fig:distExtRSL} y en la \autoref{fig:distExtRSG}. También 
	añadimos los diagramas de 
	barras y de persistencia asociados que pueden verse en 
	la \autoref{fig:distExtRDL} y en la \autoref{fig:distExtRDG}.

	\begin{figure}[!htbp]
		\fbox{\minipage{0.225\textwidth}
			\begin{figure}[H]
				\resizebox{1.0\textwidth}{!}{\begin{tikzpicture}[
					roundnode/.style={circle, draw=black, thick, fill=white, minimum size=7mm},
					]
					%Nodes
					\node           (label) 	at	(1,4) 		{Filtración 0 (t=1.0)};
					\node[roundnode]      (n0) 	at 	(4,1)           {0};
					\node[roundnode]      (n1) 	at 	(4,-1)          {1};
					\node[roundnode]      (n2) 	at 	(2,2)           {2};
					\node[roundnode]      (n3) 	at 	(2,0)           {3};
					\node[roundnode]      (n4)      at 	(2,-2)		{4};
					\node[roundnode]      (n5)      at 	(0,3)		{5};
					\node[roundnode]      (n6)      at 	(0,1)		{6};
					\node[roundnode]      (n7) 	at 	(0,-1)           {7};
					\node[roundnode]      (n8) 	at 	(0,-3)           {8};
					\node[roundnode]      (n9) 	at 	(-2,2)           {9};
					\node[roundnode]      (n10) 	at 	(-2,0)           {10};
					\node[roundnode]      (n11) 	at 	(-2,-2)           {11};
					\node 		(label2) 	at 	(1,-4) 		{$\beta_{0}=10,\beta_{1}=0$};
					
					\begin{pgfonlayer}{background}
					%Lines
					\draw[thick] (n11.mid) --  (n8);
					\draw[thick] (n9.mid) --  (n5);
					\end{pgfonlayer}
				\end{tikzpicture}}	
			\end{figure}	
			\begin{figure}[H]
				\resizebox{1.0\textwidth}{!}{\begin{tikzpicture}[
					roundnode/.style={circle, draw=black, thick, fill=white, minimum size=7mm},
					]
					%Nodes
					\node           (label) 	at	(1,4) 		{Filtración 7 (t=0.3)};
					\node[roundnode]      (n0) 	at 	(4,1)           {0};
					\node[roundnode]      (n1) 	at 	(4,-1)          {1};
					\node[roundnode]      (n2) 	at 	(2,2)           {2};
					\node[roundnode]      (n3) 	at 	(2,0)           {3};
					\node[roundnode]      (n4)      at 	(2,-2)		{4};
					\node[roundnode]      (n5)      at 	(0,3)		{5};
					\node[roundnode]      (n6)      at 	(0,1)		{6};
					\node[roundnode]      (n7) 	at 	(0,-1)           {7};
					\node[roundnode]      (n8) 	at 	(0,-3)           {8};
					\node[roundnode]      (n9) 	at 	(-2,2)           {9};
					\node[roundnode]      (n10) 	at 	(-2,0)           {10};
					\node[roundnode]      (n11) 	at 	(-2,-2)           {11};
					\node 		(label2) 	at 	(1,-4) 		{$\beta_{0}=3,\beta_{1}=0$};
					
					\begin{pgfonlayer}{background}
					
					\fill[fill=black!20,opacity=1] (n11.mid) -- (n8.mid) -- (n4.mid) -- cycle;
					\fill[fill=black!20,opacity=1] (n11.mid) to[bend left] (n3.mid) -- (n7.mid) -- cycle;
					\fill[fill=black!20,opacity=1] (n11.mid) -- (n7.mid) -- (n1.mid) -- cycle;
					\fill[fill=black!20,opacity=1] (n11.mid) -- (n3.mid) -- (n1.mid) -- cycle;
					\fill[fill=black!20,opacity=1] (n9.mid) -- (n5.mid) -- (n2.mid) -- cycle;
					\fill[fill=black!20,opacity=1] (n7.mid) -- (n3.mid) -- (n1.mid) -- cycle;
					\fill[fill=black!20,opacity=1] (n7.mid) to[bend right] (n0.mid) -- (n3.mid) -- cycle;
					%Lines
					\draw[thick] (n11.mid) --  (n8.mid);
					\draw[thick] (n11.mid) --  (n7.mid);
					\draw[thick] (n11.mid) --  (n4.mid);
					\draw[thick] (n11.mid) --  (n1.mid);
					\draw[thick] (n9.mid) --  (n6.mid);
					\draw[thick] (n9.mid) --  (n5.mid);
					\draw[thick] (n9.mid) --  (n2.mid);
					\draw[thick] (n8.mid) --  (n4.mid);
					\draw[thick] (n7.mid) --  (n3.mid);
					\draw[thick] (n7.mid) --  (n1.mid);
					\draw[thick] (n5.mid) --  (n2.mid);
					\draw[thick] (n3.mid) --  (n1.mid);
					\draw[thick] (n3.mid) --  (n0.mid);
					\end{pgfonlayer}
				\end{tikzpicture}}	
			\end{figure}	
		\endminipage}
		\fbox{\minipage{0.225\textwidth}
			\begin{figure}[H]
				\resizebox{1.0\textwidth}{!}{\begin{tikzpicture}[
					roundnode/.style={circle, draw=black, thick, fill=white, minimum size=7mm},
					]
					%Nodes
					\node           (label) 	at	(1,4) 		{Filtración 4 (t=0.6)};
					\node[roundnode]      (n0) 	at 	(4,1)           {0};
					\node[roundnode]      (n1) 	at 	(4,-1)          {1};
					\node[roundnode]      (n2) 	at 	(2,2)           {2};
					\node[roundnode]      (n3) 	at 	(2,0)           {3};
					\node[roundnode]      (n4)      at 	(2,-2)		{4};
					\node[roundnode]      (n5)      at 	(0,3)		{5};
					\node[roundnode]      (n6)      at 	(0,1)		{6};
					\node[roundnode]      (n7) 	at 	(0,-1)           {7};
					\node[roundnode]      (n8) 	at 	(0,-3)           {8};
					\node[roundnode]      (n9) 	at 	(-2,2)           {9};
					\node[roundnode]      (n10) 	at 	(-2,0)           {10};
					\node[roundnode]      (n11) 	at 	(-2,-2)           {11};
					\node 		(label2) 	at 	(1,-4) 		{$\beta_{0}=3,\beta_{1}=0$};
					
					\begin{pgfonlayer}{background}
					
					\fill[fill=black!20,opacity=1] (n11.mid) -- (n8.mid) -- (n4.mid) -- cycle;
					\fill[fill=black!20,opacity=1] (n11.mid) to[bend left] (n3.mid) -- (n7.mid) -- cycle;
					\fill[fill=black!20,opacity=1] (n11.mid) -- (n7.mid) -- (n1.mid) -- cycle;
					\fill[fill=black!20,opacity=1] (n11.mid) -- (n3.mid) -- (n1.mid) -- cycle;
					\fill[fill=black!20,opacity=1] (n9.mid) -- (n5.mid) -- (n2.mid) -- cycle;
					\fill[fill=black!20,opacity=1] (n7.mid) -- (n3.mid) -- (n1.mid) -- cycle;
					\fill[fill=black!20,opacity=1] (n7.mid) to[bend right] (n0.mid) -- (n3.mid) -- cycle;
					%Lines
					\draw[thick] (n11.mid) --  (n8.mid);
					\draw[thick] (n11.mid) --  (n7.mid);
					\draw[thick] (n11.mid) --  (n4.mid);
					\draw[thick] (n11.mid) --  (n1.mid);
					\draw[thick] (n9.mid) --  (n6.mid);
					\draw[thick] (n9.mid) --  (n5.mid);
					\draw[thick] (n9.mid) --  (n2.mid);
					\draw[thick] (n8.mid) --  (n4.mid);
					\draw[thick] (n7.mid) --  (n3.mid);
					\draw[thick] (n7.mid) --  (n1.mid);
					\draw[thick] (n5.mid) --  (n2.mid);
					\draw[thick] (n3.mid) --  (n1.mid);
					\draw[thick] (n3.mid) --  (n0.mid);
					\end{pgfonlayer}
				\end{tikzpicture}}	
			\end{figure}	
			\begin{figure}[H]
				\resizebox{1.0\textwidth}{!}{\begin{tikzpicture}[
					roundnode/.style={circle, draw=black, thick, fill=white, minimum size=7mm},
					]
					%Nodes
					\node           (label) 	at	(1,4) 		{Filtración 8 (t=0.2)};
					\node[roundnode]      (n0) 	at 	(4,1)           {0};
					\node[roundnode]      (n1) 	at 	(4,-1)          {1};
					\node[roundnode]      (n2) 	at 	(2,2)           {2};
					\node[roundnode]      (n3) 	at 	(2,0)           {3};
					\node[roundnode]      (n4)      at 	(2,-2)		{4};
					\node[roundnode]      (n5)      at 	(0,3)		{5};
					\node[roundnode]      (n6)      at 	(0,1)		{6};
					\node[roundnode]      (n7) 	at 	(0,-1)           {7};
					\node[roundnode]      (n8) 	at 	(0,-3)           {8};
					\node[roundnode]      (n9) 	at 	(-2,2)           {9};
					\node[roundnode]      (n10) 	at 	(-2,0)           {10};
					\node[roundnode]      (n11) 	at 	(-2,-2)           {11};
					\node 		(label2) 	at 	(1,-4) 		{$\beta_{0}=2,\beta_{1}=0$};
					
					\begin{pgfonlayer}{background}
					
					\fill[fill=black!20,opacity=1] (n11.mid) -- (n8.mid) -- (n4.mid) -- cycle;
					\fill[fill=black!20,opacity=1] (n11.mid) to[bend left] (n3.mid) -- (n7.mid) -- cycle;
					\fill[fill=black!20,opacity=1] (n11.mid) -- (n7.mid) -- (n1.mid) -- cycle;
					\fill[fill=black!20,opacity=1] (n11.mid) -- (n3.mid) -- (n1.mid) -- cycle;
					\fill[fill=black!20,opacity=1] (n9.mid) -- (n5.mid) -- (n2.mid) -- cycle;
					\fill[fill=black!20,opacity=1] (n7.mid) -- (n3.mid) -- (n1.mid) -- cycle;
					\fill[fill=black!20,opacity=1] (n7.mid) to[bend right] (n0.mid) -- (n3.mid) -- cycle;
					%Lines
					\draw[thick] (n11.mid) --  (n8.mid);
					\draw[thick] (n11.mid) --  (n7.mid);
					\draw[thick] (n11.mid) --  (n4.mid);
					\draw[thick] (n11.mid) --  (n1.mid);
					\draw[thick] (n10.mid) --  (n7.mid);
					\draw[thick] (n9.mid) --  (n6.mid);
					\draw[thick] (n9.mid) --  (n5.mid);
					\draw[thick] (n9.mid) --  (n2.mid);
					\draw[thick] (n8.mid) --  (n4.mid);
					\draw[thick] (n7.mid) --  (n3.mid);
					\draw[thick] (n7.mid) --  (n1.mid);
					\draw[thick] (n5.mid) --  (n2.mid);
					\draw[thick] (n3.mid) --  (n1.mid);
					\draw[thick] (n3.mid) --  (n0.mid);
					\end{pgfonlayer}
				\end{tikzpicture}}	
			\end{figure}	
		\endminipage}
		\fbox{\minipage{0.225\textwidth}
			\begin{figure}[H]
				\resizebox{1.0\textwidth}{!}{\begin{tikzpicture}[
					roundnode/.style={circle, draw=black, thick, fill=white, minimum size=7mm},
					]
					%Nodes
					\node           (label) 	at	(1,4) 		{Filtración 5 (t=0.5)};
					\node[roundnode]      (n0) 	at 	(4,1)           {0};
					\node[roundnode]      (n1) 	at 	(4,-1)          {1};
					\node[roundnode]      (n2) 	at 	(2,2)           {2};
					\node[roundnode]      (n3) 	at 	(2,0)           {3};
					\node[roundnode]      (n4)      at 	(2,-2)		{4};
					\node[roundnode]      (n5)      at 	(0,3)		{5};
					\node[roundnode]      (n6)      at 	(0,1)		{6};
					\node[roundnode]      (n7) 	at 	(0,-1)           {7};
					\node[roundnode]      (n8) 	at 	(0,-3)           {8};
					\node[roundnode]      (n9) 	at 	(-2,2)           {9};
					\node[roundnode]      (n10) 	at 	(-2,0)           {10};
					\node[roundnode]      (n11) 	at 	(-2,-2)           {11};
					\node 		(label2) 	at 	(1,-4) 		{$\beta_{0}=3,\beta_{1}=0$};
					
					\begin{pgfonlayer}{background}
					
					\fill[fill=black!20,opacity=1] (n11.mid) -- (n8.mid) -- (n4.mid) -- cycle;
					\fill[fill=black!20,opacity=1] (n11.mid) to[bend left] (n3.mid) -- (n7.mid) -- cycle;
					\fill[fill=black!20,opacity=1] (n11.mid) -- (n7.mid) -- (n1.mid) -- cycle;
					\fill[fill=black!20,opacity=1] (n11.mid) -- (n3.mid) -- (n1.mid) -- cycle;
					\fill[fill=black!20,opacity=1] (n9.mid) -- (n5.mid) -- (n2.mid) -- cycle;
					\fill[fill=black!20,opacity=1] (n7.mid) -- (n3.mid) -- (n1.mid) -- cycle;
					\fill[fill=black!20,opacity=1] (n7.mid) to[bend right] (n0.mid) -- (n3.mid) -- cycle;
					%Lines
					\draw[thick] (n11.mid) --  (n8.mid);
					\draw[thick] (n11.mid) --  (n7.mid);
					\draw[thick] (n11.mid) --  (n4.mid);
					\draw[thick] (n11.mid) --  (n1.mid);
					\draw[thick] (n9.mid) --  (n6.mid);
					\draw[thick] (n9.mid) --  (n5.mid);
					\draw[thick] (n9.mid) --  (n2.mid);
					\draw[thick] (n8.mid) --  (n4.mid);
					\draw[thick] (n7.mid) --  (n3.mid);
					\draw[thick] (n7.mid) --  (n1.mid);
					\draw[thick] (n5.mid) --  (n2.mid);
					\draw[thick] (n3.mid) --  (n1.mid);
					\draw[thick] (n3.mid) --  (n0.mid);
					\end{pgfonlayer}
				\end{tikzpicture}}	
			\end{figure}	
			\begin{figure}[H]
				\resizebox{1.0\textwidth}{!}{\begin{tikzpicture}[
					roundnode/.style={circle, draw=black, thick, fill=white, minimum size=7mm},
					]
					%Nodes
					\node           (label) 	at	(1,4) 		{Filtración 9 (t=0.1)};
					\node[roundnode]      (n0) 	at 	(4,1)           {0};
					\node[roundnode]      (n1) 	at 	(4,-1)          {1};
					\node[roundnode]      (n2) 	at 	(2,2)           {2};
					\node[roundnode]      (n3) 	at 	(2,0)           {3};
					\node[roundnode]      (n4)      at 	(2,-2)		{4};
					\node[roundnode]      (n5)      at 	(0,3)		{5};
					\node[roundnode]      (n6)      at 	(0,1)		{6};
					\node[roundnode]      (n7) 	at 	(0,-1)           {7};
					\node[roundnode]      (n8) 	at 	(0,-3)           {8};
					\node[roundnode]      (n9) 	at 	(-2,2)           {9};
					\node[roundnode]      (n10) 	at 	(-2,0)           {10};
					\node[roundnode]      (n11) 	at 	(-2,-2)           {11};
					\node 		(label2) 	at 	(1,-4) 		{$\beta_{0}=1,\beta_{1}=1$};
					
					\begin{pgfonlayer}{background}
					
					\fill[fill=black!20,opacity=1] (n11.mid) -- (n8.mid) -- (n4.mid) -- cycle;
					\fill[fill=black!20,opacity=1] (n11.mid) to[bend left] (n3.mid) -- (n7.mid) -- cycle;
					\fill[fill=black!20,opacity=1] (n11.mid) -- (n7.mid) -- (n4.mid) -- cycle;
					\fill[fill=black!20,opacity=1] (n11.mid) -- (n7.mid) -- (n1.mid) -- cycle;
					\fill[fill=black!20,opacity=1] (n11.mid) -- (n3.mid) -- (n1.mid) -- cycle;
					\fill[fill=black!20,opacity=1] (n10.mid) -- (n7.mid) -- (n3.mid) -- cycle;
					\fill[fill=black!20,opacity=1] (n10.mid) -- (n7.mid) -- (n0.mid) -- cycle;
					\fill[fill=black!20,opacity=1] (n10.mid) -- (n3.mid) -- (n1.mid) -- cycle;
					\fill[fill=black!20,opacity=1] (n10.mid) -- (n3.mid) -- (n0.mid) -- cycle;
					\fill[fill=black!20,opacity=1] (n9.mid) -- (n6.mid) -- (n2.mid) -- cycle;
					\fill[fill=black!20,opacity=1] (n9.mid) -- (n5.mid) -- (n2.mid) -- cycle;
					\fill[fill=black!20,opacity=1] (n9.mid) -- (n5.mid) -- (n0.mid) -- cycle;
					\fill[fill=black!20,opacity=1] (n7.mid) -- (n3.mid) -- (n1.mid) -- cycle;
					\fill[fill=black!20,opacity=1] (n7.mid) to[bend right] (n0.mid) -- (n3.mid) -- cycle;
					\fill[fill=black!20,opacity=1] (n5.mid) to[bend left] (n0.mid) -- (n2.mid) -- cycle;
					%Lines
					\draw[thick] (n11.mid) --  (n8.mid);
					\draw[thick] (n11.mid) --  (n7.mid);
					\draw[thick] (n11.mid) --  (n4.mid);
					\draw[thick] (n11.mid) --  (n1.mid);
					\draw[thick] (n10.mid) --  (n7.mid);
					\draw[thick] (n10.mid) --  (n6.mid);
					\draw[thick] (n10.mid) --  (n3.mid);
					\draw[thick] (n10.mid) --  (n1.mid);
					\draw[thick] (n10.mid) --  (n0.mid);
					\draw[thick] (n9.mid) --  (n6.mid);
					\draw[thick] (n9.mid) --  (n5.mid);
					\draw[thick] (n9.mid) --  (n2.mid);
					\draw[thick] (n9.mid) --  (n0.mid);
					\draw[thick] (n8.mid) --  (n4.mid);
					\draw[thick] (n7.mid) --  (n4.mid);
					\draw[thick] (n7.mid) --  (n3.mid);
					\draw[thick] (n7.mid) --  (n1.mid);
					\draw[thick] (n6.mid) --  (n2.mid);
					\draw[thick] (n5.mid) --  (n2.mid);
					\draw[thick] (n3.mid) --  (n1.mid);
					\draw[thick] (n3.mid) --  (n0.mid);
					\draw[thick] (n2.mid) --  (n0.mid);
					\end{pgfonlayer}
				\end{tikzpicture}}	
			\end{figure}	
	\endminipage}
		\fbox{\minipage{0.225\textwidth}
			\begin{figure}[H]
				\resizebox{1.0\textwidth}{!}{\begin{tikzpicture}[
					roundnode/.style={circle, draw=black, fill=white, thick, minimum size=7mm},
					]
					%Nodes
					\node           (label) 	at	(1,4) 		{Filtración 6 (t=0.4)};
					\node[roundnode]      (n0) 	at 	(4,1)           {0};
					\node[roundnode]      (n1) 	at 	(4,-1)          {1};
					\node[roundnode]      (n2) 	at 	(2,2)           {2};
					\node[roundnode]      (n3) 	at 	(2,0)           {3};
					\node[roundnode]      (n4)      at 	(2,-2)		{4};
					\node[roundnode]      (n5)      at 	(0,3)		{5};
					\node[roundnode]      (n6)      at 	(0,1)		{6};
					\node[roundnode]      (n7) 	at 	(0,-1)           {7};
					\node[roundnode]      (n8) 	at 	(0,-3)           {8};
					\node[roundnode]      (n9) 	at 	(-2,2)           {9};
					\node[roundnode]      (n10) 	at 	(-2,0)           {10};
					\node[roundnode]      (n11) 	at 	(-2,-2)           {11};
					\node 		(label2) 	at 	(1,-4) 		{$\beta_{0}=3,\beta_{1}=0$};
					
					\begin{pgfonlayer}{background}
					
					\fill[fill=black!20,opacity=1] (n11.mid) -- (n8.mid) -- (n4.mid) -- cycle;
					\fill[fill=black!20,opacity=1] (n11.mid) to[bend left] (n3.mid) -- (n7.mid) -- cycle;
					\fill[fill=black!20,opacity=1] (n11.mid) -- (n7.mid) -- (n1.mid) -- cycle;
					\fill[fill=black!20,opacity=1] (n11.mid) -- (n3.mid) -- (n1.mid) -- cycle;
					\fill[fill=black!20,opacity=1] (n9.mid) -- (n5.mid) -- (n2.mid) -- cycle;
					\fill[fill=black!20,opacity=1] (n7.mid) -- (n3.mid) -- (n1.mid) -- cycle;
					\fill[fill=black!20,opacity=1] (n7.mid) to[bend right] (n0.mid) -- (n3.mid) -- cycle;
					%Lines
					\draw[thick] (n11.mid) --  (n8.mid);
					\draw[thick] (n11.mid) --  (n7.mid);
					\draw[thick] (n11.mid) --  (n4.mid);
					\draw[thick] (n11.mid) --  (n1.mid);
					\draw[thick] (n9.mid) --  (n6.mid);
					\draw[thick] (n9.mid) --  (n5.mid);
					\draw[thick] (n9.mid) --  (n2.mid);
					\draw[thick] (n8.mid) --  (n4.mid);
					\draw[thick] (n7.mid) --  (n3.mid);
					\draw[thick] (n7.mid) --  (n1.mid);
					\draw[thick] (n5.mid) --  (n2.mid);
					\draw[thick] (n3.mid) --  (n1.mid);
					\draw[thick] (n3.mid) --  (n0.mid);
					\end{pgfonlayer}
				\end{tikzpicture}}	
			\end{figure}	
			\begin{figure}[H]
				\resizebox{1.0\textwidth}{!}{\begin{tikzpicture}[
					roundnode/.style={circle, draw=black, thick, fill=white, minimum size=7mm},
					]
					%Nodes
					\node           (label) 	at	(1,4) 		{Filtración 10 (t=0.0)};
					\node[roundnode]      (n0) 	at 	(4,1)           {0};
					\node[roundnode]      (n1) 	at 	(4,-1)          {1};
					\node[roundnode]      (n2) 	at 	(2,2)           {2};
					\node[roundnode]      (n3) 	at 	(2,0)           {3};
					\node[roundnode]      (n4)      at 	(2,-2)		{4};
					\node[roundnode]      (n5)      at 	(0,3)		{5};
					\node[roundnode]      (n6)      at 	(0,1)		{6};
					\node[roundnode]      (n7) 	at 	(0,-1)           {7};
					\node[roundnode]      (n8) 	at 	(0,-3)           {8};
					\node[roundnode]      (n9) 	at 	(-2,2)           {9};
					\node[roundnode]      (n10) 	at 	(-2,0)           {10};
					\node[roundnode]      (n11) 	at 	(-2,-2)           {11};
					\node 		(label2) 	at 	(1,-4) 		{$\beta_{0}=1,\beta_{1}=0$};
					
					\begin{pgfonlayer}{background}
					
					\fill[fill=black!20,opacity=1] (n11.mid) -- (n8.mid) -- (n4.mid) -- cycle;
					\fill[fill=black!20,opacity=1] (n11.mid) -- (n8.mid) -- (n1.mid) -- cycle;
					\fill[fill=black!20,opacity=1] (n11.mid) -- (n7.mid) -- (n4.mid) -- cycle;
					\fill[fill=black!20,opacity=1] (n11.mid) to[bend left] (n3.mid) -- (n7.mid) -- cycle;
					\fill[fill=black!20,opacity=1] (n10.mid) to[bend right] (n4.mid) -- (n7.mid) -- cycle;
					\fill[fill=black!20,opacity=1] (n10.mid) -- (n7.mid) -- (n3.mid) -- cycle;
					\fill[fill=black!20,opacity=1] (n10.mid) -- (n7.mid) -- (n1.mid) -- cycle;
					\fill[fill=black!20,opacity=1] (n10.mid) -- (n6.mid) -- (n3.mid) -- cycle;
					\fill[fill=black!20,opacity=1] (n10.mid) to[bend left] (n2.mid) -- (n6.mid) -- cycle;
					\fill[fill=black!20,opacity=1] (n9.mid) to[bend right] (n3.mid) -- (n6.mid) -- cycle;
					\fill[fill=black!20,opacity=1] (n9.mid) -- (n6.mid) -- (n2.mid) -- cycle;
					\fill[fill=black!20,opacity=1] (n9.mid) -- (n6.mid) -- (n0.mid) -- cycle;
					\fill[fill=black!20,opacity=1] (n9.mid) -- (n5.mid) -- (n2.mid) -- cycle;
					\fill[fill=black!20,opacity=1] (n9.mid) -- (n5.mid) -- (n0.mid) -- cycle;
					\fill[fill=black!20,opacity=1] (n8.mid) to[bend right] (n1.mid) -- (n4.mid) -- cycle;
					\fill[fill=black!20,opacity=1] (n7.mid) to[bend right] (n0.mid) -- (n3.mid) -- cycle;
					\fill[fill=black!20,opacity=1] (n7.mid) -- (n4.mid) -- (n1.mid) -- cycle;
					\fill[fill=black!20,opacity=1] (n7.mid) -- (n3.mid) -- (n1.mid) -- cycle;
					\fill[fill=black!20,opacity=1] (n6.mid) to[bend left] (n1.mid) -- (n3.mid) -- cycle;
					\fill[fill=black!20,opacity=1] (n6.mid) -- (n3.mid) -- (n0.mid) -- cycle;
					\fill[fill=black!20,opacity=1] (n5.mid) to[bend left] (n0.mid) -- (n2.mid) -- cycle;
					%Lines
					\draw[thick] (n11.mid) --  (n8.mid);
					\draw[thick] (n11.mid) --  (n7.mid);
					\draw[thick] (n11.mid) --  (n4.mid);
					\draw[thick] (n11.mid) --  (n1.mid);
					\draw[thick] (n10.mid) --  (n7.mid);
					\draw[thick] (n10.mid) --  (n6.mid);
					\draw[thick] (n10.mid) --  (n3.mid);
					\draw[thick] (n10.mid) --  (n1.mid);
					\draw[thick] (n10.mid) --  (n0.mid);
					\draw[thick] (n9.mid) --  (n6.mid);
					\draw[thick] (n9.mid) --  (n5.mid);
					\draw[thick] (n9.mid) --  (n2.mid);
					\draw[thick] (n9.mid) --  (n0.mid);
					\draw[thick] (n8.mid) --  (n4.mid);
					\draw[thick] (n7.mid) --  (n4.mid);
					\draw[thick] (n7.mid) --  (n3.mid);
					\draw[thick] (n7.mid) --  (n1.mid);
					\draw[thick] (n6.mid) --  (n3.mid);
					\draw[thick] (n6.mid) --  (n2.mid);
					\draw[thick] (n6.mid) --  (n0.mid);
					\draw[thick] (n5.mid) --  (n2.mid);
					\draw[thick] (n4.mid) --  (n1.mid);
					\draw[thick] (n3.mid) --  (n1.mid);
					\draw[thick] (n3.mid) --  (n0.mid);
					\draw[thick] (n2.mid) --  (n0.mid);
					\end{pgfonlayer}
				\end{tikzpicture}}	
			\end{figure}	
		\endminipage}
		\caption{Pasos en la filtración del complejo simplicial 
		asociado a \ref{fig:distExtR} y números de Betti 
		correspondientes (interpretación local).}
		\label{fig:distExtRSL}
	\end{figure}
	\begin{figure}[!htbp]
		\minipage{0.5\textwidth}
			\begin{figure}[H]
				\resizebox{1.0\textwidth}{!}{\includegraphics{Images/DiagramaBarrasEjEXTLOCAL.png}}
			\end{figure}
		\endminipage
		\minipage{0.5\textwidth}
			\begin{figure}[H]
				\resizebox{1.0\textwidth}{!}{\includegraphics{Images/DiagramaPersistenciaEjEXTLOCAL.png}}
			\end{figure}
		\endminipage
		\caption{Diagrama de barras (izquierda) y diagrama de 
		persistencia (derecha) asociados a la filtración ilustrada en 
		\ref{fig:distExtRSL} (interpretación local).}
		\label{fig:distExtRDL}
	\end{figure}
	\begin{figure}[!htbp]
		\fbox{\minipage{0.225\textwidth}
			\begin{figure}[H]
				\resizebox{1.0\textwidth}{!}{\begin{tikzpicture}[
					roundnode/.style={circle, draw=black, thick, fill=white, minimum size=7mm},
					]
					%Nodes
					\node           (label) 	at	(1,4) 		{Filtración 0 (t=1.0)};
					\node[roundnode]      (n0) 	at 	(4,1)           {0};
					\node[roundnode]      (n1) 	at 	(4,-1)          {1};
					\node[roundnode]      (n2) 	at 	(2,2)           {2};
					\node[roundnode]      (n3) 	at 	(2,0)           {3};
					\node[roundnode]      (n4)      at 	(2,-2)		{4};
					\node[roundnode]      (n5)      at 	(0,3)		{5};
					\node[roundnode]      (n6)      at 	(0,1)		{6};
					\node[roundnode]      (n7) 	at 	(0,-1)           {7};
					\node[roundnode]      (n8) 	at 	(0,-3)           {8};
					\node[roundnode]      (n9) 	at 	(-2,2)           {9};
					\node[roundnode]      (n10) 	at 	(-2,0)           {10};
					\node[roundnode]      (n11) 	at 	(-2,-2)           {11};
					\node 		(label2) 	at 	(1,-4) 		{$\beta_{0}=10,\beta_{1}=0$};
					
					\begin{pgfonlayer}{background}
					%Lines
					\draw[thick] (n11.mid) --  (n8);
					\draw[thick] (n9.mid) --  (n5);
					\end{pgfonlayer}
				\end{tikzpicture}}	
			\end{figure}	
			\begin{figure}[H]
				\resizebox{1.0\textwidth}{!}{\begin{tikzpicture}[
					roundnode/.style={circle, draw=black, thick, fill=white, minimum size=7mm},
					]
					%Nodes
					\node           (label) 	at	(1,4) 		{Filtración 7 (t=0.3)};
					\node[roundnode]      (n0) 	at 	(4,1)           {0};
					\node[roundnode]      (n1) 	at 	(4,-1)          {1};
					\node[roundnode]      (n2) 	at 	(2,2)           {2};
					\node[roundnode]      (n3) 	at 	(2,0)           {3};
					\node[roundnode]      (n4)      at 	(2,-2)		{4};
					\node[roundnode]      (n5)      at 	(0,3)		{5};
					\node[roundnode]      (n6)      at 	(0,1)		{6};
					\node[roundnode]      (n7) 	at 	(0,-1)           {7};
					\node[roundnode]      (n8) 	at 	(0,-3)           {8};
					\node[roundnode]      (n9) 	at 	(-2,2)           {9};
					\node[roundnode]      (n10) 	at 	(-2,0)           {10};
					\node[roundnode]      (n11) 	at 	(-2,-2)           {11};
					\node 		(label2) 	at 	(1,-4) 		{$\beta_{0}=3,\beta_{1}=0$};
					
					\begin{pgfonlayer}{background}
					
					\fill[fill=black!20,opacity=1] (n11.mid) -- (n8.mid) -- (n4.mid) -- cycle;
					\fill[fill=black!20,opacity=1] (n11.mid) to[bend left] (n3.mid) -- (n7.mid) -- cycle;
					\fill[fill=black!20,opacity=1] (n11.mid) -- (n7.mid) -- (n1.mid) -- cycle;
					\fill[fill=black!20,opacity=1] (n11.mid) -- (n3.mid) -- (n1.mid) -- cycle;
					\fill[fill=black!20,opacity=1] (n9.mid) -- (n5.mid) -- (n2.mid) -- cycle;
					\fill[fill=black!20,opacity=1] (n7.mid) -- (n3.mid) -- (n1.mid) -- cycle;
					\fill[fill=black!20,opacity=1] (n7.mid) to[bend right] (n0.mid) -- (n3.mid) -- cycle;
					%Lines
					\draw[thick] (n11.mid) --  (n8.mid);
					\draw[thick] (n11.mid) --  (n7.mid);
					\draw[thick] (n11.mid) --  (n4.mid);
					\draw[thick] (n11.mid) --  (n1.mid);
					\draw[thick] (n9.mid) --  (n6.mid);
					\draw[thick] (n9.mid) --  (n5.mid);
					\draw[thick] (n9.mid) --  (n2.mid);
					\draw[thick] (n8.mid) --  (n4.mid);
					\draw[thick] (n7.mid) --  (n3.mid);
					\draw[thick] (n7.mid) --  (n1.mid);
					\draw[thick] (n5.mid) --  (n2.mid);
					\draw[thick] (n3.mid) --  (n1.mid);
					\draw[thick] (n3.mid) --  (n0.mid);
					\end{pgfonlayer}
				\end{tikzpicture}}	
			\end{figure}	
		\endminipage}
		\fbox{\minipage{0.225\textwidth}
			\begin{figure}[H]
				\resizebox{1.0\textwidth}{!}{\begin{tikzpicture}[
					roundnode/.style={circle, draw=black, thick, fill=white, minimum size=7mm},
					]
					%Nodes
					\node           (label) 	at	(1,4) 		{Filtración 4 (t=0.6)};
					\node[roundnode]      (n0) 	at 	(4,1)           {0};
					\node[roundnode]      (n1) 	at 	(4,-1)          {1};
					\node[roundnode]      (n2) 	at 	(2,2)           {2};
					\node[roundnode]      (n3) 	at 	(2,0)           {3};
					\node[roundnode]      (n4)      at 	(2,-2)		{4};
					\node[roundnode]      (n5)      at 	(0,3)		{5};
					\node[roundnode]      (n6)      at 	(0,1)		{6};
					\node[roundnode]      (n7) 	at 	(0,-1)           {7};
					\node[roundnode]      (n8) 	at 	(0,-3)           {8};
					\node[roundnode]      (n9) 	at 	(-2,2)           {9};
					\node[roundnode]      (n10) 	at 	(-2,0)           {10};
					\node[roundnode]      (n11) 	at 	(-2,-2)           {11};
					\node 		(label2) 	at 	(1,-4) 		{$\beta_{0}=3,\beta_{1}=0$};
					
					\begin{pgfonlayer}{background}
					
					\fill[fill=black!20,opacity=1] (n11.mid) -- (n8.mid) -- (n4.mid) -- cycle;
					\fill[fill=black!20,opacity=1] (n11.mid) to[bend left] (n3.mid) -- (n7.mid) -- cycle;
					\fill[fill=black!20,opacity=1] (n11.mid) -- (n7.mid) -- (n1.mid) -- cycle;
					\fill[fill=black!20,opacity=1] (n11.mid) -- (n3.mid) -- (n1.mid) -- cycle;
					\fill[fill=black!20,opacity=1] (n9.mid) -- (n5.mid) -- (n2.mid) -- cycle;
					\fill[fill=black!20,opacity=1] (n7.mid) -- (n3.mid) -- (n1.mid) -- cycle;
					\fill[fill=black!20,opacity=1] (n7.mid) to[bend right] (n0.mid) -- (n3.mid) -- cycle;
					%Lines
					\draw[thick] (n11.mid) --  (n8.mid);
					\draw[thick] (n11.mid) --  (n7.mid);
					\draw[thick] (n11.mid) --  (n4.mid);
					\draw[thick] (n11.mid) --  (n1.mid);
					\draw[thick] (n9.mid) --  (n6.mid);
					\draw[thick] (n9.mid) --  (n5.mid);
					\draw[thick] (n9.mid) --  (n2.mid);
					\draw[thick] (n8.mid) --  (n4.mid);
					\draw[thick] (n7.mid) --  (n3.mid);
					\draw[thick] (n7.mid) --  (n1.mid);
					\draw[thick] (n5.mid) --  (n2.mid);
					\draw[thick] (n3.mid) --  (n1.mid);
					\draw[thick] (n3.mid) --  (n0.mid);
					\end{pgfonlayer}
				\end{tikzpicture}}	
			\end{figure}	
			\begin{figure}[H]
				\resizebox{1.0\textwidth}{!}{\begin{tikzpicture}[
					roundnode/.style={circle, draw=black, thick, fill=white, minimum size=7mm},
					]
					%Nodes
					\node           (label) 	at	(1,4) 		{Filtración 8 (t=0.2)};
					\node[roundnode]      (n0) 	at 	(4,1)           {0};
					\node[roundnode]      (n1) 	at 	(4,-1)          {1};
					\node[roundnode]      (n2) 	at 	(2,2)           {2};
					\node[roundnode]      (n3) 	at 	(2,0)           {3};
					\node[roundnode]      (n4)      at 	(2,-2)		{4};
					\node[roundnode]      (n5)      at 	(0,3)		{5};
					\node[roundnode]      (n6)      at 	(0,1)		{6};
					\node[roundnode]      (n7) 	at 	(0,-1)           {7};
					\node[roundnode]      (n8) 	at 	(0,-3)           {8};
					\node[roundnode]      (n9) 	at 	(-2,2)           {9};
					\node[roundnode]      (n10) 	at 	(-2,0)           {10};
					\node[roundnode]      (n11) 	at 	(-2,-2)           {11};
					\node 		(label2) 	at 	(1,-4) 		{$\beta_{0}=2,\beta_{1}=0$};
					
					\begin{pgfonlayer}{background}
					
					\fill[fill=black!20,opacity=1] (n11.mid) -- (n8.mid) -- (n4.mid) -- cycle;
					\fill[fill=black!20,opacity=1] (n11.mid) to[bend left] (n3.mid) -- (n7.mid) -- cycle;
					\fill[fill=black!20,opacity=1] (n11.mid) -- (n7.mid) -- (n1.mid) -- cycle;
					\fill[fill=black!20,opacity=1] (n11.mid) -- (n3.mid) -- (n1.mid) -- cycle;
					\fill[fill=black!20,opacity=1] (n9.mid) -- (n5.mid) -- (n2.mid) -- cycle;
					\fill[fill=black!20,opacity=1] (n7.mid) -- (n3.mid) -- (n1.mid) -- cycle;
					\fill[fill=black!20,opacity=1] (n7.mid) to[bend right] (n0.mid) -- (n3.mid) -- cycle;
					%Lines
					\draw[thick] (n11.mid) --  (n8.mid);
					\draw[thick] (n11.mid) --  (n7.mid);
					\draw[thick] (n11.mid) --  (n4.mid);
					\draw[thick] (n11.mid) --  (n1.mid);
					\draw[thick] (n10.mid) --  (n7.mid);
					\draw[thick] (n9.mid) --  (n6.mid);
					\draw[thick] (n9.mid) --  (n5.mid);
					\draw[thick] (n9.mid) --  (n2.mid);
					\draw[thick] (n8.mid) --  (n4.mid);
					\draw[thick] (n7.mid) --  (n3.mid);
					\draw[thick] (n7.mid) --  (n1.mid);
					\draw[thick] (n5.mid) --  (n2.mid);
					\draw[thick] (n3.mid) --  (n1.mid);
					\draw[thick] (n3.mid) --  (n0.mid);
					\end{pgfonlayer}
				\end{tikzpicture}}	
			\end{figure}	
		\endminipage}
		\fbox{\minipage{0.225\textwidth}
			\begin{figure}[H]
				\resizebox{1.0\textwidth}{!}{\begin{tikzpicture}[
					roundnode/.style={circle, draw=black, thick, fill=white, minimum size=7mm},
					]
					%Nodes
					\node           (label) 	at	(1,4) 		{Filtración 5 (t=0.5)};
					\node[roundnode]      (n0) 	at 	(4,1)           {0};
					\node[roundnode]      (n1) 	at 	(4,-1)          {1};
					\node[roundnode]      (n2) 	at 	(2,2)           {2};
					\node[roundnode]      (n3) 	at 	(2,0)           {3};
					\node[roundnode]      (n4)      at 	(2,-2)		{4};
					\node[roundnode]      (n5)      at 	(0,3)		{5};
					\node[roundnode]      (n6)      at 	(0,1)		{6};
					\node[roundnode]      (n7) 	at 	(0,-1)           {7};
					\node[roundnode]      (n8) 	at 	(0,-3)           {8};
					\node[roundnode]      (n9) 	at 	(-2,2)           {9};
					\node[roundnode]      (n10) 	at 	(-2,0)           {10};
					\node[roundnode]      (n11) 	at 	(-2,-2)           {11};
					\node 		(label2) 	at 	(1,-4) 		{$\beta_{0}=3,\beta_{1}=0$};
					
					\begin{pgfonlayer}{background}
					
					\fill[fill=black!20,opacity=1] (n11.mid) -- (n8.mid) -- (n4.mid) -- cycle;
					\fill[fill=black!20,opacity=1] (n11.mid) to[bend left] (n3.mid) -- (n7.mid) -- cycle;
					\fill[fill=black!20,opacity=1] (n11.mid) -- (n7.mid) -- (n1.mid) -- cycle;
					\fill[fill=black!20,opacity=1] (n11.mid) -- (n3.mid) -- (n1.mid) -- cycle;
					\fill[fill=black!20,opacity=1] (n9.mid) -- (n5.mid) -- (n2.mid) -- cycle;
					\fill[fill=black!20,opacity=1] (n7.mid) -- (n3.mid) -- (n1.mid) -- cycle;
					\fill[fill=black!20,opacity=1] (n7.mid) to[bend right] (n0.mid) -- (n3.mid) -- cycle;
					%Lines
					\draw[thick] (n11.mid) --  (n8.mid);
					\draw[thick] (n11.mid) --  (n7.mid);
					\draw[thick] (n11.mid) --  (n4.mid);
					\draw[thick] (n11.mid) --  (n1.mid);
					\draw[thick] (n9.mid) --  (n6.mid);
					\draw[thick] (n9.mid) --  (n5.mid);
					\draw[thick] (n9.mid) --  (n2.mid);
					\draw[thick] (n8.mid) --  (n4.mid);
					\draw[thick] (n7.mid) --  (n3.mid);
					\draw[thick] (n7.mid) --  (n1.mid);
					\draw[thick] (n5.mid) --  (n2.mid);
					\draw[thick] (n3.mid) --  (n1.mid);
					\draw[thick] (n3.mid) --  (n0.mid);
					\end{pgfonlayer}
				\end{tikzpicture}}	
			\end{figure}	
			\begin{figure}[H]
				\resizebox{1.0\textwidth}{!}{\begin{tikzpicture}[
					roundnode/.style={circle, draw=black, thick, fill=white, minimum size=7mm},
					]
					%Nodes
					\node           (label) 	at	(1,4) 		{Filtración 9 (t=0.1)};
					\node[roundnode]      (n0) 	at 	(4,1)           {0};
					\node[roundnode]      (n1) 	at 	(4,-1)          {1};
					\node[roundnode]      (n2) 	at 	(2,2)           {2};
					\node[roundnode]      (n3) 	at 	(2,0)           {3};
					\node[roundnode]      (n4)      at 	(2,-2)		{4};
					\node[roundnode]      (n5)      at 	(0,3)		{5};
					\node[roundnode]      (n6)      at 	(0,1)		{6};
					\node[roundnode]      (n7) 	at 	(0,-1)           {7};
					\node[roundnode]      (n8) 	at 	(0,-3)           {8};
					\node[roundnode]      (n9) 	at 	(-2,2)           {9};
					\node[roundnode]      (n10) 	at 	(-2,0)           {10};
					\node[roundnode]      (n11) 	at 	(-2,-2)           {11};
					\node 		(label2) 	at 	(1,-4) 		{$\beta_{0}=1,\beta_{1}=1$};
					
					\begin{pgfonlayer}{background}
					
					\fill[fill=black!20,opacity=1] (n11.mid) -- (n8.mid) -- (n4.mid) -- cycle;
					\fill[fill=black!20,opacity=1] (n11.mid) to[bend left] (n3.mid) -- (n7.mid) -- cycle;
					\fill[fill=black!20,opacity=1] (n11.mid) -- (n7.mid) -- (n4.mid) -- cycle;
					\fill[fill=black!20,opacity=1] (n11.mid) -- (n7.mid) -- (n1.mid) -- cycle;
					\fill[fill=black!20,opacity=1] (n11.mid) -- (n3.mid) -- (n1.mid) -- cycle;
					\fill[fill=black!20,opacity=1] (n10.mid) -- (n7.mid) -- (n3.mid) -- cycle;
					\fill[fill=black!20,opacity=1] (n10.mid) -- (n7.mid) -- (n0.mid) -- cycle;
					\fill[fill=black!20,opacity=1] (n10.mid) -- (n3.mid) -- (n1.mid) -- cycle;
					\fill[fill=black!20,opacity=1] (n10.mid) -- (n3.mid) -- (n0.mid) -- cycle;
					\fill[fill=black!20,opacity=1] (n9.mid) -- (n6.mid) -- (n2.mid) -- cycle;
					\fill[fill=black!20,opacity=1] (n9.mid) -- (n5.mid) -- (n2.mid) -- cycle;
					\fill[fill=black!20,opacity=1] (n9.mid) -- (n5.mid) -- (n0.mid) -- cycle;
					\fill[fill=black!20,opacity=1] (n7.mid) -- (n3.mid) -- (n1.mid) -- cycle;
					\fill[fill=black!20,opacity=1] (n7.mid) to[bend right] (n0.mid) -- (n3.mid) -- cycle;
					\fill[fill=black!20,opacity=1] (n5.mid) to[bend left] (n0.mid) -- (n2.mid) -- cycle;
					%Lines
					\draw[thick] (n11.mid) --  (n8.mid);
					\draw[thick] (n11.mid) --  (n7.mid);
					\draw[thick] (n11.mid) --  (n4.mid);
					\draw[thick] (n11.mid) --  (n1.mid);
					\draw[thick] (n10.mid) --  (n7.mid);
					\draw[thick] (n10.mid) --  (n6.mid);
					\draw[thick] (n10.mid) --  (n3.mid);
					\draw[thick] (n10.mid) --  (n1.mid);
					\draw[thick] (n10.mid) --  (n0.mid);
					\draw[thick] (n9.mid) --  (n6.mid);
					\draw[thick] (n9.mid) --  (n5.mid);
					\draw[thick] (n9.mid) --  (n2.mid);
					\draw[thick] (n9.mid) --  (n0.mid);
					\draw[thick] (n8.mid) --  (n4.mid);
					\draw[thick] (n7.mid) --  (n4.mid);
					\draw[thick] (n7.mid) --  (n3.mid);
					\draw[thick] (n7.mid) --  (n1.mid);
					\draw[thick] (n6.mid) --  (n2.mid);
					\draw[thick] (n5.mid) --  (n2.mid);
					\draw[thick] (n3.mid) --  (n1.mid);
					\draw[thick] (n3.mid) --  (n0.mid);
					\draw[thick] (n2.mid) --  (n0.mid);
					\end{pgfonlayer}
				\end{tikzpicture}}	
			\end{figure}	
	\endminipage}
		\fbox{\minipage{0.225\textwidth}
			\begin{figure}[H]
				\resizebox{1.0\textwidth}{!}{\begin{tikzpicture}[
					roundnode/.style={circle, draw=black, fill=white, thick, minimum size=7mm},
					]
					%Nodes
					\node           (label) 	at	(1,4) 		{Filtración 6 (t=0.4)};
					\node[roundnode]      (n0) 	at 	(4,1)           {0};
					\node[roundnode]      (n1) 	at 	(4,-1)          {1};
					\node[roundnode]      (n2) 	at 	(2,2)           {2};
					\node[roundnode]      (n3) 	at 	(2,0)           {3};
					\node[roundnode]      (n4)      at 	(2,-2)		{4};
					\node[roundnode]      (n5)      at 	(0,3)		{5};
					\node[roundnode]      (n6)      at 	(0,1)		{6};
					\node[roundnode]      (n7) 	at 	(0,-1)           {7};
					\node[roundnode]      (n8) 	at 	(0,-3)           {8};
					\node[roundnode]      (n9) 	at 	(-2,2)           {9};
					\node[roundnode]      (n10) 	at 	(-2,0)           {10};
					\node[roundnode]      (n11) 	at 	(-2,-2)           {11};
					\node 		(label2) 	at 	(1,-4) 		{$\beta_{0}=3,\beta_{1}=0$};
					
					\begin{pgfonlayer}{background}
					
					\fill[fill=black!20,opacity=1] (n11.mid) -- (n8.mid) -- (n4.mid) -- cycle;
					\fill[fill=black!20,opacity=1] (n11.mid) to[bend left] (n3.mid) -- (n7.mid) -- cycle;
					\fill[fill=black!20,opacity=1] (n11.mid) -- (n7.mid) -- (n1.mid) -- cycle;
					\fill[fill=black!20,opacity=1] (n11.mid) -- (n3.mid) -- (n1.mid) -- cycle;
					\fill[fill=black!20,opacity=1] (n9.mid) -- (n5.mid) -- (n2.mid) -- cycle;
					\fill[fill=black!20,opacity=1] (n7.mid) -- (n3.mid) -- (n1.mid) -- cycle;
					\fill[fill=black!20,opacity=1] (n7.mid) to[bend right] (n0.mid) -- (n3.mid) -- cycle;
					%Lines
					\draw[thick] (n11.mid) --  (n8.mid);
					\draw[thick] (n11.mid) --  (n7.mid);
					\draw[thick] (n11.mid) --  (n4.mid);
					\draw[thick] (n11.mid) --  (n1.mid);
					\draw[thick] (n9.mid) --  (n6.mid);
					\draw[thick] (n9.mid) --  (n5.mid);
					\draw[thick] (n9.mid) --  (n2.mid);
					\draw[thick] (n8.mid) --  (n4.mid);
					\draw[thick] (n7.mid) --  (n3.mid);
					\draw[thick] (n7.mid) --  (n1.mid);
					\draw[thick] (n5.mid) --  (n2.mid);
					\draw[thick] (n3.mid) --  (n1.mid);
					\draw[thick] (n3.mid) --  (n0.mid);
					\end{pgfonlayer}
				\end{tikzpicture}}	
			\end{figure}	
			\begin{figure}[H]
				\resizebox{1.0\textwidth}{!}{\begin{tikzpicture}[
					roundnode/.style={circle, draw=black, thick, fill=white, minimum size=7mm},
					]
					%Nodes
					\node           (label) 	at	(1,4) 		{Filtración 10 (t=0.0)};
					\node[roundnode]      (n0) 	at 	(4,1)           {0};
					\node[roundnode]      (n1) 	at 	(4,-1)          {1};
					\node[roundnode]      (n2) 	at 	(2,2)           {2};
					\node[roundnode]      (n3) 	at 	(2,0)           {3};
					\node[roundnode]      (n4)      at 	(2,-2)		{4};
					\node[roundnode]      (n5)      at 	(0,3)		{5};
					\node[roundnode]      (n6)      at 	(0,1)		{6};
					\node[roundnode]      (n7) 	at 	(0,-1)           {7};
					\node[roundnode]      (n8) 	at 	(0,-3)           {8};
					\node[roundnode]      (n9) 	at 	(-2,2)           {9};
					\node[roundnode]      (n10) 	at 	(-2,0)           {10};
					\node[roundnode]      (n11) 	at 	(-2,-2)           {11};
					\node 		(label2) 	at 	(1,-4) 		{$\beta_{0}=1,\beta_{1}=0$};
					
					\begin{pgfonlayer}{background}
					
					\fill[fill=black!20,opacity=1] (n11.mid) -- (n8.mid) -- (n4.mid) -- cycle;
					\fill[fill=black!20,opacity=1] (n11.mid) -- (n8.mid) -- (n1.mid) -- cycle;
					\fill[fill=black!20,opacity=1] (n11.mid) -- (n7.mid) -- (n4.mid) -- cycle;
					\fill[fill=black!20,opacity=1] (n11.mid) to[bend left] (n3.mid) -- (n7.mid) -- cycle;
					\fill[fill=black!20,opacity=1] (n10.mid) to[bend right] (n4.mid) -- (n7.mid) -- cycle;
					\fill[fill=black!20,opacity=1] (n10.mid) -- (n7.mid) -- (n3.mid) -- cycle;
					\fill[fill=black!20,opacity=1] (n10.mid) -- (n7.mid) -- (n1.mid) -- cycle;
					\fill[fill=black!20,opacity=1] (n10.mid) -- (n6.mid) -- (n3.mid) -- cycle;
					\fill[fill=black!20,opacity=1] (n10.mid) to[bend left] (n2.mid) -- (n6.mid) -- cycle;
					\fill[fill=black!20,opacity=1] (n9.mid) to[bend right] (n3.mid) -- (n6.mid) -- cycle;
					\fill[fill=black!20,opacity=1] (n9.mid) -- (n6.mid) -- (n2.mid) -- cycle;
					\fill[fill=black!20,opacity=1] (n9.mid) -- (n6.mid) -- (n0.mid) -- cycle;
					\fill[fill=black!20,opacity=1] (n9.mid) -- (n5.mid) -- (n2.mid) -- cycle;
					\fill[fill=black!20,opacity=1] (n9.mid) -- (n5.mid) -- (n0.mid) -- cycle;
					\fill[fill=black!20,opacity=1] (n8.mid) to[bend right] (n1.mid) -- (n4.mid) -- cycle;
					\fill[fill=black!20,opacity=1] (n7.mid) to[bend right] (n0.mid) -- (n3.mid) -- cycle;
					\fill[fill=black!20,opacity=1] (n7.mid) -- (n4.mid) -- (n1.mid) -- cycle;
					\fill[fill=black!20,opacity=1] (n7.mid) -- (n3.mid) -- (n1.mid) -- cycle;
					\fill[fill=black!20,opacity=1] (n6.mid) to[bend left] (n1.mid) -- (n3.mid) -- cycle;
					\fill[fill=black!20,opacity=1] (n6.mid) -- (n3.mid) -- (n0.mid) -- cycle;
					\fill[fill=black!20,opacity=1] (n5.mid) to[bend left] (n0.mid) -- (n2.mid) -- cycle;
					%Lines
					\draw[thick] (n11.mid) --  (n8.mid);
					\draw[thick] (n11.mid) --  (n7.mid);
					\draw[thick] (n11.mid) --  (n4.mid);
					\draw[thick] (n11.mid) --  (n1.mid);
					\draw[thick] (n10.mid) --  (n7.mid);
					\draw[thick] (n10.mid) --  (n6.mid);
					\draw[thick] (n10.mid) --  (n3.mid);
					\draw[thick] (n10.mid) --  (n1.mid);
					\draw[thick] (n10.mid) --  (n0.mid);
					\draw[thick] (n9.mid) --  (n6.mid);
					\draw[thick] (n9.mid) --  (n5.mid);
					\draw[thick] (n9.mid) --  (n2.mid);
					\draw[thick] (n9.mid) --  (n0.mid);
					\draw[thick] (n8.mid) --  (n4.mid);
					\draw[thick] (n7.mid) --  (n4.mid);
					\draw[thick] (n7.mid) --  (n3.mid);
					\draw[thick] (n7.mid) --  (n1.mid);
					\draw[thick] (n6.mid) --  (n3.mid);
					\draw[thick] (n6.mid) --  (n2.mid);
					\draw[thick] (n6.mid) --  (n0.mid);
					\draw[thick] (n5.mid) --  (n2.mid);
					\draw[thick] (n4.mid) --  (n1.mid);
					\draw[thick] (n3.mid) --  (n1.mid);
					\draw[thick] (n3.mid) --  (n0.mid);
					\draw[thick] (n2.mid) --  (n0.mid);
					\end{pgfonlayer}
				\end{tikzpicture}}	
			\end{figure}	
		\endminipage}
		\caption{Pasos en la filtración del complejo simplicial 
		asociado a \ref{fig:distExtR} y números de Betti 
		correspondientes (interpretación global).}
		\label{fig:distExtRSG}
	\end{figure}
	\begin{figure}[!htbp]
		\minipage{0.5\textwidth}
			\begin{figure}[H]
				\resizebox{1.0\textwidth}{!}{\includegraphics{Images/DiagramaBarrasEjEXTGLOBAL.png}}
			\end{figure}
		\endminipage
		\minipage{0.5\textwidth}
			\begin{figure}[H]
				\resizebox{1.0\textwidth}{!}{\includegraphics{Images/DiagramaPersistenciaEjEXTGLOBAL.png}}
			\end{figure}
		\endminipage
		\caption{Diagrama de barras (izquierda) y diagrama de 
		persistencia (derecha) asociados a la filtración ilustrada en 
		\ref{fig:distExtRSG} (interpretación global).}
		\label{fig:distExtRDG}
	\end{figure}

	En este último caso, si observamos detenidamente los pasos en la 
	filtración del complejo simplicial en la \autoref{fig:distIntRSL} y en 
	la \autoref{fig:distIntRSG}, podemos ver que no hay diferencia alguna 
	entre ambas interpretaciones. Tampoco podemos apreciar ninguna 
	diferencia entre los diagramas que mostramos en 
	la \autoref{fig:distExtRDL} y en la \autoref{fig:distExtRDG}.

	Esta concordancia, que inicialmente puede parecer sorprendente, 
	queda explicada por la disparidad en la distribución de los pesos, y 
	por tanto, de las importancias de la red. Esta gran desviación provoca 
	que los complejos simpliciales asociados a cada filtración se 
	mantengan constantes a lo largo del proceso de filtración, y que al 
	final de este proceso cambien rápidamente.

	No obstante, si quisiéramos profundizar en las diferencias entre ambas 
	interpretaciones aplicadas a este ejemplo, únicamente tendríamos que 
	cambiar la 
	granularidad de las filtraciones, esto es, modificar el valor del 
	parámetro que regula la construcción del complejo simplicial (véase la 
	definición \ref{def:SimpAut}). Como nuestro objetivo es de 
	comparar el funcionamiento de la homología persistente en función de 
	la distribución de los pesos de la red, mantendremos la granularidad 
	que hemos empleado en los ejemplos anteriores.
	\end{ejem}

	A la vista de los anteriores ejemplos, discutimos a continuación el 
	funcionamiento de la homología persistente en función de la 
	distribución de los pesos de la red, así como del desempeño de ambas 
	interpretaciones a lo largo de dichos ejemplos.

	En lo referente a la distribución de los pesos, observamos que la 
	homología persistente nos aporta información de mayor calidad en 
	distribuciones homogéneas. Con esto nos referimos a que, en aquellas 
	distribuciones más equitativas, la homología persistente refleja un 
	mayor número de combinaciones entre neuronas. Además, estas 
	combinaciones persisten a lo largo de un mayor número de filtraciones 
	que en otras distribuciones menos equitativas. Este resultado, aunque
	esperable, nos sugiere que la homología persistente puede constituir 
	una sólida herramienta para el análisis de redes neuronales. Ahora 
	bien, para argumentar que la homología persistente realmente refleja 
	el grado de complejidad adquirido por las redes (tal y como teoriza el 
	autor en \cite{Articulo-Watanabe}) son necesarios una serie de 
	experimentos con redes de un tamaño mayor a las expuestas en el 
	presente trabajo. Desafortunadamente, la realización de estos 
	experimentos queda fuera del alcance de este trabajo.

	En cuanto al desempeño de ambas interpretaciones, podemos observar 
	que, en líneas generales, la interpretación local (que se corresponde 
	con la original del autor) captura un número mayor de clases de 
	homología, lo que le permite aportar una mayor información sobre las 
	redes estudiadas. Por otra parte, como ya hemos comentado 
	anteriormente, esta ventaja puede ser a la vez un inconveniente. Esto 
	se debe a que, al tener un mayor grado de sensibilidad, la 
	interpretación local captura un número mayor de clases de homología 
	que representan ruido, ensuciando así los diagramas obtenidos. Al 
	igual que con la distribución de los pesos, son necesarios más 
	experimentos para profundizar en las diferencias entre ambas 
	interpretaciones antes de descartar cualquiera de ellas.

	\chapter{Conclusiones}
	A lo largo de este trabajo fin de grado he expuesto las dos 
	interpretaciones del artículo \cite{Articulo-Watanabe} que, con la 
	inestimable ayuda de mis tutores, he podido analizar en profundidad. 
	Por desgracia, finalizo esta memoria sin decantarme por ninguna de las 
	dos, pues ambas tienen sus ventajas e inconvenientes, y la realización 
	de los experimentos necesarios hubiera alargado excesivamente esta 
	memoria.

	La inteligencia 
	artificial es un tema que siempre me ha parecido fascinante y este 
	trabajo me ha permitido explorar una dimensión del tema que me 
	resultaba completamente desconocida: las matemáticas subyacentes. Con 
	estas 
	matemáticas he tenido la ocasión de descubrir la topología algebraica. 
	Desconocía totalmente esta disciplina y la he encontrado de una 
	belleza sublime, pues combina dos de las ramas que me parecen más 
	interesantes del grado: la topología y el álgebra. Además, la 
	realización de este trabajo ha requerido la revisión de multitud de 
	conceptos de estas materias, así como de Teoría de Grafos y Análisis 
	Real; y ha sido en esta revisión y en la investigación posterior donde 
	he podido apreciar la importancia de los conceptos que 
	he aprendido en el transcurso de esta titulación. 
	
	A título personal, creo que como estudiante de la doble titulación el 
	tema de este trabajo se ha ajustado perfectamente a mi formación, y me 
	gustaría expresar mi más sincera gratitud con el departamento de 
	Matemáticas y Computación, y con mis tutores José 
	Luis y Julio en particular, por brindarme la oportunidad de trabajar 
	en un tema que me ha resultado apasionante de principio a fin, ya que 
	combina a la perfección matemáticas con inteligencia artificial. He 
	disfrutado verdaderamente con la labor de investigación realizada en 
	este trabajo y espero tener la oportunidad de repetir esta experiencia 
	en el futuro. 
	\nocite{*}
	\bibliographystyle{babplain}
	\bibliography{Referencias}
\end{document} 

